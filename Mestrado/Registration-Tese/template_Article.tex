%%% Exemplo de utiliza��o da classe ITA
%%%
%%%   por        F�bio Fagundes Silveira   -  ffs [at] ita [dot] br
%%%              Benedito C. O. Maciel     -  bcmaciel [at] ita [dot] br
%%%              Giovani Volnei Meinertz   -  giovani [at] ita [dot] br
%%%
%%%  IMPORTANTE: O texto contido neste exemplo nao significa absolutamente nada.  :-)
%%%              O intuito aqui eh demonstrar os comandos criados na classe e suas
%%%              respectivas utilizacoes.
%%%
%%%  $Id: ExemploTeseITA.tex 17 2006-02-07 17:59:17Z ffs $
%%%  $HeadURL: file:///opt/repositorioITALUS/classeITA/tags/versao-2.1/ExemploTeseITA.tex $
%%%
%%% ITALUS
%%% Technological Institute of Aeronautics --- ITA
%%% Sao Jose dos Campos, Brazil
%%% HomePages:        http://www.comp.ita.br/italus
%%%                   http://groups.yahoo.com/group/italus/
%%% Discussion list: italus {at} yahoogroups.com
%%%
%++++++++++++++++++++++++++++++++++++++++++++++++++++++++++++++++++++++++++++++
% Parametros da classe ITA:
%   msc   = Tese de Mestrado    --> no ITA, Dissertacao eh Tese ...  :-)
%   dsc   = Tese de Doutorado
%   quali = Exame de Qualificacao
%   dv    = 'Draft Version'     --> imprime 'Versao Preliminar + data no rodape
%   fem   = Doutora
%   eng   = para teses em ingl�s
%++++++++++++++++++++++++++++++++++++++++++++++++++++++++++++++++++++++++++++++


\documentclass[msc]{ita}    % ITA.cls based on standard book.cls
\usepackage[section]{placeins}

 \usepackage{amsmath}
\usepackage{float}
\usepackage{pgfplots}
\usepackage{tikz}
\usetikzlibrary{arrows, decorations.markings}

% for double arrows a la chef
% adapt line thickness and line width, if needed
\tikzstyle{vecArrow} = [thick, decoration={markings,mark=at position
   1 with {\arrow[semithick]{open triangle 60}}},
   double distance=1.4pt, shorten >= 5.5pt,
   preaction = {decorate},
   postaction = {draw,line width=1.4pt, white,shorten >= 4.5pt}]
\tikzstyle{innerWhite} = [semithick, white,line width=1.4pt, shorten >= 4.5pt]


\usepackage{xcolor,colortbl}
\pgfplotsset{compat=1.8}
\usepackage{pgfkeys}

%\excludecomment{comment}  %do not show comments
%++++++++++++++++++++++++++++++++++++++++++++++++++++++++++++++++++++++++++++++
% Identificacoes...
%++++++++++++++++++++++++++++++++++++++++++++++++++++++++++++++++++++++++++++++
\course{Engenharia Eletr�nica e Computa��o}
\dept{Ci�ncia da Computa��o}
\area{Inform�tica}

% Autor do trabalho: Nome Sobrenome
\author{Bruno}{Duarte Corr�a}

% Endereco do Autor -> utilizado no verso da folha de rosto
% Obrigat�rio para Teses
\itaauthoraddress{Av. Juscelino Kubistcheck, 6701 ap 14 bl
21}{12.220-000}{S�o Jos� dos Campos--SP}

% Titulo da Tese/Dissertacao
\title{Avalia��o de t�cnicas de reconhecimento de padr�es em ambientes
aeronauticos}

% Orientador
\advisor{Prof.~Dr.}{Lu�s Gonzaga Trabasso}

% Co-orientador opcional
\coadvisor{}{Nelson Jos� Issa de Macedo}{EMBRAER}

% Chefe da divisao de Pos-Graduacao
% Obrigat�rio para Teses
\boss{Prof.~Dr.}{Luiz Carlos Sandoval G�es}

% Banca Examinadora
% Obrigat�rio para Teses
\examiner{Prof. Dr.}{Lu�s Gonzaga Trabasso}{Presidente}{ITA}
\examiner{Prof. Dr.}{Ricardo Bedin Fran�a}{Membro Externo}{EMBRAER}%
\examiner{Prof. Dr.}{Em�lia}{Membro}{ITA}%
%
%\examiner{Prof. Dr.}{Cicrano Fulano}{Membro Externo}{UXXX}%
%\examiner{Prof. Dr.}{Beltrano Cicrano}{Membro Externo}{UYYY}%
%\examiner{Prof. Dr.}{Fulano de Tal}{Membro}{ITA}%
%\examiner{Prof. Dr.}{Beltrano Fulano}{Membro}{ITA}%

% Data da defesa
\date{Fevereiro}{2015}

% Palavras-Chaves informadas pela Biblioteca -> utilizada na CIP
% Obrigat�rio para Teses
\kwcip{augmented reality}
\kwcip{opencv}
\kwcip{computer vision}
\kwcip{maintenance}
\kwcip{feature matching}
\kwcip{descriptors}
\kwcip{markerless}

% Glossario
\makeglossary
\frontmatter

\begin{document}

% Folha de Rosto
\maketitle

% Dedicatoria
\begin{itadedication}
Dedico esse trabalho primeiramente a Deus e a todas as pessoas que me apoiaram,
acreditaram e incentivaram, sem os quais com toda certeza n�o teria obtido os
resultados que esperava.


\end{itadedication}

% Agradecimentos
%\begin{itathanks}
%\input{parts/agradecimentos}
%\end{itathanks}

% Ep�grafe
\thispagestyle{empty}
\ifhyperref\pdfbookmark[0]{\nameepigraphe}{epigrafe}\fi
\begin{flushright}
\begin{spacing}{1}
\mbox{}\vfill
{\sffamily\itshape

``Persistence is the shortest path to success.''\\}
--- \textsc{Charles Chaplin}
\end{spacing}
\end{flushright}

% Resumo
\begin{abstract}
O reconhecimento de objetos em uma cena para posterior uso em realidade aumentada 
depende de diversas vari�veis, causando a necessidade do uso de t�cnicas 
espec�ficas para cada cen�rio, sendo portanto, um estudo de fronteiras para a melhor escolha 
do algoritmo de reconhecimento, de acordo com a aplica��o em quest�o de grande
valia para o meio acad�mico. 
Esta tese se prop�e a pesquisar, categorizar e tra�ar fronteiras das t�cnicas
conhecidas, tendo como caso de uso a manuten��o de aeronaves feita dentro de
centros fechados, utilizando as t�cnicas BRISK,FAST,FREAK,GFTT,MSER,
 ORB,STAR,SURF,SIFT em uma an�lise aplicada com imagens reais de janelas de
 inspe��o do Embraer ERJ-190 para reconhecimento de objetos e posteriores
 aplica��es em manuten��o.
 Comparando todas as t�cnicas quanto � cad�ncia e � precis�o de reconhecimento
 de caracter�sticas, � poss�vel selecionar GFTT e ORB
 como t�cnicas mais apropriadas ao contexto, por terem seus resultados de
 varia��o de rota��o, escala, briho e \emph{blur} dentro de uma faixa esperada
 para o contexto de manuten��o.
 


\end{abstract}
% Palavras Chave
% No manual nao consta palavras-chaves
%\keywords{Teses, Estilos, Italus}

% Abstract
\begin{englishabstract}
There are several augmented reality techniques, although each one has its flaws due to the environment or other external constraints. The study of boundaries and constraints can provide to the developer more decision power while choosing the appropriate technique. This essay provides a method to, based on common parameters and situation, chose the appropriated technique.
\end{englishabstract}
% Keywords
% Idem Palavras-chaves ...
%\englishkeywords{Theses, Styles, Italus}

% sumario
\tableofcontents
% lista de figuras
\listoffigures
% lista de tabelas
\listoftables
% lista de abreviaturas
\listofabbreviations
\begin{table}[h]
\begin{tabular}{|l|l|lll}
\label{table:acronim}
\cline{1-2}
HMD & Head-Mounted Display   &  &  &  \\ \cline{1-2}
AR  & Augmented Reality      &  &  &  \\ \cline{1-2}
VR  & Virtual Reality      &  &  &  \\ \cline{1-2}
DoG & Difference of Gaucians &  &  &  \\ \cline{1-2}
LoG & Laplacian of Gaucians  &  &  &  \\ \cline{1-2}
\end{tabular}
\end{table}
% lista de simbolos
%\listofsymbols
%\input{parts/listasimbolos}

\mainmatter
% Os capitulos comecam aqui

\chapter{Introdu��o}
	\label{ch:introducao}
	\section{Motiva��o}

O reconhecimento de estruturas e sistemas de forma autom�tica no campo da
manuten��o, pode propiciar a constru��o de ferramentas de capacita��o, dentre
v�rias outras possibilidades, auxiliando al�m de garantir maior confiabilidade
no diagn�stico de problemas. 
Um dos mais b�sicos problemas atualmente limitando o ramo da Realidade Aumentada
� a etapa de registro.
A Realidade Aumentada prev� imers�o entre o mundo virtual e o mundo real e por
isso para que a experi�ncia seja coerente � necess�rio que os dois mundos estejam bem sincronizados e propriamente alinhados.
Em algumas situa��es tal sincronia aumenta a experi�ncia, entretanto, tal
alinhamento � primordial, por exemplo em aplica��es m�dicas em uma aplica��o de
biopsia.Se o objeto n�o estiver no espa�o e tempo real, a informa��o
fornecida ao cirurgi�o poder� por em risco a vida do paciente. Na maioria das aplica��es de tempo real, problemas
 de registro podem invalidar o uso da Realidade Aumentada.
Um outro problema que pode ocorrer com falhas de registro � acentuado por um fen�meno conhecido como
 visual capture \cite{Welch78} que � a tend�ncia do c�rebro em capturar com
 mais qualidade est�mulos visuais, do que qualquer outro sentido. 
  Nesses casos, o sentido da vis�o tende a sobrepor os outros sentidos.
Assim como um ventr�loco consegue enganar quem assiste um show acreditando que o
som sai da boca do boneco o usu�rio de uma aplica��o de realidade aumentada tender� a acreditar no que v�, 
mesmo que esteja defasado no espa�o/tempo.
No caso do erro se tornar sistem�tico o usu�rio tende a se acostumar
inconscientemente adaptar-se ao erro, corrigindo o efeito.
Erros de registro s�o dif�ceis de controlar adequadamente devido � grande precis�o requerida das diversas fontes 
de erro. As fontes de erro podem ser divididas em est�ticas e din�micas sendo as est�ticas contornadas com calibra��o
 pr�via de sensores entretanto os erros din�micos s�o mais dif�ceis porque s�o suscept�veis a tempo diferen�a de tempo
  entre o real e o apresentado na tela e com o ac�mulo de erro.
O reconhecimento de objetos na cena permeia tamb�m:
\begin{itemize}
  \item O contexto da cena, sendo que com conhecimento pr�vio do cen�rio se torna bem mais f�cil;
  \item O material do qual o objeto � feito, porque caso seja feito de materiais reflexivos, os algoritmos
   podem confundir o reflexo de outros objetos com informa��es a reconhecer;
  \item O tamanho do objeto, pois de acordo com a escala do objeto, muitas informa��es que poderiam ser
   boas para o reconhecimento podem estar pr�ximas demais dificultando o posteior casamento de informa��es. \ldots
\end{itemize}

Portanto para que as diversas fontes de erros din�micos n�o sejam um impeditivo para o reconhecimento, de 
acordo com a cena, algoritmos diferentes devem ser selecionados por terem peculiaridades e caracter�sticas
 que garantam um registro direcionados ao tipo de desafio que encontrar�o, al�m de j� ter informa��es 
 pr�vias, o que facilita na sele��o de caracter�sticas.


	\section{Objetivos}
O presente trabalho tem como objetivo geral avaliar o ambiente de manuten��o
aeron�utico, no contexto de janelas de inspe��o, tra�ando estrat�gias de
reconhecimento de items de manuten��o.
Para a consecu��o do objetivo geral, foram definidos os seguintes objetivos espec�ficos:


\begin{itemize}
\item Avaliar os algoritmos cl�ssicos de reconhecimento;
\item Aplicar os algoritmos cl�ssicos � situa��es reais;
\item Selecionar algoritmo mais adequado para o contexto.
\end{itemize}
	
	
Esta tese prop�e o uso da realidade aumentada no cen�rio de manuten��o de
aeronaves, tendo como objetivo determinar a melhor estrat�gia de reconhecimento
de caracter�sticas dos objetos no contexto, para que posteriormente seja
utilizado em ferramentas de auxilio na manuten��o.
	\section{Justificativa}

Muitas s�o as abordagens de compara��o entre detectores de caracter�sticas
ultimamente.
Uma an�lise de performance entre descritores locais feita em
\cite{performance_evaluation} levanta a metodologia, bem como a m�trica
utilizada nesse trabalho.
Uma an�lise comparativa feita com tr�s descritores bin�rios
(ORB,BRIEF e BRISK)\cite{binaryfeatures} utilizando os mais conhecidos
detectores (ORB, MSER, SIFT, SURF, FAST e BRISK) � realizada comparando-se o
efeito de transforma��es geom�tricas e fotom�tricas.

Tamb�m um estudo comparativo de desempenho demonstrado em \cite{lowlevelfeature}
apresenta resultados para FAST-SIFT comparando varia��es de blur, ilumina��o,
escala e rota��o.

A an�lise do desempenho de v�rios descritores foi feita no contexto de navega��o
rob�tica em \cite{robotnavigation} decidindo pelo melhor detector e descritor
para navega��o de rob�s.

O desempenho dos detectores (FREAK vs. SURF vs. BRISK) � examinado no contexto
de detec��o de pedestres em \cite{Pedestrian}.
Estudos sobre a descritores locais tamb�m s�o feitos em
\cite{localfeaturedetector}, descrevendo e comparando os principais descritores
quanto � suas peculiaridades.

Uma an�lise de sete combina��es de conhecidos detectores e descritores, a saber,
SIFT com SIFT, SURF com SURF, MSER com SIFT, BRISK com FREAK, BRISK com
BRISK, ORB com ORB e FAST com BRIEF � apresentado em \cite{Well-Known}

O presente trabalho apresenta a compara��o de nove combina��es de detectores e
descritores, a saber, BRISK com BRISK, STAR com BRIEF, MSER com SIFT, ORB com
ORB, FAST com BRIEF, FREAK com FREAK, GFTT com BRIEF, SURF com SURF e SIFT com
SIFT e prop�e um estudo de caso do ambiente de aeron�utico com um m�todo pr�
estabelecido de sele��o de limites de restri��es, com janelas de decis�o pr� de
tal forma a eliminar as combina��es que n�o estejam dentro da regi�o desejada.

	\section{Contextualiza��o}
\label{sec:restricao}

O contexto dessa tese prev� o cen�rio de manuten��o com o uso de realidade aumentada como uma ferramenta
 para aux�lio nas tarefas rotineiras. Algumas vari�veis devem ser
 consideradas para garantir a viabilidade de implanta��o da abordagem:
\begin{description}
\item [Velocidade de reconhecimento] para que a aplica��o seja utilizada pelo
usu�rio com um taxa aceit�vel, garantindo assim uma experi�ncia ;
\item [Qualidade do reconhecimento] de objetos para que sejam encontrados pela
t�cnica com resultados compar�veis a um mec�nico;
\item [Invari�ncia a par�metros ambientais] para que seja poss�vel emular
situa��es reais do dia a dia como altera��o entre momentos do dia, ambientes
esfuma�ados, etc \ldots
\end{description}



\subsection{Cen�rio}

O uso da realidade aumentada em manuten��o de aeronaves pode trazer ganho no
fornecimento de informa��es de procedimentos, na previs�o de
falhas ou no reconhecimento de regi�es com falha.

 Como caso de uso ser� adotado a janela de inspe��o frontal, como mostrado na
 Figura~\ref{fig:ERJ190}, localizada na aeronave Embraer ERJ-190. 
 
\begin{figure}[h!]
\centering
\includegraphics[scale=0.8]{images/ERJ190}
\caption{Posicionamento da janela de inspe��o. Fonte http://www.aero-news.net/}
\label{fig:ERJ190}
\end{figure}

\section{Vari�veis de contorno}
\label{sec:variaveiscontorno}
O cen�rio de reconhecimento de objetos dentro da aeronave traz alguns desafios que devem ser contornados
\begin{itemize}
\item Pouca ilumina��o em ambientes internos
\item Objetos muito parecidos entre si
\item Alguns objetos com textura
\item Objeto brilhante
\end{itemize}

 %\section{Caso de uso} 
	\section{Recursos e M�todos}

\definecolor{LightCyan}{rgb}{0.88,1,1}


%A metodologia � a aplica��o de procedimentos e t�cnicas que devem ser observadas
para constru��o do conhecimento de comprovar sua validade e utilidade nos
diversos �mbitos da sociedade. Em em um n�vel aplicado, examina, descreve e
avalia m�todos e t�cnicas de pesquisa que possibilitam a coleta e o
processamento de informa��es, visando ao encaminhamento e � resolu��o de temas
de investiga��o \cite{MetodologiaCientifica}.
Pesquisa cient�fica � a realiza��o de um estudo planejado, sendo o m�todo de
abordagem do problema, o que caracteriza o aspecto cient�fico da investiga��o.
Sua finalidade � descobrir respostas para quest�es mediante a aplica��o do
m�todo cient�fico. A pesquisa sempre parte de um problema, uma interroga��o ou
uma situa��o para a qual o repert�rio de conhecimento dispon�vel n�o gera
resposta adequada. Para solucionar esse problema, s�o levantadas hip�teses que
podem ser confirmadas ou refutadas pela pesquisa. Portanto, toda pesquisa se
baseia em uma teoria que sirva como ponto de partida para a investiga��o.
Utilizando-se de formas tradicionais de classifica��o das pesquisas, temos uma
compila��o como mostrado na Figura~\ref{fig:tiposdepesquisa}
\cite{MetodologiaCientifica}.


\begin{figure}[h!]
\centering
\includegraphics[scale=0.5]{images/tiposdepesquisa}
\caption{Categorias de pesquisas cient�ficas. Fonte
\cite{MetodologiaCientifica} }
\label{fig:tiposdepesquisa}
\end{figure}


A tabela~\ref{table:classificacaopesquisa} sumariza como o presente trabalho se
categoriza segundo o m�todo tradicional de classifica��o dos crit�rios


\definecolor{LightCyan}{rgb}{0.88,1,1}


 \begin{table}[H]
  \centering
      \caption{Classifica��o da pesquisa}
\label{table:classificacaopesquisa}
    \begin{tabular}{|l|l|}
    \hline
    \rowcolor{LightCyan}
     Ponto de Vista & Tipo de pesquisa utilizada   \\
     \hline
    Natureza   & Pesquisa Aplicada     \\
    Abordagem do Problema   & Pesquisa Quantitativa        \\
    Objetivos  & Pesquisa Explicativa \\
    Procedimentos T�cnicos  & Pesquisa Bibliogr�fica, Pesquisa
    Experimental 
    \\
    \hline
    \end{tabular}

\end{table}


\begin{enumerate}
  \item  \textbf{Natureza}\newline
  A pesquisa aplicada objetiva gerar conhecimentos para aplica��o pr�tica,
  dirigidos � solu��o de problemas espec�ficos \cite{MetodologiaCientifica}.
  Assim, a pesquisa sobre reconhecimentos de padr�o aplicados � realidade
  aeron�utica, tem como objetivo selecionar as t�cnicas mais adequadas para
  situa��es reais em futuras aplica��es de AR para manuten��o, portanto se
  categoriza como uma aplica��o pr�tica a um problema espec�fico.
  
  \item \textbf{Abordagem do Problema}\newline
  A pesquisa quantitativa, tem como objetivo garantir a precis�o dos
  resultados, evitando contradi��es no processo de an�lise e interpreta��o, para tanto � feita uma
  hip�tese pr�via e tra�ada uma estrat�gia para prov�-la.
    O estudo da melhor t�cnica de reconhecimento de padr�es � realizada com
  experimentos emp�ricos e seu ambiente � artificial, emulando situa��es que,
  caso feitos de forma natural, demandariam muito tempo e custo, pois deveriam
  ser realizados em diversas aeronaves e em diversos ambientes, desde cen�rios
  noturnos at� cen�rio com neve.  
  
  \item \textbf{Objetivos}\newline
  A pesquisa explicativa tem por objetivo explicar
  os porqu�s das coisas e suas causas, por meio do registro, da an�lise, da classifica��o e da
  interpreta��o dos fen�menos observados. Visa identificar os fatores que
  determinam ou contribuam para a ocorr�ncia dos fen�menos; ''aprofunda o
  conhecimento da realidade porque explica a raz�o, o porqu� das
  coisas.'' \cite{ElaborarPesquisa}.

    
  \item \textbf{Procedimentos T�cnicos}\newline
  Este trabalho utiliza as seguintes formas de procedimentos: Pesquisa
  Bibliogr�fica e Pesquisa Experimental.
  A pesquisa bibliogr�fica � elaborada a partir de material j� publicado,
  constitu�do principalmente de: livros, revistas, publica��es em peri�dicos e
  artigos cient�ficos, jornais, boletins, monografias, disserta��es,teses,
  material cartogr�fico e internet, sempre com o objetivo de proporcionar ao
  pesquisador um contato direto com todo material j� existente sobre o assunto
  da pesquisa. Para elaborar o presente trabalho, foi feita uma intensa
  busca por t�cnicas de reconhecimento de padr�o com o objetivo de posterior uso
  em aplica��es de AR e sobre o uso de AR na manuten��o e compara��es
  de t�cnicas em outros contextos que n�o o aeron�utico.
  Na Pesquisa Experimental determinamos um objeto de estudo, selecionamos as
  vari�veis que seriam capazes de influenci�-lo e definimos as formas de
  controle e de observa��o dos efeitos que a vari�vel produz no objeto. No presente trabalho, foram gerados
  experimentos refazendo situa��es reais, avaliando o impacto de vari�veis
  ambientais por meio de aproxima��es matem�ticas aos eventos reais, como por exemplo, varia��o de brilho,
   escala ou rota��o do ponto de vista do usu�rio em rela��o ao objeto de
   estudo.
  
  
\end{enumerate}


Com uma abordagem pragm�tica, temos a tabela~\ref{table:metodosresumo}, que
exemplifica para cada um dos objetivos os recursos utilizados e os m�todos com
os quais ser�o atingidos.

\begin{table}[H]
  \centering
      \caption{Resumo de Recursos e M�todos}
      \label{table:metodosresumo}
    \begin{tabular}{|m{150pt}|m{100pt}|m{150pt}|}
     \rowcolor{LightCyan}
    \hline
    Objetivo & Recurso & M�todo\\
     \hline
     Avaliar os algoritmos cl�ssicos de reconhecimento & Artigos, Livros e Sites
     & Pesquisa Bibliogr�fica
     \\
     \hline
    Aplicar os algoritmos cl�ssicos � situa��es reais   & Recursos fornecidos
    pelo OpenCV & Simula��es
    \\
    \hline
    Selecionar algoritmo mais adequado para o contexto & Dados coletados de
    simula��o & An�lise de cad�ncia e qualidade de reconhecimento\\
   \hline
     \end{tabular}
\end{table}





	\section{Estrutura}
Este trabalho fundamenta-se em 9 cap�tulos, conforme descritos abaixo:

\begin{itemize}
\item Cap�tulo~\ref{ch:introducao} apresenta a motiva��o do presente trabalho,
levantando as necessidades e limita��es impostas pelo ambiente e o fato de
haverem diversas t�cnicas de reconhecimento de caracter�sticas e a necessidade
de selecionar a adequada para o contexto, bem como descrever o escopo do
trabalho e o contexto em que os testes foram feitos 

\item Cap�tulo~\ref{ch:fundamentacaoteorica} tem informa��es suficientes para o entendimento das
an�lises, descrevendo conceitos b�sicos e as t�cnicas de reconhecimento que
foram comparadas;

\item Cap�tulo~\ref{ch:propostao} descreve a metodologia de an�lise adotada bem
como o prot�tipo desenvolvido para a an�lise

\item Cap�tulo~\ref{ch:analisederesultados} descreve os resultados obtidos comparando-se as
t�cnicas, selecionando qual a t�cnica mais adequada para reconhecimento de
caracter�sticas para o caso de uso descrito;

\item Cap�tulo~\ref{ch:conclusao} conclui o trabalho apresentando como os
objetivos foram atingidos, limita��es da pesquisa e poss�veis
trabalhos futuros.
\end{itemize}

\chapter{Fundamenta��o Te�rica}
	\label{ch:fundamentacaoteorica}
	\section{Realidade Aumentada}
A realidade aumentada como citado em \cite{SurveyAR} � uma t�cnica de vis�o
computacional em que valendo-se de artefatos do mundo real tem por objetivo causar sensa��o de imers�o do usu�rio em um ambiente aumentado por artefatos virtuais, ao contr�rio de ambientes puramente virtuais como � comum em aplica��es de realidade virtual.
Idealmente o mundo virtual se torna imersivo o suficiente para que o usu�rio n�o consiga distinguir o real do virtual.
Alguns autores definem AR como tendo a necessidade de utilizar-se interfaces visuais port�teis para que a usabilidade tenha mais coer�ncia com a proposta inicial de garantir uma experi�ncia imersiva.
As imagens s�o obtidas por c�meras e o resultado apresentado em dispositivos como projetores ou 
displays como monitores, tablets ou head-mounted display (HMD). Como mostrado em
\cite{Devices}
\section{Head-Mounted Displays}
� um equipamento utilizado na cabe�a de forma que as duas m�os do usu�rio fiquem livres e tem por objetivo exibir imagens e �udio, sendo uma interface muito utilizada tanto em RV quanto em RA.
Os HMD basicamente s�o dispositivos constitu�dos de duas telas posicionadas frente ao olho do usu�rio.
Com duas telas, a tecnologia pode ser empregada para exibir imagens estereosc�picas apresentando os respectivos pontos de vista de cada olho para cada tela, o que contribui em muito na experi�ncia de imers�o.
Os HMDs funcionam tamb�m como dispositivos de entrada de dados, porque cont�m sensores de rastreamento que medem a posi��o e orienta��o da cabe�a, transmitindo esses dados ao computador.
Existem dois tipos de HMDs: Feed-Through e See-Through
\subsection{Feed-Through}
S�o dispositivos que � um sistema fechado de visualiza��o de imagens, em que o
usu�rio consegue enxergar somente o que � mostrado no display, sendo assim o
resultado apresentado � sempre a soma da imagem real com informa��es
superpostas. Como mostrado na figure~\ref{fig:feedthrough}

\begin{figure}[h!]
\centering
\includegraphics[scale=0.8]{images/feedthrough}
\caption{Arquitetura do Feed-Trough.}
\label{fig:feedthrough}
\end{figure}


\subsection{See-Through}
S�o dispositivos constru�dos com lentes transl�cidas em que o usu�rio enxerga o
mundo real e com algum tipo de sistema que sobrepoe na lente as informa��es
adicionais. Como mostrado na figure~\ref{fig:seethrough}

\begin{figure}[h!]
\centering
\includegraphics[scale=0.8]{images/seethrough}
\caption{Arquitetura do See-Trough.}
\label{fig:seethrough}
\end{figure}


\subsection{Projetores}
O uso de projetores possibilita uma abordagem de realidade aumentada diferente porque pode ser utilizada para cobrir superf�cies largas, projetando sobre objetos como carros, pessoas, pr�dios, etc�.
Um problema dessa abordagem � que a calibra��o se faz necess�ria em v�rias situa��es.

\subsection{Monitores}
O uso de monitores reduz bastante o custo da aplica��o apesar de ter perda de
imers�o por ser um m�todo de visualiza��o indireta, o que implica o usu�rio
ficar olhando na dire��o do monitor, entretanto existe a possibilidade de
compartilhar os resultados da RA com mais de uma pessoa ao mesmo tempo. Como
mostrado na figure~\ref{fig:monitores}

\begin{figure}[h!]
\centering
\includegraphics[scale=0.7]{images/monitores}
\caption{Realidade Aumentada com projetores.}
\label{fig:monitores}
\end{figure}

\section{Caracter�sticas Locais}
Caracter�sticas locais s�o padr�es em imagens que diferenciam de seu vizinho imediato, geralmente � caracterizada por mudan�as em propriedades das imagens ou diversas propriedades simultanamente. As propriedades mais comuns s�o intensidade, cor e texturas.

\subsection{Propriedades da Caracter�stica Local ideal}
Algoritmos de reconhecimento baseam-se em compara��o de caracter�sticas recuperadas da cena. A recupera��o e compara��o de pontos tem um custo computacional relevante perto do tempo de execu��o da aplica��o, portanto selecionar o menor n�mero poss�vel de caracter�sticas aumenta o desempenho e diminui o tempo de resposta da aplica��o.
Garantir que estamos selecionando boas caracter�sticas pode ser crucial na efic�cia do reconhecimento.
Segundo \cite{localfeaturedetector} , boas caracter�sticas devem ter as
seguintes propriedades:

\begin{itemize}
	\item \textbf{Repetibilidade}: Dadas duas imagens do mesmo objeto ou cena,
	tomadas em condi��es ou pontos de vista diferentes, uma porcentagem alta de caracter�sticas deve ser reconhecidas se estiverem vis�veis.
	\item \textbf{Distin��o}: Os padr�es reconhecidos t�m de ser poss�veis de serem
	distinguidos entre si para facilitar o casamento.
	\item \textbf{Localidade}: As caracter�sticas devem ser locais para reduzir a
	probabilidade de oclus�o.
	\item \textbf{Quantidade}: O n�mero de pontos detectados tem que ser o
	suficiente para que mesmo objetos pequenos tenham minimamente caracter�sticas que possam ser localizados e para que o 
	objeto possa sofrer oclus�o e ainda assim seja reconhecido.
	\item \textbf{Exatid�o}: As caracter�sticas detectadas tem que ser localizadas
	com o m�ximo de exatid�o poss�vel com respeito tanto referente � posi��o quanto
	� escala.
	\item \textbf{Efici�ncia}:De prefer�ncia  a detec��o deve ser o mais r�pido
	poss�vel.
\end{itemize}

\subsection{Features}
Antes de compreender como � feito o reconhecimento e registro de imagens � importante nos perguntar
 como n�s conseguimos reconhecer objetos em uma cena, como conseguimos comparar facilmente objetos em duas 
 imagens distintas. Somos treinados desde cedo a diferenciar formas geom�tricas, perceber escalas diferentes 
 ou mesmo reconhecer o mesmo objeto independente de como est� posicionado na cena buscando padr�es que 
 categorizem e diferenciem o objeto. Instintivamente conseguimos reconhecer boas caracter�sticas e localizar objetos.
Na imagem~\ref{fig:padroes} temos uma imagem de um pr�dio e seis recortes, dos quais
conseguimos facilmente reconhecer com precis�o a de letra E e F, as de letra A,B,C,D
 podemos identificar poss�veis localiza��es mas n�o podemos dizer com certeza onde est�o na imagem.

\begin{figure}[h!]
\centering
\includegraphics[scale=1.0]{images/features}
\caption{Reconhecimento de padr�es.\cite{understandingfeatures}}
\label{fig:padroes}
\end{figure}
As caracter�sticas E e F s�o o que chamamos de good features pois o n�vel de
certeza � bem alto.

\begin{figure}[h!]
\centering
\includegraphics[scale=1.0]{images/featuresregioes}
\caption{Regi�es de reconhecimento de padr�es.\cite{understandingfeatures}}
\label{fig:featuresregioes}
\end{figure}

A Imagem~\ref{fig:featuresregioes} ilustra os tipos de caracter�sticas. A regi�o
azul n�o possibilita diferenciar onde est� na imagem, a regi�o preta pode ser confundida com qualquer uma das regi�es
 ao deslocarmos horizontalmente, a imagem vermelha nos possibilita diferenciar e reconhecer o canto da imagem verde com precis�o milim�trica.
Podemos ent�o concluir que uma caracter�stica � boa para ser utilizada como par�metro de entrada
 para algoritmos de reconhecimento, quanto maior foi o n�vel de certeza da sua localiza��o, 
 o que facilita o \textbf{Feature Detection}.
Para localizar o mesmo objeto em outra imagem � necess�rio identificarmos a regi�o onde se encontra, 
caso contr�rio no exemplo da imagem~\ref{fig:padroes} seria imposs�vel
localizar uma janela espec�fica. Tal descri��o de contexto � chamada de
\textbf{Feature Description}. Uma vez de posse da caracter�stica e do seu
contexto � poss�vel reconhecer o objeto de fato.



\section{Cad�ncia}
� a medida do n�mero de quadros individuais que um determinado dispositivo �ptico ou eletr�nico processa e exibe por unidade de tempo. Em geral a cad�ncia � medida em fps.
Em cinema a cad�ncia de proje��o padr�o desde 1929 foi fixada em 24fps mas no per�odo do cinema mudo a maioria dos filmes eram rodados com cad�ncia entre 16 e 20fps.
Em v�deo, os principais sistemas lidam com cad�ncia entre 25fps(PAL) e 30fps(NTSC).
As aplica��es devem ter cad�ncia toler�vel dependendo de seu uso, segundo \cite{Tang93whydo} para aplica��es interaticas o m�nimo toler�vel � de 5fps enquanto para aplica��es de anima��es fluidas de 30fps.
Sendo a cad�ncia, a freq��ncia entre frames, deve ser contabilizado o tempo de gerar a informa��o e o tempo de dispor a informa��o no dispositivo �ptico.
O tempo de cada frame � calculado como mostrado na equa��o~\ref{eq:fps} 

\begin{equation}
t_{frame} =  \frac{1}{fps}
\label{eq:fps}
\end{equation}

No caso de cad�ncia m�nima de 5fps, temos quadros com tempo menor que 200ms,
portanto as an�lises devem ser balisadas a tempos menores.

Nowaday cameras have problems of distortion that can be treated because there are constant distortions and calibrations issues.

Radial distortion as "barrel" or "fish-eye" effect

barrel

Barrel lens distortion is an effect associated with wide-angle lenses and, in particular, zoom wide-angles. This effect causes images to be spherized, which means the edges of images look curved and bowed to the human eye. It almost appears as though the photo image has been wrapped around a curved surface.

Tangential Distortion


\subsection[Reconhecimento]{Reconhecimento}
\begin{frame}
\frametitle{Reconhecimento}

\begin{figure}[H]
\centering
\includegraphics[scale=0.4]{images/reconhecimento}
\caption{Ilustra��o do procedimento de reconhecimento com caracter�sticas
locais.
Fonte \cite{localfeaturedetector}}
\label{fig:reconhecimento}
\end{figure}

\end{frame}

\begin{frame}
Processo de reconhecimento como ilustrado na
imagem~\ref{fig:reconhecimento}:
\begin{itemize}
	\item Encontrar um grupo de \emph{keypoints} distintos;
	\item Definir uma regi�o em torno de cada \emph{keypoint};
	\item Extrair e normalizar o conte�do da regi�o;
	\item Calcular um descritor para a regi�o normalizada;
	\item Encontrar correspond�ncias de descritores.  
\end{itemize}



\end{frame}





 
\chapter{Aplica��o e Sele��o de Algoritmos}
	\label{ch:propostao}
	\section{Metodologia}
\subsection{Defini��o de par�metros}

Os testes s�o realizados adicionando vari�veis de forma artificial por meio de
transforma��es afins, de forma a emular comportamentos ou situa��es encontradas
no ambiente de manuten��o:
\begin{itemize}
	\item Varia��o de escala para simular a aproxima��o dentro da janela de
	inspe��o;
	\item Varia��o de rota��o para simular a movimenta��o durante a manuten��o;
	\item Varia��o de ilumina��o para simular manuten��es feitas em hor�rios do dia
	e ilumina��o diferente, como por exemplo, ambientes com neve com brilho muito
	maior;
	\item Adi��o de \emph{Blur} para emular ambientes com muita poeira ou
	esfuma�ados.
\end{itemize}

A defini��o das faixas de par�metros utilizadas foi feita com inspe��o em campo.

\section{An�lise de dados}
\label{sec:analisededados}

A an�lise de desempenho de cada t�cnica deve ser balizada por par�metros que garantam que a aplica��o seja vi�vel para o
 contexto citado na Se��o \ref{sec:restricao}. Para tal, deve-se considerar os
 par�metros estimados abaixo definidos por inspe��o: 

\begin{itemize}
	\item Cad�ncia > 5fps ou seja, tempo < 200ms \cite{Tang93whydo};
	\item Robustez � escala:  0.5 < escala < 1.2;
	\item Robustez � rota��o:  0$^{\circ}$ < rota��o < 30$^{\circ}$; 
	\item Robustez � ilumina��o: -50 < ilumina��o < 50
	\item Robustez � \emph{blur}: \emph{kernelsize} < 4
\end{itemize}

Uma an�lise mais precisa dos par�metros n�o faz parte do escopo desse trabalho
pois envolve pesquisa de campo em situa��es adversas. Portanto, tais par�metros
foram escolhidos para validar o m�todo de sele��o das t�cnicas, sendo assim,
identificados por experi�ncia como usuais para o contexto, entretanto, n�o h�
perda de generalidade no m�todo, posto que pode ser
executado da mesma maneira com outros valores. 



	O escopo de desenvolvimento dessa tese tem por tra�ar um m�todo de escolha do
algoritmo a ser utilizado dependente de aspectos descritos na se��o 2.1 Vari�veis de Contorno, os testes
 ser�o feitos em duas etapas: 
Defini��o de par�metros de contorno e avalia��o dos algoritmos mais conhecidos no contexto aeron�utico

\section{Testes de defini��o de fronteiras de par�metros}

\section{Testes de reconhecimento de padr�es}

Utilizando os par�metros definidos na etapa anterior ser�o rodados todos os algoritmos para avaliar qual se encaixa 
melhor em cada uma das situa��es.
O prot�tipo para testes ser� realizado utilizando o OpenCV por j� disponibilizar os algoritmos mais utilizados para 
o reconhecimento de padr�es e algoritmos de vis�o computacional.
Nesse prot�tipo ser� avaliado a ader�ncia dos algoritmos:Fast, GoodFeaturesToTrack, Mser, Star, Sift, Surf.
Segundo os seguintes crit�rios:\par
\textbf{Velocidade} - para garantir que a aplica��o poder� ser utilizada como
una ferramenta de tempo real � imprescind�vel buscar uma aplica��o que rode naturalmente a 30
 frames por segundo. Tal realidade � facilmente atingida em computadores com v�rios n�cleos 
 como os core i7 mas � importante lembrar que o cen�rio de realidade aumentada para ter um 
 contexto coerente com a proposta de imers�o � composto por dispositivos portateis que via 
 de realidade regra n�o tem poder de processamento t�o alto.\par
\textbf{Qualidade} - ccaracter�sticas detectadas s�o geralmente utilizadas
posteriormente em etapas de rastrio e casamento de padr�o. Para os testes de rastreio
 ser�o adicionadas �s imagens algumas transforma��es afins, como escala, rota��o e ilumina��o
  e ent�o estimar a qualidade das caracter�sticas.\par
\textbf{Ilumina��o e invari�ncia � escala} - Detectores de caracter�sticas tem
que ter a habilidade de reconhecer caracter�sticas independente do tamanho do objeto. A invari�ncia deve ser verdadeira para varia��o de ilumina��o. Varia��es pequenas de ilumina��o e contraste n�o devem afetar o detector significativamente. As c�meras atuais em geral tem controle autom�tico de ganho que ajusta automaticamente a esposi��o evitando sub ou super esposi��o.
\par

Ser�o utilizados as seguintes transforma��es que tem por finalidade avaliar as
varia��es de ambiente
\begin{itemize}
  \item \textbf{Rota��o: }Robustez � rota��o tamb�m � uma caracter�stica
  importante pois nem sempre encontraremos situa��es em que o dispositivo de RA
  sr� posicionado na mesma horienta��o o tempo todo, garantindo uma maior
  autonomia a quem executa a manuten��o
  \item \textbf{Brilho:} A varia��o de brilho pode emular tanto varia��es
  temporais, como dia e noite, mas tamb�m pode emular situa��es de muita
  luminosidade como neve por exemplo
  \item \textbf{Blur: } A robustes a Blur auxilia na resposta do algoritmo
  frente � movimentos bruscos
  \item \textbf{Escala: } A robustez � escala do objeto garante que o mesmo ser�
  reconhecido tanto ao longe quanto em situa��es bem pr�ximas
\end{itemize}


\section{Ambiente}
Os testes tem por objetivo garantir as restri��es descritas na se��o 3. Constraints, portanto � o desempenho e o tempo de reconhecimento s�o uma caracter�stica relevante.
Os resultados obtidos pelos testes descritos nessa tese foram executado em uma m�quina com a seguinte configura��o:

Processador: Core 2-duo 2.2GHz
Mem�ria: 4GB
Placa de V�deo: NVidia GForce 9300M GS

Biblioteca: OpenCV 2.4.10

\subsection{OpenCV}
Distribu�do sob licensa BSD e portanto livre para uso acad�mico e comercial. 
Possui interfaces para C++, C, Python e Java suportando Windows, Linux, Mac OS, iOS 
e Android. OpenCV foi desenvolvido para resultados eficientes e com foco em aplica��es
 de tempo real, podendo tomar vantagem de processamentos paralelos, utilizando-se de OpenCL 
 aproveitando-se da acelera��o de hardware ou mesmo de plataformas heterog�neas. Adotado ao redor
  do mundo em v�rias pesquisas com uma comunidade de  mais de 47 mil pessoas com mais de 9 milh�es
   de downloads. Adotado como plataforma de desenvolvimento para aplica��es de arte interativa a 
   aplica��es de rob�tica avan�ada.
   
 \section{Caso de uso}

Como caso de uso ser� adotado a janela de inspe��o frontal [COLOCAR AQUI UMA 
REFER�NCIA MELHOR] e selecionado uma pe�a. 
A sele��o da pe�a ser� feita de tal forma que contenha informa��es relevantes para estressar 
os limites dos algoritmos de registro, portanto um estudo pr�vio das caracter�sticas dos 
algoritmos para abstrair as caracter�sticas mais relevantes se faz necess�rio. 



ESCREVER as peculiaridades do ambiente para poder decidir qual abordagem vou tomar


	

\chapter{An�lise de Resultados}
	\label{ch:analisederesultados}
	\newcommand{\gtype}{pres}


\newcommand{\minscale}{0.2}
\newcommand{\maxscale}{1.2}
\newcommand{\minangle}{0}
\newcommand{\maxangle}{30}
\newcommand{\minblur}{0}
\newcommand{\maxblur}{4}
\newcommand{\minbright}{-50}
\newcommand{\maxbright}{50}
\newcommand{\correctmatches}{0.6}
\newcommand{\cadencia}{200}
\newcommand{\maxtime}{10000}


\newcommand{\miny}{-10}
\newcommand{\maxy}{3}
\newcommand{\minx}{-150}
\newcommand{\maxx}{400}
\newcommand{\ylimit}{0.6}
\newcommand{\ylabel}{Percent of correct matches}
\newcommand{\graphwidth}{0.8}


A abordagem de testes e an�lise de desempenho � feita como descrito em
\cite{performance_evaluation}.Como descrito na Se��o \ref{sec:analisededados},
existem restri��es para garantir que a t�cnica utilizada responde segundo as necessidades da aplica��o.
Nos gr�ficos a seguir s�o tra�adas janelas de restri��o, sendo que os valores
que se encontram dentro dela s�o adequadas � situa��o.
As t�cnicas tem suas implementa��es no OpenCV com a possibilidade de
configurar alguns par�metros entretanto para garantir que a compara��o inicial
 seja independente de peculiaridades de implementa��o ou mesmo das diversas
 possibilidades de configura��o foi feita a compara��o utilizando os
 construtores padr�o, com as configura��es iniciais.
 

\section{Taxa de Acertos}

Foram avaliados, a raz�o entre a quantidade de caracter�sticas corretamente
localizadas e o n�mero de caracter�sticas na imagem inicial. Idealmente o valor
dessa medida deve se aproximar de 100\%.
Inicialmente foram considerados os resultados que se demonstraram dentro de
 pelo menos mais de 50\%  da janela delimitada, pois a an�lise deve ser feita
 considerando tanto o par�metro de taxa de acertos quanto o tempo gasto para
 realiz�-lo.

 
 Como mostrado na imagem \ref{graph:scaleresultpres}, para os testes de Escala
 est�o dentro da janela as t�cnicas FREAK,ORB,GFTT.
 
\begin{figure}[H]
\centering

%scale 
\begin{tikzpicture}
	\pgfplotsset{small}
\begin{axis}
	[
	   width=\graphwidth\textwidth,
    ylabel=$\ylabel$, % Set the labels
    xlabel=$Scale Factor$,
	legend entries={$BRISK$,$FAST$,$FREAK$,$GFTT$,$MSER$,$ORB$,$STAR$,$SURF$,$SIFT$},
	legend pos=outer north east,
	title= Robustness to scaling
    ]
	\addplot table [x=Argument, y=BRISK    , col sep=comma]	{graphs/scale-all-\gtype.csv}; 
	\addplot table [x=Argument, y=FAST     , col sep=comma]	{graphs/scale-all-\gtype.csv}; 
	\addplot table [x=Argument, y=FREAK    , col sep=comma]	{graphs/scale-all-\gtype.csv}; 
	\addplot table [x=Argument, y=GFTT     , col sep=comma]	{graphs/scale-all-\gtype.csv};
	 \addplot table [x=Argument, y=MSER     , col	sep=comma]	{graphs/scale-all-\gtype.csv};
	  \addplot table [x=Argument, y=ORB      , col sep=comma]	{graphs/scale-all-\gtype.csv}; 
	\addplot table [x=Argument, y=STAR     , col sep=comma]	{graphs/scale-all-\gtype.csv}; 
	\addplot table [x=Argument, y=SURF     , col sep=comma]	{graphs/scale-all-\gtype.csv}; 
	\addplot table [x=Argument, y=SIFT     , col sep=comma]	{graphs/scale-all-\gtype.csv}; 
	
	
	%eixo horizontal
	\addplot[red,sharp plot,update limits=false] coordinates{(\minx,\ylimit)(\maxx,\ylimit)};
	 \fill [opacity=0.4,red!25] (axis	cs:\minx,\miny)	rectangle (axis	
	 cs:\maxx,\ylimit);
	
	
	%eixo vertical
	\addplot[blue,sharp plot,update limits=false] coordinates {(\minscale,\miny)(\minscale,\maxy)} ;
		\fill [opacity=0.4,blue!25] (axis cs:0,-1) rectangle (axis	cs:\minscale,\maxy);
	
	\addplot[blue,sharp plot,update limits=false] coordinates {(\maxscale,\miny)(\maxscale,\maxy)} ;
	\fill [opacity=0.4,blue!25] (axis cs:\maxscale,\miny) rectangle (axis	cs:\maxx,\maxy);
	
\end{axis}
%AQUI
\end{tikzpicture}


\caption{Resultado de performance de varia��o de escala}
\label{graph:scaleresultpres}
\end{figure}


 Como mostrado na imagem~\ref{graph:rotresultpres}, para os testes de Rota��o
 est�o dentro da janela as t�cnicas BRISK,FAST,FREAK,GFTT,MSER,ORB,STAR,SURF,SIFT


%rotation
\begin{figure}[H]
\centering
\begin{tikzpicture}
	\pgfplotsset{small}
\begin{axis}
	[
	   width=\graphwidth\textwidth,
    ylabel=$\ylabel$, % Set the labels
    xlabel=$Angle(Degree)$,
	legend entries={$BRISK$,$FAST$,$FREAK$,$GFTT$,$MSER$,$ORB$,$STAR$,$SURF$,$SIFT$},	
	legend pos=outer north east,
	title= Rotation Invariance 
    ]
	\addplot table [x=Argument, y=BRISK    , col sep=comma]	{graphs/rot-all-\gtype.csv}; 
	\addplot table [x=Argument, y=FAST     , col sep=comma]	{graphs/rot-all-\gtype.csv}; 
	\addplot table [x=Argument, y=FREAK    , col sep=comma]	{graphs/rot-all-\gtype.csv}; 
	\addplot table [x=Argument, y=GFTT     , col sep=comma]	{graphs/rot-all-\gtype.csv};
	 \addplot table [x=Argument, y=MSER     , col	sep=comma]	{graphs/rot-all-\gtype.csv};
	  \addplot table [x=Argument, y=ORB      , col sep=comma]	{graphs/rot-all-\gtype.csv}; 
	\addplot table [x=Argument, y=STAR     , col sep=comma]	{graphs/rot-all-\gtype.csv}; 
	\addplot table [x=Argument, y=SURF     , col sep=comma]	{graphs/rot-all-\gtype.csv}; 
	\addplot table [x=Argument, y=SIFT     , col sep=comma]	{graphs/rot-all-\gtype.csv}; 

	%eixo horizontal
	\addplot[red,sharp plot,update limits=false] coordinates
	{(-100,\ylimit) (400,\ylimit)};
		\fill [opacity=0.4,red!25] (axis cs:-100,\miny) rectangle (axis
	cs:\maxx,\ylimit);
	
	
		%eixo vertical
	\addplot[blue,sharp plot,update limits=false] coordinates {(\minangle,\miny)
	(\minangle,\maxy)} ;
		\fill [opacity=0.4,blue!25] (axis cs:-100,\miny) rectangle (axis
	cs:\minangle,\maxy);
	
	\addplot[blue,sharp plot,update limits=false] coordinates {(\maxangle,\miny)
	(\maxangle,\maxy)} ;
	\fill [opacity=0.4,blue!25] (axis cs:\maxangle,\miny) rectangle (axis
	cs:\maxx,\maxy);

	
\end{axis}
%AQUI
\end{tikzpicture}
\caption{Resultado de performance de varia��o de rota��o}
\label{graph:rotresultpres}
\end{figure}


Como mostrado na imagem~\ref{graph:blurresultpres}, para os testes de Blur
est�o dentro da janela as t�cnicas FAST,FREAK,GFTT,MSER,ORB,STAR,SURF,SIFT


%blur
\begin{figure}[H]
\centering
\begin{tikzpicture}
	\pgfplotsset{small}
\begin{axis}
	[
	   width=\graphwidth\textwidth,
       ylabel=$\ylabel$, % Set the labels
    xlabel=$Kernel size$,
	legend entries={$BRISK$,$FAST$,$FREAK$,$GFTT$,$MSER$,$ORB$,$STAR$,$SURF$,$SIFT$},	
	legend pos=outer north east,
	title= Robustness to blur]
	\addplot table [x=Argument, y=BRISK    , col sep=comma]	{graphs/blur-all-\gtype.csv}; 
	\addplot table [x=Argument, y=FAST     , col sep=comma]	{graphs/blur-all-\gtype.csv}; 
	\addplot table [x=Argument, y=FREAK    , col sep=comma]	{graphs/blur-all-\gtype.csv}; 
	\addplot table [x=Argument, y=GFTT     , col sep=comma]	{graphs/blur-all-\gtype.csv}; 
	\addplot table [x=Argument, y=MSER     , col	sep=comma]	{graphs/blur-all-\gtype.csv}; 
	\addplot table [x=Argument, y=ORB      , col sep=comma]	{graphs/blur-all-\gtype.csv}; 
	\addplot table [x=Argument, y=STAR     , col sep=comma]	{graphs/blur-all-\gtype.csv}; 
	\addplot table [x=Argument, y=SURF     , col sep=comma]	{graphs/blur-all-\gtype.csv}; 
	\addplot table [x=Argument, y=SIFT     , col sep=comma]	{graphs/blur-all-\gtype.csv}; 
	 
	%eixo horizontal
	\addplot[red,sharp plot,update limits=false] coordinates{(\minx,\ylimit)(\maxx,\ylimit)};
	 \fill [opacity=0.4,red!25] (axis	cs:\minx,\ylimit) rectangle(axis
	 cs:\maxx,\miny);


				%eixo vertical
	\addplot[blue,sharp plot,update limits=false] coordinates
	{(\minblur,\miny)(\minblur,\maxy)} ; \fill [opacity=0.4,blue!25] (axis	cs:\minx,\miny) rectangle (axis cs:\minblur,\maxy);
	
	\addplot[blue,sharp plot,update limits=false] coordinates
	{(\maxblur,\miny)(\maxblur,\maxy)} ; \fill [opacity=0.4,blue!25] (axis	cs:\maxblur,\miny) rectangle (axis cs:\maxx,\maxy);


\end{axis}
%AQUI
\end{tikzpicture}


\caption{Resultado de performance de varia��o de blur}
\label{graph:blurresultpres}
\end{figure}


 Como mostrado na imagem~\ref{graph:brightresultpres}, para os testes de
 Ilumina��o est�o dentro da janela as t�cnicas BRISK,FAST,FREAK,GFTT,MSER,ORB,STAR,SURF,SIFT


%bright
\begin{figure}[H]
\centering
\begin{tikzpicture}
	\pgfplotsset{small}
\begin{axis}
	[
	   width=\graphwidth\textwidth,
       ylabel=$\ylabel$, % Set the labels
    xlabel=$Change of brightness$,
	legend entries={$BRISK$,$FAST$,$FREAK$,$GFTT$,$MSER$,$ORB$,$STAR$,$SURF$,$SIFT$},	
	legend pos=outer north east,
	title= Brightness Invariance 
    ]
	\addplot table [x=Argument, y=BRISK    , col sep=comma]	{graphs/bright-all-\gtype.csv}; 
	\addplot table [x=Argument, y=FAST     , col sep=comma]	{graphs/bright-all-\gtype.csv}; 
	\addplot table [x=Argument, y=FREAK    , col sep=comma]	{graphs/bright-all-\gtype.csv}; 
	\addplot table [x=Argument, y=GFTT     , col sep=comma]	{graphs/bright-all-\gtype.csv}; 
	 \addplot table [x=Argument, y=MSER    , col sep=comma] {graphs/bright-all-\gtype.csv}; 
	 \addplot table [x=Argument, y=ORB     , col sep=comma]	{graphs/bright-all-\gtype.csv}; 
	\addplot table [x=Argument, y=STAR     , col sep=comma]	{graphs/bright-all-\gtype.csv}; 
	\addplot table [x=Argument, y=SURF     , col sep=comma]	{graphs/bright-all-\gtype.csv}; 
	\addplot table [x=Argument, y=SIFT     , col sep=comma]	{graphs/bright-all-\gtype.csv}; 
	
	
	
	%eixo horizontal
	\addplot[red,sharp plot,update limits=false] coordinates{(\minx,\ylimit)
	(\maxx,\ylimit)}; 
	\fill [opacity=0.4,red!25] (axis cs:\minx,\ylimit) rectangle (axis
	cs:\maxx,\maxy);
		
	
						%eixo vertical
	\addplot[blue,sharp plot,update limits=false] coordinates
	{(\minbright,\miny)(\minbright,\maxy)} ; \fill [opacity=0.4,blue!25] (axis
	cs:\minx,\miny) rectangle (axis cs:\minbright,\maxy);
	
	\addplot[blue,sharp plot,update limits=false] coordinates
	{(\maxbright,\miny)(\maxbright,\maxy)} ; \fill [opacity=0.4,blue!25] (axis
	cs:\maxbright,\miny) rectangle (axis cs:\maxx,\maxy);
		
	
\end{axis}
%AQUI
\end{tikzpicture}


\caption{Resultado de performance de varia��o de blur}
\label{graph:brightresultpres}
\end{figure}

\section{An�lise de Tempo}

A an�lise de tempo, ao contr�rio da an�lise de taxa de acertos, � excludente,
pois como descrito em \cite{Tang93whydo}, os usu�rios n�o toleram aplica��es com
menos do que 5fps.




\renewcommand{\gtype}{time}

\renewcommand{\maxy}{3000}
\renewcommand{\miny}{-155}



\renewcommand{\ylimit}{200}
\renewcommand{\ylabel}{Time(Ms)}


 Como mostrado na imagem~\ref{graph:scaleresulttime}, para os testes de
 Escala est�o dentro da janela, as t�cnicas BRISK,GFTT,ORB,STAR



%scale
\begin{figure}[H]
\centering
\begin{tikzpicture}
	\pgfplotsset{small}
	%scale
\begin{axis}
	[
	   width=\graphwidth\textwidth,
      ylabel=$\ylabel$, % Set the labels
    xlabel=$Scale Factor$,
	legend entries={$BRISK$,$FAST$,$FREAK$,$GFTT$,$MSER$,$ORB$,$STAR$,$SURF$,$SIFT$},
	legend pos=outer north east,
	title= Robustness to scaling
    ]
	\addplot table [x=Argument, y=BRISK    , col sep=comma]	{graphs/scale-all-\gtype.csv}; 
	\addplot table [x=Argument, y=FAST     , col sep=comma]	{graphs/scale-all-\gtype.csv}; 
	\addplot table [x=Argument, y=FREAK    , col sep=comma]	{graphs/scale-all-\gtype.csv}; 
	\addplot table [x=Argument, y=GFTT     , col sep=comma]	{graphs/scale-all-\gtype.csv};
	 \addplot table [x=Argument, y=MSER     , col	sep=comma]	{graphs/scale-all-\gtype.csv};
	  \addplot table [x=Argument, y=ORB      , col sep=comma]	{graphs/scale-all-\gtype.csv}; 
	\addplot table [x=Argument, y=STAR     , col sep=comma]	{graphs/scale-all-\gtype.csv}; 
	\addplot table [x=Argument, y=SURF     , col sep=comma]	{graphs/scale-all-\gtype.csv}; 
	\addplot table [x=Argument, y=SIFT     , col sep=comma]	{graphs/scale-all-\gtype.csv}; 
	
	%eixo horizontal
	\addplot[red,sharp plot,update limits=false] coordinates{(\miny,\ylimit)
	(\maxx,\ylimit)}; 
	\fill [opacity=0.4,red!25] (axis cs:\miny,\ylimit) rectangle (axis
	cs:\maxx,\maxy);
	
	
	%eixo vertical
	\addplot[blue,sharp plot,update limits=false] coordinates {(\minscale,\miny)(\minscale,\maxy)} ;
		\fill [opacity=0.4,blue!25] (axis cs:\minx,\miny) rectangle (axis		cs:\minscale,\maxy);
	
	\addplot[blue,sharp plot,update limits=false] coordinates {(\maxscale,\miny)	(\maxscale,\maxy)} ;
	\fill [opacity=0.4,blue!25] (axis cs:\maxscale,\miny) rectangle (axis	cs:\maxx,\maxy);
	
\end{axis}
\end{tikzpicture}


\caption{An�lise de tempo de varia��o de escala}
\label{graph:scaleresulttime}

\end{figure}

Como mostrado na imagem~\ref{graph:scaleresulttime}, para os testes de
 Rota��o est�o dentro da janela, as t�cnicas BRISK,GFTT,ORB,STAR

%rotation
\begin{figure}[H]
\centering
\begin{tikzpicture}
	\pgfplotsset{small}
\begin{axis}
	[
	   width=\graphwidth\textwidth,
      ylabel=$\ylabel$, % Set the labels
    xlabel=$Angle(Degree)$,
	legend entries={$BRISK$,$FAST$,$FREAK$,$GFTT$,$MSER$,$ORB$,$STAR$,$SURF$,$SIFT$},
	legend pos=outer north east,
	title= Rotation Invariance 
    ]
	\addplot table [x=Argument, y=BRISK    , col sep=comma]	{graphs/rot-all-\gtype.csv}; 
	\addplot table [x=Argument, y=FAST     , col sep=comma]	{graphs/rot-all-\gtype.csv}; 
	\addplot table [x=Argument, y=FREAK    , col sep=comma]	{graphs/rot-all-\gtype.csv}; 
	\addplot table [x=Argument, y=GFTT     , col sep=comma]	{graphs/rot-all-\gtype.csv};
\addplot table [x=Argument, y=MSER     , col	sep=comma]	{graphs/rot-all-\gtype.csv}; 
\addplot table [x=Argument, y=ORB      , col sep=comma]	{graphs/rot-all-\gtype.csv}; 
	\addplot table [x=Argument, y=STAR     , col sep=comma]	{graphs/rot-all-\gtype.csv}; 
	\addplot table [x=Argument, y=SURF     , col sep=comma]	{graphs/rot-all-\gtype.csv}; 
	\addplot table [x=Argument, y=SIFT     , col sep=comma]	{graphs/rot-all-\gtype.csv}; 

	
	%eixo horizontal
	\addplot[red,sharp plot,update limits=false] coordinates{(-100,\ylimit)
	(400,\ylimit)}; 
	\fill [opacity=0.4,red!25] (axis cs:-100,\ylimit) rectangle (axis
	cs:400,\maxy);
	
	
		%eixo vertical
	\addplot[blue,sharp plot,update limits=false] coordinates {(\minangle,-100)
	(\minangle,\maxy)} ;
		\fill [opacity=0.4,blue!25] (axis cs:-100,-100) rectangle (axis
		cs:\minangle,\maxy);
	
	\addplot[blue,sharp plot,update limits=false] coordinates {(\maxangle,-100)
	(\maxangle,\maxy)} ;
	\fill [opacity=0.4,blue!25] (axis cs:\maxangle,-100) rectangle (axis
	cs:400,\maxy);

	
\end{axis}
\end{tikzpicture}


\caption{An�lise de tempo de varia��o de rota��o}
\label{graph:rotresulttime}

\end{figure}


Como mostrado na imagem~\ref{graph:blurresulttime}, para os testes de
 Blur est�o dentro da janela, as t�cnicas BRISK,GFTT,ORB,STAR

%blur
\begin{figure}[H]
\centering
\begin{tikzpicture}
	\pgfplotsset{small}
\begin{axis}
	[
	   width=\graphwidth\textwidth,
      ylabel=$\ylabel$, % Set the labels
    xlabel=$Kernel size$,
	legend entries={$BRISK$,$FAST$,$FREAK$,$GFTT$,$MSER$,$ORB$,$STAR$,$SURF$,$SIFT$},
	legend pos=outer north east,
	title= Robustness to blur
    ]
	\addplot table [x=Argument, y=BRISK    , col sep=comma]	{graphs/blur-all-\gtype.csv}; 
	\addplot table [x=Argument, y=FAST     , col sep=comma]	{graphs/blur-all-\gtype.csv}; 
	\addplot table [x=Argument, y=FREAK    , col sep=comma]	{graphs/blur-all-\gtype.csv}; 
	\addplot table [x=Argument, y=GFTT     , col sep=comma]	{graphs/blur-all-\gtype.csv};
	 \addplot table [x=Argument, y=MSER     , col	sep=comma]	{graphs/blur-all-\gtype.csv}; 
	 \addplot table [x=Argument, y=ORB      , col sep=comma]	{graphs/blur-all-\gtype.csv}; 
	\addplot table [x=Argument, y=STAR     , col sep=comma]	{graphs/blur-all-\gtype.csv}; 
	\addplot table [x=Argument, y=SURF     , col sep=comma]	{graphs/blur-all-\gtype.csv}; 
	\addplot table [x=Argument, y=SIFT     , col sep=comma]	{graphs/blur-all-\gtype.csv}; 
	
	%eixo horizontal
	\addplot[red,sharp plot,update limits=false] coordinates{(\minx,\ylimit)
	(\maxx,\ylimit)}; 
	\fill [opacity=0.4,red!25] (axis cs:\minx,\ylimit) rectangle (axis
	cs:\maxx,\maxy);

				%eixo vertical
	\addplot[blue,sharp plot,update limits=false] coordinates
	{(\minblur,\miny)(\minblur,\maxy)} ; \fill [opacity=0.4,blue!25] (axis
	cs:\minx,\miny) rectangle (axis cs:\minblur,\maxx);
	
	\addplot[blue,sharp plot,update limits=false] coordinates
	{(\maxblur,\miny)(\maxblur,\maxy)} ; \fill [opacity=0.4,blue!25] (axis
	cs:\maxblur,\miny) rectangle (axis cs:\maxx,\maxy);


\end{axis}
\end{tikzpicture}


\caption{An�lise de tempo de varia��o de blur}
\label{graph:blurresulttime}

\end{figure}


Como mostrado na imagem~\ref{graph:brightresulttime}, para os testes de
 Ilumina��o est�o dentro da janela, as t�cnicas BRISK,GFTT,ORB,STAR

%bright
\begin{figure}[H]
\centering
\begin{tikzpicture}
	\pgfplotsset{small}
\begin{axis}
	[
	   width=\graphwidth\textwidth,
      ylabel=$\ylabel$, % Set the labels
    xlabel=$Change of brightness$,
	legend entries={$BRISK$,$FAST$,$FREAK$,$GFTT$,$MSER$,$ORB$,$STAR$,$SURF$,$SIFT$},
	legend pos=outer north east,
	title= Brightness Invariance 
    ]
	\addplot table [x=Argument, y=BRISK    , col sep=comma]	{graphs/bright-all-\gtype.csv}; 
	\addplot table [x=Argument, y=FAST     , col sep=comma]	{graphs/bright-all-\gtype.csv}; 
	\addplot table [x=Argument, y=FREAK    , col sep=comma]	{graphs/bright-all-\gtype.csv}; 
	\addplot table [x=Argument, y=GFTT     , col sep=comma]	{graphs/bright-all-\gtype.csv}; 
	\addplot table [x=Argument, y=MSER     , col sep=comma]	{graphs/bright-all-\gtype.csv}; 
	\addplot table [x=Argument, y=ORB      , col sep=comma]	{graphs/bright-all-\gtype.csv}; 
	\addplot table [x=Argument, y=STAR     , col sep=comma]	{graphs/bright-all-\gtype.csv}; 
	\addplot table [x=Argument, y=SURF     , col sep=comma]	{graphs/bright-all-\gtype.csv}; 
	\addplot table [x=Argument, y=SIFT     , col sep=comma]	{graphs/bright-all-\gtype.csv}; 
	
	%eixo horizontal
	\addplot[red,sharp plot,update limits=false] coordinates{(\minx,\ylimit)
	(\maxx,\ylimit)}; 
	\fill [opacity=0.4,red!25] (axis cs:\minx,\ylimit) rectangle (axis
	cs:\maxx,\maxy);
		
	
						%eixo vertical
	\addplot[blue,sharp plot,update limits=false] coordinates
	{(\minbright,\miny)(\minbright,\maxy)} ; \fill [opacity=0.4,blue!25] (axis
	cs:\minx,\miny) rectangle (axis cs:\minbright,\maxy);
	
	\addplot[blue,sharp plot,update limits=false] coordinates
	{(\maxbright,\miny)(\maxbright,\maxy)} ; \fill [opacity=0.4,blue!25] (axis
	cs:\maxbright,\miny) rectangle (axis cs:\maxx,\maxy);
		
	
\end{axis}
\end{tikzpicture}

\caption{An�lise de tempo de varia��o de ilumina��o}
\label{graph:brightresulttime}
\end{figure}


\subsection{Sele��o da T�cnica}

\definecolor{Gray}{gray}{0.85}
\newcolumntype{a}{>{\columncolor{Gray}}c}
As restri��es das an�lises feitas como mostrado
na tabela~\ref{table:techdecision}, nos levam a escolher GFTT e ORB por serem os
�nicos que garantem resultados dentro da janela esperada para todos os testes.


 \begin{table}[H]
  \centering
  \caption{Decis�o de t�cnica � utilizar}
\label{table:techdecision}
  \resizebox{\textwidth}{!}{  
\begin{tabular}{ | l | l | l | l | a | l | a | l | l | l | }
\hline
	 & BRISK & FAST & FREAK & GFTT & MSER & ORB & STAR & SURF & SIFT \\ \hline
	Precis�o Escala &  &  & X & X &  & X &  &  &  \\ \hline
	Precis�o Rota��o & X & X & X & X & X & X & X & X & X \\ \hline
	Precis�o Blur &  & X & X & X & X & X & X & X & X \\ \hline
	Precis�o Ilumina��o & X & X & X & X & X & X & X & X & X \\ \hline
	Tempo Escala & X &  &  & X &  & X & X &  &  \\ \hline
	Tempo Rota��o & X &  &  & X &  & X & X &  &  \\ \hline
	Tempo Blur & X &  &  & X &  & X & X &  &  \\ \hline
	Tempo Ilumina��o & X &  &  & X &  & X & X &  &  \\ \hline
\end{tabular}
}

\end{table}


\newcommand{\D}{8} % number of dimensions (config option)
\newcommand{\U}{2} % number of scale units (config option)

\newdimen\R % maximal diagram radius (config option)
\R=3.5cm 
\newdimen\L % radius to put dimension labels (config option)
\L=4cm

\newcommand{\A}{360/\D} % calculated angle between dimension axes  

\begin{figure}[htbp]
 \centering

\begin{tikzpicture}[scale=1]
  \path (0:0cm) coordinate (O); % define coordinate for origin

  % draw the spiderweb
  \foreach \X in {1,...,\D}{
    \draw (\X*\A:0) -- (\X*\A:\R);
  }

  \foreach \Y in {0,...,\U}{
    \foreach \X in {1,...,\D}{
      \path (\X*\A:\Y*\R/\U) coordinate (D\X-\Y);
      \fill (D\X-\Y) circle (1pt);
    }
    \draw [opacity=0.3] (0:\Y*\R/\U) \foreach \X in {1,...,\D}{
        -- (\X*\A:\Y*\R/\U)
    } -- cycle;
  }

  % define labels for each dimension axis (names config option)
  \path (1*\A:\L) node (L1) {\tiny Precis�o Escala};
  \path (2*\A:\L) node (L2) {\tiny Precis�o Rota��o};
  \path (3*\A:\L) node (L3) {\tiny Precis�o Blur};
  \path (4*\A:\L) node (L4) {\tiny Precis�o Ilumina��o};
  
  \path (5*\A:\L) node (L5) {\tiny Tempo Escala};
  \path (6*\A:\L) node (L6) {\tiny Tempo Rota��o};
  \path (7*\A:\L) node (L7) {\tiny Tempo Blur};
  \path (8*\A:\L) node (L8) {\tiny Tempo Ilumina��o};


  % for each sample case draw a path around the web along concrete values
  % for the individual dimensions. Each node along the path is labeled
  % with an identifier using the following scheme:
  %
  %   D<d>-<v>, dimension <d> a number between 1 and \D (#dimensions) and
  %             value <v> a number between 0 and \U (#scale units)
  %
  % The paths will be drawn half-opaque, so that overlapping parts will be
  % rendered in a composite color.

  % Example Case 1 (red)
  %
  % D1 (Security): 0/7; D2 (Content Quality): 5/7; D3 (Performance): 0/7;
  % D4 (Stability): 6/7; D5 (Usability): 0/7; D6 (Generality): 5/7;
  % D7 (Popularity): 0/7
  \draw [color=red,line width=1.5pt,opacity=0.5]
    (D1-0) --
    (D2-1) --
    (D3-1) --
    (D4-0) --
    (D5-0) --
    (D6-1) --    
    (D7-0) --
    (D8-1) --cycle;


  % Example Case 2 (green)
  %
  % D1 (Security): 2/7; D2 (Content Quality): 2/7; D3 (Performance): 5/7;
  % D4 (Stability): 1/7; D5 (Usability): 4/7; D6 (Generality): 1/7;
  % D7 (Popularity): 7/7
  \draw [color=green,line width=1.5pt,opacity=0.5]
    (D1-0) --
    (D2-1) --
    (D3-1) --
    (D4-1) --
    (D5-1) --
    (D6-0) --    
    (D7-0) --
    (D8-1) --cycle;

  % Example Case 3 (blue)
  %
  % D1 (Security): 1/7; D2 (Content Quality): 7/7; D3 (Performance): 4/7;
  % D4 (Stability): 4/7; D5 (Usability): 3/7; D6 (Generality): 5/7;
  % D7 (Popularity): 2/7
  \draw [color=blue,line width=1.5pt,opacity=0.5]
    (D1-0) --
    (D2-1) --
    (D3-1) --
    (D4-0) --
    (D5-1) --
    (D6-1) --    
    (D7-0) --
    (D8-1) --cycle;

\end{tikzpicture}
\caption{Spiderweb Diagram (\D~Dimensions, \U-Notch Scale, 3 Samples)}
\label{fig:spiderweb}
\end{figure}










\chapter{Conclus�o}
	\label{ch:conclusao}
	\section{Conclus�o}
\begin{frame}
\frametitle{Conclus�o}

\subsection{Atendimento dos objetivos}

\textbf {Avaliar os algoritmos cl�ssicos de reconhecimento}

O trabalho apresentou os algoritmos cl�ssicos, descrevendo seu funcionamento e
realizando testes de desempenho com os mesmos.

\textbf{Aplicar os algoritmos cl�ssicos � situa��es reais}

Os algoritmos foram implementados em um prot�tipo desenvolvido em C++ e OpenCV,
configurado para rodar simula��es dos algoritmos BRISK, FAST, FREAK, GFTT, MSER,
ORB, STAR, SURF, SIFT, gerando ao final de cada simula��o resultados de precis�o
de reconhecimento e de tempo gasto.

\end{frame}

\begin{frame}
\frametitle{Conclus�o}

\textbf {Selecionar algoritmo mais adequado para o contexto}

An�lise comparativa entre os algoritmos utilizando as restri��es e
criado nos gr�ficos, janelas de decis�o, gerando uma matriz de decis�o com os
resultados de todos os testes, tanto de qualidade de reconhecimento quanto de
tempo de execu��o.





\end{frame}
	\section{Proposta de Trabalhos Futuros}

A utiliza��o de Realidade Aumentada no campo da manuten��o pode trazer muitos
ganhos no que tange � usabilidade levando ao usu�rio uma quantidade de
informa��es que, da maneira tradicional, por meio da inspe��o e consulta em
manuais, seria invi�vel.
Este trabalho teve como foco o reconhecimento de padr�es em um cen�rio
aeron�utico espec�fico. Como pr�ximos passos temos:
\begin{itemize}

\item Adequar a aplica��o para dispositivos m�veis como \emph{tablets},
celulares ou �culos de realidade aumentada de forma a dar mais flexibilidade ao
condutor da manuten��o;

\item Realizar o casamento de padr�es com v�deos e imagens em tempo real,
utilizando as t�cnicas identificadas, otimizando a aplica��o quanto ao
desempenho;
 
\item Adaptar a aplica��o para utilizar processamento paralelo e processamento
em GPU, visto os algoritmos serem recursivos e localmente independentes;

\item Analisar por meio de testes em campo com poss�veis usu�rios para abstrair
par�metros de usabilidade, como por exemplo determinar que tipo de informa��o
seria �til ao usu�rio ou mesmo que tipo de conFigura��o de dispositivo seria o
mais adequado para uma aplica��o desse porte.
\end{itemize}


% Referencias Bibliograficas
\begin{spacing}{1.0}
\begin{flushleft}
%\bibliographystyle{alpha}%Choose a bibliograhpic style
\bibliography{bibliography}
\end{flushleft}
\end{spacing}


% Apendices
%\appendix
%\chapter{T�picos de �lgebra Linear}
%\input{apendiceA}

% Anexos
%\annex
%\chapter{Exemplo de um Primeiro Anexo}
%\input{anexoA}

% Referencias Bibliograficas



% Glossario
\itaglossary
\printglossary

% Folha de Registro do Documento
% Valores dos campos do formulario
\FRDitadata{24 de dezembro de 1969}
\FRDitadocnro{CTA/ITA - IEC/TM-002/1969}
\FRDitaorgaointerno{Divis�o de Ci�ncia da Computa��o -- ITA/IEC}
\FRDitapalavrasautor{Teses; Estilos; Italus}
\FRDitapalavrasresult{Teses e Disserta��es; Estilos; Usu�rios}
\FRDitaresumo{O reconhecimento de objetos em uma cena para posterior uso em realidade aumentada 
depende de diversas vari�veis, causando a necessidade do uso de t�cnicas 
espec�ficas para cada cen�rio, sendo portanto, um estudo de fronteiras para a melhor escolha 
do algoritmo de reconhecimento, de acordo com a aplica��o em quest�o de grande
valia para o meio acad�mico. 
Esta tese se prop�e a pesquisar, categorizar e tra�ar fronteiras das t�cnicas
conhecidas, tendo como caso de uso a manuten��o de aeronaves feita dentro de
centros fechados, utilizando as t�cnicas BRISK,FAST,FREAK,GFTT,MSER,
 ORB,STAR,SURF,SIFT em uma an�lise aplicada com imagens reais de janelas de
 inspe��o do Embraer ERJ-190 para reconhecimento de objetos e posteriores
 aplica��es em manuten��o.
 Comparando todas as t�cnicas quanto � cad�ncia e � precis�o de reconhecimento
 de caracter�sticas, � poss�vel selecionar GFTT e ORB
 como t�cnicas mais apropriadas ao contexto, por terem seus resultados de
 varia��o de rota��o, escala, briho e \emph{blur} dentro de uma faixa esperada
 para o contexto de manuten��o.
 

}
%  Primeiro Parametro: Nacional ou Internacional -- N/I
%  Segundo parametro: Ostensivo, Reservado, Confidencial ou Secreto -- O/R/C/S
\FRDitaOpcoes{N}{S}
% Cria o formulario
\itaFRD

\end{document}
% Fim do Documento.
