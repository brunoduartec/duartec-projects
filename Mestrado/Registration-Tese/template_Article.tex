%%% Exemplo de utiliza��o da classe ITA
%%%
%%%   por        F�bio Fagundes Silveira   -  ffs [at] ita [dot] br
%%%              Benedito C. O. Maciel     -  bcmaciel [at] ita [dot] br
%%%              Giovani Volnei Meinertz   -  giovani [at] ita [dot] br
%%%
%%%  IMPORTANTE: O texto contido neste exemplo nao significa absolutamente nada.  :-)
%%%              O intuito aqui eh demonstrar os comandos criados na classe e suas
%%%              respectivas utilizacoes.
%%%
%%%  $Id: ExemploTeseITA.tex 17 2006-02-07 17:59:17Z ffs $
%%%  $HeadURL: file:///opt/repositorioITALUS/classeITA/tags/versao-2.1/ExemploTeseITA.tex $
%%%
%%% ITALUS
%%% Technological Institute of Aeronautics --- ITA
%%% Sao Jose dos Campos, Brazil
%%% HomePages:        http://www.comp.ita.br/italus
%%%                   http://groups.yahoo.com/group/italus/
%%% Discussion list: italus {at} yahoogroups.com
%%%
%++++++++++++++++++++++++++++++++++++++++++++++++++++++++++++++++++++++++++++++
% Parametros da classe ITA:
%   msc   = Tese de Mestrado    --> no ITA, Dissertacao eh Tese ...  :-)
%   dsc   = Tese de Doutorado
%   quali = Exame de Qualificacao
%   dv    = 'Draft Version'     --> imprime 'Versao Preliminar + data no rodape
%   fem   = Doutora
%   eng   = para teses em ingl�s
%++++++++++++++++++++++++++++++++++++++++++++++++++++++++++++++++++++++++++++++


\documentclass[msc]{ita}    % ITA.cls based on standard book.cls

%++++++++++++++++++++++++++++++++++++++++++++++++++++++++++++++++++++++++++++++
% Identificacoes...
%++++++++++++++++++++++++++++++++++++++++++++++++++++++++++++++++++++++++++++++
\course{Engenharia Eletr�nica e Computa��o}
\dept{Ci�ncia da Computa��o}
\area{Inform�tica}

% Autor do trabalho: Nome Sobrenome
\author{Bruno}{Duarte Corr�a}

% Endereco do Autor -> utilizado no verso da folha de rosto
% Obrigat�rio para Teses
\itaauthoraddress{Av. ABC, 1000}{12.000-000}{S�o Jos� dos Campos--SP}

% Titulo da Tese/Dissertacao
\title{Avalia��o de t�cnicas de reconhecimento de padr�es em ambientes
aeronauticos}

% Orientador
\advisor{Prof.~Dr.}{Fulano de Tal}

% Co-orientador opcional
\coadvisor{Prof.~Dr.}{Beltrano de Tal}{OVNI}

% Chefe da divisao de Pos-Graduacao
% Obrigat�rio para Teses
\boss{Prof.~Dr.}{Cicrano de Tal}

% Banca Examinadora
% Obrigat�rio para Teses
\examiner{Prof. Dr.}{Fulano Beltrano}{Presidente}{ITA}%
\examiner{Prof. Dr.}{Cicrano Fulano}{Membro Externo}{UXXX}%
\examiner{Prof. Dr.}{Beltrano Cicrano}{Membro Externo}{UYYY}%
\examiner{Prof. Dr.}{Fulano de Tal}{Membro}{ITA}%
\examiner{Prof. Dr.}{Beltrano Fulano}{Membro}{ITA}%

% Data da defesa
\date{Fevereiro}{2015}

% Palavras-Chaves informadas pela Biblioteca -> utilizada na CIP
% Obrigat�rio para Teses
\kwcip{Teses}
\kwcip{Estilos}
\kwcip{Italus}

% Glossario
\makeglossary
\frontmatter

\begin{document}

% Folha de Rosto
\maketitle

% Dedicatoria
\begin{itadedication}
Aqui pode ser escrita uma dedicat�ria. N�o � obrigat�ria.
\end{itadedication}

% Agradecimentos
\begin{itathanks}
\input{parts/agradecimentos}
\end{itathanks}

% Ep�grafe
\thispagestyle{empty}
\ifhyperref\pdfbookmark[0]{\nameepigraphe}{epigrafe}\fi
\begin{flushright}
\begin{spacing}{1}
\mbox{}\vfill
{\sffamily\itshape
``If I have seen farther than others,\\
it is because I stood on the shoulders of giants.''\\}
--- \textsc{Sir~Isaac Newton}
\end{spacing}
\end{flushright}

% Resumo
\begin{abstract}
O reconhecimento de objetos em uma cena para posterior uso em realidade aumentada 
depende de diversas vari�veis, causando a necessidade do uso de t�cnicas 
espec�ficas para cada cen�rio, sendo portanto, um estudo de fronteiras para a melhor escolha 
do algoritmo de reconhecimento, de acordo com a aplica��o em quest�o de grande
valia para o meio acad�mico. 
Esta tese se prop�e a pesquisar, categorizar e tra�ar fronteiras das t�cnicas
conhecidas, tendo como caso de uso a manuten��o de aeronaves feita dentro de
centros fechados, utilizando as t�cnicas BRISK,FAST,FREAK,GFTT,MSER,
 ORB,STAR,SURF,SIFT em uma an�lise aplicada com imagens reais de janelas de
 inspe��o do Embraer ERJ-190 para reconhecimento de objetos e posteriores
 aplica��es em manuten��o.
 Comparando todas as t�cnicas quanto � cad�ncia e � precis�o de reconhecimento
 de caracter�sticas, � poss�vel selecionar GFTT e ORB
 como t�cnicas mais apropriadas ao contexto, por terem seus resultados de
 varia��o de rota��o, escala, briho e \emph{blur} dentro de uma faixa esperada
 para o contexto de manuten��o.
 


\end{abstract}
% Palavras Chave
% No manual nao consta palavras-chaves
%\keywords{Teses, Estilos, Italus}

% Abstract
\begin{englishabstract}
There are several augmented reality techniques, although each one has its flaws due to the environment or other external constraints. The study of boundaries and constraints can provide to the developer more decision power while choosing the appropriate technique. This essay provides a method to, based on common parameters and situation, chose the appropriated technique.
\end{englishabstract}
% Keywords
% Idem Palavras-chaves ...
%\englishkeywords{Theses, Styles, Italus}

% sumario
\tableofcontents
% lista de figuras
\listoffigures
% lista de tabelas
\listoftables
% lista de abreviaturas
\listofabbreviations
\begin{table}[h]
\begin{tabular}{|l|l|lll}
\label{table:acronim}
\cline{1-2}
HMD & Head-Mounted Display   &  &  &  \\ \cline{1-2}
AR  & Augmented Reality      &  &  &  \\ \cline{1-2}
VR  & Virtual Reality      &  &  &  \\ \cline{1-2}
DoG & Difference of Gaucians &  &  &  \\ \cline{1-2}
LoG & Laplacian of Gaucians  &  &  &  \\ \cline{1-2}
\end{tabular}
\end{table}
% lista de simbolos
%\listofsymbols
%\input{parts/listasimbolos}

\mainmatter
% Os capitulos comecam aqui

\chapter{Motiva��o}
\section{Motiva��o}

O reconhecimento de estruturas e sistemas de forma autom�tica no campo da
manuten��o, pode propiciar a constru��o de ferramentas de capacita��o, dentre
v�rias outras possibilidades, auxiliando al�m de garantir maior confiabilidade
no diagn�stico de problemas. 
Um dos mais b�sicos problemas atualmente limitando o ramo da Realidade Aumentada
� a etapa de registro.
A Realidade Aumentada prev� imers�o entre o mundo virtual e o mundo real e por
isso para que a experi�ncia seja coerente � necess�rio que os dois mundos estejam bem sincronizados e propriamente alinhados.
Em algumas situa��es tal sincronia aumenta a experi�ncia, entretanto, tal
alinhamento � primordial, por exemplo em aplica��es m�dicas em uma aplica��o de
biopsia.Se o objeto n�o estiver no espa�o e tempo real, a informa��o
fornecida ao cirurgi�o poder� por em risco a vida do paciente. Na maioria das aplica��es de tempo real, problemas
 de registro podem invalidar o uso da Realidade Aumentada.
Um outro problema que pode ocorrer com falhas de registro � acentuado por um fen�meno conhecido como
 visual capture \cite{Welch78} que � a tend�ncia do c�rebro em capturar com
 mais qualidade est�mulos visuais, do que qualquer outro sentido. 
  Nesses casos, o sentido da vis�o tende a sobrepor os outros sentidos.
Assim como um ventr�loco consegue enganar quem assiste um show acreditando que o
som sai da boca do boneco o usu�rio de uma aplica��o de realidade aumentada tender� a acreditar no que v�, 
mesmo que esteja defasado no espa�o/tempo.
No caso do erro se tornar sistem�tico o usu�rio tende a se acostumar
inconscientemente adaptar-se ao erro, corrigindo o efeito.
Erros de registro s�o dif�ceis de controlar adequadamente devido � grande precis�o requerida das diversas fontes 
de erro. As fontes de erro podem ser divididas em est�ticas e din�micas sendo as est�ticas contornadas com calibra��o
 pr�via de sensores entretanto os erros din�micos s�o mais dif�ceis porque s�o suscept�veis a tempo diferen�a de tempo
  entre o real e o apresentado na tela e com o ac�mulo de erro.
O reconhecimento de objetos na cena permeia tamb�m:
\begin{itemize}
  \item O contexto da cena, sendo que com conhecimento pr�vio do cen�rio se torna bem mais f�cil;
  \item O material do qual o objeto � feito, porque caso seja feito de materiais reflexivos, os algoritmos
   podem confundir o reflexo de outros objetos com informa��es a reconhecer;
  \item O tamanho do objeto, pois de acordo com a escala do objeto, muitas informa��es que poderiam ser
   boas para o reconhecimento podem estar pr�ximas demais dificultando o posteior casamento de informa��es. \ldots
\end{itemize}

Portanto para que as diversas fontes de erros din�micos n�o sejam um impeditivo para o reconhecimento, de 
acordo com a cena, algoritmos diferentes devem ser selecionados por terem peculiaridades e caracter�sticas
 que garantam um registro direcionados ao tipo de desafio que encontrar�o, al�m de j� ter informa��es 
 pr�vias, o que facilita na sele��o de caracter�sticas.



\chapter{Objetivo}
\section{Objetivos}
O presente trabalho tem como objetivo geral avaliar o ambiente de manuten��o
aeron�utico, no contexto de janelas de inspe��o, tra�ando estrat�gias de
reconhecimento de items de manuten��o.
Para a consecu��o do objetivo geral, foram definidos os seguintes objetivos espec�ficos:


\begin{itemize}
\item Avaliar os algoritmos cl�ssicos de reconhecimento;
\item Aplicar os algoritmos cl�ssicos � situa��es reais;
\item Selecionar algoritmo mais adequado para o contexto.
\end{itemize}
	
	
Esta tese prop�e o uso da realidade aumentada no cen�rio de manuten��o de
aeronaves, tendo como objetivo determinar a melhor estrat�gia de reconhecimento
de caracter�sticas dos objetos no contexto, para que posteriormente seja
utilizado em ferramentas de auxilio na manuten��o.


\chapter{Restri��es}
\section{Vari�veis de contorno}
\label{sec:variaveiscontorno}
O cen�rio de reconhecimento de objetos dentro da aeronave traz alguns desafios que devem ser contornados
\begin{itemize}
\item Pouca ilumina��o em ambientes internos
\item Objetos muito parecidos entre si
\item Alguns objetos com textura
\item Objeto brilhante
\end{itemize}
\section{Restri��es}
O contexto dessa tese prev� o cen�rio de manuten��o com o uso de realidade aumentada como uma ferramenta para auxiliar nas tarefas rotineiras portanto algumas vari�veis devem ser consideradas para garantir a viabilidade de implanta��o da abordagem:
\begin{itemize}
\item Velocidade de reconhecimento
\item Qualidade do reconhecimento
\item Invari�ncia quanto � par�metros ambientais
\end{itemize}

\chapter{Conceitos}
\section{Realidade Aumentada}
A realidade aumentada como citado em \cite{SurveyAR} � uma t�cnica de vis�o
computacional em que valendo-se de artefatos do mundo real tem por objetivo causar sensa��o de imers�o do usu�rio em um ambiente aumentado por artefatos virtuais, ao contr�rio de ambientes puramente virtuais como � comum em aplica��es de realidade virtual.
Idealmente o mundo virtual se torna imersivo o suficiente para que o usu�rio n�o consiga distinguir o real do virtual.
Alguns autores definem AR como tendo a necessidade de utilizar-se interfaces visuais port�teis para que a usabilidade tenha mais coer�ncia com a proposta inicial de garantir uma experi�ncia imersiva.
As imagens s�o obtidas por c�meras e o resultado apresentado em dispositivos como projetores ou 
displays como monitores, tablets ou head-mounted display (HMD). Como mostrado em
\cite{Devices}
\section{Head-Mounted Displays}
� um equipamento utilizado na cabe�a de forma que as duas m�os do usu�rio fiquem livres e tem por objetivo exibir imagens e �udio, sendo uma interface muito utilizada tanto em RV quanto em RA.
Os HMD basicamente s�o dispositivos constitu�dos de duas telas posicionadas frente ao olho do usu�rio.
Com duas telas, a tecnologia pode ser empregada para exibir imagens estereosc�picas apresentando os respectivos pontos de vista de cada olho para cada tela, o que contribui em muito na experi�ncia de imers�o.
Os HMDs funcionam tamb�m como dispositivos de entrada de dados, porque cont�m sensores de rastreamento que medem a posi��o e orienta��o da cabe�a, transmitindo esses dados ao computador.
Existem dois tipos de HMDs: Feed-Through e See-Through
\subsection{Feed-Through}
S�o dispositivos que � um sistema fechado de visualiza��o de imagens, em que o
usu�rio consegue enxergar somente o que � mostrado no display, sendo assim o
resultado apresentado � sempre a soma da imagem real com informa��es
superpostas. Como mostrado na figure~\ref{fig:feedthrough}

\begin{figure}[h!]
\centering
\includegraphics[scale=0.8]{images/feedthrough}
\caption{Arquitetura do Feed-Trough.}
\label{fig:feedthrough}
\end{figure}


\subsection{See-Through}
S�o dispositivos constru�dos com lentes transl�cidas em que o usu�rio enxerga o
mundo real e com algum tipo de sistema que sobrepoe na lente as informa��es
adicionais. Como mostrado na figure~\ref{fig:seethrough}

\begin{figure}[h!]
\centering
\includegraphics[scale=0.8]{images/seethrough}
\caption{Arquitetura do See-Trough.}
\label{fig:seethrough}
\end{figure}


\subsection{Projetores}
O uso de projetores possibilita uma abordagem de realidade aumentada diferente porque pode ser utilizada para cobrir superf�cies largas, projetando sobre objetos como carros, pessoas, pr�dios, etc�.
Um problema dessa abordagem � que a calibra��o se faz necess�ria em v�rias situa��es.

\subsection{Monitores}
O uso de monitores reduz bastante o custo da aplica��o apesar de ter perda de
imers�o por ser um m�todo de visualiza��o indireta, o que implica o usu�rio
ficar olhando na dire��o do monitor, entretanto existe a possibilidade de
compartilhar os resultados da RA com mais de uma pessoa ao mesmo tempo. Como
mostrado na figure~\ref{fig:monitores}

\begin{figure}[h!]
\centering
\includegraphics[scale=0.7]{images/monitores}
\caption{Realidade Aumentada com projetores.}
\label{fig:monitores}
\end{figure}

\section{Caracter�sticas Locais}
Caracter�sticas locais s�o padr�es em imagens que diferenciam de seu vizinho imediato, geralmente � caracterizada por mudan�as em propriedades das imagens ou diversas propriedades simultanamente. As propriedades mais comuns s�o intensidade, cor e texturas.

\subsection{Propriedades da Caracter�stica Local ideal}
Algoritmos de reconhecimento baseam-se em compara��o de caracter�sticas recuperadas da cena. A recupera��o e compara��o de pontos tem um custo computacional relevante perto do tempo de execu��o da aplica��o, portanto selecionar o menor n�mero poss�vel de caracter�sticas aumenta o desempenho e diminui o tempo de resposta da aplica��o.
Garantir que estamos selecionando boas caracter�sticas pode ser crucial na efic�cia do reconhecimento.
Segundo \cite{localfeaturedetector} , boas caracter�sticas devem ter as
seguintes propriedades:

\begin{itemize}
	\item \textbf{Repetibilidade}: Dadas duas imagens do mesmo objeto ou cena,
	tomadas em condi��es ou pontos de vista diferentes, uma porcentagem alta de caracter�sticas deve ser reconhecidas se estiverem vis�veis.
	\item \textbf{Distin��o}: Os padr�es reconhecidos t�m de ser poss�veis de serem
	distinguidos entre si para facilitar o casamento.
	\item \textbf{Localidade}: As caracter�sticas devem ser locais para reduzir a
	probabilidade de oclus�o.
	\item \textbf{Quantidade}: O n�mero de pontos detectados tem que ser o
	suficiente para que mesmo objetos pequenos tenham minimamente caracter�sticas que possam ser localizados e para que o 
	objeto possa sofrer oclus�o e ainda assim seja reconhecido.
	\item \textbf{Exatid�o}: As caracter�sticas detectadas tem que ser localizadas
	com o m�ximo de exatid�o poss�vel com respeito tanto referente � posi��o quanto
	� escala.
	\item \textbf{Efici�ncia}:De prefer�ncia  a detec��o deve ser o mais r�pido
	poss�vel.
\end{itemize}

\subsection{Features}
Antes de compreender como � feito o reconhecimento e registro de imagens � importante nos perguntar
 como n�s conseguimos reconhecer objetos em uma cena, como conseguimos comparar facilmente objetos em duas 
 imagens distintas. Somos treinados desde cedo a diferenciar formas geom�tricas, perceber escalas diferentes 
 ou mesmo reconhecer o mesmo objeto independente de como est� posicionado na cena buscando padr�es que 
 categorizem e diferenciem o objeto. Instintivamente conseguimos reconhecer boas caracter�sticas e localizar objetos.
Na imagem~\ref{fig:padroes} temos uma imagem de um pr�dio e seis recortes, dos quais
conseguimos facilmente reconhecer com precis�o a de letra E e F, as de letra A,B,C,D
 podemos identificar poss�veis localiza��es mas n�o podemos dizer com certeza onde est�o na imagem.

\begin{figure}[h!]
\centering
\includegraphics[scale=1.0]{images/features}
\caption{Reconhecimento de padr�es.\cite{understandingfeatures}}
\label{fig:padroes}
\end{figure}
As caracter�sticas E e F s�o o que chamamos de good features pois o n�vel de
certeza � bem alto.

\begin{figure}[h!]
\centering
\includegraphics[scale=1.0]{images/featuresregioes}
\caption{Regi�es de reconhecimento de padr�es.\cite{understandingfeatures}}
\label{fig:featuresregioes}
\end{figure}

A Imagem~\ref{fig:featuresregioes} ilustra os tipos de caracter�sticas. A regi�o
azul n�o possibilita diferenciar onde est� na imagem, a regi�o preta pode ser confundida com qualquer uma das regi�es
 ao deslocarmos horizontalmente, a imagem vermelha nos possibilita diferenciar e reconhecer o canto da imagem verde com precis�o milim�trica.
Podemos ent�o concluir que uma caracter�stica � boa para ser utilizada como par�metro de entrada
 para algoritmos de reconhecimento, quanto maior foi o n�vel de certeza da sua localiza��o, 
 o que facilita o \textbf{Feature Detection}.
Para localizar o mesmo objeto em outra imagem � necess�rio identificarmos a regi�o onde se encontra, 
caso contr�rio no exemplo da imagem~\ref{fig:padroes} seria imposs�vel
localizar uma janela espec�fica. Tal descri��o de contexto � chamada de
\textbf{Feature Description}. Uma vez de posse da caracter�stica e do seu
contexto � poss�vel reconhecer o objeto de fato.



\section{Cad�ncia}
� a medida do n�mero de quadros individuais que um determinado dispositivo �ptico ou eletr�nico processa e exibe por unidade de tempo. Em geral a cad�ncia � medida em fps.
Em cinema a cad�ncia de proje��o padr�o desde 1929 foi fixada em 24fps mas no per�odo do cinema mudo a maioria dos filmes eram rodados com cad�ncia entre 16 e 20fps.
Em v�deo, os principais sistemas lidam com cad�ncia entre 25fps(PAL) e 30fps(NTSC).
As aplica��es devem ter cad�ncia toler�vel dependendo de seu uso, segundo \cite{Tang93whydo} para aplica��es interaticas o m�nimo toler�vel � de 5fps enquanto para aplica��es de anima��es fluidas de 30fps.
Sendo a cad�ncia, a freq��ncia entre frames, deve ser contabilizado o tempo de gerar a informa��o e o tempo de dispor a informa��o no dispositivo �ptico.
O tempo de cada frame � calculado como mostrado na equa��o~\ref{eq:fps} 

\begin{equation}
t_{frame} =  \frac{1}{fps}
\label{eq:fps}
\end{equation}

No caso de cad�ncia m�nima de 5fps, temos quadros com tempo menor que 200ms,
portanto as an�lises devem ser balisadas a tempos menores.

Nowaday cameras have problems of distortion that can be treated because there are constant distortions and calibrations issues.

Radial distortion as "barrel" or "fish-eye" effect

barrel

Barrel lens distortion is an effect associated with wide-angle lenses and, in particular, zoom wide-angles. This effect causes images to be spherized, which means the edges of images look curved and bowed to the human eye. It almost appears as though the photo image has been wrapped around a curved surface.

Tangential Distortion


\subsection[Reconhecimento]{Reconhecimento}
\begin{frame}
\frametitle{Reconhecimento}

\begin{figure}[H]
\centering
\includegraphics[scale=0.4]{images/reconhecimento}
\caption{Ilustra��o do procedimento de reconhecimento com caracter�sticas
locais.
Fonte \cite{localfeaturedetector}}
\label{fig:reconhecimento}
\end{figure}

\end{frame}

\begin{frame}
Processo de reconhecimento como ilustrado na
imagem~\ref{fig:reconhecimento}:
\begin{itemize}
	\item Encontrar um grupo de \emph{keypoints} distintos;
	\item Definir uma regi�o em torno de cada \emph{keypoint};
	\item Extrair e normalizar o conte�do da regi�o;
	\item Calcular um descritor para a regi�o normalizada;
	\item Encontrar correspond�ncias de descritores.  
\end{itemize}



\end{frame}






\chapter{Metodologia}
\section{Metodologia}
\subsection{Defini��o de par�metros}

Os testes s�o realizados adicionando vari�veis de forma artificial por meio de
transforma��es afins, de forma a emular comportamentos ou situa��es encontradas
no ambiente de manuten��o:
\begin{itemize}
	\item Varia��o de escala para simular a aproxima��o dentro da janela de
	inspe��o;
	\item Varia��o de rota��o para simular a movimenta��o durante a manuten��o;
	\item Varia��o de ilumina��o para simular manuten��es feitas em hor�rios do dia
	e ilumina��o diferente, como por exemplo, ambientes com neve com brilho muito
	maior;
	\item Adi��o de \emph{Blur} para emular ambientes com muita poeira ou
	esfuma�ados.
\end{itemize}

A defini��o das faixas de par�metros utilizadas foi feita com inspe��o em campo.

\section{An�lise de dados}
\label{sec:analisededados}

A an�lise de desempenho de cada t�cnica deve ser balizada por par�metros que garantam que a aplica��o seja vi�vel para o
 contexto citado na Se��o \ref{sec:restricao}. Para tal, deve-se considerar os
 par�metros estimados abaixo definidos por inspe��o: 

\begin{itemize}
	\item Cad�ncia > 5fps ou seja, tempo < 200ms \cite{Tang93whydo};
	\item Robustez � escala:  0.5 < escala < 1.2;
	\item Robustez � rota��o:  0$^{\circ}$ < rota��o < 30$^{\circ}$; 
	\item Robustez � ilumina��o: -50 < ilumina��o < 50
	\item Robustez � \emph{blur}: \emph{kernelsize} < 4
\end{itemize}

Uma an�lise mais precisa dos par�metros n�o faz parte do escopo desse trabalho
pois envolve pesquisa de campo em situa��es adversas. Portanto, tais par�metros
foram escolhidos para validar o m�todo de sele��o das t�cnicas, sendo assim,
identificados por experi�ncia como usuais para o contexto, entretanto, n�o h�
perda de generalidade no m�todo, posto que pode ser
executado da mesma maneira com outros valores. 




\chapter{Prot�tipos}
O escopo de desenvolvimento dessa tese tem por tra�ar um m�todo de escolha do
algoritmo a ser utilizado dependente de aspectos descritos na se��o 2.1 Vari�veis de Contorno, os testes
 ser�o feitos em duas etapas: 
Defini��o de par�metros de contorno e avalia��o dos algoritmos mais conhecidos no contexto aeron�utico

\section{Testes de defini��o de fronteiras de par�metros}

\section{Testes de reconhecimento de padr�es}

Utilizando os par�metros definidos na etapa anterior ser�o rodados todos os algoritmos para avaliar qual se encaixa 
melhor em cada uma das situa��es.
O prot�tipo para testes ser� realizado utilizando o OpenCV por j� disponibilizar os algoritmos mais utilizados para 
o reconhecimento de padr�es e algoritmos de vis�o computacional.
Nesse prot�tipo ser� avaliado a ader�ncia dos algoritmos:Fast, GoodFeaturesToTrack, Mser, Star, Sift, Surf.
Segundo os seguintes crit�rios:\par
\textbf{Velocidade} - para garantir que a aplica��o poder� ser utilizada como
una ferramenta de tempo real � imprescind�vel buscar uma aplica��o que rode naturalmente a 30
 frames por segundo. Tal realidade � facilmente atingida em computadores com v�rios n�cleos 
 como os core i7 mas � importante lembrar que o cen�rio de realidade aumentada para ter um 
 contexto coerente com a proposta de imers�o � composto por dispositivos portateis que via 
 de realidade regra n�o tem poder de processamento t�o alto.\par
\textbf{Qualidade} - ccaracter�sticas detectadas s�o geralmente utilizadas
posteriormente em etapas de rastrio e casamento de padr�o. Para os testes de rastreio
 ser�o adicionadas �s imagens algumas transforma��es afins, como escala, rota��o e ilumina��o
  e ent�o estimar a qualidade das caracter�sticas.\par
\textbf{Ilumina��o e invari�ncia � escala} - Detectores de caracter�sticas tem
que ter a habilidade de reconhecer caracter�sticas independente do tamanho do objeto. A invari�ncia deve ser verdadeira para varia��o de ilumina��o. Varia��es pequenas de ilumina��o e contraste n�o devem afetar o detector significativamente. As c�meras atuais em geral tem controle autom�tico de ganho que ajusta automaticamente a esposi��o evitando sub ou super esposi��o.
\par

Ser�o utilizados as seguintes transforma��es que tem por finalidade avaliar as
varia��es de ambiente
\begin{itemize}
  \item \textbf{Rota��o: }Robustez � rota��o tamb�m � uma caracter�stica
  importante pois nem sempre encontraremos situa��es em que o dispositivo de RA
  sr� posicionado na mesma horienta��o o tempo todo, garantindo uma maior
  autonomia a quem executa a manuten��o
  \item \textbf{Brilho:} A varia��o de brilho pode emular tanto varia��es
  temporais, como dia e noite, mas tamb�m pode emular situa��es de muita
  luminosidade como neve por exemplo
  \item \textbf{Blur: } A robustes a Blur auxilia na resposta do algoritmo
  frente � movimentos bruscos
  \item \textbf{Escala: } A robustez � escala do objeto garante que o mesmo ser�
  reconhecido tanto ao longe quanto em situa��es bem pr�ximas
\end{itemize}


\section{Ambiente}
Os testes tem por objetivo garantir as restri��es descritas na se��o 3. Constraints, portanto � o desempenho e o tempo de reconhecimento s�o uma caracter�stica relevante.
Os resultados obtidos pelos testes descritos nessa tese foram executado em uma m�quina com a seguinte configura��o:

Processador: Core 2-duo 2.2GHz
Mem�ria: 4GB
Placa de V�deo: NVidia GForce 9300M GS

Biblioteca: OpenCV 2.4.10

\subsection{OpenCV}
Distribu�do sob licensa BSD e portanto livre para uso acad�mico e comercial. 
Possui interfaces para C++, C, Python e Java suportando Windows, Linux, Mac OS, iOS 
e Android. OpenCV foi desenvolvido para resultados eficientes e com foco em aplica��es
 de tempo real, podendo tomar vantagem de processamentos paralelos, utilizando-se de OpenCL 
 aproveitando-se da acelera��o de hardware ou mesmo de plataformas heterog�neas. Adotado ao redor
  do mundo em v�rias pesquisas com uma comunidade de  mais de 47 mil pessoas com mais de 9 milh�es
   de downloads. Adotado como plataforma de desenvolvimento para aplica��es de arte interativa a 
   aplica��es de rob�tica avan�ada.
   
 \section{Caso de uso}

Como caso de uso ser� adotado a janela de inspe��o frontal [COLOCAR AQUI UMA 
REFER�NCIA MELHOR] e selecionado uma pe�a. 
A sele��o da pe�a ser� feita de tal forma que contenha informa��es relevantes para estressar 
os limites dos algoritmos de registro, portanto um estudo pr�vio das caracter�sticas dos 
algoritmos para abstrair as caracter�sticas mais relevantes se faz necess�rio. 



ESCREVER as peculiaridades do ambiente para poder decidir qual abordagem vou tomar



\chapter{Resultados}
\newcommand{\minscale}{0.2}
\newcommand{\maxscale}{1.2}
\newcommand{\minangle}{0}
\newcommand{\maxangle}{30}
\newcommand{\minblur}{0}
\newcommand{\maxblur}{4}
\newcommand{\minbright}{-50}
\newcommand{\maxbright}{50}
\newcommand{\correctmatches}{0.6}
\newcommand{\cadencia}{200}
\newcommand{\maxtime}{10000}

\newcommand{\miny}{-10}
\newcommand{\maxy}{3}
\newcommand{\minx}{-150}
\newcommand{\maxx}{400}
\newcommand{\ylimit}{0.6}
\newcommand{\ylabel}{Percent of correct matches}



\newcommand{\tec}{brisk}
\newcommand{\gtype}{pres}



\newcommand{\graphwidth}{0.4}


A abordagem de testes e an�lise de desempenho � feita como descrito em
\cite{performance_evaluation}.

Para que sejam respeitados as restri��es citadas na Se��o~\ref{sec:restricao} e
para que a aplica��o seja coerente para uso cotidiano no contexto de manuten��o
de aeronaves, a velocidade de reconhecimento � muito importante.


\section{An�lise de par�metros pontuais}
As implementa��es existentes no OpenCV para as t�cnicas utilizadas nessa tese
tem par�metros configur�veis, o que pode garantir uma maior ader�ncia �s
necessidades dependendo do contexto, portanto antes de compararmos as t�cnicas
entre si � importante garantirmos que a configura��o utilizada de cada uma se
encaixa nas restri��es citadas em \ref{sec:restricao}. Cada t�cnica possibilita variar
alguns par�metros que ser� analisado para encontrar a configura��o mais
adequada.




\pagebreak

\subsection{BRISK}
A assinatura do m�todo � \newline
BRISK(int thresh=30, int octaves=3, float patternScale=1.0f);

\renewcommand{\tec}{brisk}
\renewcommand{\gtype}{pres}


\renewcommand{\miny}{-10}
\renewcommand{\maxy}{3}
\renewcommand{\minx}{-150}
\renewcommand{\maxx}{400}
\renewcommand{\ylimit}{0.6}
\renewcommand{\ylabel}{Percent of correct matches}

 
\begin{figure}[H]
\centering

\begin{tikzpicture}
	\pgfplotsset{small}
	\matrix{
	%scale
	
	 
	
\begin{axis}
	[
	   width=\graphwidth\textwidth,
    ylabel=$\ylabel$, % Set the labels
    xlabel=$Scale Factor$,
	legend entries={$d=40$,$d=50$,$d=60$,$d=70$},	
	title= Robustness to scaling
    ]
%	\addplot table [x=Argument, y=BRISK-40, col sep=comma]
	% {graphs/scale-\gtype.csv}; \addplot table [x=Argument, y=BRISK-50, col	sep=comma]{graphs/scale-\gtype.csv};
%	\addplot table [x=Argument,	y=BRISK-60, col sep=comma]	 
	 %{graphs/scale-\gtype.csv}; \addplot table [x=Argument, y=BRISK-70, col
	 % sep=comma]  {graphs/scale-all-\gtype.csv}; 
	 \addplot table [x=Argument, y=BRISK-40, col sep=comma]	{graphs/scale-all-\gtype.csv}; 
	\addplot table [x=Argument, y=BRISK-50, col	sep=comma]{graphs/scale-all-\gtype.csv};
	 \addplot table [x=Argument,y=BRISK-60, col sep=comma]	 {graphs/scale-all-\gtype.csv};
	  \addplot table [x=Argument, y=BRISK-70, col	 sep=comma] {graphs/scale-all-\gtype.csv};
	
	
	%eixo horizontal
	\addplot[red,sharp plot,update limits=false] coordinates{(\minx,\ylimit)(\maxx,\ylimit)};
	 \fill [opacity=0.4,red!25] (axis	cs:\minx,\miny)	rectangle (axis	
	 cs:\maxx,\ylimit);
	
	
	%eixo vertical
	\addplot[blue,sharp plot,update limits=false] coordinates {(\minscale,\miny)(\minscale,\maxy)} ;
		\fill [opacity=0.4,blue!25] (axis cs:0,-1) rectangle (axis	cs:\minscale,\maxy);
	
	\addplot[blue,sharp plot,update limits=false] coordinates {(\maxscale,\miny)(\maxscale,\maxy)} ;
	\fill [opacity=0.4,blue!25] (axis cs:\maxscale,\miny) rectangle (axis	cs:\maxx,\maxy);
	
\end{axis}
&
%rotation
\begin{axis}
	[
	   width=\graphwidth\textwidth,
    ylabel=$\ylabel$, % Set the labels
    xlabel=$Angle(Degree)$,
	title= Rotation Invariance 
    ]
	\addplot table [x=Argument, y=BRISK-40, col sep=comma]	{graphs/rot-all-\gtype.csv};
	 \addplot table [x=Argument, y=BRISK-50, col	sep=comma]{graphs/rot-all-\gtype.csv};
	  \addplot table [x=Argument,	y=BRISK-60, col sep=comma]	 {graphs/rot-all-\gtype.csv}; 
	  \addplot table	[x=Argument, y=BRISK-70, col	 sep=comma] {graphs/rot-all-\gtype.csv};

	%eixo horizontal
	\addplot[red,sharp plot,update limits=false] coordinates
	{(-100,\ylimit) (400,\ylimit)};
		\fill [opacity=0.4,red!25] (axis cs:-100,\miny) rectangle (axis
	cs:\maxx,\ylimit);
	
	
		%eixo vertical
	\addplot[blue,sharp plot,update limits=false] coordinates {(\minangle,\miny)
	(\minangle,\maxy)} ;
		\fill [opacity=0.4,blue!25] (axis cs:-100,\miny) rectangle (axis
	cs:\minangle,\maxy);
	
	\addplot[blue,sharp plot,update limits=false] coordinates {(\maxangle,\miny)
	(\maxangle,\maxy)} ;
	\fill [opacity=0.4,blue!25] (axis cs:\maxangle,\miny) rectangle (axis
	cs:\maxx,\maxy);

	
\end{axis}
\\
%blur
\begin{axis}
	[
	   width=\graphwidth\textwidth,
       ylabel=$\ylabel$, % Set the labels
    xlabel=$Kernel size$,
	title= Robustness to blur]
	\addplot table [x=Argument, y=BRISK-40, col sep=comma]	{graphs/blur-all-\gtype.csv};
	 \addplot table [x=Argument, y=BRISK-50, col	sep=comma]{graphs/blur-all-\gtype.csv};
	  \addplot table [x=Argument,	y=BRISK-60, col sep=comma]	 {graphs/blur-all-\gtype.csv}; 
	  \addplot table	[x=Argument, y=BRISK-70, col	 sep=comma] {graphs/blur-all-\gtype.csv};
	 
	%eixo horizontal
	\addplot[red,sharp plot,update limits=false] coordinates{(\minx,\ylimit)(\maxx,\ylimit)};
	 \fill [opacity=0.4,red!25] (axis	cs:\minx,\ylimit) rectangle(axis
	 cs:\maxx,\miny);


				%eixo vertical
	\addplot[blue,sharp plot,update limits=false] coordinates
	{(\minblur,\miny)(\minblur,\maxy)} ; \fill [opacity=0.4,blue!25] (axis	cs:\minx,\miny) rectangle (axis cs:\minblur,\maxy);
	
	\addplot[blue,sharp plot,update limits=false] coordinates
	{(\maxblur,\miny)(\maxblur,\maxy)} ; \fill [opacity=0.4,blue!25] (axis	cs:\maxblur,\miny) rectangle (axis cs:\maxx,\maxy);


\end{axis}
&
%bright
\begin{axis}
	[
	   width=\graphwidth\textwidth,
       ylabel=$\ylabel$, % Set the labels
    xlabel=$Change of brightness$,
	title= Brightness Invariance 
    ]
	\addplot table [x=Argument, y=BRISK-40, col sep=comma]	{graphs/bright-all-\gtype.csv};
	 \addplot table [x=Argument, y=BRISK-50, col	sep=comma]{graphs/bright-all-\gtype.csv};
	  \addplot table [x=Argument,	y=BRISK-60, col sep=comma]	 {graphs/bright-all-\gtype.csv}; 
	  \addplot table	[x=Argument, y=BRISK-70, col	 sep=comma] {graphs/bright-all-\gtype.csv};
	
		%eixo horizontal
	\addplot[red,sharp plot,update limits=false] coordinates	{(\minx,\ylimit)(\maxx,\ylimit)};
	 \fill [opacity=0.4,red!25] (axis cs:\minx,\ylimit) rectangle(axis cs:\maxx,\miny);
		
		
						%eixo vertical
	\addplot[blue,sharp plot,update limits=false] coordinates
	{(\minbright,\miny)(\minbright,\maxy)} ; 
	\fill [opacity=0.4,blue!25] (axis	cs:\minx,\miny) rectangle (axis
	cs:\minbright,\maxy);
	
	\addplot[blue,sharp plot,update limits=false] coordinates
	{(\maxbright,\miny)(\maxbright,\maxy)} ; \fill [opacity=0.4,blue!25] (axis
	cs:\maxbright,\miny) rectangle (axis cs:\maxx,\maxy);
		
	
\end{axis}
\\
}

\end{tikzpicture}
\label{graph:briskresultpres}
\caption{Resultado de testes individuais de preformance com BRISK}
\end{figure}




%TIME

\renewcommand{\tec}{brisk}
\renewcommand{\gtype}{time}

\renewcommand{\maxy}{3000}
\renewcommand{\miny}{-155}



\renewcommand{\ylimit}{200}
\renewcommand{\ylabel}{Time(Ms)}


\begin{figure}[H]
\centering
\begin{tikzpicture}
	\pgfplotsset{small}
	\matrix{
	%scale
\begin{axis}
	[
	   width=\graphwidth\textwidth,
      ylabel=$\ylabel$, % Set the labels
    xlabel=$Scale Factor$,
	legend entries={$d=40$,$d=50$,$d=60$,$d=70$},
	title= Robustness to scaling
    ]
	\addplot table [x=Argument, y=BRISK-40, col sep=comma]	{graphs/scale-all-\gtype.csv};
	 \addplot table [x=Argument, y=BRISK-50, col	sep=comma]{graphs/scale-all-\gtype.csv};
	  \addplot table [x=Argument,	y=BRISK-60, col sep=comma]	 {graphs/scale-all-\gtype.csv}; 
	  \addplot table	[x=Argument, y=BRISK-70, col	 sep=comma] {graphs/scale-all-\gtype.csv};
	
	%eixo horizontal
	\addplot[red,sharp plot,update limits=false] coordinates{(\miny,\ylimit)
	(\maxx,\ylimit)}; 
	\fill [opacity=0.4,red!25] (axis cs:\miny,\ylimit) rectangle (axis
	cs:\maxx,\maxy);
	
	
	%eixo vertical
	\addplot[blue,sharp plot,update limits=false] coordinates {(\minscale,\miny)(\minscale,\maxy)} ;
		\fill [opacity=0.4,blue!25] (axis cs:\minx,\miny) rectangle (axis		cs:\minscale,\maxy);
	
	\addplot[blue,sharp plot,update limits=false] coordinates {(\maxscale,\miny)	(\maxscale,\maxy)} ;
	\fill [opacity=0.4,blue!25] (axis cs:\maxscale,\miny) rectangle (axis	cs:\maxx,\maxy);
	
\end{axis}
&
%rotation
\begin{axis}
	[
	   width=\graphwidth\textwidth,
      ylabel=$\ylabel$, % Set the labels
    xlabel=$Angle(Degree)$,
	title= Rotation Invariance 
    ]
	\addplot table [x=Argument, y=BRISK-40, col sep=comma]	{graphs/rot-all-\gtype.csv};
	 \addplot table [x=Argument, y=BRISK-50, col	sep=comma]{graphs/rot-all-\gtype.csv};
	  \addplot table [x=Argument,	y=BRISK-60, col sep=comma]	 {graphs/rot-all-\gtype.csv}; 
	  \addplot table	[x=Argument, y=BRISK-70, col	 sep=comma]	  {graphs/rot-all-\gtype.csv};

	
	%eixo horizontal
	\addplot[red,sharp plot,update limits=false] coordinates{(-100,\ylimit)
	(400,\ylimit)}; 
	\fill [opacity=0.4,red!25] (axis cs:-100,\ylimit) rectangle (axis
	cs:400,\maxy);
	
	
		%eixo vertical
	\addplot[blue,sharp plot,update limits=false] coordinates {(\minangle,-100)
	(\minangle,\maxy)} ;
		\fill [opacity=0.4,blue!25] (axis cs:-100,-1) rectangle (axis
	cs:\minangle,\maxy);
	
	\addplot[blue,sharp plot,update limits=false] coordinates {(\maxangle,-1)
	(\maxangle,\maxy)} ;
	\fill [opacity=0.4,blue!25] (axis cs:\maxangle,-1) rectangle (axis
	cs:400,\maxy);

	
\end{axis}
\\
%blur
\begin{axis}
	[
	   width=\graphwidth\textwidth,
      ylabel=$\ylabel$, % Set the labels
    xlabel=$Kernel size$,
	title= Robustness to blur
    ]
	\addplot table [x=Argument, y=BRISK-40, col sep=comma]	{graphs/blur-all-\gtype.csv};
	 \addplot table [x=Argument, y=BRISK-50, col	sep=comma]{graphs/blur-all-\gtype.csv};
	  \addplot table [x=Argument,	y=BRISK-60, col sep=comma]	 {graphs/blur-all-\gtype.csv}; 
	  \addplot table	[x=Argument, y=BRISK-70, col	 sep=comma]	  {graphs/blur-all-\gtype.csv};
	
	%eixo horizontal
	\addplot[red,sharp plot,update limits=false] coordinates{(\minx,\ylimit)
	(\maxx,\ylimit)}; 
	\fill [opacity=0.4,red!25] (axis cs:\minx,\ylimit) rectangle (axis
	cs:\maxx,\maxy);

				%eixo vertical
	\addplot[blue,sharp plot,update limits=false] coordinates
	{(\minblur,\miny)(\minblur,\maxy)} ; \fill [opacity=0.4,blue!25] (axis
	cs:\minx,\miny) rectangle (axis cs:\minblur,\maxx);
	
	\addplot[blue,sharp plot,update limits=false] coordinates
	{(\maxblur,\miny)(\maxblur,\maxy)} ; \fill [opacity=0.4,blue!25] (axis
	cs:\maxblur,\miny) rectangle (axis cs:\maxx,\maxy);


\end{axis}
&
%bright
\begin{axis}
	[
	   width=\graphwidth\textwidth,
      ylabel=$\ylabel$, % Set the labels
    xlabel=$Change of brightness$,
	title= Brightness Invariance 
    ]
	\addplot table [x=Argument, y=BRISK-40, col sep=comma]	{graphs/bright-all-\gtype.csv};
	 \addplot table [x=Argument, y=BRISK-50, col	sep=comma]{graphs/bright-all-\gtype.csv};
	  \addplot table [x=Argument,	y=BRISK-60, col sep=comma]	 {graphs/bright-all-\gtype.csv}; 
	  \addplot table	[x=Argument, y=BRISK-70, col	 sep=comma]	  {graphs/bright-all-\gtype.csv};
	
	%eixo horizontal
	\addplot[red,sharp plot,update limits=false] coordinates{(\minx,\ylimit)
	(\maxx,\ylimit)}; 
	\fill [opacity=0.4,red!25] (axis cs:\minx,\ylimit) rectangle (axis
	cs:\maxx,\maxy);
		
		
						%eixo vertical
	\addplot[blue,sharp plot,update limits=false] coordinates
	{(\minbright,\miny)(\minbright,\maxy)} ; \fill [opacity=0.4,blue!25] (axis
	cs:\minx,\miny) rectangle (axis cs:\minbright,\maxy);
	
	\addplot[blue,sharp plot,update limits=false] coordinates
	{(\maxbright,\miny)(\maxbright,\maxy)} ; \fill [opacity=0.4,blue!25] (axis
	cs:\maxbright,\miny) rectangle (axis cs:\maxx,\maxy);
		
	
\end{axis}
\\
}

\end{tikzpicture}
\label{graph:briskresulttime}
\caption{Resultado de testes individuais de tempo com BRISK}
\end{figure}

\subsection{An�lise}
Fazendo uma an�lise de tend�ncia variando o par�metro thresh de 10 � 40 podemos
inferir que A taxa de acertos:
\begin{itemize}
\item escala � diretamente proporcional
\item rota��o � inversamente proporcional
\item blur � diretamente proporcional
\item ilumina��o � diretamente proporcional
\end{itemize}
O tempo de execu��o todos os par�metros s�o inversamente proporcionais





\subsection{FAST}
A assinatura do m�todo � 
\begin{center}
FastFeatureDetector( int threshold=1, bool nonmaxSuppression=true );
\end{center}

variando o par�metro treshhold em 10,20,30,40 

\renewcommand{\tec}{fast}
\renewcommand{\gtype}{pres}


\renewcommand{\miny}{-10}
\renewcommand{\maxy}{3}
\renewcommand{\minx}{-150}
\renewcommand{\maxx}{400}
\renewcommand{\ylimit}{0.6}
\renewcommand{\ylabel}{Percent of correct matches}


\begin{figure}[H]
\centering

\begin{tikzpicture}
	\pgfplotsset{small}
	\matrix{
	%scale
\begin{axis}
	[
	   width=\graphwidth\textwidth,
    ylabel=$\ylabel$, % Set the labels
    xlabel=$Scale Factor$,
	legend entries={$d=10$,$d=20$,$d=30$,$d=40$},
	title= Robustness to scaling
    ]
	\addplot table[x=Argument, y=FAST-10, colsep=comma]{graphs/scale-all-\gtype.csv};
	 \addplot table[x=Argument, y=FAST-20, col sep=comma]{graphs/scale-all-\gtype.csv};
	  \addplot table[x=Argument,	y=FAST-30, col sep=comma]	 {graphs/scale-all-\gtype.csv}; 
	  \addplot table[x=Argument, y=FAST-40, col	 sep=comma] {graphs/scale-all-\gtype.csv};

	  
	
	%eixo horizontal
	\addplot[red,sharp plot,update limits=false] coordinates{(\minx,\ylimit)(\maxx,\ylimit)};
	 \fill [opacity=0.4,red!25] (axis	cs:\minx,\miny)	rectangle (axis	
	 cs:\maxx,\ylimit);
	
	
	%eixo vertical
	\addplot[blue,sharp plot,update limits=false] coordinates {(\minscale,\miny)(\minscale,\maxy)} ;
		\fill [opacity=0.4,blue!25] (axis cs:0,-1) rectangle (axis	cs:\minscale,\maxy);
	
	\addplot[blue,sharp plot,update limits=false] coordinates {(\maxscale,\miny)(\maxscale,\maxy)} ;
	\fill [opacity=0.4,blue!25] (axis cs:\maxscale,\miny) rectangle (axis	cs:\maxx,\maxy);
	
\end{axis}
&
%rotation
\begin{axis}
	[
	   width=\graphwidth\textwidth,
    ylabel=$\ylabel$, % Set the labels
    xlabel=$Angle(Degree)$,
	title= Rotation Invariance 
    ]
	\addplot table [x=Argument, y=FAST-10, colsep=comma]{graphs/rot-all-\gtype.csv};
	 \addplot table [x=Argument, y=FAST-20, col sep=comma]{graphs/rot-all-\gtype.csv};
	  \addplot table [x=Argument,	y=FAST-30, col sep=comma]	 {graphs/rot-all-\gtype.csv}; 
	  \addplot table	[x=Argument, y=FAST-40, col	 sep=comma] {graphs/rot-all-\gtype.csv};




	
	%eixo horizontal
	\addplot[red,sharp plot,update limits=false] coordinates
	{(-100,\ylimit) (400,\ylimit)};
		\fill [opacity=0.4,red!25] (axis cs:-100,\miny) rectangle (axis
	cs:\maxx,\ylimit);
	
	
		%eixo vertical
	\addplot[blue,sharp plot,update limits=false] coordinates {(\minangle,\miny)
	(\minangle,\maxy)} ;
		\fill [opacity=0.4,blue!25] (axis cs:-100,\miny) rectangle (axis
	cs:\minangle,\maxy);
	
	\addplot[blue,sharp plot,update limits=false] coordinates {(\maxangle,\miny)
	(\maxangle,\maxy)} ;
	\fill [opacity=0.4,blue!25] (axis cs:\maxangle,\miny) rectangle (axis
	cs:\maxx,\maxy);

	
\end{axis}
\\
%blur
\begin{axis}
	[
	   width=\graphwidth\textwidth,
       ylabel=$\ylabel$, % Set the labels
    xlabel=$Kernel size$,
	title= Robustness to blur
    ]
	\addplot table [x=Argument, y=FAST-10, colsep=comma]{graphs/blur-all-\gtype.csv};
	 \addplot table [x=Argument, y=FAST-20, col sep=comma]{graphs/blur-all-\gtype.csv};
	  \addplot table [x=Argument,	y=FAST-30, col sep=comma]	 {graphs/blur-all-\gtype.csv}; 
	  \addplot table	[x=Argument, y=FAST-40, col	 sep=comma] {graphs/blur-all-\gtype.csv};


	
	%eixo horizontal
	\addplot[red,sharp plot,update limits=false] coordinates{(\minx,\ylimit)(\maxx,\ylimit)};
	 \fill [opacity=0.4,red!25] (axis	cs:\minx,\ylimit) rectangle(axis
	 cs:\maxx,\miny);


				%eixo vertical
	\addplot[blue,sharp plot,update limits=false] coordinates
	{(\minblur,\miny)(\minblur,\maxy)} ; \fill [opacity=0.4,blue!25] (axis	cs:\minx,\miny) rectangle (axis cs:\minblur,\maxy);
	
	\addplot[blue,sharp plot,update limits=false] coordinates
	{(\maxblur,\miny)(\maxblur,\maxy)} ; \fill [opacity=0.4,blue!25] (axis	cs:\maxblur,\miny) rectangle (axis cs:\maxx,\maxy);

\end{axis}
&
%bright
\begin{axis}
	[
	   width=\graphwidth\textwidth,
       ylabel=$\ylabel$, % Set the labels
    xlabel=$Change of brightness$,
	title= Brightness Invariance 
    ]
	\addplot table[x=Argument, y=FAST-10, col sep=comma]{graphs/bright-all-\gtype.csv};
	 \addplot table[x=Argument, y=FAST-20, col	sep=comma]{graphs/bright-all-\gtype.csv};
	  \addplot table[x=Argument,y=FAST-30, col sep=comma] {graphs/bright-all-\gtype.csv}; 
	  \addplot table[x=Argument, y=FAST-40, col	 sep=comma] {graphs/bright-all-\gtype.csv};

	
	
	
	
	%eixo horizontal
	\addplot[red,sharp plot,update limits=false] coordinates	{(\minx,\ylimit)(\maxx,\ylimit)};
	 \fill [opacity=0.4,red!25] (axis cs:\minx,\ylimit) rectangle(axis cs:\maxx,\miny);
		
		
						%eixo vertical
	\addplot[blue,sharp plot,update limits=false] coordinates
	{(\minbright,\miny)(\minbright,\maxy)} ; 
	\fill [opacity=0.4,blue!25] (axis	cs:\minx,\miny) rectangle (axis
	cs:\minbright,\maxy);
	
	\addplot[blue,sharp plot,update limits=false] coordinates
	{(\maxbright,\miny)(\maxbright,\maxy)} ; \fill [opacity=0.4,blue!25] (axis
	cs:\maxbright,\miny) rectangle (axis cs:\maxx,\maxy);
		
	
\end{axis}
\\
}

\end{tikzpicture}

\label{graph:fastresult}
\caption{Resultado de testes individuais de preformance com FAST}
\end{figure}


%TIME

\renewcommand{\tec}{FAST}
\renewcommand{\gtype}{time}

\renewcommand{\maxy}{3000}
\renewcommand{\miny}{-155}



\renewcommand{\ylimit}{200}
\renewcommand{\ylabel}{Time(Ms)}


\begin{figure}[H]
\centering
\begin{tikzpicture}
	\pgfplotsset{small}
	\matrix{
	%scale
\begin{axis}
	[
	   width=\graphwidth\textwidth,
      ylabel=$\ylabel$, % Set the labels
    xlabel=$Scale Factor$,
	legend entries={$d=10$,$d=20$,$d=30$,$d=40$},
	title= Robustness to scaling
    ]
	\addplot table [x=Argument, y=FAST-10, colsep=comma]{graphs/scale-all-\gtype.csv};
	 \addplot table [x=Argument, y=FAST-20, col sep=comma]{graphs/scale-all-\gtype.csv};
	  \addplot table [x=Argument,	y=FAST-30, col sep=comma]	 {graphs/scale-all-\gtype.csv}; 
	  \addplot table	[x=Argument, y=FAST-40, col	 sep=comma] {graphs/scale-all-\gtype.csv};
	 
	  
	
	%eixo horizontal
	\addplot[red,sharp plot,update limits=false] coordinates{(\miny,\ylimit)
	(\maxx,\ylimit)}; 
	\fill [opacity=0.4,red!25] (axis cs:\miny,\ylimit) rectangle (axis
	cs:\maxx,\maxy);
	
	
	%eixo vertical
	\addplot[blue,sharp plot,update limits=false] coordinates {(\minscale,\miny)(\minscale,\maxy)} ;
		\fill [opacity=0.4,blue!25] (axis cs:\minx,\miny) rectangle (axis		cs:\minscale,\maxy);
	
	\addplot[blue,sharp plot,update limits=false] coordinates {(\maxscale,\miny)	(\maxscale,\maxy)} ;
	\fill [opacity=0.4,blue!25] (axis cs:\maxscale,\miny) rectangle (axis	cs:\maxx,\maxy);
	
\end{axis}
&
%rotation
\begin{axis}
	[
	   width=\graphwidth\textwidth,
      ylabel=$\ylabel$, % Set the labels
    xlabel=$Angle(Degree)$,
	title= Rotation Invariance 
    ]
	\addplot table [x=Argument, y=FAST-10, colsep=comma]{graphs/rot-all-\gtype.csv};
	 \addplot table [x=Argument, y=FAST-20, col sep=comma]{graphs/rot-all-\gtype.csv};
	  \addplot table [x=Argument,	y=FAST-30, col sep=comma]	 {graphs/rot-all-\gtype.csv}; 
	  \addplot table	[x=Argument, y=FAST-40, col	 sep=comma] {graphs/rot-all-\gtype.csv};


	
	%eixo horizontal
	\addplot[red,sharp plot,update limits=false] coordinates{(\miny,\ylimit)
	(\maxx,\ylimit)}; 
	\fill [opacity=0.4,red!25] (axis cs:\miny,\ylimit) rectangle (axis
	cs:\maxx,\maxy);
	
	
		%eixo vertical
	\addplot[blue,sharp plot,update limits=false] coordinates {(\minangle,\miny)
	(\minangle,\maxy)} ;
		\fill [opacity=0.4,blue!25] (axis cs:\minx,\miny) rectangle (axis
	cs:\minangle,\maxy);
	
	\addplot[blue,sharp plot,update limits=false] coordinates {(\maxangle,\miny)
	(\maxangle,\maxy)} ;
	\fill [opacity=0.4,blue!25] (axis cs:\maxangle,\miny) rectangle (axis
	cs:\maxx,\maxy);

	
\end{axis}
\\
%blur
\begin{axis}
	[
	   width=\graphwidth\textwidth,
      ylabel=$\ylabel$, % Set the labels
    xlabel=$Kernel size$,
	title= Robustness to blur
    ]
	\addplot table [x=Argument, y=FAST-10, colsep=comma]{graphs/blur-all-\gtype.csv};
	 \addplot table [x=Argument, y=FAST-20, col sep=comma]{graphs/blur-all-\gtype.csv};
	  \addplot table [x=Argument,	y=FAST-30, col sep=comma]	 {graphs/blur-all-\gtype.csv}; 
	  \addplot table	[x=Argument, y=FAST-40, col	 sep=comma] {graphs/blur-all-\gtype.csv};


	
	%eixo horizontal
	\addplot[red,sharp plot,update limits=false] coordinates{(\minx,\ylimit)
	(\maxx,\ylimit)}; 
	\fill [opacity=0.4,red!25] (axis cs:\minx,\ylimit) rectangle (axis
	cs:\maxx,\maxy);

				%eixo vertical
	\addplot[blue,sharp plot,update limits=false] coordinates
	{(\minblur,\miny)(\minblur,\maxy)} ; \fill [opacity=0.4,blue!25] (axis
	cs:\minx,\miny) rectangle (axis cs:\minblur,\maxx);
	
	\addplot[blue,sharp plot,update limits=false] coordinates
	{(\maxblur,\miny)(\maxblur,\maxy)} ; \fill [opacity=0.4,blue!25] (axis
	cs:\maxblur,\miny) rectangle (axis cs:\maxx,\maxy);


\end{axis}
&
%bright
\begin{axis}
	[
	   width=\graphwidth\textwidth,
      ylabel=$\ylabel$, % Set the labels
    xlabel=$Change of brightness$,
	title= Brightness Invariance 
    ]
	\addplot table [x=Argument, y=FAST-10, col sep=comma]	{graphs/bright-all-\gtype.csv};
	 \addplot table [x=Argument, y=FAST-20, col	sep=comma]{graphs/bright-all-\gtype.csv};
	  \addplot table [x=Argument,	y=FAST-30, col sep=comma]	 {graphs/bright-all-\gtype.csv}; 
	  \addplot table	[x=Argument, y=FAST-40, col	 sep=comma] {graphs/bright-all-\gtype.csv};

	
	
	
	%eixo horizontal
	\addplot[red,sharp plot,update limits=false] coordinates{(\minx,\ylimit)
	(\maxx,\ylimit)}; 
	\fill [opacity=0.4,red!25] (axis cs:\minx,\ylimit) rectangle (axis
	cs:\maxx,\maxy);
		
		
						%eixo vertical
	\addplot[blue,sharp plot,update limits=false] coordinates
	{(\minbright,\miny)(\minbright,\maxy)} ; \fill [opacity=0.4,blue!25] (axis
	cs:\minx,\miny) rectangle (axis cs:\minbright,\maxy);
	
	\addplot[blue,sharp plot,update limits=false] coordinates
	{(\maxbright,\miny)(\maxbright,\maxy)} ; \fill [opacity=0.4,blue!25] (axis
	cs:\maxbright,\miny) rectangle (axis cs:\maxx,\maxy);
		
	
\end{axis}
\\
}

\end{tikzpicture}

\label{graph:fastresulttime}
\caption{Resultado de testes individuais de tempo com FAST}
\end{figure}


\subsection{FREAK}
A assinatura do m�todo � 
\begin{center}
TODO
\end{center}



\renewcommand{\tec}{freak}
\renewcommand{\gtype}{pres}


\renewcommand{\miny}{-10}
\renewcommand{\maxy}{3}
\renewcommand{\minx}{-150}
\renewcommand{\maxx}{400}
\renewcommand{\ylimit}{0.6}
\renewcommand{\ylabel}{Percent of correct matches}

\begin{figure}[H]
\centering
\begin{tikzpicture}
	\pgfplotsset{small}
	\matrix{
	%scale
\begin{axis}
	[
	   width=\graphwidth\textwidth,
    ylabel=$\ylabel$, % Set the labels
    xlabel=$Scale Factor$,
	legend entries={$d=1200$,$d=1500$,$d=1800$,$d=2100$},
	title= Robustness to scaling
    ]

	\addplot table [x=Argument, y=FREAK-1800, col sep=comma]	{graphs/scale-all-\gtype.csv};
	 \addplot table [x=Argument, y=FREAK-2100, col	sep=comma] {graphs/scale-all-\gtype.csv};
		\addplot table [x=Argument, y=FREAK-2400, col sep=comma]	{graphs/scale-all-\gtype.csv};
		 \addplot table [x=Argument, y=FREAK-2700, col	sep=comma]		 {graphs/scale-all-\gtype.csv};
	

	  
	
	%eixo horizontal
	\addplot[red,sharp plot,update limits=false] coordinates{(\minx,\ylimit)(\maxx,\ylimit)};
	 \fill [opacity=0.4,red!25] (axis	cs:\minx,\miny)	rectangle (axis	
	 cs:\maxx,\ylimit);
	
	
	%eixo vertical
	\addplot[blue,sharp plot,update limits=false] coordinates {(\minscale,\miny)(\minscale,\maxy)} ;
		\fill [opacity=0.4,blue!25] (axis cs:0,-1) rectangle (axis	cs:\minscale,\maxy);
	
	\addplot[blue,sharp plot,update limits=false] coordinates {(\maxscale,\miny)(\maxscale,\maxy)} ;
	\fill [opacity=0.4,blue!25] (axis cs:\maxscale,\miny) rectangle (axis	cs:\maxx,\maxy);
	
\end{axis}
&
%rotation
\begin{axis}
	[
	   width=\graphwidth\textwidth,
    ylabel=$\ylabel$, % Set the labels
    xlabel=$Angle(Degree)$,
	title= Rotation Invariance 
    ]
	\addplot table [x=Argument, y=FREAK-1800, col sep=comma]	{graphs/rot-all-\gtype.csv};
	 \addplot table [x=Argument, y=FREAK-2100, col	sep=comma] {graphs/rot-all-\gtype.csv};
		\addplot table [x=Argument, y=FREAK-2400, col sep=comma]	{graphs/rot-all-\gtype.csv};
		 \addplot table [x=Argument, y=FREAK-2700, col	sep=comma]
		 {graphs/rot-all-\gtype.csv};
		
	%eixo horizontal
	\addplot[red,sharp plot,update limits=false] coordinates
	{(-100,\ylimit) (400,\ylimit)};
		\fill [opacity=0.4,red!25] (axis cs:-100,\miny) rectangle (axis
	cs:\maxx,\ylimit);
	
	
		%eixo vertical
	\addplot[blue,sharp plot,update limits=false] coordinates {(\minangle,\miny)
	(\minangle,\maxy)} ;
		\fill [opacity=0.4,blue!25] (axis cs:-100,\miny) rectangle (axis
	cs:\minangle,\maxy);
	
	\addplot[blue,sharp plot,update limits=false] coordinates {(\maxangle,\miny)
	(\maxangle,\maxy)} ;
	\fill [opacity=0.4,blue!25] (axis cs:\maxangle,\miny) rectangle (axis
	cs:\maxx,\maxy);

	
\end{axis}
\\
%blur
\begin{axis}
	[
	   width=\graphwidth\textwidth,
       ylabel=$\ylabel$, % Set the labels
    xlabel=$Kernel size$,
	title= Robustness to blur
    ]
	\addplot table [x=Argument, y=FREAK-1800, col sep=comma]	{graphs/blur-all-\gtype.csv};
	 \addplot table [x=Argument, y=FREAK-2100, col	sep=comma] {graphs/blur-all-\gtype.csv};
		\addplot table [x=Argument, y=FREAK-2400, col sep=comma]	{graphs/blur-all-\gtype.csv};
		 \addplot table [x=Argument,  y=FREAK-2700, col	sep=comma] {graphs/blur-all-\gtype.csv};
	
	%eixo horizontal
	\addplot[red,sharp plot,update limits=false] coordinates{(\minx,\ylimit)(\maxx,\ylimit)};
	 \fill [opacity=0.4,red!25] (axis	cs:\minx,\ylimit) rectangle(axis
	 cs:\maxx,\miny);


				%eixo vertical
	\addplot[blue,sharp plot,update limits=false] coordinates
	{(\minblur,\miny)(\minblur,\maxy)} ; \fill [opacity=0.4,blue!25] (axis	cs:\minx,\miny) rectangle (axis cs:\minblur,\maxy);
	
	\addplot[blue,sharp plot,update limits=false] coordinates
	{(\maxblur,\miny)(\maxblur,\maxy)} ; \fill [opacity=0.4,blue!25] (axis	cs:\maxblur,\miny) rectangle (axis cs:\maxx,\maxy);


\end{axis}
&
%bright
\begin{axis}
	[
	   width=\graphwidth\textwidth,
       ylabel=$\ylabel$, % Set the labels
    xlabel=$Change of brightness$,
	title= Brightness Invariance 
    ]
	\addplot table [x=Argument, y=FREAK-1800, col sep=comma]	{graphs/bright-all-\gtype.csv};
	 \addplot table [x=Argument, y=FREAK-2100, col	sep=comma] {graphs/bright-all-\gtype.csv};
		\addplot table [x=Argument, y=FREAK-2400, col sep=comma]	{graphs/bright-all-\gtype.csv};
		 \addplot table [x=Argument,  y=FREAK-2700, col	sep=comma] {graphs/bright-all-\gtype.csv};

	
	%eixo horizontal
	\addplot[red,sharp plot,update limits=false] coordinates	{(\minx,\ylimit)(\maxx,\ylimit)};
	 \fill [opacity=0.4,red!25] (axis cs:\minx,\ylimit) rectangle(axis cs:\maxx,\miny);
		
		
						%eixo vertical
	\addplot[blue,sharp plot,update limits=false] coordinates
	{(\minbright,\miny)(\minbright,\maxy)} ; 
	\fill [opacity=0.4,blue!25] (axis	cs:\minx,\miny) rectangle (axis
	cs:\minbright,\maxy);
	
	\addplot[blue,sharp plot,update limits=false] coordinates
	{(\maxbright,\miny)(\maxbright,\maxy)} ; \fill [opacity=0.4,blue!25] (axis
	cs:\maxbright,\miny) rectangle (axis cs:\maxx,\maxy);
		
	
\end{axis}
\\
}

\end{tikzpicture}
\label{graph:freakresultpres}
\caption{Resultado de testes individuais de preformance com FREAK}
\end{figure}


%TIME

\renewcommand{\tec}{FREAK}
\renewcommand{\gtype}{time}

\renewcommand{\maxy}{3000}
\renewcommand{\miny}{-155}



\renewcommand{\ylimit}{200}
\renewcommand{\ylabel}{Time(Ms)}


\begin{figure}[H]
\centering
\begin{tikzpicture}
	\pgfplotsset{small}
	\matrix{
	%scale
\begin{axis}
	[
	   width=\graphwidth\textwidth,
      ylabel=$\ylabel$, % Set the labels
    xlabel=$Scale Factor$,
	legend entries={$d=1200$,$d=1500$,$d=1800$,$d=2100$},
	title= Robustness to scaling
    ]
	\addplot table [x=Argument, y=FREAK-1800, col sep=comma]	{graphs/scale-all-\gtype.csv};
	 \addplot table [x=Argument, y=FREAK-2100, col	sep=comma] {graphs/scale-all-\gtype.csv};
		\addplot table [x=Argument, y=FREAK-2400, col sep=comma]	{graphs/scale-all-\gtype.csv};
		 \addplot table [x=Argument,  y=FREAK-2700, col	sep=comma] {graphs/scale-all-\gtype.csv};
 
	  	
	%eixo horizontal
	\addplot[red,sharp plot,update limits=false] coordinates{(\miny,\ylimit)
	(\maxx,\ylimit)}; 
	\fill [opacity=0.4,red!25] (axis cs:\miny,\ylimit) rectangle (axis
	cs:\maxx,\maxy);
	
	
	%eixo vertical
	\addplot[blue,sharp plot,update limits=false] coordinates {(\minscale,\miny)(\minscale,\maxy)} ;
		\fill [opacity=0.4,blue!25] (axis cs:\minx,\miny) rectangle (axis		cs:\minscale,\maxy);
	
	\addplot[blue,sharp plot,update limits=false] coordinates {(\maxscale,\miny)	(\maxscale,\maxy)} ;
	\fill [opacity=0.4,blue!25] (axis cs:\maxscale,\miny) rectangle (axis	cs:\maxx,\maxy);
	
\end{axis}
&
%rotation
\begin{axis}
	[
	   width=\graphwidth\textwidth,
      ylabel=$\ylabel$, % Set the labels
    xlabel=$Angle(Degree)$,
	title= Rotation Invariance 
    ]
	\addplot table [x=Argument, y=FREAK-1800, col sep=comma]	{graphs/rot-all-\gtype.csv};
	 \addplot table [x=Argument, y=FREAK-2100, col	sep=comma] {graphs/rot-all-\gtype.csv};
		\addplot table [x=Argument, y=FREAK-2400, col sep=comma]	{graphs/rot-all-\gtype.csv};
		 \addplot table [x=Argument, y=FREAK-2700, col	sep=comma] {graphs/rot-all-\gtype.csv};
	
	%eixo horizontal
	\addplot[red,sharp plot,update limits=false] coordinates{(-100,\ylimit)
	(400,\ylimit)}; 
	\fill [opacity=0.4,red!25] (axis cs:-100,\ylimit) rectangle (axis
	cs:400,\maxy);
	
	
		%eixo vertical
	\addplot[blue,sharp plot,update limits=false] coordinates {(\minangle,-100)
	(\minangle,\maxy)} ;
		\fill [opacity=0.4,blue!25] (axis cs:-100,-1) rectangle (axis
	cs:\minangle,\maxy);
	
	\addplot[blue,sharp plot,update limits=false] coordinates {(\maxangle,-1)
	(\maxangle,\maxy)} ;
	\fill [opacity=0.4,blue!25] (axis cs:\maxangle,-1) rectangle (axis
	cs:400,\maxy);

	
\end{axis}
\\
%blur
\begin{axis}
	[
	   width=\graphwidth\textwidth,
      ylabel=$\ylabel$, % Set the labels
    xlabel=$Kernel size$,
	title= Robustness to blur
    ]
	\addplot table [x=Argument, y=FREAK-1800, col sep=comma]	{graphs/blur-all-\gtype.csv};
	 \addplot table [x=Argument, y=FREAK-2100, col	sep=comma] {graphs/blur-all-\gtype.csv};
		\addplot table [x=Argument, y=FREAK-2400, col sep=comma]	{graphs/blur-all-\gtype.csv};
		 \addplot table [x=Argument,  y=FREAK-2700, col	sep=comma] {graphs/blur-all-\gtype.csv};
	%eixo horizontal
	\addplot[red,sharp plot,update limits=false] coordinates{(\minx,\ylimit)
	(\maxx,\ylimit)}; 
	\fill [opacity=0.4,red!25] (axis cs:\minx,\ylimit) rectangle (axis
	cs:\maxx,\maxy);

				%eixo vertical
	\addplot[blue,sharp plot,update limits=false] coordinates
	{(\minblur,\miny)(\minblur,\maxy)} ; \fill [opacity=0.4,blue!25] (axis
	cs:\minx,\miny) rectangle (axis cs:\minblur,\maxx);
	
	\addplot[blue,sharp plot,update limits=false] coordinates
	{(\maxblur,\miny)(\maxblur,\maxy)} ; \fill [opacity=0.4,blue!25] (axis
	cs:\maxblur,\miny) rectangle (axis cs:\maxx,\maxy);


\end{axis}
&
%bright
\begin{axis}
	[
	   width=\graphwidth\textwidth,
      ylabel=$\ylabel$, % Set the labels
    xlabel=$Change of brightness$,
	title= Brightness Invariance 
    ]
	\addplot table [x=Argument, y=FREAK-1800, col sep=comma]	{graphs/bright-all-\gtype.csv};
	 \addplot table [x=Argument, y=FREAK-2100, col	sep=comma] {graphs/bright-all-\gtype.csv};
		\addplot table [x=Argument, y=FREAK-2400, col sep=comma]	{graphs/bright-all-\gtype.csv};
		 \addplot table [x=Argument,  y=FREAK-2700, col	sep=comma] {graphs/bright-all-\gtype.csv};
	
	%eixo horizontal
	\addplot[red,sharp plot,update limits=false] coordinates{(\minx,\ylimit)
	(\maxx,\ylimit)}; 
	\fill [opacity=0.4,red!25] (axis cs:\minx,\ylimit) rectangle (axis
	cs:\maxx,\maxy);
		
		
						%eixo vertical
	\addplot[blue,sharp plot,update limits=false] coordinates
	{(\minbright,\miny)(\minbright,\maxy)} ; \fill [opacity=0.4,blue!25] (axis
	cs:\minx,\miny) rectangle (axis cs:\minbright,\maxy);
	
	\addplot[blue,sharp plot,update limits=false] coordinates
	{(\maxbright,\miny)(\maxbright,\maxy)} ; \fill [opacity=0.4,blue!25] (axis
	cs:\maxbright,\miny) rectangle (axis cs:\maxx,\maxy);
		
	
\end{axis}
\\
}

\end{tikzpicture}
\label{graph:freakresulttime}
\caption{Resultado de testes individuais de tempo com FREAK}
\end{figure}


\subsection{MSER}
A assinatura do m�todo � 
\begin{center}
MSER( int _delta=5, int _min_area=60, int _max_area=14400,
          double _max_variation=0.25, double _min_diversity=.2,
          int _max_evolution=200, double _area_threshold=1.01,
          double _min_margin=0.003, int _edge_blur_size=5 );
\end{center}


\renewcommand{\tec}{mser}
\renewcommand{\gtype}{pres}

\renewcommand{\miny}{-10}
\renewcommand{\maxy}{3}
\renewcommand{\minx}{-150}
\renewcommand{\maxx}{400}
\renewcommand{\ylimit}{0.6}
\renewcommand{\ylabel}{Percent of correct matches}

\begin{figure}[H]
\centering
\begin{tikzpicture}
	\pgfplotsset{small}
	\matrix{
	%scale
\begin{axis}
	[
	   width=\graphwidth\textwidth,
    ylabel=$\ylabel$, % Set the labels
    xlabel=$Scale Factor$,
	legend entries={$d=15$,$d=18$,$d=21$,$d=24$},
	title= Robustness to scaling
    ]
	\addplot table [x=Argument, y=MSER-15, col sep=comma]	{graphs/scale-all-\gtype.csv}; 
	\addplot table [x=Argument, y=MSER-18, col	sep=comma]	{graphs/scale-all-\gtype.csv}; 
	\addplot table [x=Argument, y=MSER-21, col	sep=comma]	{graphs/scale-all-\gtype.csv}; 
	\addplot table [x=Argument, y=MSER-24, col	sep=comma] {graphs/scale-all-\gtype.csv}; 
  
	
	%eixo horizontal
	\addplot[red,sharp plot,update limits=false] coordinates{(\minx,\ylimit)(\maxx,\ylimit)};
	 \fill [opacity=0.4,red!25] (axis	cs:\minx,\miny)	rectangle (axis	
	 cs:\maxx,\ylimit);
	
	
	%eixo vertical
	\addplot[blue,sharp plot,update limits=false] coordinates {(\minscale,\miny)(\minscale,\maxy)} ;
		\fill [opacity=0.4,blue!25] (axis cs:0,-1) rectangle (axis	cs:\minscale,\maxy);
	
	\addplot[blue,sharp plot,update limits=false] coordinates {(\maxscale,\miny)(\maxscale,\maxy)} ;
	\fill [opacity=0.4,blue!25] (axis cs:\maxscale,\miny) rectangle (axis	cs:\maxx,\maxy);
	
\end{axis}
&
%rotation
\begin{axis}
	[
	   width=\graphwidth\textwidth,
    ylabel=$\ylabel$, % Set the labels
    xlabel=$Angle(Degree)$,
	title= Rotation Invariance 
    ]
	\addplot table [x=Argument, y=MSER-15, col sep=comma]	{graphs/rot-all-\gtype.csv}; 
	\addplot table [x=Argument, y=MSER-18, col	sep=comma]	{graphs/rot-all-\gtype.csv}; 
	\addplot table [x=Argument, y=MSER-21, col	sep=comma]	{graphs/rot-all-\gtype.csv}; 
	\addplot table [x=Argument, y=MSER-24, col	sep=comma] {graphs/rot-all-\gtype.csv}; 
	 			
	%eixo horizontal
	\addplot[red,sharp plot,update limits=false] coordinates
	{(-100,\ylimit) (400,\ylimit)};
		\fill [opacity=0.4,red!25] (axis cs:-100,\miny) rectangle (axis
	cs:\maxx,\ylimit);
	
	
		%eixo vertical
	\addplot[blue,sharp plot,update limits=false] coordinates {(\minangle,\miny)
	(\minangle,\maxy)} ;
		\fill [opacity=0.4,blue!25] (axis cs:-100,\miny) rectangle (axis
	cs:\minangle,\maxy);
	
	\addplot[blue,sharp plot,update limits=false] coordinates {(\maxangle,\miny)
	(\maxangle,\maxy)} ;
	\fill [opacity=0.4,blue!25] (axis cs:\maxangle,\miny) rectangle (axis
	cs:\maxx,\maxy);

	
\end{axis}
\\
%blur
\begin{axis}
	[
	   width=\graphwidth\textwidth,
       ylabel=$\ylabel$, % Set the labels
    xlabel=$Kernel size$,
	title= Robustness to blur
    ]
	\addplot table [x=Argument, y=MSER-15, col sep=comma]	{graphs/blur-all-\gtype.csv}; 
	\addplot table [x=Argument, y=MSER-18, col	sep=comma]	{graphs/blur-all-\gtype.csv}; 
	\addplot table [x=Argument, y=MSER-21, col	sep=comma]	{graphs/blur-all-\gtype.csv}; 
	\addplot table [x=Argument, y=MSER-24, col	sep=comma] {graphs/blur-all-\gtype.csv}; 
	 	
	
	%eixo horizontal
	\addplot[red,sharp plot,update limits=false] coordinates{(\minx,\ylimit)(\maxx,\ylimit)};
	 \fill [opacity=0.4,red!25] (axis	cs:\minx,\ylimit) rectangle(axis
	 cs:\maxx,\miny);


				%eixo vertical
	\addplot[blue,sharp plot,update limits=false] coordinates
	{(\minblur,\miny)(\minblur,\maxy)} ; \fill [opacity=0.4,blue!25] (axis	cs:\minx,\miny) rectangle (axis cs:\minblur,\maxy);
	
	\addplot[blue,sharp plot,update limits=false] coordinates
	{(\maxblur,\miny)(\maxblur,\maxy)} ; \fill [opacity=0.4,blue!25] (axis	cs:\maxblur,\miny) rectangle (axis cs:\maxx,\maxy);


\end{axis}
&
%bright
\begin{axis}
	[
	   width=\graphwidth\textwidth,
       ylabel=$\ylabel$, % Set the labels
    xlabel=$Change of brightness$,
	title= Brightness Invariance 
    ]
	\addplot table [x=Argument, y=MSER-15, col sep=comma]	{graphs/bright-all-\gtype.csv}; 
	\addplot table [x=Argument, y=MSER-18, col	sep=comma]	{graphs/bright-all-\gtype.csv}; 
	\addplot table [x=Argument, y=MSER-21, col	sep=comma]	{graphs/bright-all-\gtype.csv}; 
	\addplot table [x=Argument, y=MSER-24, col	sep=comma] {graphs/bright-all-\gtype.csv}; 
	 	
	
	%eixo horizontal
	\addplot[red,sharp plot,update limits=false] coordinates	{(\minx,\ylimit)(\maxx,\ylimit)};
	 \fill [opacity=0.4,red!25] (axis cs:\minx,\ylimit) rectangle(axis cs:\maxx,\miny);
		
		
						%eixo vertical
	\addplot[blue,sharp plot,update limits=false] coordinates
	{(\minbright,\miny)(\minbright,\maxy)} ; 
	\fill [opacity=0.4,blue!25] (axis	cs:\minx,\miny) rectangle (axis
	cs:\minbright,\maxy);
	
	\addplot[blue,sharp plot,update limits=false] coordinates
	{(\maxbright,\miny)(\maxbright,\maxy)} ; \fill [opacity=0.4,blue!25] (axis
	cs:\maxbright,\miny) rectangle (axis cs:\maxx,\maxy);
		
	
\end{axis}
\\
}

\end{tikzpicture}
\label{graph:mserresultpres}
\caption{Resultado de testes individuais de preformance com MSER}
\end{figure}


%TIME

\renewcommand{\tec}{MSER}
\renewcommand{\gtype}{time}

\renewcommand{\maxy}{8000}
\renewcommand{\miny}{-1000}



\renewcommand{\ylimit}{200}
\renewcommand{\ylabel}{Time(Ms)}

\begin{figure}[H]
\centering
\begin{tikzpicture}
	\pgfplotsset{small}
	\matrix{
	%scale
\begin{axis}
	[
	   width=\graphwidth\textwidth,
      ylabel=$\ylabel$, % Set the labels
    xlabel=$Scale Factor$,
	legend entries={$d=15$,$d=18$,$d=21$,$d=24$},
	title= Robustness to scaling
    ]
	\addplot table [x=Argument, y=MSER-15, col sep=comma]	{graphs/scale-all-\gtype.csv}; 
	\addplot table [x=Argument, y=MSER-18, col	sep=comma]	{graphs/scale-all-\gtype.csv}; 
	\addplot table [x=Argument, y=MSER-21, col	sep=comma]	{graphs/scale-all-\gtype.csv}; 
	\addplot table [x=Argument, y=MSER-24, col	sep=comma] {graphs/scale-all-\gtype.csv}; 
 	 
	  
	
	%eixo horizontal
	\addplot[red,sharp plot,update limits=false] coordinates{(\minx,\ylimit)
	(\maxx,\ylimit)}; 
	\fill [opacity=0.4,red!25] (axis cs:\minx,\ylimit) rectangle (axis
	cs:\maxx,\maxy);
	
	
	%eixo vertical
	\addplot[blue,sharp plot,update limits=false] coordinates {(\minscale,\miny)(\minscale,\maxy)} ;
		\fill [opacity=0.4,blue!25] (axis cs:\minx,\miny) rectangle (axis		cs:\minscale,\maxy);
	
	\addplot[blue,sharp plot,update limits=false] coordinates {(\maxscale,\miny)	(\maxscale,\maxy)} ;
	\fill [opacity=0.4,blue!25] (axis cs:\maxscale,\miny) rectangle (axis	cs:\maxx,\maxy);
	
\end{axis}
&
%rotation
\begin{axis}
	[
	   width=\graphwidth\textwidth,
      ylabel=$\ylabel$, % Set the labels
    xlabel=$Angle(Degree)$,
	title= Rotation Invariance 
    ]
	\addplot table [x=Argument, y=MSER-15, col sep=comma]	{graphs/rot-all-\gtype.csv}; 
	\addplot table [x=Argument, y=MSER-18, col	sep=comma]	{graphs/rot-all-\gtype.csv}; 
	\addplot table [x=Argument, y=MSER-21, col	sep=comma]	{graphs/rot-all-\gtype.csv}; 
	\addplot table [x=Argument, y=MSER-24, col	sep=comma] {graphs/rot-all-\gtype.csv}; 
	 	
	%eixo horizontal
	\addplot[red,sharp plot,update limits=false] coordinates{(\minx,\ylimit)
	(\maxx,\ylimit)}; 
	\fill [opacity=0.4,red!25] (axis cs:\miny,\ylimit) rectangle (axis
	cs:\maxx,\maxy);
	
	
		%eixo vertical
	\addplot[blue,sharp plot,update limits=false] coordinates {(\minangle,\miny)
	(\minangle,\maxy)} ;
		\fill [opacity=0.4,blue!25] (axis cs:\minx,\miny) rectangle (axis
	cs:\minangle,\maxy);
	
	\addplot[blue,sharp plot,update limits=false] coordinates {(\maxangle,\miny)
	(\maxangle,\maxy)} ;
	\fill [opacity=0.4,blue!25] (axis cs:\maxangle,\miny) rectangle (axis
	cs:\maxx,\maxy);

	
\end{axis}
\\
%blur
\begin{axis}
	[
	   width=\graphwidth\textwidth,
      ylabel=$\ylabel$, % Set the labels
    xlabel=$Kernel size$,
	title= Robustness to blur
    ]
	\addplot table [x=Argument, y=MSER-15, col sep=comma]	{graphs/blur-all-\gtype.csv}; 
	\addplot table [x=Argument, y=MSER-18, col	sep=comma]	{graphs/blur-all-\gtype.csv}; 
	\addplot table [x=Argument, y=MSER-21, col	sep=comma]	{graphs/blur-all-\gtype.csv}; 
	\addplot table [x=Argument, y=MSER-24, col	sep=comma] {graphs/blur-all-\gtype.csv}; 
	 		
	%eixo horizontal
	\addplot[red,sharp plot,update limits=false] coordinates{(\minx,\ylimit)
	(\maxx,\ylimit)}; 
	\fill [opacity=0.4,red!25] (axis cs:\minx,\ylimit) rectangle (axis
	cs:\maxx,\maxy);

				%eixo vertical
	\addplot[blue,sharp plot,update limits=false] coordinates
	{(\minblur,\miny)(\minblur,\maxy)} ; 
	\fill [opacity=0.4,blue!25] (axis	cs:\minx,\miny) rectangle (axis
	cs:\minblur,\maxy);
	
	\addplot[blue,sharp plot,update limits=false] coordinates
	{(\maxblur,\miny)(\maxblur,\maxy)} ; \fill [opacity=0.4,blue!25] (axis
	cs:\maxblur,\miny) rectangle (axis cs:\maxx,\maxy);


\end{axis}
&
%bright
\begin{axis}
	[
	   width=\graphwidth\textwidth,
      ylabel=$\ylabel$, % Set the labels
    xlabel=$Change of brightness$,
	title= Brightness Invariance 
    ]
	\addplot table [x=Argument, y=MSER-15, col sep=comma]	{graphs/bright-all-\gtype.csv}; 
	\addplot table [x=Argument, y=MSER-18, col	sep=comma]	{graphs/bright-all-\gtype.csv}; 
	\addplot table [x=Argument, y=MSER-21, col	sep=comma]	{graphs/bright-all-\gtype.csv}; 
	\addplot table [x=Argument, y=MSER-24, col	sep=comma] {graphs/bright-all-\gtype.csv}; 
	
	%eixo horizontal
	\addplot[red,sharp plot,update limits=false] coordinates{(\minx,\ylimit)
	(\maxx,\ylimit)}; 
	\fill [opacity=0.4,red!25] (axis cs:\minx,\ylimit) rectangle (axis
	cs:\maxx,\maxy);
		
		
						%eixo vertical
	\addplot[blue,sharp plot,update limits=false] coordinates
	{(\minbright,\miny)(\minbright,\maxy)} ; \fill [opacity=0.4,blue!25] (axis
	cs:\minx,\miny) rectangle (axis cs:\minbright,\maxy);
	
	\addplot[blue,sharp plot,update limits=false] coordinates
	{(\maxbright,\miny)(\maxbright,\maxy)} ; \fill [opacity=0.4,blue!25] (axis
	cs:\maxbright,\miny) rectangle (axis cs:\maxx,\maxy);
		
	
\end{axis}
\\
}

\end{tikzpicture}
\label{graph:mserresulttime}
\caption{Resultado de testes individuais de tempo com MSER}
\end{figure}

\input{parts/results/orbresult}
\input{parts/results/siftresult}
\input{parts/results/starresult}
\input{parts/results/surfresult}





 





\section{An�lise de resultados}

Ap�s a an�lise pontual de cada algoritmo, podemos compar�-los entre si e avaliar
o que melhor se adequa ao contexto proposto.

Os algoritmos SIFT e SURF apesar de retornarem bons resultados para o contexto,
s�o pagos e portanto ser�o desconsiderados para a abordagem proposta.

 Uma an�lise pr�via dos algoritmos selecionados tem como tempo de resposta por
frame a imagem~\ref{fig:performance6algorithms} que nos mostra claramente que a
t�cnica FREAK n�o respondem em tempo �bil para serem considerados tempo
real, portanto ser�o desconsiderados.


A imagem~\ref{fig:time_sift_surf} que relaciona como as t�cnicas: BRISK, FAST,
FREAK, ORB,STAR,SURF e SIFT respondem a varia��o


\begin{figure}[H]
\centering
\includegraphics[scale=0.3]{images/time_sift_surf}
\caption{Performance quanto � varia��es de escala, ilumina��o, rota��o e
gaussian blur}
\label{fig:time_sift_surf}
\end{figure}


\begin{figure}[H]
\centering
\includegraphics[scale=0.3]{images/time}
\caption{Performance quanto � varia��es de escala, ilumina��o, rota��o e
gaussian blur sem os algoritmos SIFT e SURF}
\label{fig:performancealgorithms}
\end{figure}






\section{Homografia}

Ap�s a etapa de extra��o de caracter�sticas da imagem padr�o e da imagem de
compara��o � importante fazer o casamento de padr�es para que com um n�mero
significativo de caracter�sticas, o objeto seja reconhecido. As imagens
imagem~\ref{fig:blurhomography}, mostram correspond�ncias entre imagens variando
as transforma��es propostas pelo prot�tipo em compara��o com imagens sem as
transforma��es.

\begin{figure}[H]
\centering
\includegraphics[scale=0.5]{images/ORB-BLUR-HOMOGRAPHY}
\caption{Homografia com varia��o varia��o de Blur.}
\label{fig:blurhomography}
\end{figure}

\begin{figure}[H]
\centering
\includegraphics[scale=0.5]{images/ORB-BRIGHT-HOMOGRAPHY}
\caption{Homografia com varia��o varia��o de ilumina��o.}
\label{fig:brighthomography}
\end{figure}




\begin{figure}[H]
\centering
\includegraphics[scale=0.5]{images/ORB-ROTATION-HOMOGRAPHY}
\caption{Homografia com varia��o varia��o de Rota��o.}
\label{fig:rotationhomography}
\end{figure}



\begin{figure}[H]
\centering
\includegraphics[scale=0.8]{images/ORB-SCALE-HOMOGRAPHY}
\caption{Homografia com varia��o varia��o de Escala.}
\label{fig:scalehomography}
\end{figure}



\cite{ISMAR2012}



\chapter{Conclus�o}
\section{Conclus�o}
\begin{frame}
\frametitle{Conclus�o}

\subsection{Atendimento dos objetivos}

\textbf {Avaliar os algoritmos cl�ssicos de reconhecimento}

O trabalho apresentou os algoritmos cl�ssicos, descrevendo seu funcionamento e
realizando testes de desempenho com os mesmos.

\textbf{Aplicar os algoritmos cl�ssicos � situa��es reais}

Os algoritmos foram implementados em um prot�tipo desenvolvido em C++ e OpenCV,
configurado para rodar simula��es dos algoritmos BRISK, FAST, FREAK, GFTT, MSER,
ORB, STAR, SURF, SIFT, gerando ao final de cada simula��o resultados de precis�o
de reconhecimento e de tempo gasto.

\end{frame}

\begin{frame}
\frametitle{Conclus�o}

\textbf {Selecionar algoritmo mais adequado para o contexto}

An�lise comparativa entre os algoritmos utilizando as restri��es e
criado nos gr�ficos, janelas de decis�o, gerando uma matriz de decis�o com os
resultados de todos os testes, tanto de qualidade de reconhecimento quanto de
tempo de execu��o.





\end{frame}

% Referencias Bibliograficas
\begin{spacing}{1.0}
\begin{flushleft}
\bibliographystyle{abnt-cite-style}%Choose a bibliograhpic style
\bibliography{bibliography}
\end{flushleft}
\end{spacing}


% Apendices
%\appendix
%\chapter{T�picos de �lgebra Linear}
%\input{apendiceA}

% Anexos
%\annex
%\chapter{Exemplo de um Primeiro Anexo}
%\input{anexoA}

% Referencias Bibliograficas



% Glossario
\itaglossary
\printglossary

% Folha de Registro do Documento
% Valores dos campos do formulario
\FRDitadata{24 de dezembro de 1969}
\FRDitadocnro{CTA/ITA - IEC/TM-002/1969}
\FRDitaorgaointerno{Divis�o de Ci�ncia da Computa��o -- ITA/IEC}
\FRDitapalavrasautor{Teses; Estilos; Italus}
\FRDitapalavrasresult{Teses e Disserta��es; Estilos; Usu�rios}
\FRDitaresumo{O reconhecimento de objetos em uma cena para posterior uso em realidade aumentada 
depende de diversas vari�veis, causando a necessidade do uso de t�cnicas 
espec�ficas para cada cen�rio, sendo portanto, um estudo de fronteiras para a melhor escolha 
do algoritmo de reconhecimento, de acordo com a aplica��o em quest�o de grande
valia para o meio acad�mico. 
Esta tese se prop�e a pesquisar, categorizar e tra�ar fronteiras das t�cnicas
conhecidas, tendo como caso de uso a manuten��o de aeronaves feita dentro de
centros fechados, utilizando as t�cnicas BRISK,FAST,FREAK,GFTT,MSER,
 ORB,STAR,SURF,SIFT em uma an�lise aplicada com imagens reais de janelas de
 inspe��o do Embraer ERJ-190 para reconhecimento de objetos e posteriores
 aplica��es em manuten��o.
 Comparando todas as t�cnicas quanto � cad�ncia e � precis�o de reconhecimento
 de caracter�sticas, � poss�vel selecionar GFTT e ORB
 como t�cnicas mais apropriadas ao contexto, por terem seus resultados de
 varia��o de rota��o, escala, briho e \emph{blur} dentro de uma faixa esperada
 para o contexto de manuten��o.
 

}
%  Primeiro Parametro: Nacional ou Internacional -- N/I
%  Segundo parametro: Ostensivo, Reservado, Confidencial ou Secreto -- O/R/C/S
\FRDitaOpcoes{N}{S}
% Cria o formulario
\itaFRD

\end{document}
% Fim do Documento.
