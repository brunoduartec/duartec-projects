\section{Estrutura do Texto}
Este trabalho fundamenta-se em 5 cap�tulos, conforme descritos abaixo:

\begin{itemize}
\item Cap�tulo~\ref{ch:introducao} apresenta a motiva��o do presente trabalho,
levantando as necessidades e limita��es impostas pelo ambiente, o fato de
haver diversas t�cnicas de reconhecimento de caracter�sticas, a necessidade
de selecionar a adequada para o contexto, bem como descreve o escopo do
trabalho e o contexto em que os testes s�o feitos; 

\item Cap�tulo~\ref{ch:fundamentacaoteorica} tem informa��es suficientes para o entendimento das
an�lises, descrevendo conceitos b�sicos e t�cnicas de reconhecimento que
foram comparadas;

\item Cap�tulo~\ref{ch:propostao} descreve a metodologia de an�lise adotada bem
como o prot�tipo desenvolvido;

\item Cap�tulo~\ref{ch:analisederesultados} descreve os resultados obtidos comparando as
t�cnicas, selecionando qual a mais adequada para reconhecimento de
caracter�sticas para o caso de uso descrito;

\item Cap�tulo~\ref{ch:conclusao} conclui o trabalho apresentando como os
objetivos foram atingidos e poss�veis trabalhos futuros.
\end{itemize}