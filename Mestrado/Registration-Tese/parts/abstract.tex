Recognition of objects in a scene for later use in augmented reality depends on several variables, causing the need 
to use several specific techniques for each scenario, and therefore a border
analysis to the best choice of the registration algorithm according to
application in question, of great value to academia.
This thesis aims to examine, categorize and draw boundaries of the known techniques having as use case maintenance of 
aircraft made in closed center, using the techniques BRISK,FAST,FREAK,GFTT,MSER,
ORB,STAR,SURF,SIFT in an applied analysis with images of the Embraer ERJ-190 to recognition of objects and future usage in maintenance. Comparing all of the techniques, using cadency and recognition precision, it is possible to chose GFTT and ORB as the most appropriate ones because its results to the variation of
rotation, scale, brightness and blur fulfils the constraints needed
