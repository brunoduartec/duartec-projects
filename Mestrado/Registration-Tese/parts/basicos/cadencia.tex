\section{Cad�ncia}
� a medida do n�mero de quadros individuais que um determinado dispositivo �ptico ou eletr�nico processa e exibe
 por unidade de tempo. Em geral a cad�ncia � medida em fps.
Em cinema, a cad�ncia de proje��o padr�o desde 1929 foi fixada em 24fps, sendo
no per�odo do cinema mudo a maioria dos filmes eram rodados com cad�ncia entre 16 e 20fps.
Em v�deo, os principais sistemas lidam com cad�ncia entre 25fps(PAL) e 30fps(NTSC).
As aplica��es devem ter cad�ncia toler�vel dependendo de seu uso, segundo
\cite{Tang93whydo} para aplica��es interativas o m�nimo toler�vel � de 5fps enquanto para aplica��es
 de anima��es fluidas de 30fps.
Sendo a cad�ncia a freq��ncia entre frames, deve ser contabilizado o tempo de
gera��o de informa��es e o tempo de dispor a informa��o no dispositivo �ptico.
O tempo de cada frame � calculado  o inverso do n�mero de fps.
%como mostrado na
%equa��o~\ref{eq:fps}
%\begin{equation}
%t_{frame} =  \frac{1}{fps}
%\label{eq:fps}
%\end{equation}
No caso de cad�ncia m�nima de 5fps, temos quadros com tempo menor que 200ms,
portanto as an�lises devem ser balizadas a tempos menores.