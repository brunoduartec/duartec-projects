Este trabalho fundamenta-se em 9 cap�tulos, conforme descritos abaixo:

\begin{itemize}
\item Cap�tulo 1 apresenta a motiva��o do presente trabalho, levantando as 

necessidades e limita��es impostas pelo ambiente e o fato de haverem 

diversas t�cnicas de reconhecimento de caracter�sticas e a necessidade de 

selecionar a adequada para o contexto;

\item Cap�tulo 2 descreve o escopo do trabalho;

\item Cap�tulo 3 descreve o contexto em que ser�o feitos os testes e an�lises
bem

como o cen�rio em que as amostras ser�o retiradas;

\item Cap�tulo 4 tem informa��es suficientes para o entendimento das
an�lises,

descrevendo conceitos b�sicos e as t�cnicas de reconhecimento que foram 

comparadas;

\item Cap�tulo 5 descreve a metodologia de an�lise adotada;

\item Cap�tulo 6 descreve os prot�tipos desenvolvidos para a an�lise;

\item Cap�tulo 7 descreve os resultados obtidos comparando-se as t�cnicas;

\item Cap�tulo 8 conclui a partir dos dados apresentados no cap�tulo de
resultados

qual a t�cnica mais adequada para reconhecimento de caracter�sticas para o 

caso de uso descrito;

\item Cap�tulo 9 descreve poss�veis trabalhos futuros.
\end{itemize}