\subsection{FREAK - \emph{Fast Retina Keypoint}}

FREAK \cite{FREAK} � um descritor bin�rio, composto por tr�s etapas:

\textbf{Amostragem de padr�o}

Prop�e uma abordagem biol�gica para o reconhecimento de caracter�sticas,
emulando o funcionamento da retina para amostragem de padr�es, como demonstrado
na Figura \ref{fig:freak-sampler}, sendo um padr�o circular com maior densidade
de pontos pr�ximo do centro, decrescendo exponencialmente.

\begin{figure}[H]
\centering
\includegraphics[scale=1.0]{images/freak-sampler}
\caption{Padr�o de amostragem do descritor FREAK. Fonte \cite{FREAK}}
\label{fig:freak-sampler}
\end{figure}


Cada amostra � suavizada por um filtro kernel Gaussiano em que o raio do c�rculo
ilustra o tamanho do desvio padr�o do kernel.
Como pode ser observado na Figura~\ref{fig:freak-retina}, o padr�o de amostragem
corresponde � distribui��o de receptores na retina.

\begin{figure}[H]
\centering
\includegraphics[scale=1.0]{images/freak-retina}
\caption{Distribui��o de receptores na retina. Fonte \cite{FREAK}}
\label{fig:freak-retina}
\end{figure}


\textbf{Compensa��o de Orienta��o}

Para estimar a rota��o dos \emph{keypoints}, s�o somados os gradientes locais
assim como no BRISK, entretanto ao inv�s de considerar os pontos de longa
dist�ncia, � considerado um padr�o de 45 pontos como mostrado na
Figura~\ref{fig:freak-rotation}


\begin{figure}[H]
\centering
\includegraphics[scale=1.0]{images/freak-rotation}
\caption{Pares selecionados para calcular a orienta��o. Fonte \cite{FREAK}}
\label{fig:freak-rotation}
\end{figure}

Apesar de ter menos precis�o para recuperar informa��es de rota��o, como o
n�mero de pontos, o descritor � bem menor do que BRISK e a quantidade de mem�ria
armazenada � em geral 5 vezes menor.



\textbf{Compara��o de pares de amostragem}

Os pares de pontos s�o selecionados considerando a densidade maior no centro, 
como podemos observar na Figura~\ref{fig:freak-retina} (a). 
Os pares come�am a ser comparados pelas extremidades e para dentro do centro, 
dessa forma otimizamos o reconhecimento pois com menos pontos podemos descartar casos em que 
a dist�ncia estiver maior do que um \emph{threshold}, caso contr�rio, devemos
recuperar a informa��o integral do descritor e utilizar os outros 128
bits do descritor.


