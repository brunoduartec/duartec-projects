\subsection{GFTT - Good Features To Track}

O m�todo � desenvolvido para detectar padr�es sem bordas. A matriz de gradientes
G � calculada para cada pixel como mostrado na equa��o \ref{eq:gftt}

\begin{equation}
G = \sum_{\Omega}\begin{bmatrix}
I_{x}^2 & I_{x}I_{y} \\ 
I_{x}I_{y}^2& I_{y}^2 
\end{bmatrix}
\label{eq:gftt}
\end{equation}

\label{sec:gftt}

Sendo o valor de inteisidades $I(x,y)$ de uma imagem de tons de cinza e suas
derivadas parciais $Ix$,$Iy$ de uma determinada regi�o $\Omega$. A matriz de
gradientes � implementada por meio de uma imagem integral para
$I_{x}^2$,$I_{y}^2$ e $I_{x}I_{y}$. Devido ao uso de imagens integrais a
complexidade computacional da matriz de gradientes � constante e independente do
tamanho de $\Omega$. Uma boa caracter�stica pode ser identificada pelo menor
autovalor de G. Pontos fortes aparecem em geral nas bordas, onde problemas com
movimento s�o mais comuns, o que leva para o problema de abertura geral. P�s
processamento � aplicado por uma supress�o de n�o m�ximos com um
\emph{threshold} em $q*max(\lambda(x,y))$, em que $q$ � uma constante para
garantir a qualidade.

