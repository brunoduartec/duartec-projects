\section{Proposta de Trabalhos Futuros}

A utiliza��o de Realidade Aumentada no campo da manuten��o pode trazer muitos
ganhos no que tange � usabilidade levando ao usu�rio uma quantidade de
informa��es que, da maneira tradicional, por meio inspe��o e consulta em
manuais, seria invi�vel.
Este trabalho teve como foco o reconhecimento de padr�es em um cen�rio
aeron�utico espec�fico, como pr�ximos passos temos:
\begin{itemize}

\item Adequar a aplica��o para dispositivos m�veis como tablets, celulares ou
mesmo dispositivos HMD de forma a dar mais flexibilidade ao condutor da
manuten��o;

\item Realizar o casamento de padr�es com v�deos e imagens em tempo real,
utilizando as t�cnicas identificadas, otimizando a aplica��o para se tornar o mais tempo
real e aceit�vel poss�vel;
 
\item Adaptar a aplica��o para utilizar processamento paralelo e processamento
em GPU, visto os algoritmos serem recursivos e localmente independentes;

\item Analisar por meio de testes em campo com poss�veis usu�rios para abstrair
par�metros de usabilidade, como por exemplo determinar que tipo de informa��o
seria �til ao usu�rio ou mesmo que tipo de conFigura��o de dispositivo seria o
mais adequado para uma aplica��o desse porte.
\end{itemize}