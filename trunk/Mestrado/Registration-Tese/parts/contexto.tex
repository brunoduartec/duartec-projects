\section{Contextualiza��o}
\label{sec:restricao}

O contexto dessa tese prev� o cen�rio de manuten��o com o uso de realidade aumentada como uma ferramenta
 para aux�lio nas tarefas rotineiras. Algumas vari�veis devem ser
 consideradas para garantir a viabilidade de implanta��o da abordagem:
\begin{description}
\item [Velocidade de reconhecimento] para que a aplica��o seja utilizada pelo
usu�rio com um taxa aceit�vel, garantindo assim uma experi�ncia ;
\item [Qualidade do reconhecimento] de objetos para que sejam encontrados pela
t�cnica com resultados compar�veis a um mec�nico;
\item [Invari�ncia a par�metros ambientais] para que seja poss�vel emular
situa��es reais do dia a dia como altera��o entre momentos do dia, ambientes
esfuma�ados, etc \ldots
\end{description}



\subsection{Cen�rio}

O uso da realidade aumentada em manuten��o de aeronaves pode trazer ganho no
fornecimento de informa��es de procedimentos, na previs�o de
falhas ou no reconhecimento de regi�es com falha.

 Como caso de uso ser� adotado a janela de inspe��o frontal, como mostrado na
 Figura~\ref{fig:ERJ190}, localizada na aeronave Embraer ERJ-190. 
 
\begin{figure}[h!]
\centering
\includegraphics[scale=0.8]{images/ERJ190}
\caption{Posicionamento da janela de inspe��o. Fonte http://www.aero-news.net/}
\label{fig:ERJ190}
\end{figure}

\section{Vari�veis de contorno}
\label{sec:variaveiscontorno}
O cen�rio de reconhecimento de objetos dentro da aeronave traz alguns desafios que devem ser contornados
\begin{itemize}
\item Pouca ilumina��o em ambientes internos
\item Objetos muito parecidos entre si
\item Alguns objetos com textura
\item Objeto brilhante
\end{itemize}

 %\section{Caso de uso}