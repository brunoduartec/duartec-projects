O reconhecimento de objetos em uma cena para posterior uso em realidade aumentada 
depende de diversas vari�veis, causando a necessidade do uso de t�cnicas 
espec�ficas para cada cen�rio, sendo portanto, um estudo de fronteiras para a melhor escolha 
do algoritmo de reconhecimento, de acordo com a aplica��o em quest�o de grande
valia para o meio acad�mico. 
Esta tese se prop�e a pesquisar, categorizar e tra�ar fronteiras das t�cnicas
conhecidas, tendo como caso de uso manuten��o de aeronaves feita dentro de
centros fechados, utilizando as t�cnicas BRISK,FAST,FREAK,GFTT,MSER,
 ORB,STAR,SURF,SIFT em uma an�lise aplicada com imagens reais de janelas de
 inspe��o do Embraer ERJ-190 para reconhecimento de objetos e poster aplica��es
 em manuten��o.
 Comparando todas as t�cnicas quanto � cad�ncia e � precis�o de reconhecimento
 de caracter�sticas, � poss�vel selecionar GFTT e ORB
 como t�cnicas mais apropriadas ao contexto, por terem seus resultados de
 varia��o de rota��o, escala, briho e blur dentro de uma faixa esperada para o
 contexto de manuten��o.
 

