Para compreender o presente trabalho se faz necess�rio elucidar fundamentos
b�sicos, fornecendo assim subs�dios necess�rios para o leitor avaliar com mais
propriedade os resultados apresentados. Os conceitos apresentados seguem uma
ordem crescente de conhecimento para que seja constru�do um arcabolso de
conhecimento suficiente.
O cap�tulo inicia com uma explana��o sobre realidade virtual e realidade
aumentada, elucida o uso de dispositivos para imers�o. Para a realiza��o de
aplica��es de realidade aumentada o uso de sensores para abstrair informa��es do
ambiente � fundamental, sendo as c�meras de baixo custo os mais comuns,
portanto, � descrito no cap�tulo os problemas de distor��o inerentes a tais
dispositivos e a modelagem matem�tica adotada para as c�meras utilizada em todas
as abordagens presentes nesse trabalho.
A abordagem utilizada nesse trabalho � de reconhecimento de caracter�sticas
locais, diferente do conceito comum de reconhecer padr�es como ret�ngulos,
circulos, ou contornos, portanto conceituar o reconhecimento de caracter�sticas
se faz t�o importante.
� tamb�m apresentada conceitua��o b�sica de cada um dos algoritmos utilizados e
de par�metros utilizados para a an�lise dos resultados

\section{Realidade Virtual}

A Realidade Virtual (RV) � uma �interface avan�ada do usu�rio�
para acessar aplica��es executadas no computador, propiciando a
visualiza��o, movimenta��o e intera��o do usu�rio, em tempo real,
em ambientes tridimensionais gerados por computador, como mostra a
figura~\ref{fig:realidadevirtual}.
O sentido da vis�o costuma ser preponderante em aplica��es de realidade
virtual, mas os outros sentidos, como tato, audi��o, etc. tamb�m
podem ser usados para enriquecer a experi�ncia do usu�rio.\cite{realidadevirtual}

\begin{figure}[h!]
\centering
\includegraphics[scale=0.5]{images/realidade-virtual}
\caption{Aplica��o de realidade virtual}
\label{fig:realidadevirtual}
\end{figure}


\section{Realidade Aumentada}
A realidade aumentada como citado em \cite{SurveyAR} � uma t�cnica de vis�o
computacional em que valendo-se de artefatos do mundo real tem por objetivo causar sensa��o de imers�o 
do usu�rio em um ambiente aumentado por artefatos virtuais, ao contr�rio de ambientes puramente virtuais 
como � comum em aplica��es de realidade virtual.
Idealmente o mundo virtual se torna imersivo o suficiente para que o usu�rio n�o consiga distinguir o real do virtual.
Alguns autores definem AR como tendo a necessidade de utilizar-se interfaces visuais port�teis para que a 
usabilidade tenha mais coer�ncia com a proposta inicial de garantir uma experi�ncia imersiva.
As imagens s�o obtidas por c�meras e o resultado apresentado em dispositivos como projetores ou 
displays como monitores, tablets ou \emph{head-mounted display} (HMD).
Realidade aumentada pode ser realizada com ou sem marcadores para facilitar o
reconhecimento e posicionamento de entidades. No presente trabalho � utilizada a
abordagem sem marcadores para que aplica��es no cen�rios de manuten��o se torne
mais flex�vel no que tange � aplicabilidade e configura��o inicial, n�o sendo
necess�rio modificar o ambiente.A figura~\ref{diagram:pipelinera} apresenta um
pipeline b�sico de realidade aumentada de forma can�nica. O presente trabalho,
trata de aspectos at� a etapa de reconhecimento.
As etapas s�o representadas por:
\begin{itemize}
  \item \textbf{Captura:} Etapa de obten��o de imagens, feita por sensores como
  c�meras;
  \item \textbf{Prepara��o:} Etapa de prepara��o da imagem, aplicando filtros
  para a etapa de detec��o; 
  \item \textbf{Detec��o:} Etapa de detec��o de padr�es, em que s�o removido
  informa��es das imagems;
  \item \textbf{Reconhecimento:} Etapa de reconhecimento de padr�es e compara��o
  das caracter�sticas reconhecidas na etapa anterior;
  \item \textbf{Rastreio:} Etapa que garante-se que a imagem reconhecida
  continua no contexto, reconhecendo apesar de movimenta��es ou outras
  varia��es;
  \item \textbf{Apresenta�ao:} Etapa em que s�o desenhado na tela representa��es
  dos objetos reconhecidos de acordo com a aplica��o.
\end{itemize}

\begin{figure}[H]
\centering
\begin{tikzpicture}[scale=1.5 ,transform shape]

  \node[draw,rectangle] (a) {Captura};
  \node[draw,rectangle,below of=a] (b) {Prepara��o};
  \node[draw,rectangle,below of=b] (c) {Detec��o};
  \node[draw,rectangle,below of=c] (d) {Reconhecimento};
  \node[draw,rectangle,below of=d] (e) {Rastreio};
  \node[draw,rectangle,below of=e] (f) {Apresenta��o};

  % 1st pass: draw arrows
  \draw[vecArrow] (a) to (b);
  \draw[vecArrow] (b) to (c);
  \draw[vecArrow] (c) to (d);
  \draw[vecArrow] (d) to (e);
  \draw[vecArrow] (e) to (f);
    
    
  % 2nd pass: copy all from 1st pass, and replace vecArrow with innerWhite
  \draw[innerWhite] (a) to (b);
  \draw[innerWhite] (b) to (c);
  \draw[innerWhite] (c) to (d);
  \draw[innerWhite] (d) to (e);
  \draw[innerWhite] (e) to (f);

  % Note: If you have no branches, the 2nd pass is not needed

\end{tikzpicture}
  \caption{Pipeline Can�nico de Realidade Aumentada}
  \label{diagram:pipelinera}

\end{figure}




\section{Dispositivos}
 Para que a experi�ncia de imers�o seja
 completa, podem ser usados alguns dispositivos como �culos especiais, monitores
 ou projetores \cite{Devices}.

\subsection{\emph{Head-Mounted Displays}}
� um equipamento utilizado na cabe�a de forma que as duas m�os do usu�rio fiquem livres e tem por objetivo 
exibir imagens e �udio, sendo uma interface muito utilizada tanto em VR quanto
em AR.
Os HMD basicamente s�o dispositivos constitu�dos por duas telas posicionadas
frente ao olho do usu�rio.
A tecnologia pode ser empregada para exibir imagens estereosc�picas
apresentando os respectivos pontos de vista de cada olho para cada tela, o que contribui em muito na experi�ncia de imers�o.
Esses dispositivos funcionam tamb�m como entrada de dados,
porque cont�m sensores de rastreamento que medem a posi��o e orienta��o da
cabe�a, transmitindo esses dados ao computador.
Existem dois tipos de HMDs: \emph{Feed-Through} e \emph{See-Through}

\subsubsection{\emph{Feed-Through}}
S�o dispositivos que representam um sistema fechado de visualiza��o de
imagens, como mostrado na Figura~\ref{fig:feedthrough}, em que o usu�rio
consegue enxergar somente o que � mostrado no \emph{display}, sendo assim, o
resultado apresentado � sempre a soma da Figura real com informa��es superpostas.

\begin{figure}[H]
\centering
\includegraphics[scale=0.8]{images/feedthrough}
\caption{Arquitetura do \emph{Feed-Trough}. Fonte \cite{Devices}}
\label{fig:feedthrough}
\end{figure}


\subsubsection{\emph{See-Through}}
S�o dispositivos constru�dos com lentes transl�cidas em que o usu�rio enxerga o
mundo real e com algum tipo de sistema que sobrepoe na lente as informa��es
adicionais. Como mostrado na Figura~\ref{fig:seethrough}:

\begin{figure}[H]
\centering
\includegraphics[scale=0.8]{images/seethrough}
\caption{Arquitetura do \emph{See-Trough}. Fonte \cite{Devices}}
\label{fig:seethrough}
\end{figure}


\subsection{Projetores}
O uso de projetores possibilita uma abordagem de realidade aumentada diferente porque pode ser
 utilizada para cobrir superf�cies largas, projetando sobre objetos como carros, pessoas, pr�dios, etc�.
Um problema dessa abordagem � que a calibra��o se faz necess�ria em situa��es
como superf�cies irregulares ou n�o paralelas ao projetor.

\subsection{Monitores}
O uso de monitores reduz bastante o custo da aplica��o apesar de ter perda de
imers�o por ser um m�todo de visualiza��o indireta, o que implica o usu�rio
ficar olhando na dire��o do monitor. Entretanto, existe a possibilidade de
compartilhar os resultados da AR com mais de uma pessoa ao mesmo tempo. Como
mostrado na Figura~\ref{fig:monitores}:

\begin{figure}[H]
\centering
\includegraphics[scale=0.7]{images/monitores}
\caption{Realidade Aumentada com projetores. Fonte \cite{Devices}}
\label{fig:monitores}
\end{figure}

Nowaday cameras have problems of distortion that can be treated because there are constant distortions and calibrations issues.

Radial distortion as "barrel" or "fish-eye" effect

barrel

Barrel lens distortion is an effect associated with wide-angle lenses and, in particular, zoom wide-angles. This effect causes images to be spherized, which means the edges of images look curved and bowed to the human eye. It almost appears as though the photo image has been wrapped around a curved surface.

Tangential Distortion


\subsection[Reconhecimento]{Reconhecimento}
\begin{frame}
\frametitle{Reconhecimento}

\begin{figure}[H]
\centering
\includegraphics[scale=0.4]{images/reconhecimento}
\caption{Ilustra��o do procedimento de reconhecimento com caracter�sticas
locais.
Fonte \cite{localfeaturedetector}}
\label{fig:reconhecimento}
\end{figure}

\end{frame}

\begin{frame}
Processo de reconhecimento como ilustrado na
imagem~\ref{fig:reconhecimento}:
\begin{itemize}
	\item Encontrar um grupo de \emph{keypoints} distintos;
	\item Definir uma regi�o em torno de cada \emph{keypoint};
	\item Extrair e normalizar o conte�do da regi�o;
	\item Calcular um descritor para a regi�o normalizada;
	\item Encontrar correspond�ncias de descritores.  
\end{itemize}



\end{frame}

\section{Cad�ncia}
� a medida do n�mero de quadros individuais que um determinado dispositivo �ptico ou eletr�nico processa e exibe
 por unidade de tempo. Em geral a cad�ncia � medida em fps.
Em cinema, a cad�ncia de proje��o padr�o desde 1929 foi fixada em 24fps, sendo
no per�odo do cinema mudo a maioria dos filmes eram rodados com cad�ncia entre 16 e 20fps.
Em v�deo, os principais sistemas lidam com cad�ncia entre 25fps(PAL) e 30fps(NTSC).
As aplica��es devem ter cad�ncia toler�vel dependendo de seu uso, segundo
\cite{Tang93whydo} para aplica��es interativas o m�nimo toler�vel � de 5fps enquanto para aplica��es
 de anima��es fluidas de 30fps.
Sendo a cad�ncia a freq��ncia entre frames, deve ser contabilizado o tempo de
gera��o de informa��es e o tempo de dispor a informa��o no dispositivo �ptico.
O tempo de cada frame � calculado  o inverso do n�mero de fps.
%como mostrado na
%equa��o~\ref{eq:fps}
%\begin{equation}
%t_{frame} =  \frac{1}{fps}
%\label{eq:fps}
%\end{equation}
No caso de cad�ncia m�nima de 5fps, temos quadros com tempo menor que 200ms,
portanto as an�lises devem ser balizadas a tempos menores.







