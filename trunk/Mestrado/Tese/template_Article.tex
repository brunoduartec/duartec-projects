\documentclass[]{article}
%\usepackage[scaled=.92]{helvet}
\usepackage{times}
\usepackage{graphicx}
\usepackage[section] {placeins}
\usepackage{float}
%%* Dica: use esse pacote pra aceitar caracteres acentuados, Ç etc sem sofrimento
%%* obs: válido se estiverem escrevendo com codificação Windows, caso contrário
%%* pode ser necessário trocar ansinew por utf8 ou algo assim.
\usepackage[ansinew]{inputenc}

%% use this for zero \parindent and non-zero \parskip, intelligently.
\usepackage{parskip}

%% the 'caption' package provides a nicer-looking replacement
\usepackage[labelfont=bf,textfont=it]{caption}

\usepackage{url}
\usepackage{listings}
\usepackage{color}
\usepackage{graphicx}

\lstset{language=C++,breaklines=true,morekeywords={float3,float4},keywordstyle=\color[rgb]{0,0,1},        commentstyle=\color[rgb]{0.133,0.545,0.133} }


%opening
\title{}
\author{Bruno Duarte Corr�a}

\begin{document}

\maketitle

\begin{abstract}
There are several augmented reality techniques, although each one has its flaws due to the environment or other external constraints. The study of boundaries and constraints can provide to the developer more decision power while choosing the appropriate technique. This essay provides a method to, based on common parameters and situation, chose the appropriated technique.


\end{abstract}

%% The ``\keywordlist'' command prints out the keywords.


\section{Introduction}
Maintenance and Maintainability are two important parameters that influence directly the airplane life cycle and operational cost. They have to be considered in a new airplane design so as in an airplane modification. These considerations are essential for the design success.

Maintainability means the ease with which the airplane can be safely repaired in the least amount of time. It provides design characteristics that facilitates the airplane maintenance and results in minimum maintenance costs in monitoring, dispatchability, availability, servicing and other operations. Maintenance are corrective, predictive and preventive actions for replace or maintain the airplane in airworthy conditions. It includes inspections, servicing, modifications, tests and repairs.

It is necessary to have in mind that maintainability does not eliminate the needs for aircraft maintenance and it should not be overestimated. This could affect indirectly the costs-benefits. Maintainability and optimized procedures of maintenance turn the airplane economically viable and more competitive.

For EMBRAER ERJ 145 family the Direct Maintenance Cost (DMC) increased strongly in last years. This happened in part because systems technologies are older and maintenance always increases with the airplane life time. The EMBRAER ERJ-145 was conceived in 1990 decade and its basics systems became obsolete despite some modifications that had been incorporated. But the main cause of DMC increase is related with the propulsion system maintenance cost. The propulsion system has an expensive maintenance due to it obsolescence and the way that the supplier contract was made. It represents approximately 50 \% of the maintenance cost.

Other systems from EMBRAER ERJ 145 airplane also have an expensive maintenance due to supplier contracts, like avionics and landing gear systems. Further, the high prices of EMBRAER limited life parts can also be pointed as causes of the high maintenance cost.

According to ERJ 145's Owner's Operators Guide nowadays the ERJ-145's maintenance program is optimized for operators flying 2,500 flight hours (FH) and 2,500 flight cycles (FC) per year. Consequently operators with lower or higher utilizations will find maintenance planning more complicated. The ERJ 145's Owner's Operators Guide also says that the ERJ 145 has a base maintenance cycle of 20,000 FH and many aircraft have been though their first base cycle. The ERJ 145 structure was designed for three or even four base maintenance cycles during their service lives. The points described above are better discussed in the next items.

First, important concepts are presented and after that, the possibilities of DMC reduction are shown.




\section{Concepts}
\subsection{Camera}
\subsection{Camera Calibration}
\subsection{SLAM}
\subsection{Structure from Motion (SfM)}
\subsection{Edge Recognition}

\section{Method}
The recognition of an object in a scene with precision can be a trick problem to be solved, and to increase the experience with some virtual artifacts calls for a high level of localization accuracy.

This paper presents an approach based a pre-known 3d representation of the scene.

The real time reconstruction of the scene are fundamental to reduce the cumulative error added to the recognition added after each frame because of interest point recognition hardness caused by light variation, textures and other issues studied in this essay.
As proposed in \cite{ISMAR2012}

\begin{figure}[ht!]
\centering
\includegraphics[width=90mm]{images/algorithm.jpg}
\caption{Algorithm Diagram}
\label{algorithm}
\end{figure}


O caminho feliz é ditado por aquisição de imagem, reconhecimento de bordas, reconhecimento de pontos de interesse, model match reconhecimento de 



\subsection{AR Common Flaws}

Jitter
Occlusion

\subsection{Camera Calibration}

\subsubsection{Error Reduction Approach}

Todo frame com a movimentação da camera ou do objeto o rastreamento descasa um pouco, prejudicando a precisão.
Dois métodos são utilizados para reduzir o erro de rastreamento
SLAM utilizado para conferir uma calibracao boa recuperando informacoes de movimento e tentando inferir a posicao da camera


Tenho que ver como vou medir o erro a cada frame, se vai ser um minimo quadrados de pontos importantes com o modelo ou algo mais robusto


\subsection{Image Acquisition}

Etapa simples, tenho que definir qual a latencia de aquisicao de imagens que terei uma quantidade de informacoes suficiente, aqui cabe colocar uma variável para calcular o erro final

\subsection{Edge Detection}

Obter as bordas para depois conseguir retirar os pontos de interesse, aqui cabem filtros ou reconhecedores de padrao.

Também é outra fonte de erro adicionada ao final do modelo.

\cite{Drummond99real-timetracking}


\subsection{Extract Interest Points}

Dependendo do reconhecedor de padrão conseguirei retirar um tipo de ponto de interesse diferente.

Aqui terei uma miriade de reconhecedores que podem adicionar erros ou incertezas ao modelo

\subsection{Model Match}


Eu acho que é a parte que vai mais dar trabalho porque eu vou ter que provavelmente estimar pose para definir projecoes diferentes

\subsubsection{Choosing the apropriate approach to 3d track}



\subsubsection{Real Time Model Reconstruction}

Quando reconhecermos e 


\subsubsection{SLAM}




\section{Boundaries}
Nessa seção serão descritos as fronteiras que desejam ser estudadas bem como porque elas são relevantes


\section{Study Case}
\input{parts/results}

\section{Conclusion}
As the project goals was to increase the revenue and increase the attractive of the 145, all the interior modifications aimed to be expressive enough to worth the pain. Considering the world reality when it is released, could some aggressive proposal be done but it was validated what gave us a great revenue margin. Some subjective validations could not be done in another way than in a mockup which brought to the project a level of perception not reached otherwise.

\bibliographystyle{sbgames}
\bibliography{bibliography}
\end{document}
