\documentclass[]{article}
%\usepackage[scaled=.92]{helvet}
\usepackage{times}
\usepackage{graphicx}
\usepackage[section] {placeins}
\usepackage{float}
%%* Dica: use esse pacote pra aceitar caracteres acentuados, Ç etc sem sofrimento
%%* obs: válido se estiverem escrevendo com codificação Windows, caso contrário
%%* pode ser necessário trocar ansinew por utf8 ou algo assim.
\usepackage[ansinew]{inputenc}

%% use this for zero \parindent and non-zero \parskip, intelligently.
\usepackage{parskip}

%% the 'caption' package provides a nicer-looking replacement
\usepackage[labelfont=bf,textfont=it]{caption}

\usepackage{url}
\usepackage{listings}
\usepackage{color}

\lstset{language=C++,
breaklines=true,
morekeywords={float3,float4},keywordstyle=\color[rgb]{0,0,1},
        commentstyle=\color[rgb]{0.133,0.545,0.133} }


%opening
\title{}
\author{Bruno Duarte Corrêa}

\begin{document}

\maketitle

\begin{abstract}

There are several augmented reality techniques, although each one has its flaws due to the environment or other external constraints. The study of boundaries and constraints can provide 


\end{abstract}

%% The ``\keywordlist'' command prints out the keywords.


\section{Introduction}
Based primarily on the parameters cited on the spec phase leading to decrease the maintenance cost and increase the profit.


The main modification guided by market forecasting the possibility of a more passenger reality in the world by increasing to 60 seats adding a new fuselage section.


To reduce the airplane weight all the illumination were replaced to LED instead of regular lamps what indirectly decreases the maintenance costs. The replacement of the seat was guided to decrease the pitch in the cabin and enable the increase of seats without increasing the fuselage so much, choosing a 29" pitch seat.


An operator galley usage analysis guided the decision of decrease its number but not its necessary volume.
All decisions made were also mockup driven to validate comfort and ergonomics.


\section{Model Based}
\subsection{Arestas}
	\subsubsection{Descritores de Arestas}

\subsection{Fluxo Optico}
\subsection{Textura}


\section{Reconstruction}
reconstruction test



\cite{DesignAI}

\section{Hybrid}
\input{parts/hybrid}


\section{Boundaries}
Nessa seção serão descritos as fronteiras que desejam ser estudadas bem como porque elas são relevantes

\section{Comparisons}
\input{parts/comparisons}


\section{Conclusion}
%% ---------- Conclusion ---------- %%

The study main objective was to verify the technical and economical viability of the 145 modification on structures point of view, as mentioned before, most of the studies were a demand from aeronautics group due to the high integration between those two.

In the end, the analysis has shown the only small changes were necessary to guarantee the aircraft good safety margin and fulfill the regulations. The costs of those changes were smaller than initially thought as well as the necessary reinforcements.

\bibliographystyle{sbgames}
\bibliography{bibliography}
\end{document}
