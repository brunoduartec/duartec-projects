The electrical system supplies DC and AC power for all loads of the aircraft. The current system is mainly comprised of four 400 A, 28 VDC generators (two per engine), one 400 A, 28 VDC APU starter/generator, two nickel-cadmium batteries, one lead-acid backup battery and an external power source receptacle. A schematic of the electrical distribution system is showed below.

\begin{figure}[H] % Example image
\center{\includegraphics[width=400px]{Pictures/Electrical_Systems/ERJ145_DC_Ele_Distr_Sys_Schematic.eps}}
\caption{Electrical system schematic}
\label{fig:145eletrical}
\end{figure}

The electrical system of the ERJ 145 family is remarkably simple and reliable, which helps to explain the absence of major complaints from operators concerning this system.
During the first years of the ERJ 145 fleet, the bearings of the electrical generators suffered a spread crisis of infant mortality, raising the number of unscheduled removals. This issue was later solved by the manufacturer Goodrich, thus achieving normal MTBUR rates.

Since then, only minor modifications were applied to new aircraft, as new main Ni-Cd batteries and a new static inverter, while keeping the same architecture.

The Electrical Systems team focused on modifications that could reduce maintenance costs (due to scheduled and unscheduled tasks) and respond demands from other groups, as can be seen below.
