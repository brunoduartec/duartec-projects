% 10.6 - Conclusion
The electrical team worked on modifications focusing on reducing maintenance cost. To achieve that, several tests and certification processes are necessary, as well as buying new components to substitute the current ones. All these items have their costs, which will impact the manufacturer, the supplier or the airline. The team's job was to predict these costs with the information provided during the EC and EP studies, predict the maintenance cost reduction to the aircraft's operator and use these numbers to make the decision whether to insert the modification in the modernization package or not.

Besides cost information, the team also studied which tests would be necessary to develop and certificate the proposed modifications, and acquired the information necessary to guarantee the technical possibility of implementing them.

The results, as discussed along the chapter, were changing the original APU starter/generator by a brushless one and switching the static inverter's and the main batteries' suppliers. These changes are estimated to reduce the maintenance cost and grant the operator some profit over the initial investment after 10 years.
