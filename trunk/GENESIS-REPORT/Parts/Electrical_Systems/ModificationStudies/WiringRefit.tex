As previously predicted during the Conceptual Studies, the ERJ-145's wiring will need to be refit due to other teams' modifications. During Preliminary Studies, it was defined which of these modifications will be performed, and what kind of changes regarding wiring will be needed.

The interiors and aeronautics teams worked together on a fuselage extension of 1.056 m, which demands wiring extension of the same length. The interiors team worked on two modifications that require some wiring refit or addition: substitution of the lighting components from lamps to LED and insertion of USB jacks to allow passengers to charge their electronic devices.

Extending the fuselage also requires extending the electrical wires passing through it, which can basically be performed by two ways: inserting connectors and adding a wiring section, or removing the wiring and inserting a new extended one. Both options require the removal of the entire wiring, because the insertion of connectors can not be performed inside the aircraft. The first option adds failure modes to the electrical system, possibly compromising current safety requirements. Moreover, this alternative is more expensive due to the need for a more careful removal of the wiring (which must be installed again after the modifications).

The man-hour costs for the insertion of a new extended wiring are cheaper and easier to perform. In addition, its installation cost will be diluted by the application of some necessary modifications studied by other technical teams, such as flight control cables removal and hydraulic tubes extension. These modifications will require removing the interior and the floor to be performed, reducing the cost to remove and insert electrical wiring.

The wires required to be removed are the lighting wires, power cables (from the 5 generators) and wires that supply the electrical components contained in the back of the airplane (wiring identification number 101 and 102). The hardware cost for these wires with an increase of 1.056 m is US\$ 2,300.00 (lighting wiring not taken into account).

Removing all the wiring means a risk of breaking connectors during the procedure. It was estimated a total of 100 damaged connectors, totaling US\$ 3,000.00 to replace these.

The LED addition opens an opportunity to resize the lighting wire diameter. Since the LED requires less power than the current lamps, the wire can be thinner. Currently there are 150 wires of  1.0 $mm^{2}$ each (20 AWG size), with 22 m length supplying the fuselage, wing and tail lighting. The new LED lighting system will require 130 wires of 0.5 $mm^{2}$ each (22 AWG size), reducing the lighting wiring weight by 5.5 kg. The hardware cost for these wires is US\$ 700.00.

The last wiring modification is inserting wires to provide power to USB jacks. They will join the 101 and 102 cables, which have space to grow in the floor compartment. The jacks will be placed in the armrests, which already have place for wiring inside. Each jack consumes 0.45 A at 28 V and requires 1.0 $mm^{2}$ wires. The total hardware cost for this wiring is US\$ 800.00.

The total wiring hardware cost is US\$ 6,800.00 for all the modifications. The man-hour estimations to perform manufacturing, wiring removal and wiring insertion are 400 MH, 250 MH and 400 MH respectively. This sums a total of US\$ 52,500.00, thus, the total cost for this modification is US\$ 59,300.00. To mitigate the risks of these cost estimations, an upper error margin of 10\% was created for the man-hour costs, and a 5\% one for hardware costs, resulting on a maximum cost of US\$ 57,750.00 for manufacturing and assembling and a maximum cost of US\$ 7,140.00 for hardware, resulting in a maximum total cost of US\$ 64,890.00.

The development of the new wiring involves calculation of impedance increase, test of the APU start with the new impedance, and some changes in the maintenance manual. These tests' costs are estimated in US\$ 25,000.00. With a 5\% margin, the total non concurrent cost is US\$ 26,250.00.
