% 10.2.2 - APU Starter Generator and GCU
A brushless starter/generator (S/G) has lower maintenance costs than brushed models. The brushless design eliminates the need for brush replacements and the contamination associated with brush wear which reduces the lifetime of the bearings. It has also a lower weight and, as the S/G is located in the tail of the aircraft, it contributes for the CG position. Brushless starter/generators are also more efficient than traditional units.

Three different suppliers were initially considered for the study: Innovative Power Solutions (IPS), Astronics and Skurka. IPS has a family of 28 VDC brushless starter/generators with outputs to 600 A, and is the supplier of all the three brushless S/G for the Learjet 85. Astronics has an off-the-shelf 300 A unit, and Skurka is currently developing a 28 VDC brushless S/G rated at 28-30 VDC for a continuous load of 150-400 A.

The model used as example is the SGA1-300-2, a 300 A 28 VDC brushless S/G manufactured by IPS. The ERJ 145 S/G is rated at 400 A, and another model with equivalent power capacity would be used for replacement. It consists of an eight pole rotor and a 48 slot stator which has two 3-phase windings separated by 30 electrical degrees. The S/G 6-phase output is rectified to DC by the IPS model SGCU1-300-1 starter/generator control unit. Thus, instead of two cables connecting the S/G to the APU contactor, six power cables would connect the brushless S/G to its control unit.

Typically, control units for brushless starter/generators are heavier and bigger than the traditional units. They have all the functionalities of standard AGCUs, but also rectifies the power from the S/G when it is operating as a generator. The dimensions of the current AGCU in the ERJ 145 are 6.22" x 4.42" x 4.25" (L x W x H) with a weight of 2.9 lbs, and it is located under the cockpit floor, approximately below the observer seat. The control unit SGCU1-300-1 for the IPS starter generator measures 12" x 8.25" x 7" and weighs 18 lbs. This extra weight, placed under the cockpit, also contributes for the CG position.

During a visit at the production line of the ERJ 145 XR, it was observed that there was no enough space under the cockpit floor for a starter generator control unit with the dimensions specified previously, as can be seen in the Figure \ref{AGCU_Location}. It was necessary to find another place for this component inside the aircraft.

\begin{figure}[H] % Example image
\center{\includegraphics[width=400px]{Pictures/Electrical_Systems/AGCU_Location.eps}}
\caption{Current AGCU location}
\label{fig:AGCU_Location}
\end{figure}

In order to find a suitable location for the new control unit, a mockup of this component was made with the exact dimensions of the model SGCU1-300-1 (IPS), as can be seen below. The AGCU mockup was used within the cabin mockup for the decision making with the Interiors group, and it was decided that the control unit will be located in a rack behind the trolley at the forward fuselage, with the TCAS and GPS units.

\begin{figure}[H] % Example image
\center{\includegraphics[width=400px]{Pictures/Electrical_Systems/AGCU_Mockup.eps}}
\caption{AGCU mockup (IPS model SGCU1-300-1)}
\label{fig:AGCU_mockup}
\end{figure}

The maintenance savings after 10 years of operation due to the elimination of brush replacement is approximately US\$ 8125.00, considering a hour time window of 400 FH and 2 MH to discard and restore the S/G brushes. The use of a brushless S/G reduces the DMC by 0.154 \%, and its respective AGCU reduces the DMC by 0.055 \%.
