% 10.2.3 - Ni-Cd Batteries
Two battery technologies were analyzed: nickel-cadmium and lead-acid. Ni-Cd is characterized by a greater lifetime (approximately 5 years), has better performance at low temperatures and has a lower weight. Pb-acid batteries have lower maintenance costs, are cheaper and suffer less overheating and thermal runaway issues than Ni-Cd.

The supplier Concorde has a Pb-acid battery (model RG-442) already certified for several versions of the ERJ 145/135, with a STC approved by FAA. However, the market price of this battery is very high (US\$ 5,000.00), mainly due to its development and certification costs for its application in the ERJ 145 family. Currently, there are no records of operators using this kind of battery within their fleets.

Another issue concerning the use of Pb-acid batteries is the fact that ERJ 145 operators are used to Ni-Cd models. These batteries does not suffer with low ambient temperatures, and there is no need to remove the Ni-Cd batteries when the aircraft is in a cold environment.

Besides the Marathon model BTSP-4445L (certified with the aircraft), other two part numbers are already available for the ERJ 145 family: Marathon M3-44-8 and Saft 442CH1, both rated at 44 Ah, 24 V. Instead of the 500 FH inspection interval for the BTSP model, 442CH1 has a check interval of 1000 FH and the M3 battery must pass a check every 2000 FH. These two batteries have also higher MTBUR rates than the BTSP model (9548 FH): 10509 FH for the Saft model 442CH1 and 16280 FH for the Marathon M3-44-8 battery.

With a higher MTBUR and check interval, the use of the Marathon model M3-44-8 instead of the BTSP-4445L reduces the DMC by 0.12 \%.
