% ---------- Fuselage ---------- %
\subsubsection{Fuselage}

In order to fit the additional seat rows it was necessary to stretch the fuselage by adding a 1056 mm long fuselage section. Due to the weight and balance characteristics of the ERJ-145 it was decided that this additional section had to be positioned behind of the aircraft CG, as shown in the Figure \ref{fig:fig4_Addtional_fuselage_section}.

\begin{figure}[H] % Example image
\center{\includegraphics[width=400px]{Pictures/Structures/ModificationStudies/fig4_Addtional_fuselage_section.eps}}
\caption{ERJ-145 with the highlighted addtional fuselage section}
\label{fig:fig4_Addtional_fuselage_section}
\end{figure}

This increase in length also increased the distance between the fuselage junction point and the tail and the CG of the whole aft sections of the aircraft, as shown in Figure \ref{fig:fig5_Distances_between_fuselage}.

\begin{figure}[H] % Example image
\center{\includegraphics[width=400px]{Pictures/Structures/ModificationStudies/fig5_Distances_between_fuselage.eps}}
\caption{Distances between the fuselage junctions and the aft section CG and stabilizers}
\label{fig:fig5_Distances_between_fuselage}
\end{figure}

The increment of these distances consequently increased the arm between the fuselage junction and the tail loads and the rear sections weight load, therefore increasing the compression loads in this fuselage portion.

In order to bear this increased loads, this fuselage section had to be strengthened. To minimize the cost and the impact of this modification, it was decided that the new fuselage section would be reinforced and fitted between the central fuselage sections 3 and 4.

The compression load increase is directly proportional to the arm length increase, therefore the increment in the structure thickness can be estimated by a simple rule of three.

% Table generated by Excel2LaTeX from sheet 'Plan2'
\begin{table}[htbp]
  \centering
  \caption{Skin reinforcement analysis}
    \begin{tabular}{rcccc}
    \toprule
    \multicolumn{5}{c}{\textbf{Skin}} \\
    \midrule
          & \textbf{Critical Load} & \textbf{Thickness [mm]} & \textbf{Arm Increase} & \textbf{New Thickness [mm]} \\
    \textbf{Upper} & Hor. Stab. Deflection & 1.60  & 8.7\% & 1.74 \\
    \textbf{Lateral} & Vert. Stab. Deflection & 1.60  & 8.7\% & 1.74 \\
    \textbf{Lower} & Landing & 2.54  & 23.9\% & 3.15 \\
    \bottomrule
    \end{tabular}%
  \label{tab:skin_reinforcement}%
\end{table}%

Analogously, a thickness increase was applied to the stringers according to their type and position. Stringers were also added in the lower portion of the central fuselage 3, between the junction and the wing stub, to help distribute the additional load.

% Table generated by Excel2LaTeX from sheet 'Plan2'
\begin{table}[htbp]
  \centering
  \caption{Stringers reinforcement analysis}
    \begin{tabular}{rcccc}
    \toprule
    \multicolumn{5}{c}{\textbf{Skin}} \\
    \midrule
          & \textbf{Critical Load} & \textbf{Thickness [mm]} & \textbf{Arm Increase} & \textbf{New Thickness [mm]} \\
    \textbf{Upper} & Hor. Stab. Deflection & 1.27  & 8.7\% & 1.38 \\
    \textbf{Lateral} & Vert. Stab. Deflection & 1.27  & 8.7\% & 1.38 \\
    \textbf{Lower} & Landing & 3.0   & 23.9\% & 3.72 \\
    \bottomrule
    \end{tabular}%
  \label{tab:stringers_reinforcement}%
\end{table}%

The frames have the same thickness of the other fuselage sections. Their positioning and quantities were determined in a way that the two frames criterion was met, with minimum cost and weight and that allowed the addition of windows.

The Figure \ref{fig:fig6_Plug_frame_positioning} shows the final frame positioning.

\begin{figure}[H] % Example image
\center{\includegraphics[width=250px]{Pictures/Structures/ModificationStudies/fig6_Plug_frame_positioning.eps}}
\caption{Plug frame positioning}
\label{fig:fig6_Plug_frame_positioning}
\end{figure}

Based on the actual cost and weight of existing fuselage segments it was possible to estimate this values for the new segment. With the addiction of the reinforcements a new weight was calculated, and based on that value the final cost was determined. The results are shown in Tables \ref{tab:tab5_cost_weight} and \ref{tab:tab5_plug_cost}.

% Table generated by Excel2LaTeX from sheet 'Plan3'
\begin{table}[htbp]
  \centering
  \caption{Cost and weight analysis}
    \begin{tabular}{ccc}
    \toprule
    \multicolumn{3}{c}{\textbf{Existing Sections}} \\
    \midrule
    Lenght [m] & Weight [kg] & Cost [US\textbackslash{}\$ thousands] \\
    2.853 & 184   & 56 \\
    1.532 & 105   & 30 \\
    \bottomrule
    \end{tabular}%
  \label{tab:tab5_cost_weight}%
\end{table}%

% Table generated by Excel2LaTeX from sheet 'Plan3'
\begin{table}[htbp]
  \centering
  \caption{Plug Cost and weight}
    \begin{tabular}{ccc}
    \toprule
    \multicolumn{3}{c}{\textbf{New Sections}} \\
    \midrule
    Lenght [m] & Weight [kg] & Cost [US\textbackslash{}\$ thousands] \\
    1.056 & 77    & 21 \\
    With reinforcements & 87    & 24 \\
    \bottomrule
    \end{tabular}%
  \label{tab:tab5_plug_cost}%
\end{table}%
