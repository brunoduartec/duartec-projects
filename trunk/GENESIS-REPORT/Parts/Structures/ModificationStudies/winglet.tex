% ---------- Winglet ---------- %
\subsubsection{Winglet}

The add of the winglet in the wingtip increases the lift produced there. So, the bending moment along the wingspan will increase. It is necessary to analyze if the current structure is able to resist this increase of bending moment.

First, an estimation of the new lift distribution over the wing is done using a simple panel method:

\begin{figure}[H] % Example image
\center{\includegraphics[width=400px]{Pictures/Structures/ModificationStudies/fig1_Lift_distribution_estimation.eps}}
\caption{Lift distribution estimation}
\label{fig:fig1_Lift_distribution_estimation}
\end{figure}

With the lift distribution is possible to determine the new bending moment distribution and the increase of the bending moment (in percentage) along the wing span:

\begin{figure}[H] % Example image
\center{\includegraphics[width=400px]{Pictures/Structures/ModificationStudies/fig2_Bending_moment_distribution.eps}}
\caption{Bending moment distribution}
\label{fig:fig2_Bending_moment_distribution}
\end{figure}

\begin{figure}[H] % Example image
\center{\includegraphics[width=400px]{Pictures/Structures/ModificationStudies/fig3_Bending_moment_increase_winglet.eps}}
\caption{Bending moment increase due to winglet}
\label{fig:fig3_Bending_moment_increase_winglet}
\end{figure}

The increase of bending moment is determined just up to 60 \% of the wing span because although the percentage of increase of the bending moment close to the ting tip is very high, the absolute value of the increase is very low. On this region, the structural elements are not dimensioned by the bending moment, but by other failures modes such as shear or buckling. So it is irrelevant the consequences of the increase of bending moment on this region.

The next step is to list all wing part numbers and analyze if the safety margin of each part number is higher or lower than the bending moment increase. Then all parts numbers whose safety margin is lower than the bending moment increase is listed and an estimation of the reinforcement mass is made.

\begin{table}[H]
\caption{Part number analysis}
\center{\includegraphics[width=400px]{Pictures/Structures/ModificationStudies/tab1_Part_number_analysis.eps}}
\label{tab:tab1_Part_number_analysis}
\end{table}

% Table generated by Excel2LaTeX from sheet 'Plan1'
\begin{table}[H]
  \centering
  \caption{Reinforcements mass}
    \begin{tabular}{lc}
    \toprule
    \textbf{Item} & \textbf{Mass} \\
    \midrule
    Spar I & 0.2 \\
    Spar II & 0.1 \\
    Spar III & 0.2 \\
    Main Box Upper Skin & 0.1 \\
    Main Box Ribs & 0.4 \\
    Upper Skin & 0.1 \\
    Ribs  & 0.1 \\
    \textbf{Total} & \textbf{1.1} \\
    \bottomrule
    \end{tabular}%
  \label{tab:ta2_Reinforcements_mass}%
\end{table}%

For the wing stub, is made the same proceeding, but no structural part needed reinforcements. So, all the reinforcements necessary will be placed in the wing.

Finally is possible to conclude that a reinforcement mass of 1.1 kg per wing is necessary. As this extra wing mass added is very low, it will not affect the aircraft performance, because it is irrelevant comparing to all aircraft weight. Furthermore, all the reinforcements necessary is able to be done in an existing aircraft.

Winglet Installation

To add the winglet in the wingtip. some modifications are necessary to be done. However, EMBRAER has already done this modification to fit the same winglet in the wingtip of the Legacy and the ERJ-145XR. So, the same procedure will be done. Consequently, the wingtip will be more reinforced and it will aid to support the extra loads that will be produced in this region.
