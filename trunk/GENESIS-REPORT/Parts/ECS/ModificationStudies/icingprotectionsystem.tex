The ERJ's 145 ice protection is a thermal heating system type using the engine air bleed as the energy source.

These types of ice protection systems are commonly used in turbo fan engined aircrafts as the ERJ 145 does because of its high energy demand.

In ERJ 145 the bleed air is taken from the $9^{th}$ and 14$^{th}$ stages of engine's compressor and feeds all the pneumatic system. Part of this air is used to heat the leading edges of wings and horizontal stabilizer and also the engine's air intake lips.

Nowadays the anti-icing system operates with a pressure about 18 $\pm$2 psig and the temperature around 120$\celsius$ on the surfaces, demanding a maximum airflow of 47.8 ppm (pounds per minute) for each wing and 87 ppm for the whole horizontal stab.

The proposal made for the Preliminary Design (PD) phase was to elaborate a AMESim model to validate the system behavior and to include a new control system in order to reduce the airflow demand and then reduce the fuel consumption due to the bleed air.

This control system should measure the temperature of the leading edge and operate the AIV (Anti Ice Valve) to minimize the air mass demand that just works as an ON-OFF system nowadays. It could optimize the energy of the system for each type of ice condition. After a talk with an Embraer employee we found that it was not feasible because would be necessary to change the whole measurement system, once the actual system just indicates that the aircraft is in an icing condition but doesn't indicates what kind of ice is forming on the surfaces and according to this employee, there is no available equipment in the actual market that provides this kind of information. This way, an alternative is to change the AIV operating temperature to reduce the airflow, trying to keep the total energy provided to the leading edges.

Some information about the anti-icing system in flight tests data were provided in order to elaborate the model and to make a comparison between the real system behavior and the model results.

The following table shows some results about ERJ's 145 A/I system.

% Table generated by Excel2LaTeX from sheet 'Plan1'
\begin{table}[htbp]
  \centering
  \caption{Results of EMB-145 Integrated Bleed Pneumatic System Simulation}
    \begin{tabular}{rrrrrrr}
    \toprule
    \textbf{Altitude} & \textbf{Tcool out} & \textbf{Wing} & \textbf{Stab} & \textbf{Lip} & \textbf{ECU} & \textbf{Cooler} \\
    \midrule
    \textbf{[Kft]} & \textbf{[C]} & \textbf{[ppm]} & \textbf{[ppm]} & \textbf{[ppm]} & \textbf{[ppm]} & \textbf{[ppm]} \\
    5.0   & 266.0 & 47.83 & 25.14 & 8.66  & 28.40 & 74.60 \\
    10.0  & 266.0 & 44.56 & 23.38 & 8.11  & 27.62 & 70.01 \\
    15.0  & 266.0 & 41.70 & 21.87 & 7.62  & 26.41 & 66.84 \\
    20.0  & 266.0 & 39.25 & 20.69 & 7.20  & 25.45 & 62.27 \\
    22.0  & 266.0 & 38.39 & 20.13 & 7.04  & 24.90 & 60.90 \\
    \bottomrule
    \end{tabular}%
  \label{tab:ResultsIBPS2}%
\end{table}%

From this table the maximum temperature of the wing skin is. This is the result to be reached in the AMESim model.

The first model elaborated was the wing A/I model. The model elaborated in AMESim Imagine Lab counts on the thermal and pneumatic libraries and the source used was a mass flow and temperature source. For having a better fidelity in the model, the wing was divided into three sections, considering the Piccolo's tube diameter different for each one. This discretization can result in a little bit different values from the real behavior of the system but it's an acceptable result once the parameters have been adjusted the most reliable as possible. This model can be seen below.

\begin{figure}[H] % Example image
    \center{\includegraphics[width=400px]{Pictures/ECS/AntiiceFigure1.eps}}
    \caption{AMESim A/I model for a wing section}
    \label{fig:AMESimAImodel}
\end{figure}

\begin{figure}[H] % Example image
    \center{\includegraphics[width=400px]{Pictures/ECS/AntiiceFigure2.eps}}
    \caption{AMESim complete model for the Anti-Icing System}
    \label{fig:AMESimCompleteModel}
\end{figure}

A model of the atmosphere was elaborated and included into the AMESim model considering the variation of the temperature and the pressure as functions of flight altitude and also the aircraft speed (for estimating the convectional heat exchange coefficient) as follow.

\begin{figure}[H] % Example image
    \center{\includegraphics[width=400px]{Pictures/ECS/AntiiceFigure3.eps}}
    \caption{Ambient temperature as a function of flight altitude}
    \label{fig:AmbTempFuncFlAt}
\end{figure}

\begin{figure}[H] % Example image
    \center{\includegraphics[width=400px]{Pictures/ECS/AntiiceFigure4.eps}}
    \caption{Ambient pressure as a function of flight altitude}
    \label{fig:AmbPressFuncFlAt}
\end{figure}

\begin{figure}[H] % Example image
    \center{\includegraphics[width=400px]{Pictures/ECS/AntiiceFigure5.eps}}
    \caption{Convective heat coefficient as a function of air speed}
    \label{fig:ConvHeatCoeff}
\end{figure}

According to the data provided some parameters for this model were considered. After dividing the wing into three sections and consequently the piccolo's tube, the respectively values of internal diameter, external diameter and length for each section of the tube are:
section 1 - $\phi_{int}$=28.1 mm, $\phi_{ext}$=38.1 mm, L=2,340 mm; section 2 - $\phi_{int}$=21.75 mm, $\phi_{ext}$=31.75 mm, L=5,660 mm; section 3 - $\phi_{int}$=19.05 mm, $\phi_{ext}$=9.05 mm, L=2,536 mm.  These values were based on the values of diameters and lengths of piccolo's tubes of Embraer's Phenom and Legacy.

In each section a number of exhaust holes were considered in the tubes to estimate the total exhaust area for each section.

Considering the tube's material as being Titanium (density, 4.11 g/c$m^{3}$) the masses for each section are 5.0, 9.8 and 2.4 kg. The mass of leading edge was estimated by the Aeronautical Team and a value of 130 kg was adopted. Then, considering each section with 1/3 of the total leading edge mass, the value used for each section was 43.3 kg.

With all these parameters the model has been elaborated and the simulation has been successfully run. The results are shown below.

\begin{figure}[H] % Example image
    \center{\includegraphics[width=400px]{Pictures/ECS/AntiiceFigure6.eps}}
    \caption{Results of AMESim simulation - Wing leading edge temperature}
    \label{fig:ResAMESimWingLeadEdge}
\end{figure}

In an attempt of increasing the AIV operating temperature in order to reduce the mass flow, a new simulation has been performed considering an operating bleed temperature of 299$\celsius$.

With these results, an analysis was made to verify how much the AIV operating bleed temperature could be increased to reach the necessary energy demanded by the system in an icing condition. Considering the bleed temperature increase up to 299$\celsius$  the minimum airflow required to the system is 47.9 ppm and the leading edge mean temperature of 123$\celsius$. It leads us to a reduction of 3.0 \% in the airflow demand to the wing A/I.

Despite this reduction the increase of AIV operating temperature requires from the bleed system, temperatures already in the range that the system operates nowadays, what means that this reduction can just be called "variation" and this variation already occurs in the actual system once that it operates between 266 and 299$\celsius$. This way, to reach a real reduction of airflow with a temperature increase it's necessary to elevate the temperature above 299$\celsius$  what become unfeasible because it can affect the system components reliability (higher temperatures can deteriorate the life cycle of the components).

Furthermore the A/I system is rarely activated in the aircraft normal cycles. Then the modification of the A/I operating logic is something that cannot be justified to change and as a conclusion the system will keep the same.
