As was mentioned in Conceptual Studies Report, the pneumatic system is responsible to provide the air with controlled pressure and temperature to engine starter, air conditioning and pressurization system and Icing and Rain Protection system (wing leading edge, horizontal stabilizer and engine nacelles). In order to supply all the pressurized and heated air required by others Environmental Control Subsystems, the pneumatic system has a distribution system composed by duct, valves and connections dimensioned according the subsystems needs.


\begin{figure}[H] % Example image
\center{\includegraphics[width=400px]{Pictures/ECS/PneumaticConection.eps}}
\caption{ERJ-145 Pneumatic System}
\label{fig:PneumaticConection}
\end{figure}

Obs.: More details, as the engine-pneumatic system connection, can be found in Conceptual Studies Report.

The pneumatic distribution lines are thermally insulated , in order to avoid that the high temperature could affect others systems at the same environment, as flammable fluids and electrical/electronics equipments.
 In order to absorb the possible movement between the airframe and engine, and allow the thermal expansion as well, some flexible joints are installed in bleed distribution line. All flexible joints have a thermal switch associated, with the purpose of offering a leakage detection of duct connection. Those switches are more or less densely installed according to the rotor non-contained zone. Once one of those switches are activated an input signal is sent to EICAS and a warning message is shown.


\begin{figure}[H] % Example image
\center{\includegraphics[width=200px]{Pictures/ECS/leakswitcheslocation.eps}}
\caption{Leak Switches Location}
\label{fig:leakswitcheslocation}
\end{figure}

Currently all the monitored joints are only classified in three parts, as shown in EICAS messages: <BLD 1 LEAK> (corresponding to all odds joints), <BLD 2 LEAK> (joints 2 to 26) and <BLD APU LEAK> (28, 30 and 32 joints). This classification is not enough to inform the exact location of a possible leakage in bleed ducts, what make a maintenance team spending time in looking for the leakage position, increasing the access time, witch is responsible for 72,6\% of all maintenance lead time.
The purpose of this modifications is increase the number of this parts classifications from 3 to 12 parts with their respective messages on the EICAS system. This will improve the access time, once the messages will have more accurate information about the failure, and this can eliminate some steps on troubleshooting.

The new set of joints was chosen by its proximity and/or by which have the same inspection window for two or more joints, what will make the check easier.
Following this principle the new grouping of monitored joints is as folow:

<BLD 1 LEAK> (Joints 1, 3, 5, 7 and 9)

\begin{figure}[H] % Example image
\center{\includegraphics[width=200px]{Pictures/ECS/joint5.eps}}
\caption{Joint 1 (Access 195 AL), Joint 3 (Access 195 CL)}
\label{fig:joint5}
\end{figure}


\begin{figure}[H] % Example image
\center{\includegraphics[width=200px]{Pictures/ECS/joint1.eps}}
\caption{Joint 5 (Access 193 GR)}
\label{fig:joint1}
\end{figure}


\begin{figure}[H] % Example image
\center{\includegraphics[width=200px]{Pictures/ECS/joint7.eps}}
\caption{Joint 7 (Access 193 AL), Joint 9 (Access 193 JR)}
\label{fig:joint7}
\end{figure}
<BLD 3 LEAK> (Joints 11, 13 and 15)


\begin{figure}[H] % Example image
\center{\includegraphics[width=200px]{Pictures/ECS/joint11.eps}}
\caption{Joints 11, 13 and 15 (Access 271 AF)}
\label{fig:joint11}
\end{figure}
<BLD 4 LEAK> (Joints 12, 14 and 16)


\begin{figure}[H] % Example image
\center{\includegraphics[width=200px]{Pictures/ECS/joint12.eps}}
\caption{Joints 12, 14 and 16 (Access 271 AF)}
\label{fig:joint12}
\end{figure}

<BLD 5 LEAK> (Joints 17 and 19)


 \begin{figure}[H] % Example image
\center{\includegraphics[width=200px]{Pictures/ECS/joint17.eps}}
\caption{Joint 17 (Access 271 DLW) Joint 19 (Access 271 CLW)}
\label{fig:joint17}
\end{figure}

<BLD 6 LEAK> (Joints 20 and 22)

\begin{figure}[H] % Example image
\center{\includegraphics[width=200px]{Pictures/ECS/joint20.eps}}
\caption{Joint 20 (Access 272 DRW) Joint 22 (Access 271 ERW)}
\label{fig:joint20}
\end{figure}

 <BLD 7 LEAK> (Joints 23, 25 and 27)



 \begin{figure}[H] % Example image
\center{\includegraphics[width=200px]{Pictures/ECS/joint23.eps}}
\caption{Joints 23, 25 and 27 (Access 272 DR)}
\label{fig:joint23}
\end{figure}

 <BLD 8 LEAK> (Joint 26)
<BLD 9 LEAK> (Joint 29)



\begin{figure}[H] % Example image
\center{\includegraphics[width=200px]{Pictures/ECS/322AL.eps}}
\caption{Joint 29 (Access 322 AL)}
\label{fig:322AL}
\end{figure}



<BLD 10 LEAK> (Joint 24)
<BLD 11 LEAK> (Joint 21)
<BLD APU LEAK> (Joints 28, 30 and 32)

This way, the tasks where the search of leaking joints are needed will be minimized, once that the new messages will inform more accurate locations. This way only 3 access windows will be needed to reach the leaking point of pneumatic system that need to be analyzed.

\textbf{Installation}

In this modification there will be no changes on switches part numbers, responsible for signal joints leakages. The modification will be a logical alteration in the way how these switches are connected to each other. Currently they are connected in parallel between the DAU 2 (Data Acquisition Unit 2) from avionic system to ground, where, once a switch is set, a corresponding ground signal to DAU is sent. In the new configuration these parallel connections will be formed by a smaller number of same switches sending more messages of leakage events to a greater number of DAU inputs.


\textbf{Modification Goal.}

The average time for this maintenance task, without changes, is 8 hours of two technician's work. With the new bleed monitoring system configuration, the time planned will decrease to 4 hours of only one technician's work. This time decreasing will result in a 75\% reduction of the DMC of this maintenance activity, with the possibility to transfer this task to an overnight maintenance routine.





