During the Preliminary Studies phase the Environmental Control System team was concerned to develop, as main modifications, 3 alterations aiming the reduce of fuel consumption (Anti-Icing Operation Logic Control System, Bleed Temperature Control and Cabin Temperature Control) and one modification aiming the maintenance cost reduction (Bleed Leaking Monitoring System).

The fuel consumption modifications implementation feasibility analysis and validation, had its behavior modeled and simulated by AMESim$\copyright$, a specific software intended for modeling purposes. For the Bleed Leakage Monitoring System modification was analyzed how its maintenance routine is currently made e through a comparison with current architecture decisions were made in order to minimize the task time.

For Bleed Temperature Control was analyzed though its AMESim$\copyright$ model, that the inclusion of such control will not bring any fan air flow reduction, not bringing any benefits consequently for fuel economy, and if consider a maintenance benefit of this modification, the costs are higher than return in DMC costs reduction.

For Anti-Icing Operation Logic Control the conclusion arrived at through analysis of modeling data was that type of control, by air flow, provides a little economy, comparing with the modification investment, even more the case of system where its functioning is in only a short period of time of all aircraft flight.

In Cabin Temperature Control the modeling could bring important information about the quantity of  cooling air is necessary to maintain a comfortable cabin, even with passenger increasing, and could confirm that, together with other modification in electrical and interior systems, the air conditioning pack update will not be necessary and the current system is capable to provide the appropriated level of comfort and safety systems cooling in new aircraft configuration. 
As a Bleed Leakage Monitoring System, was achieved a reduction of 75\% of task runtime, comparing with the current configuration, what gives the possibility to include this task in an overnight maintenance routine, eliminating a lot of disorders, related to reallocation and canceling flight by the aircraft operator.

In summary, the Preliminary Studies made in Environmental Control System, is as a decision not to include the modifications aiming the fuel consumption reduction Bleed Temperature Control and Anti-Icing Operation Logic Control, due its implementations costs will be higher than return in benefits, avoiding the AMS Controller inclusion needs,  that will not be necessary any air conditioning pack update and that the updating of Bleed Leakage Monitoring System will bring maintenance returns that can offset the initial investment.
