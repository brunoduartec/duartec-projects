\subsubsection{Planning}
The main propose of the mockup it is concept validations, so its fast prototyping it is indispensable. 
Its structure in 2 weeks, was built in paperboard and covered with a brown paper to bring the feeling of being inside a 145.
The represented section of the airplane selected was the one in front of the passenger door where lies the new 2 seats taking place of some galleys and wardrobes.
The mockup doesn't have the pretension to faithfully represent an airplane but just simulate situations and enable a immersive analysis.

\subsubsection{Validation}
The decisions made by the interior team was greatly based on validation tests made in the mockup taking into consideration subjective aspects such as comfort and ergonomics.

\subsubsection{Pitch}

\begin{figure}[H] % Example image
\center{\includegraphics[width=200px]{Pictures/Interior/standard.eps}}
\caption{Standard Seat.}
\label{fig:standardseat}
\end{figure}

\begin{figure}[H] % Example image
\center{\includegraphics[width=200px]{Pictures/Interior/slim_adaptada.eps}}
\caption{Slim Seat adapted.}
\label{fig:adaptedseat}
\end{figure}




\subsubsection{Galley size and position}
As cited in Galleys section we determined the volume of the galley we must have, but to determine its position and how the passengers and COMISSARIAS would be, some ergonomics studies were made.


\begin{figure}[H] % Example image
\center{\includegraphics[width=200px]{Pictures/Interior/galley_position_validation_solo.eps}}
\caption{Galley position.}
\label{fig:galleypos_solo}
\end{figure}

To validate the service with the hall-trolley in the new LOPA configuration, a maneuver simulation was made.

\begin{figure}[H] % Example image
\center{\includegraphics[width=400px]{Pictures/Interior/galley_position_validation.eps}}
\caption{Galley position service simulation}
\label{fig:galleypos_service}
\end{figure}



\subsubsection{Wardrobe}
The wardrobe and space for a new half-trolley was simulated to verify cabin comfort

\begin{figure}[H] % Example image
\center{\includegraphics[width=200px]{Pictures/Interior/newtrolley.eps}}
\caption{ Wardrobe and trolley location}
\label{fig:newtrolley}
\end{figure}

To place electrical systems in a place where the maintenance can be made in few time, an electrical rack mockup was made, simulating the most common electrical systems, such as TCAS, GPS and the new starter/generator control unit (AGCU), and a sidewall inspection door was made to simulate the cabling installation and maintenance.


\subsubsection{Conclusion}
The mockup as a engineering tool was a very useful to validate subjective aspects and noticed new problems that even the more complex and well conceived draw or sketch would be easy to noticed.
The use of low cost material such as paper and styrofoam without concerning to the appeal , brings to the mockup user a big agility of decision, with only the necessary characteristics to take decisions.




