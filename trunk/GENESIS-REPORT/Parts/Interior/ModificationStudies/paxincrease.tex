% 5.2.1 - PAX Increase
Increasing the number of passenger is one of the main modifications and to achieve that, it is necessary to change the LOPA (Sigla!!) and the aircraft's structure. To decide the size of the fuselage, considering the replacement of the seat by one of 29" instead of the 31" standard seat, considering the choice among insert the fuselage in front of or behind the emergency exit, considering that to avoid the airplane stability getting worst the weight increase must happen below the CG and to reduce the drag and weight impact the fuselage increase must be as shorter as it can. Considering all this factors, a fuselage ring of 1.06 m (41.7 inches), the minimum necessary to accommodate 2 more row of seats, was chosen.


The reduction in the pitch size also reduces the comfort comparing to the original pitch size, therefore the aircraft's seats must also be modified. After performing some research about the current ERJ-145 operation, the seats are comfortable and there isn't any complaint about the pitch size. This information brings the opportunity to change this pitch size.

The new seat chosen is the new generation of Embraer's seats, under its specification and developed by C\&D with some minors changes discussed in the Seat section galley.\\

\textbf{LOPA}
\\

To determine the LOPA there are some preview considerations that must be taken in account.

In every configuration of the 145, there are 6 galley distribution
\begin{figure}[H] % Example image
\center{\includegraphics[width=400px]{Pictures/Interior/InteriorArrangements.eps}}
\caption{Galleys distribuctions}
\label{fig:galleysdistribuction}
\end{figure}

As the study shown in the galley section, the number of galleys can be reduced, so the space occupied can be used to place new seats, as shown in the image \ref{fig:frontlopa}


\begin{figure}[H] % Example image
\center{\includegraphics[width=300px]{Pictures/Interior/front_lopa_seats.eps}}
\caption{Front LOPA new arrangement}
\label{fig:frontlopa}
\end{figure}

In the region in front of the service door there is a double seat with VLAUS inches of pitch.To check if the seats would be comfortable, and if the galleys can be properly used, that section was built at the mockup, and some tests were made and recorded as pictures and video, showing how the passengers would feel, and how the flight atendants would use the galleys. These tests are best described in the mockup section.
The figure \ref{fig:LOPA} shows the LOPA, emphasizing the regions where new passengers were added.

\begin{itemize}
  \item 6 below the windows
  \item 2 in front of the service door
  \item 2 in front of the passenger door
\end{itemize}

\begin{figure}[H] % Example image
\center{\includegraphics[scale = 0.6]{Pictures/Interior/LOPA.eps}}
\caption{LOPA configuration}
\label{fig:LOPA}
\end{figure}

Increasing the pax number, the amount of overhead bins was increased to keep the same internal volume per passenger as it is now, conserving the same autonomy to the passenger.In the new ring of fuselage, there will be Overhead bins for the extra passengers in that area. Also, new oberhead bins will be placed in the area of the two new seats in the front of the service door. Hence, the capability of hand lugage for each passenger will not be reduced.


\textbf{Seat}
\\
To be possible to increase the number of passengers, it is necessary to change the aircraft's LOPA. Besides resizing the galley, reducing the size of pitch from 31 inches to 29 inches is also needed to achieve the project's goal. One concern the fact that decreasing the pitch can also bring discomfort to the passengers, so the solution was the choice for a new seat that provides an enjoyable experience for the passengers even reducing the pitch size.

\begin{figure}[H]
\center{\includegraphics[width=200px]{Pictures/Interior/SlimSeat_Isometric.eps}}
\caption{Galleys distribuctions}
\label{fig:slimseatisometric}
\end{figure}

The C\&D Zodiac, under Embraer's specification, has developed a new seat, known as Slim Seat, as seen in Figure \ref{fig:slimseatisometric} and Figure \ref{fig:slimseatside}, which has certain characteristics what matters to the EMB-145's modernization, such as 2.5 kg less in each dual seat configuration in comparison to the original version, called Elite, which have 24 kg.

\begin{figure}[H]
\center{\includegraphics[width=200px]{Pictures/Interior/SlimSeat_Side.eps}}
\caption{Galleys distribuctions}
\label{fig:slimseatside}
\end{figure}

The Slim Seat was initially developed for the ERJ-190's aircraft family, so some changes are necessary to proper compatibility with the EMB-145. These changes are summarized in reducing the width of the seat and the side rail's fixation. Fortunately, all these modifications have been made and, therefore, it is not necessary to redesign these seats.


Another important feature is that the new design allow to transmit an actual pitch's size one more inch bigger than the real pitch's size, so the 29 inches pitch chose indeed seems as a 30 inches pitch. The Figure \ref{fig:slimelitecomparison} illustrates the comparison between the original seat and new one. As can be seen, there is no big difference in the passenger's space. Also, the mockup bilt was used to make some tests os the pich and seat sizes, and confirm the choice. These tests are best described at the mockup section. '

\begin{figure}[H]
\center{\includegraphics[width=400px]{Pictures/Interior/EliteVsSlim_Comparison.eps}}
\caption{Comparison between Elite and Slim seats}
\label{fig:slimelitecomparison}
\end{figure}


\textbf{Galleys}
Changing the current EMB-145's galley is an important aspect in the proposed modernization. As seen in EMB-145 Airport Planning Manual (REFERENCE), there are 6 different interiors arrangements on EMB-145, as shown in

\begin{figure}[H]
\center{\includegraphics[width=400px]{Pictures/Interior/InteriorArrangements.eps}}
\caption{EMB-145 Interior Arrangements-Optional Configuration}
\label{fig:InteriorArrangements}
\end{figure}


% Table generated by Excel2LaTeX from sheet 'Sheet1'
\begin{table}[htbp]
  \centering
  \caption{EMB-145 Interior Arrangements-Option Configuration Specifications}
    \begin{tabular}{ccccccc}
    \toprule
    \textbf{ITEMS} & \textbf{OPTION 1} & \textbf{OPTION 2} & \textbf{OPTION 3} & \textbf{OPTION 4} & \textbf{OPTION 5} & \textbf{OPTION 6$^{a}$} \\
    \midrule
    PASSENGERS & 50    & 50    & 50    & 49    & 48    & 50 \\
    HALF TROLLEY & 3     & 4     & 6     & 6     & 6     & - \\
    2/3 HALF TROLLEY & -     & -     & -     & -     & -     & 6 \\
    GALLEY VOLUME (S) m$^{3}$ (cu.ft) & 1.27 (44.7) & 1.49 (52.7) & 2.42 (85.4) & 2.42 (85.4) & 2.42 (85.4) & 2.42 (85.4) \\
    STOWAGE VOLUME (S) m$^{3}$ (cu.ft) & 0.17 (6.0) & -     & -     & -     & -     & 0.69 (24.6) \\
    WARDROBE VOLUME (W) m$^{3}$ (cu.ft) & 0.93 (32.9) & 0.93 (32.9) & -     & 0.57 (20.2) & 1.07 (37.6) & 1.12 (39.6) \\
    \bottomrule
    \end{tabular}%
  \label{tab:InteriogArrangementOption}%
\end{table}%

Despite all these different arrangements, it is necessary to quantify what are the commonly used by the EMB-145's operators. To that end, the arrangements used by the operators need to be quantified.


% Table generated by Excel2LaTeX from sheet 'Sheet1'
\begin{table}[htbp]
  \centering
  \caption{Arrangements-Option chosen by Operators}
    \begin{tabular}{rcccccc}
    \toprule
    \multicolumn{1}{c}{\multirow{2}[2]{*}{\textbf{ITEM}}} & \multirow{2}[2]{*}{\textbf{QUANTITY}} & \textbf{AVERAGE VOLUME} & \textbf{AVERAGE CAPACITY} & \multirow{2}[2]{*}{\textbf{NUMBER OF OPERATORS}} & \multirow{2}[2]{*}{\textbf{TOTAL OF OPERATORS}} & \multirow{2}[2]{*}{\textbf{\%}} \\
    \midrule
    \multicolumn{1}{c}{} &       & \textbf{m$^{3}$ (cu.ft)} & \textbf{kg (lb)} &       &       &  \\
    \multicolumn{1}{c}{\textbf{WARDROBE}} & 0     & -     & -     & 98    & 378   & 25.9 \\
    \multicolumn{1}{c}{\textbf{}} & 1     & 0.88 (31.1) & 69.9 (154) & 280   &       & 74.1 \\
    \multicolumn{1}{c}{\textbf{GALLEY}} & 1     & -     & 114  (251) & 17    & 378   & 4.5 \\
    \multicolumn{1}{c}{\textbf{}} & 2     & -     & 224 (494) & 230   &       & 60.8 \\
    \multicolumn{1}{c}{\textbf{}} & 3     & -     & 344 (758) & 131   &       & 34.7 \\
    \multicolumn{1}{c}{\textbf{STOWAGE}} & 0     & -     & -     & 238   & 378   & 63 \\
    \textbf{} & 1     & 0.12 (4.2) & 14.8 (32.6) & 105   &       & 27.8 \\
    \textbf{} & 2     & 0.44 (15.5) & 64.2 (141) & 29    &       & 7.7 \\
    \textbf{} & 3     & 0.53 (18.7) & 147 (324) & 6     &       & 1.6 \\
    \bottomrule
    \end{tabular}%
  \label{tab:OpratorArrangementOption}%
\end{table}%



The Table \ref{tab:OpratorArrangementOption} contains a compilation of the operator's choices about arrangements. As can be seen 74.1 \% of the operator uses wardrobe on the LOPA configuration. Therefore, that is probably an important item on the choice. Considering galley, 60.8 \% of the operators use only 2 galleys and 34.7 \% use 3, with an average capacity of 224 kg (493.8 lb) and 344 kg (758.4 lb) respectively.


From these data, it is estimated that just 2 galleys with 224 kg (494 lb) of capacity would be sufficient to satisfy most of the operators, in view of the increasing number of passengers in return of reducing the galley allows lowering the CASM. Thus, the new LOPA arrangement bring important benefits for the cost of maintenance of the aircraft.


	Therefore will be removed all old galleys and will be installed just one galley in ERJ 145, that is the same galley that is used on EMB 120. This galley meets all necessities of operators. As well as in EMB 120 one trolley will be placed in this galley and the other trolley will stay in the compartment in the forward right side of cabin of passenger, near to the wall of the cockpit. This galley preserves all functionalities of EMB 120 galley providing the necessary support at operations in flight time and passenger capacity of ERJ 145. Some views of galley with its compartments are illustrated in the figure \ref{fig:InteriorArrangements} and figure \ref{fig:viewhalftrolley}.

\begin{figure}[H]
\center{\includegraphics[width=400px]{Pictures/Interior/galley_trolley.eps}}
\caption{View of galley with its lockers, hot jugs and half trolley of its side.}
\label{fig:InteriorArrangements}
\end{figure}



\begin{figure}[H]
\center{\includegraphics[width=400px]{Pictures/Interior/viewhalftrolley.eps}}
\caption{View of half trolley inside of galley}
\label{fig:viewhalftrolley}
\end{figure}

	To determine if the pitch for the passengers sitting right in front of the galley and the foward compartiment is comfortable, a mockup study was made. This pitch must be a bit higher, as ther will be a wall in front of the passenger, and not a seat. Twenty tree persons was told to seat in front of a wall, and adjust the pitch so they could be confortable. Their hights vary from 1.53 m to 1.94 m, covering a high range of heights. The medium desired pitch was of 33.88 inches, and the medium pitch plus one standard deviation is of 36 inches. This value would completely satisfy 84.1 /% of the population, based on our studies. This proves that the existing pitch is enought for this special case seats.




\textbf{Stowage}
In the galley there is space to store just one trolley. Thus, was necessary to create other compartment on the forward right side of cabin for to store this another trolley. In this compartment there will be space for store some clothes, one half trolley and some avionics boxes. This compartment will facilitate the maintenance of avionics devices in electronic bay, because will allow access to electronics through one removable door on front passenger seat, as shown in figure \ref{fig:viewhalftrolley}, and through of racks that can be pulled for pick up electronic equipments like the hardware of TCAS or of GPS, as shown in figure \ref{fig:MFS2}.  As it can seen in figure \ref{fig:viewhalftrolley}, this compartment is composed by one superior space that works like a wardrobe, and one inferior space for store the half trolley. The inferior space has one false back wall that can be disconnected for to access the avionics boxes through of racks, as can seen in figure 05.



\begin{figure}[H]
\center{\includegraphics[width=400px]{Pictures/Interior/mfs2.eps}}
\caption{Rack pulled for to have access to hardware of TCAS.}
\label{fig:MFS2}
\end{figure}

For to realize maintenance on avionics boxes that stay stored in this compartment, it's necessary to take out the half trolley from inside of compartment, disconnect the false back wall from inferior space and the wall on front passenger seat. After it's necessary unplug cables of avionics boxes, as can seen in figure 06, and to pull the racks where are the avionics boxes.
\begin{figure}[H]
\center{\includegraphics[width=400px]{Pictures/Interior/mfs3.eps}}
\caption{View of avionic box plugged at cable and how to access this box.}
\label{fig:MFS3}
\end{figure}


\textbf{Oxygen Masks}

By regulation the number of oxygen masks must exceed in 10 percent the number of seats, so in the new airplane configuration its number must be increased by 6 to reach the number of 66 oxygen masks without counting the passenger oxygen adds.
For those we must add the extra masks:

\begin{itemize}
  \item 3 triple masks distributed homogeneously in the cabin
  \item 1 on the toilet
  \item 2 to the flight attendant
\end{itemize}

\textbf{Certification}
The ERJ 145 like showed in the figure 07 has two exits type I and two type III. Thus, increasing the number of passenger, there is no necessity in to increase or put more emergency exits, because according with FAR Part 25 Sec 25.807, that treats about emergency exits, the quantity, disposition and types of emergency exits, that already exist on ERJ 145 satisfy these requirements.

\begin{figure}[H]
\center{\includegraphics[width=400px]{Pictures/Interior/145-exits.eps}}
\caption{Top view with the types of exits of ERJ 145.}
\label{fig:145-exits}
\end{figure}


Below, are transcribed some paragraphs from Sec 25.807 that show how the ERJ 145 attend the requirements, even increasing to 10 passenger. The figure  show how the new Lopa attends at requirements of item 04 of Sec 25.807.

\textbf{Emergency exits}

(4) For an airplane that is required to have more than one passenger emergency exit for each side of the fuselage, no passenger emergency exit shall be more than 60 feet from any adjacent passenger emergency exit on the same side of the same deck of the fuselage, as measured parallel to the airplane's longitudinal axis between the nearest exit edges.


(g) \emph{Type and number required}. The maximum number of passenger seats permitted depends on the type and number of exits installed in each side of the fuselage. Except as further restricted in paragraphs (g)(1) through (g)(9) of this section, the maximum number of passenger seats permitted for each exit of a specific type installed in each side of the fuselage is as follows:

Type A 110\\
Type B 75\\
Type C 55\\
Type I 45\\
Type II 40\\
Type III 35\\
Type IV 9\\

(5) For a passenger seating configuration of 41 to 110 seats, there must be at least two exits, one of which must be a Type I or larger exit, in each side of the fuselage.


\begin{figure}[H]
\center{\includegraphics[width=400px]{Pictures/Interior/lopameasuresevacuation.eps}}
\caption{New lopa with measures that attend at item 04 - Sec 25.807}
\label{fig:lopameasuresevacuation}
\end{figure}


The evacuation time in the certification of ERJ 145 is 85 seconds. Hence, might be necessary to do another evacuation test or combination of analysis and testing with this new configuration of 60 passenger, for to proof that is possible evacuate all passenger and crew members in 90 seconds, attending the Sec 25.803 that is transcribed below.

(c) For airplanes having a seating capacity of more than 44 passengers, it must be shown that the maximum seating capacity, including the number of crewmembers required by the operating rules for which certification is requested, can be evacuated from the airplane to the ground under simulated emergency conditions within 90 seconds. Compliance with this requirement must be shown by actual demonstration using the test criteria outlined in Appendix J of this part unless the Administrator finds that a combination of analysis and testing will provide data equivalent to that which would be obtained by actual demonstration.






