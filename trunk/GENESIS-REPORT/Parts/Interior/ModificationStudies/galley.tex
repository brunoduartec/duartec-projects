Changing the current EMB-145's galley is an important aspect in the proposed modernization. As seen in EMB-145 Airport Planning Manual (REFERENCE), there are 6 different interiors arrangements on EMB-145, as shown in

\begin{figure}[H]
\center{\includegraphics[width=400px]{Pictures/Interior/InteriorArrangements.eps}}
\caption{EMB-145 Interior Arrangements-Optional Configuration}
\label{fig:InteriorArrangements}
\end{figure}


% Table generated by Excel2LaTeX from sheet 'Sheet1'
\begin{table}[htbp]
  \centering
  \tiny
  \caption{EMB-145 Interior Arrangements-Option Configuration Specifications}
    \begin{tabular}{ccccccc}
    \toprule
    \textbf{ITEMS} & \textbf{OPTION 1} & \textbf{OPTION 2} & \textbf{OPTION 3} & \textbf{OPTION 4} & \textbf{OPTION 5} & \textbf{OPTION 6$^{a}$} \\
    \midrule
    PASSENGERS & 50    & 50    & 50    & 49    & 48    & 50 \\
    HALF TROLLEY & 3     & 4     & 6     & 6     & 6     & - \\
    2/3 HALF TROLLEY & -     & -     & -     & -     & -     & 6 \\
    GALLEY VOLUME (S) m$^{3}$ (cu.ft) & 1.27 (44.7) & 1.49 (52.7) & 2.42 (85.4) & 2.42 (85.4) & 2.42 (85.4) & 2.42 (85.4) \\
    STOWAGE VOLUME (S) m$^{3}$ (cu.ft) & 0.17 (6.0) & -     & -     & -     & -     & 0.69 (24.6) \\
    WARDROBE VOLUME (W) m$^{3}$ (cu.ft) & 0.93 (32.9) & 0.93 (32.9) & -     & 0.57 (20.2) & 1.07 (37.6) & 1.12 (39.6) \\
    \bottomrule
    \end{tabular}%
  \label{tab:InteriogArrangementOption}%
\end{table}%

Despite all these different arrangements, it is necessary to quantify what are the commonly used by the EMB-145's operators. To that end, the arrangements used by the operators need to be quantified.


% Table generated by Excel2LaTeX from sheet 'Sheet1'
\begin{table}[htbp]
  \centering
  \tiny
  \caption{Arrangements-Option chosen by Operators}
    \begin{tabular}{rcccccc}
    \toprule
    \multicolumn{1}{c}{\multirow{2}[2]{*}{\textbf{ITEM}}} & \multirow{2}[2]{*}{\textbf{QUANTITY}} & \textbf{AVERAGE VOLUME} & \textbf{AVERAGE CAPACITY} & \multirow{2}[2]{*}{\textbf{NUMBER OF OPERATORS}} & \multirow{2}[2]{*}{\textbf{TOTAL OF OPERATORS}} & \multirow{2}[2]{*}{\textbf{\%}} \\
    \midrule
    \multicolumn{1}{c}{} &       & \textbf{m$^{3}$ (cu.ft)} & \textbf{kg (lb)} &       &       &  \\
    \multicolumn{1}{c}{\textbf{WARDROBE}} & 0     & -     & -     & 98    & 378   & 25.9 \\
    \multicolumn{1}{c}{\textbf{}} & 1     & 0.88 (31.1) & 69.9 (154) & 280   &       & 74.1 \\
    \multicolumn{1}{c}{\textbf{GALLEY}} & 1     & -     & 114  (251) & 17    & 378   & 4.5 \\
    \multicolumn{1}{c}{\textbf{}} & 2     & -     & 224 (494) & 230   &       & 60.8 \\
    \multicolumn{1}{c}{\textbf{}} & 3     & -     & 344 (758) & 131   &       & 34.7 \\
    \multicolumn{1}{c}{\textbf{STOWAGE}} & 0     & -     & -     & 238   & 378   & 63 \\
    \textbf{} & 1     & 0.12 (4.2) & 14.8 (32.6) & 105   &       & 27.8 \\
    \textbf{} & 2     & 0.44 (15.5) & 64.2 (141) & 29    &       & 7.7 \\
    \textbf{} & 3     & 0.53 (18.7) & 147 (324) & 6     &       & 1.6 \\
    \bottomrule
    \end{tabular}%
  \label{tab:OpratorArrangementOption}%
\end{table}%



The Table \ref{tab:OpratorArrangementOption} contains a compilation of the operator's choices about arrangements. As can be seen 74.1 \% of the operator uses wardrobe on the LOPA configuration. Therefore, that is probably an important item on the choice. Considering galley, 60.8 \% of the operators use only 2 galleys and 34.7 \% use 3, with an average capacity of 224 kg (493.8 lb) and 344 kg (758.4 lb) respectively.


From these data, it is estimated that just 2 galleys with 224 kg (494 lb) of capacity would be sufficient to satisfy most of the operators, in view of the increasing number of passengers in return of reducing the galley allows lowering the CASM. Thus, the new LOPA arrangement bring important benefits for the cost of maintenance of the aircraft.


	Therefore will be removed all old galleys and will be installed just one galley in ERJ 145, that is the same galley that is used on EMB 120. This galley meets all necessities of operators. As well as in EMB 120 one trolley will be placed in this galley and the other trolley will stay in the compartment in the forward right side of cabin of passenger, near to the wall of the cockpit. This galley preserves all functionalities of EMB 120 galley providing the necessary support at operations in flight time and passenger capacity of ERJ 145. Some views of galley with its compartments are illustrated in the figure \ref{fig:InteriorArrangements} and figure \ref{fig:viewhalftrolley}.

\begin{figure}[H]
\center{\includegraphics[width=400px]{Pictures/Interior/galley_trolley.eps}}
\caption{View of galley with its lockers, hot jugs and half trolley of its side.}
\label{fig:InteriorArrangements}
\end{figure}



\begin{figure}[H]
\center{\includegraphics[width=400px]{Pictures/Interior/viewhalftrolley.eps}}
\caption{View of half trolley inside of galley}
\label{fig:viewhalftrolley}
\end{figure}

	To determine if the pitch for the passengers sitting right in front of the galley and the forward compartment is comfortable, a mockup study was made. This pitch must be a bit higher, as there will be a wall in front of the passenger, and not a seat. Twenty tree persons was told to seat in front of a wall, and adjust the pitch so they could be comfortable. Their heights vary from 1.53 m to 1.94 m, covering a high range of heights. The medium desired pitch was of 33.88 inches, and the medium pitch plus one standard deviation is of 36 inches. This value would completely satisfy 84.1 \% of the population, based on our studies. This proves that the existing pitch is enough for this special case seats.


