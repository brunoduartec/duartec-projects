% 5.2.3 - LED
The replacement of the lightning of the aircraft from halogen and fluorescent lamps to LED technology was a decision made during the previous phase of the project, due to benefits in maintenance, power consumption and heat transference. Now, further studies have shown what are the lamps to be replaced, where they would be placed, how much power they would save, and how much less heat they would put in the aircraft. These studies are described in this section of the report.

To decide which are the lamps to be replaced, an analysis of the reduction in maintenance costs compared to the costs of the replacement for each component was made. The results of this studies are in the maintenance section. From that, the lights that will be changed are the passenger cabin wash lights, the cockpit reading and dome lights, in the interior of the aircraft, and the landing, taxi, and anti collision lights, on the outside of the aircraft.

Despite that the gain in maintenance for the other interior lights of the aircrafts do not overcome the costs of the replacement, these lights will also be changed, due to the reduction of the heat emitted by them, to balance the heat emitted by the 10 extra passengers. These lights are the passenger cabin dome and reading lights and the cockpit anti-glare lights.

The lights that will remain unchanged are the emergency, service and stowage lights, along with the navigation, inspection, and logo lights.

The final cost of all the replacement material is of US\$ 48,095.00. The concurrent cost, that includes manufacturing and installation, is of US\$ 10,400.00, and the non-concurrent cost, that includes engineering, development, and certification makes a total of US\$ 24,000.00.

% ---------- Position ---------- %
\textbf{Position}

As the luminescence of the LED lamps will be the same as it is to the Halogen and Fluorescent ones, the position of the lamps will not change, avoiding then costs with engineering development, and different kinds of installation, other than just the bulbs replacement.

The main change will be for the wash lights, that are longer than the nowadays fluorescent lights. Fewer lamps will then be necessary, and even fewer wiring. This is because each LED wash light can be connected directly to the end of the one before it, up to 8 ft of lightning, with just one power cable.

The images below show how will the lights be fixed in its places with its dimensions, and how will they be connected on each other.


\begin{figure}[H] % Example image
\center{\includegraphics[width=200px]{Pictures/Interior/WASH1.eps}}
\caption{Dimension and fixation of the LED wash lights}
\label{fig:wash1}
\end{figure}


\begin{figure}[H] % Example image
\center{\includegraphics[width=200px]{Pictures/Interior/WASH2.eps}}
\caption{Connection scheme for the LED wash lights}
\label{fig:wash2}
\end{figure}

All the other lights will be placed just as they are today, as a simple change between the bulbs today, and the LED lamps.

% ---------- Power Usage ---------- %
\textbf{Power Usage}

The LED technology consumes a lot less power than the current lightning of the ERJ-145. That is another great advantage of this modification. The table  shows a comparison between technologies.

% Table generated by Excel2LaTeX from sheet 'Sheet1'
\begin{table}[htbp]
  \centering
  \caption{Technology comparison}
    \begin{tabular}{rrrrrrrr}
    \toprule
    Sector & Type  & \# of LEDs & Power (W) & Total (W) & \# Today & Power (W) & Total (W) \\
    \midrule
    PAX Cabin & Wash Lights & 44    & 30    & 1320  & 72    & 20    & 1440 \\
    PAX Cabin & Dome Lights & 10    & 2.52  & 25.2  & 10    & 20    & 200 \\
    PAX Cabin & Reading Lights & 60    & 1.82  & 109.2 & 50    & 11.5  & 575 \\
    Cockpit & Dome Lights & 2     & 2.52  & 5.04  & 2     & 20    & 40 \\
    Cockpit & Reading Lights & 5     & 1.52  & 7.6   & 5     & 11.5  & 57.5 \\
    Cockpit & Anti-Glare & 3     & 8.9   & 26.7  & 3     & 15    & 45 \\
    Landing & Leading Edge & 2     & 75    & 150   & 2     & 450   & 900 \\
    Landing & NLG   & 1     & 75    & 75    & 1     & 600   & 600 \\
    Taxi  & Sealed Bean & 2     & 75    & 150   & 2     & 450   & 900 \\
    Anticolision  & Strobe Light & 3     & 28    & 84    & 3     & 150   & 450 \\
    Anticolision  & Red Beacon & 2     & 28    & 56    & 2     & 150   & 300 \\
    Total &       & 134   &       & 2008.74 & 152   &       & 5507.5 \\
    \bottomrule
    \end{tabular}%
  \label{tab:LEDTechCompare}%
\end{table}%


The table above shows that, if all lights were on the entire flight, the savings in power would reach 3.5 kW. Of course, this doesn't happen, but we can assume that the consumption of power, that represents about 20 \% of electricity consumption in flight, will be reduced by half.

This power saving does not reflect in a significant amount of fuel, but it can be used to other purposes, such as IFE. A more detailed analysis of power usage and savings appear on the electrical systems report.

% ---------- Thermal Tranference ---------- %
\textbf{Thermal Transference}

One big advantage is that the heat transferred do the cabin by the lights will reduce significantly, and the heat emitted by the ten extra passengers will be balanced. Therefore, no extra air conditioning will be needed. This is better explained in the AMS report. The table shows the power dissipation for the lights in the cabin.

% Table generated by Excel2LaTeX from sheet 'Sheet1'
\begin{table}[htbp]
  \centering
  \caption{Light Power Dissipation}
    \begin{tabular}{ccc}
    \toprule
    Lamps & LED   & Fluorescent \\
    \midrule
    Wash Lights & 1250 W & 2160 W \\
    Dome Lights & 25 W  & 200 W \\
    Reading Lights & 110 W & 575 W \\
    \bottomrule
    \end{tabular}%
  \label{tab:LightDissipation}%
\end{table}%


% ---------- Maintenance ---------- %
\textbf{Maintenance}

The analysis of the maintenance cost is a bit more complex. The Mean Time Between Failure (MTBF) of the lamps before and after the replacement was considered. The data was obtained from datasheets and manufacturers data.

To obtain a cost of maintenance per hour of flight, the MTBF, in hours, was divided by a factor that represents the percentage of flight that the light is on, obtaining a value in flight hours. Then, the cost of the replacement of each lamp is divided by this value in flight hours, obtaining a value in US dollars per flight hour.

Multiplying the value above for the number o lights for each kind of lamp, a total cost per hour is obtaining. Then, to obtain the costs for ten years, the value in multiplied by 2.500 (number of flight hours in a year), and then by 10. The costs for the Lamps nowadays and after the replacement are compared. If the difference is greater than the costs of the modification, there is advantage in trading for the LED technology.

The maintenance section of the report will bring more detail. The table  shows some of the results for ten years.

% Table generated by Excel2LaTeX from sheet 'Sheet1'
\begin{table}[htbp]
  \centering
  \caption{Light Maintenance}
    \begin{tabular}{rr}
    \toprule
    Material Costs & \$39,955.00 \\
    \midrule
    Concurrent Costs & \$10,400.00 \\
    Non Concurrent Costs & \$24,000.00 \\
    Costs Per Aircraft & \$50,475.00 \\
    Maintenance savings & \$69,935.00 \\
    Total Earnings & \$19,460.00 \\
    \bottomrule
    \end{tabular}%
  \label{tab:LightMaintenance}%
\end{table}%

% ---------- Certification ---------- %
\textbf{Certification}

Some certification issues must be considered in the replacement of the lightning of the aircraft. This issues were considered in the non concurrent costs, as some tests are required. This tests are photometric, glare of the pilot, taxi and landing tests, at night. This are not in flight tests, as they are made on ground, so the costs of this tests are not high. The Navigation lights require some angle of view tests, but this are going to be kept without change, therefore, there are no needs for this.




