By decisions taken at EC, began a survey to identify the characteristics and failure mode of the aileron damper. This research was accomplished because the study of maintenance report SPMR, revealed that this component had greatly decreased his MTBUR (Medium Time Before Unschedule Remove).

From that, we performed two conversations with Embraer engineers to understand more about the process. Through conversation with the engineer Pedro Faveret, it was found that the aileron failure was due aileron damper fitting. Also, for certain airplanes, the SPMR would have required replacing the rod end of the aileron damper assembly with an improved rod end. The SPMR resulted from reports of structural failure of the aileron damper rod end , which was caused by insufficient clearance between the lugs of the aileron damper fitting and the rod end of the aileron damper. The proposed actions were intended to prevent failure of the aileron damper, which could result in failure of the aileron actuator and consequent reduced controllability of the airplane.

Regarding the proposed solution was obtained the following results:

\begin{itemize}
\item The failures of the aileron damper rod ends that prompted the SPMR were discovered during inspections performed under these requirements.
\item According to EMBRAER Service Bulletin 145-00-0038 FAA-EASA, which the SPMR references as the appropriate source of service information for the required actions, was issued to correct insufficient clearance between the lugs of the aileron damper fitting and the rod end of the aileron damper. Through meetings with EMBRAER and have determined that the actions in that service bulletin are not intended to address an unsafe condition.
\end{itemize}

Doing those actions may provide an economic benefit to operators by preventing the need for an expensive repair in the event that damage is detected during routine inspections. Since there is no unsafe condition, the proposed modification is unnecessary.

Upon further consideration, it have determined that there is no unsafe condition associated with structural failure of the rod end of the aileron damper.
Withdrawal of the SPMR does not preclude the FAA from issuing another related action or commit the FAA to any course of action in the future.
This service bulletin, EMBRAER Service Bulletin 145-00-0038 FAA-EASA, describes procedures for removing an aileron damper and modifying the hydraulic system, among other actions.
The impact of this change will be only in reducing aileron maintenance and thus avoiding the maintenance due to aileron damper.
