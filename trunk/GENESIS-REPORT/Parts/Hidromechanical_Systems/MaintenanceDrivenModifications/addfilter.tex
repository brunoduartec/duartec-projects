By decisions taken at EC, it began a survey to identify the characteristics and mode of failure of the hydraulic pump. This research was accomplished because the study of maintenance report SPMR, revealed that this bomb had greatly decreased his MTBUR (Mean Time Before Unscheduled Remove).

From that, we performed two conversations with Embraer engineers to understand more about the process. Through conversation with the engineer Pedro Faveret, it was found that the pump failure was due to poor maintenance of the operator, who had just flown with the clogged filter. This caused the failure of the pump with the passage of time.
Were then generated three alternatives for the study:

\begin{itemize}
    \item Add another filter in line
    \item Add a message on EICAS
    \item Improve the time of the maintenance check to verify the optimal time in which change the filter compensates instead of changing the pump.
\end{itemize}

Then, with the meeting with mentor José Farat, these alternatives were analyzed to verify viability along certification and solution cost.
Regarding the proposed solution was obtained the following results:

\begin{itemize}
    \item Adding one more filter in line: adding a filter in the return line of the hydraulic system causes the change in function and size of the system, which implies to certify the entire hydraulic system again.
    \item Add a message on EICAS: adding a message on the EICAS would inform the pilot when taxiing aircraft. But the cost for software certification is very high (US\$ 200,000.00 per message). As the filter clogged warning isn't already present in CMC, and the clogged filter indication is often ignored by the operator's maintenance, the more interesting it would be to do an awareness campaign on the operators on the filter change.
    \item Improve the time of the maintenance check to verify the optimal time in which change the filter compensates instead of changing the pump: This modification is the one that generates the most cost-effective, because through price comparisons in cost between pump and filter, and their respective maintenance time, there is the maximum time between checks for performing maintenance of the filter.
\end{itemize}

Therefore, the best solution is to study the optimization of the cost of checks, and also conduct an awareness campaign about filter maintenance. This is because, since when doing more frequent maintenances of the filter can increase the cost of maintaining in the short-term but decreases the long-term maintenance, as the pump will not fail due to clogging of the filter
