The Wheel Speed Transducer is a sensor expected to output a signal corresponding to the wheel speed. It has an extreme importance because its data is used by the brake control unit to adjust the braking force of the tires without wheel skid or wheel block. A malfunction of this part may generate a brake failure. Each wheel has this part thus the airplane contain 4 sensors.
A failure analysis was conducted and the result shows that the responsible for the most failures is the mechanical connection between the axis of the sensor rotational part and the wheel cap. This wheel cap is a metal clip which uses the elasticity of the metal to plug the axis to the cap moving the rotational part of the sensor. The cap of this part is shown at the figure below.

\begin{figure}[H] % Example image
\center{\includegraphics[width=400px]{Pictures/Hidromechanical_Systems/wheelspeed1.eps}}
\caption{Old connection}
\label{fig:wheelspeed1}
\end{figure}

In order to palliate this failure, the modification consists in changing the mechanical connection similar to the one available on the fleet of EMB-190. A analysis of the maintenance performance of this connection on the EMB-190 fleet is 10 times higher than the connection available in the ERJ-145. The new connection is shown at the figure below.

\begin{figure}[H] % Example image
\center{\includegraphics[width=400px]{Pictures/Hidromechanical_Systems/WST_2.eps}}
\caption{New connection}
\label{fig:WST_2}
\end{figure}

To develop this upgrade, an engineer must be available for around 6 to 10 months to follow with the supplier and the supplier might require some funds to support the tests and verification. A prototype must make some landing in order to certify to the authorities that the new connection does not affect the braking performance. The replacement of the new part will be made by a repair from the supplier made by service bulletin. Once the old part fails, the airliner ships the part to repair and then this part is upgraded.
