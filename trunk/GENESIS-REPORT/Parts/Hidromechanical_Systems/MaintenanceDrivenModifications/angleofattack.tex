The previous analysis of the ERJ-145 SPMR list was possible to find that this component was on the list of the most representative parts to be replaced having a MTBUR of 28452 flight hours. The main strategy of this modification is to reduce the maintenance cost.

It was made a huge research and contacts with experienced engineers and it was concluded that AOA SENSOR doesn't perform as well as similar parts, being easily contaminated with water inside the interconnections of the sensor. In order to have the value of the reduction in DMC cost, a research over the angle of attack sensors available nowadays in other airplanes have shown that it is possible to have a considerable increment of MTBUR up to 34858 flight hours by changing the sensor and the systems related.

The actual maintenance cost of AOA sensor per airplane is about US\$ 9,252.30 and the new cost per airplane for a new AOA sensor was estimated around US\$ 7,595.19, all these costs was estimated for 10 years maintenance.

The development of a new AOA sensor to avoid the problems that happens nowadays has a non-concurrent cost that made an unfeasible modification.
In fact if this modification was implemented there would be a reduction of the indirect costs due to reduction of flight cancellations, delays and dispatchability.
