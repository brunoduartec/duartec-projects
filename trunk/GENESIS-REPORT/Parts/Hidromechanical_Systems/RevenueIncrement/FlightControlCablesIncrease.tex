The revenue increase demands a fuselage stretching, therefore an increase in the command cables length is mandatory. This increase has the size of the fuselage increase which is around one meter. The command cable increase has an undesirable effect in force loss by friction between the grommet and the carbon-steel wire. The assumption made to calculate the friction force increase will be that the new fuselage has one grommet and the weight of this one meter cable is supported by the grommet as shown in the figure below

\begin{figure}[H] % Example image
\center{\includegraphics[width=400px]{Pictures/Hidromechanical_Systems/cable.eps}}
\caption{Cable System Arrangement}
\label{fig:Sistema}
\end{figure}


Assuming the dry static friction coefficient between Teflon and steel is 0.2 and the carbon steel density is \[7850\ kg/m^{3}\], the diameter of the cable is 3.18 mm, the friction force is calculated by the equations below.

$F_{AT}=\mu .m.g$ %Fat=mu*m*g
$m=\rho .g$ %
$V=l\left ( \pi .\frac{D^{^{2}}}{4} \right )$
$F_{AT}=\mu .\rho .l.\left ( \pi .\frac{D^{^{2}}}{4} \right ).g$

The friction force generated by the cable increase is 0.122 N, which is considered to be negligible for the airplane operation.
