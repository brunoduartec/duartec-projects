It is necessary to calculate the cost of the modification in order to know if it will be profitable and continue on the program otherwise it will be rejected. To make this consideration, it is needed to know the concurrent costs and non-concurrent costs, together with the MTBUR (Mean Time Between Unscheduled Removal) of the old part and the expected MTBUR of the new part. The assumptions made in this part are that the costs for the owner are the price invested in the removal/replace and the price paid for the SB (Service Bulletin). The removal/replace price assimilates the costs of the new part discounted by thirty percent of the price on the old part plus the man-hour of the replacement times the price of the man hour. The price paid for the SB will be considered the price spent in its non-concurrent development divided by two hundred (the number of aircrafts planned on this project). For the modification to be profitable, the difference between the replacement costs of the old part and the new part in ten years (expected time of service) must be greater than the investment made (price of the SB plus concurrent cost explained above). If it is not the case, the modification is considered as not profitable and it will be discontinued.

Some modifications are different from replacement cases, sometimes the modification will not change the part that will be improved but some other part that have a performance impact over this one. There are also modifications that are maintenance procedure, having no parts replacement and costs only the SB with the man-hour cost.
The costs associated with the fuselage stretching are not considered in this chapter, those costs are quoted in the values shown by the aeronautics team.
A list of modifications and assumptions made in each case is described below:

\begin{itemize}
\item 	Gust lock Actuator: This solution has already been developed and it is available for the airlines, so the non-concurrent costs are none. The concurrent costs are just the man-hour cost for this maintenance procedure, once there is no part replacement.
\item 	Angle of Attack sensor: Once this part has no off the shelf replacement and the complexity associated with it, the prevision is to have a high expense in non-concurrent cost. The concurrent cost will be the cost of a new part discounted by the sale of the old part plus the man-hour cost. This modification proved to be ineffective once the investment made are higher than the value saved with the modification.
\item 	Command Cables: The carbon steel cable part number replacing the old part is already available in the ERJ-145 IPC (Illustrated Parts Catalog) and then it has non-concurrent cost. The concurrent cost is the cost of the new part plus the man-hour cost.
\item	Flap Transmission Brake: This solution has already been developed and it is available for the airlines as an alternative grease for the cable available in the ERJ-145 AMM (Aircraft Maintenance Manual), so the non-concurrent costs are none. The concurrent cost is the cost of the new cables plus the man-hour cost.
\item	Aileron Damper: The removal of this part is associated with the FAA certification. The argument for this removal is the European fleet that is not equipped with this part. The non-concurrent cost is to fund this campaign and is estimated in US\$ 507,500.00. The concurrent cost is the man-hour cost of the removal discounted by the cost of the removed part. The sale of the old part costs more than the man-hour of the removal being a profit to the airliner.
\item	Spoiler Control Unit: The non-concurrent cost of this modification is negligible once this repair already has a SB. The concurrent cost is the pull up electrical resistors and the man-hour cost for this repair.
\item	Brake Temperature Sensor: This part has no replacement available then a new part must be specified with the old part manufacturer. Due to this, the non-concurrent cost to fund this development is US\$ 110,000.00 and the concurrent cost will be the cost of the new part plus the signal conditioner discounted by the sale of the old conditioner plus the man-hour cost.
\item	Brake Control Unit: A new part with a better MTBUR performance is already available then the non-concurrent cost is negligible. The concurrent cost are the acquisition of the new part discounted by the sale of the old one plus the man-hour installation cost.
\item	Wheel Speed Transducer: The faults from this part is linked with the connection between the rotational part of the sensor and the wheel. Thus, the non-concurrent cost of US\$ 95,000.00 is used to fund this development with the supplier. This part will not be repurchased, the airliner will send the old part to be repaired and upgraded via SB and the supplier will return the new connection, making the concurrent cost be the man-hour cost.
\end{itemize}


\begin{figure}[H] % Example image
\center{\includegraphics[width=400px]{Pictures/Hidromechanical_Systems/CostSum1.eps}}
\label{fig:HidroCostSum1}
\end{figure}

\begin{figure}[H] % Example image
\center{\includegraphics[width=400px]{Pictures/Hidromechanical_Systems/CostSum2.eps}}
\label{fig:HidroCostSum2}
\end{figure}

