This chapter has the intention to report chronologically the changes
suggested by staff and all rational involved in each of them, including the secondary changes, these were considered due to a primary modification.

As described in the section \ref{sec:SolutionOverview}, the strategy of Genesis team was to work in four different fronts. The modifications that were decided to be implemented on the ERJ 145 retrofit are presented below organized in their different fronts.

The Figure \ref{fig:ModificationTree} is a map that shows which fronts were considered followed by primary and secondary modifications.

\begin{figure}[H] % Example image
\center{\includegraphics[width=400px]{Pictures/Technical_Summary/ModificationTree.eps}}
\caption{ERJ 145 Modification Tree.}
\label{fig:ModificationTree}
\end{figure}

\subsection{Revenue Increment Driven Modification Studies}
The revenue increment for the operator is mainly due to the increasing of the number of
passenger. It is possible to increase passenger's number just changing the LOPA and also
changing the LOPA and the fuselage stretching. The consequences of increasing the fuselage, falls into increasing tubes of hydraulic systems, all wiring associated and flight control cables
For a given target in SPEC phase, Genesis Team decided to increase by 20\% the pax number, thus requiring change LOPA and increase the fuselage.
\\Basically the LOPA parameter, changed the amount and position of the Galley. The seats also
were changed to the slim seat, adding 4 pax.
\\The fuselage stretching and pitch decreasing of 31 inches to 29 inches helped primarily to increase 6 pax.The final result was an increase of 10 pax with an effective pitch of 30 inches.
\\All such modifications generated secondary changes such as increasing the number of oxygen masks, check the thermal capacity of the air conditioner and the temperature control system and finally the verification of evacuation passenger test emergency.
\\Indirectly revenue increase can also be a subjective parameter that is how the aircraft turns attractive for the passenger. For this parameter was installed an USB power outlet to allow passengers to charge their electronic devices such as cell phones or tablets.

\subsection{Maintenance Costs Reduction Modification Studies}
From the modification studies made by each technology, the main goal for these systems are exchange the components to improve the system, the accessibility and the reliability also, and thus will decrease the maintenance cost. 
\\First is the ECS system, with a Bleed Leakage Monitoring System. This modification will contribute to decrease the lead-time of the maintenance task by making the leakage search easier than the current architecture, this improvement reduces in 75\% the cost of this maintenance routine.
\\Second is the Hydraulics systems, an important factor for the new components is the improvement in the reliability, for example the Flap Transmission Brake give us a reduction of around the 0,3\% of DMC reduction, and also decrease the time of tasks for each system, these exchanges contribute to reduce the number of parts that will be exchanged by the operators. 
\\Third is the Interiors System, that studied the modifications to improve the system and the contributions of the exchange for LED lamps, these situation was analyzed and the conclusion is that exchange is better for all system and contribute to increase the reliability of the components and decrease the maintenance costs and considerably the number of changes.
\\Finally the Electrical team analyzed modifications to the components: Starter Generator APU, Main Batteries, Static inverter and Generator Control Unit, these modification contribute to reduce the maintenance cost in 0,35\%, improve the airplane system of navigation and the reliability of the components.
\\In conclusion of this study is that implementation of these components contribute to decrease the maintenance cost and improve other systems.

\subsection{Fuel Consumption Reduction Modification Studies}
As shown in the section \ref{sec:FuelConsumptionRedOverview}, an important factor for the operator is to decrease the fuel consumption for a given route.
\\For that, aerodynamics and ECS system modifications were installed.
\\In aerodynamics, there are four major groups of changes. The first is basically to improve
directly L/D using winglet. The second is the reduction of steps and gaps exist in the
aircraft involving wing sealing of gaps and steps, nose and main landing gear gaps seals, steps seals on movable boom parts, new fairing for the internal flap, aileron new upper fairing surface mechanism design, wing hard points removal. The third group of modifications are related to incoming and outgoing air, as nose avionics bay NACA inlet, nose eltronic bay NACA inlet, air conditioning NACA inlet, wing to fuselage fairing NACA inlet and grill outlet and wing tank vents. \\Finally the modifications listed below are general modifications as door handles removal, new ski fairing, red beacon and drain removal, leading edge polishing, removal of anti-skid strips on left wing, new top red beacon, main landing gear wheel cap and wing root leading edge and trailing edge fillets.
\\Another factor that is responsible for the fuel consumption is called bleed, that is nothing more than energy extracted from the turbine at various thermodynamic stages to supply other aircraft systems energy demands. Upgrades such as improved temperature control and anti-icing system operating logic control were installed.

\subsection{Regulation Driven Modification Studies}
These modifications turn the aircraft technically prepared to the new operating regulations. As mentioned in \ref{sec:RegulationFrontOverview}, the regulation rules motivate the team to work on an avionics system update. These rules start to be committed from 2014/2015 in some regions of the globe, and shall derail the operation from the year 2020.
\\The changes suggested by the team are software upgrades, including VNAV, LPV, RNP and ADS-B features. There will be only a single hardware change that is the antenna replacing that will pick up the ADS-B signal.
\\The VNAV provides the possibility of a vertical guidance, reduces the pilot's workload and reduces consumption by optimizing route.
\\LPV GPS uses a more accurate (WAAS) and it is able to provide information of guiding approach operations simulating an ILS operation.
\\The RNP is part of performance-based navigation with the goal of enabling a sharper operation routes allowing a distance reduction between aircraft in the same airspace and decreased consumption due to optimized flight profiles.
\\The ADS-B is a real time information transmission from the aircraft as altitude, speed and current position which are used to improve airspace control since it works with more accurate information compared with radar-based system.
\\It is important to point that the avionics system update may increases the aircraft value and reduce fuel consumption.