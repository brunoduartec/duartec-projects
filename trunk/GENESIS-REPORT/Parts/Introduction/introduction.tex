
This report aims to present the results of preliminary studies (PDR - Preliminary Design Report) of the Embraer ERJ 145 retrofit. This study has been done by the Genesis Team during the third phase of the Engineering Specialization Program (PEE - Programa de Especializa\c c\~ ao em Engenharia).

The main objective of the PDR is to present the preliminary design of the aircraft and the market scenario where the aircraft will be inserted.

This report presents and describes the modifications that compose the ERJ 145 retrofit and how these modifications will change the economic outcome of ERJ 145 Operators. It also presents the Embraer's business plan that makes this program viable.

\subsection{Embraer}
Embraer is one of the world's main aircraft manufacturers, a position achieved due to the constant and determined pursuit of full customer satisfaction. With a global customer base and important internationally renowned partners, Embraer has been contributing, for more than 40 years, to world's integration through aviation, shortening distances between people and offering the most modern technology, versatility and comfort in airplane.

Created on August 19, 1969, Embraer - Empresa Brasileira de Aeron\' autica, was a mixed capital company under government control. With the support of the Brazilian Government, the Company would transform science and technology into engineering and industrial capability. In addition to starting the production of the Bandeirante, Embraer was commissioned by the Brazilian Government to manufacture the EMB 326 Xavante, an advanced trainer and ground attack jet, under license of Italian company Aermacchi. Other developments that marked the beginning of Embraer's activities were the EMB 400 Urupema high-performance glider and the EMB 200 Ipanema agricultural airplane.

By the end of the 70's, the development of new products, such as the EMB 312 Tucano and the EMB 120 Brasilia, followed by the AMX program in cooperation with the companies Aeritalia (now Alenia) and Aermacchi, allowed Embraer to reach a new technological and industrial level.

The entry into service of the new EMBRAER 170/190 family of commercial aviation in 2004, the confirmation of Embraer's definitive presence in the executive aviation market with the launch of new products, as well as the expansion of its operations into the aviation services market, established solid foundations for the future development of the Company.

The Figure \ref{fig:EmbraerGlobalBusiness} shows the countries that have offices or units of Embraer.

\begin{figure}[H] % Example image
\center{\includegraphics[width=400px]{Pictures/Introduction/EmbraerGlobalBusiness.eps}}
\caption{Global Business.}
\label{fig:EmbraerGlobalBusiness}
\end{figure}

Figure \ref{fig:EmployeesAndEducationalLevels} and Figure \ref{fig:DataSheetEmbraer} shows the number of employees, revenue and orders
according to a recent survey, which reinforces the importance of the company in the
global aircraft scenario.

\begin{figure}[H] % Example image
\center{\includegraphics[width=450px]{Pictures/Introduction/EmployesAndEducationalLevels.eps}}
\caption{Employees and Educational Levels.}
\label{fig:EmployeesAndEducationalLevels}
\end{figure}


\begin{figure}[H] % Example image
\center{\includegraphics[width=200px]{Pictures/Introduction/DataSheetEmbraer.eps}}
\caption{Data Sheet Embraer.}
\label{fig:DataSheetEmbraer}
\end{figure}

Embraer's business is to generate value for its shareholders by fully satisfying its
customers in the global aviation market. By "generate value", we mean maximizing the
Company's value and ensuring its perpetuity, acting with integrity and social
environmental awareness.

The Company concentrates on three business segments and markets:
Commercial Aviation, Executive Aviation, and Defense Systems. The values that mold
the attitudes and unite actions to ensure the Company's perpetuity are:

\begin{itemize}
  \item Our people;
  \item Our customers;
  \item Company excellence;
  \item Boldness and innovation;
  \item Global presence;
  \item Sustainable future.
\end{itemize}
*This text was extracted from the site (EMBRAER, 2013).

\subsection{PEE}

The Engineering Specialization Program, PEE (Programa de Especialização em
Engenharia, in Portuguese) - is a corporative program wherewith Embraer aims to empower
graduates engineers to work in aeronautical engineering. In a partnership with ITA
(Technological Institute of Aeronautics), they offer a professional master's degree in
Aeronautical Engineering recognized by CAPES / MEC. The courses and activities are
held in the company's premises by professionals and consultants hired by Embraer.

\begin{figure}[H] % Example image
\center{\includegraphics[width=180px]{Pictures/Introduction/EmbraerPEELogo.eps}}
\caption{Embraer Logo.}
\label{fig:EmbraerPEELogo}
\end{figure}

The PEE lasts 15 months and has three distinct phases. Figure \ref{fig:PEEPhases} describes each
phase:

\begin{figure}[H] % Example image
\center{\includegraphics[width=450px]{Pictures/Introduction/PEEPhases.eps}}
\caption{Stages of Professional Master Program of Embraer.}
\label{fig:PEEPhases}
\end{figure}

\textbf{Phase 1:} This phase contains a set of compulsory subjects for all students,
focused on knowledge of basic subjects: Aeronautical Engineering and Electronics and
Engineering. This phase lasts for 5 months.

\textbf{Phase 2:} In this phase, students usually are divided into careers as shown above,
but PEE 19 students were divided in one group called Systems. The main idea was
gave him knowledge about Hydromechanical Systems, Electrical-Electronics Systems,
System Engineering.

\textbf{Phase 3:} As part of Phase 3 of the program, students are divided into project
teams and engage with conceptual designs of an aircraft specified by the Embraer
Technical Director of Engineering. During this phase students are constantly supervised
and observed in their technical skills, behavioral, teamwork, and leadership. To meet the
requirement of high technical standard required in projects, coordination of project
consists of senior engineers from Embraer and a team of international consultants with
industry expertise who collaborate with the company. This part is called Program
Placement and is characterized by full-time dedication to the students' projects. This
phase lasts for 4 months.

Parallel with the Program Placement, each student, individually, is dedicated to
research for the preparation of his Master's Thesis. The orientation of this thesis is held
by a professor from ITA, and may be co-oriented by an Embraer professional.

\subsection{The ERJ 145 Context}
The context of the ERJ 145 in the market scenario is based on the following
forecast tables.

\begin{figure}[H] % Example image
\center{\includegraphics[width=300px]{Pictures/Introduction/WorldFleetInService.eps}}
\caption{World Fleet in Service.}
\label{fig:WorldFleetInService}
\end{figure}

\begin{figure}[H] % Example image
\center{\includegraphics[width=450px]{Pictures/Introduction/WorldProjectedNewDeliveries.eps}}
\caption{World Projected New Deliveries.}
\label{fig:WorldProjectedNewDeliveries}
\end{figure}

It can be observed in the Figure \ref{fig:WorldFleetInService}, for 30-60 seats, a reduction of
approximately 1000 aircrafts is expected in the next 20 years on the actual fleet.

It can also be observed in the Figure \ref{fig:WorldProjectedNewDeliveries} that, in the next 20 years, only 400
aircrafts may be delivered in the 30-60 seats segment.

The first question that comes up is: Which aircrafts are going to be extinguished,
and which ones are going to remain operating?

Studies have shown that the 30-60 seats segment is migrating to the 75 seats
segment, due to the capacity demand expansion in the actual routes and to the
updates of the scope clauses condition existing in the United States.

A scope clause is part of a contract between an airline and a pilot union. Generally, these clauses are used by the union of a Major airline to limit the number and/or size of aircraft that airline may contract out to a Regional airline. The goal is to protect union jobs at the major airline from being eliminated by regional airlines operating larger aircraft

It is also known that the ERJ 145 CASM is reaching high levels, being close to
cross over the Yield levels, which means that the profit of the airliners that operates ERJ 145 is decreasing and, in some cases, reaching zero.

This increasing in the Operational Costs is mainly caused by the increasing prices of Oil in the last 10 years (Figure \ref{fig:oilPriceEvolution}) and also by the changes that have been done in the maintenance strategy. In this scenario, the operation of the aircraft is close to be considered impracticable.

\begin{figure}[H] % Example image
\center{\includegraphics[width=400px]{Pictures/Introduction/oilPriceEvolution.eps}}
\caption{Oil Price Evolution.}
\label{fig:oilPriceEvolution}
\end{figure}

Since the release of the first ERJ 145 family, Embraer has sold about 900 aircrafts.
It is known that 400 aircraft has been sold with a residual value warranty contract that
obligates Embraer to pay back their clients the warranted market value of the aircraft in
case that the real value is under this price. This value varies among the aircraft, but, for standardization purposes, it is being considered as US\$2,000,000.00 per aircraft in 2012. Therefore, in case that the operation of these aircrafts becomes impracticable, it is probable that Embraer will have to refund their clients that require their residual value warranty.

\subsection{Objective}
The objective of this project is to offer a modernization package to ERJ 145 to improve its attractiveness in order to keep the aircraft of ERJ 145 Family operational and create an economically viable business plan to this solution.

\subsection{Solution Overview}
\label{sec:SolutionOverview}

It has become clear that to make ERJ 145 attractive to the market is necessary to adequate the aircraft to the new economical and technical scenario that takes place nowadays and the scenario that will remain true in the following 10 to 15 years.

Based on the scenario and objective described in the previous sections, it was first defined the main strategy to reach the objectives.

It was identified that the team should work in 4 initiatives, called fronts, which are:

\begin{itemize}
  \item Clients' Revenue Increment,
  \item Maintenance Costs Reduction,
  \item Fuel Consumption Reduction and
  \item Regulation.
\end{itemize}

\subsubsection{Revenue Increment}
In commercial terms, market attractiveness is highly related to how much the clients can profit operating the aircraft. This front studies possible modifications that may increase the revenue of ERJ 145's operators.

\subsubsection{Maintenance Costs Reduction}
Maintenance Costs represents a very substantial part of the aircraft's operational cost. According to the image \ref{fig:145COCDistribution}, 18\% the operational cost is composed by maintenance operations.
During the very initial phase of this project, it was investigated the potential maintenance items that could be improved in ERJ 145.
It was identified that several modifications in aircraft components could be performed in order to reduce maintenance costs. It was also studied some improvements in the maintenance strategy plan like the checks period. Another issue that was studied was related to the maintenance activities like improvements in maintenance access. Simultaneously to this studies, it was also created a new business plan that can be proposed to the main suppliers of ERJ 145.


\begin{figure}[H] % Example image
\center{\includegraphics[width=450px]{Pictures/Introduction/145COCDistribution.eps}}
\caption{ERJ 145 Cash Operational Cost Distribution.}
\label{fig:145COCDistribution}
\end{figure}

\subsubsection{Fuel Consumption Reduction}
\label{sec:FuelConsumptionRedOverview}
The Fuel Consumption also represents a substantial part of the aircraft's operational cost according to the figure \ref{fig:145COCDistribution}.

During the Conceptual studies of this project, many improvements that could contribute with a reduction were analysed, like aerodynamic improvements, winglet and energy efficient improvements.

\subsubsection{Regulation}
\label{sec:RegulationFrontOverview}
In order to maintain the aircraft operational during the following 15 years, at least, the Genesis team studied all the functionalities that should be implemented to comply the rules that will be established by the different market's authorities, such as FAA and EASA.

The modifications that emerged by regulation issues are basically placed in the avionics system. The regulation rules start to be committed from 2014/2015 in some regions of the globe, and shall derail the operation from the year 2020.

\subsection{Organizational}

This section intends to show some organizational issues that supported the Genesis Team during this project.
The creation of the groups in the different phases of the project was proposed by the leaders and discussed with the Team. The division of the team members in groups were primarily based on the member's familiarity.
\\Parallel to the main group division, it was created committees to work in some important issues like standardization and organization definitions.
Organizational tools were used to improve productivity and integration of the
Team. The Genesis Team had implemented some important tools that were essential to
the success of the project. These tools are described in the sections below.

\subsubsection{Team Divisions}
In the beginning of the project, It was elected the Leader and Vice-Leader of the Genesis Team, as shown in the Figure \ref{fig:GenesisLeadershipMembers}.

\begin{figure}[H] % Example image
\center{\includegraphics[width=200px]{Pictures/Introduction/Lideres.eps}}
\caption{Genesis Leadership Members.}
\label{fig:GenesisLeadershipMembers}
\end{figure}


During the very initial phase of the project, the SPEC phase, the Genesis Team was organized in four groups:

\begin{itemize}
  \item Technical Data
  \item Market
  \item Market Forecast \& Opportunities
  \item Competitors
\end{itemize}

\begin{figure}[H] % Example image
\center{\includegraphics[width=400px]{Pictures/Introduction/Technical.eps}}
\caption{SPEC Phase  - Technical Data Group.}
\label{fig:TechnicalDataGroup}
\end{figure}

\begin{figure}[H] % Example image
\center{\includegraphics[width=400px]{Pictures/Introduction/Market_Data.eps}}
\caption{SPEC Phase  - Market Data Group.}
\label{fig:MarketDataGroup}
\end{figure}

\begin{figure}[H] % Example image
\center{\includegraphics[width=400px]{Pictures/Introduction/Market_Forecast.eps}}
\caption{SPEC Phase  - Market Forecast \& Opportunities Group.}
\label{fig:SpecForecastMembers}
\end{figure}

\begin{figure}[H] % Example image
\center{\includegraphics[width=400px]{Pictures/Introduction/Competitors.eps}}
\caption{SPEC Phase  - Competitors Group.}
\label{fig:CompetitorsGroup}
\end{figure}

Once the SPEC Phase was finished, the Genesis Team was reorganized in technical groups:

\begin{itemize}
  \item Aeronautics, Structure \& Propulsion,
  \item ECS,
  \item Hidromechanical Systems,
  \item Electrical Systems \& Avionics,
  \item Interiors,
  \item Maintenance, Maintainability, Manufacturing and Support, and
  \item Market.
\end{itemize}

\begin{figure}[H] % Example image
\center{\includegraphics[width=400px]{Pictures/Introduction/Aero.eps}}
\caption{Aeronautics, Structure \& Propulsion Group.}
\label{fig:AeronauticsGroup}
\end{figure}

\begin{figure}[H] % Example image
\center{\includegraphics[width=400px]{Pictures/Introduction/ECS.eps}}
\caption{ECS Group.}
\label{fig:ECSGroup}
\end{figure}

\begin{figure}[H] % Example image
\center{\includegraphics[width=400px]{Pictures/Introduction/Hydromec.eps}}
\caption{Hydromechanical Systems Group.}
\label{fig:HydromecGroup}
\end{figure}

\begin{figure}[H] % Example image
\center{\includegraphics[width=400px]{Pictures/Introduction/Electrical.eps}}
\caption{Electrical Systems \& Avionics Group.}
\label{fig:ElectricalSystemGroup}
\end{figure}

\begin{figure}[H] % Example image
\center{\includegraphics[width=400px]{Pictures/Introduction/Interiors.eps}}
\caption{Interiors Group.}
\label{fig:ECEPDivisionMembers}
\end{figure}

\begin{figure}[H] % Example image
\center{\includegraphics[width=400px]{Pictures/Introduction/3MASU.eps}}
\caption{Maintenance, Maintainability, Manufacturing and Support Group.}
\label{fig:ECEPDivisionMembers}
\end{figure}

\begin{figure}[H] % Example image
\center{\includegraphics[width=150px]{Pictures/Introduction/Market.eps}}
\caption{Market \& Finances Group.}
\label{fig:ECEPDivisionMembers}
\end{figure}



\subsubsection{Scrum}
The Scrum development model was implemented in the technical groups. Each technical group run its own scrum. A higher level of integration was necessary to keep all the groups synchronized. This higher level was composed by the Scrum Masters and the Team Leaders who made the role of the Product Owner.

The sprint period adopted was 2.5 days in the Spec phase and Conceptual Studies Phase, and 5 days in the Preliminary Studies Phases.
\subsubsection{Obeya}

The main practices of the Japanese process Obeya was implemented. The
concept of building a Project Room with useful and practical visual information on the
walls was highly useful to support the communication of the team, integration of the
information and tasks, and decision making practices.
\\It was also developed a Mock up of the new interior of the aircraft to validate the main decisions of the team. With the mock up, it was possible to simulate the real sensation of the passenger, result not possible with CAD simulations.

\subsubsection{5S Committee}
It was created a committee responsible for implementing the 5S into the Team. The
main considered issues were the layout of the project room, rules for cleanliness and
materials organization.

\begin{figure}[H] % Example image
\center{\includegraphics[width=400px]{Pictures/Introduction/5S.eps}}
\caption{5S Committee Members.}
\label{fig:5SCommitteeMembers}
\end{figure}

\subsubsection{Information Management Committee}

It was created a committee to discuss, define and implement standards for
documentation, information handling and communication tools.

This committee was responsible for the creation of templates, version control of
artifacts, etc.

\begin{figure}[H] % Example image
\center{\includegraphics[width=400px]{Pictures/Introduction/Info.eps}}
\caption{Information Management Committee Members.}
\label{fig:InformationManagementCommitteMembers}
\end{figure}

\subsubsection{Visual Communication Management Committee}

In order to discuss, define and implement standards for visual information and visual communication, it was created the Visual Communication Management Committee. This committee was precisely useful to support the Obeya Wally-board practices.
\begin{figure}[H] % Example image
\center{\includegraphics[width=400px]{Pictures/Introduction/Visual.eps}}
\caption{Visual Management Committee Members.}
\label{fig:VisualManagementCommitteMembers}
\end{figure}

