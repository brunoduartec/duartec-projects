Immersed in a context with 400 aircrafts with the residual value contract, the company faces a delicate situation. Based on market researches, it's expected that only 200 aircrafts will be requiring the contract clause of the residual value.

Based on this fact, and knowing that the residual value is \$ 2.000.000,00, the expected injury is \$ 400.000.000,00.

To deal with this situation, a modernization package was created as a solution. However, this solution will require a certain investment of the company in order to pay the non-recurrent values of the modifications.

An analysis of the previously defined scenario shows that this investment is quite profitable to the company.

\textbf{Scenario:}

200 aircrafts;

150 modification package sold;

50 required the residual value;

150 modification package sold for clients that do not have the residual value contract.

\textbf{Investment, Costs and Revenues}

The value of investment will be the total of all non-recurrent values of the modifications multiplied by the markup of the outsourced engineering services.

Investment = \$ 41.097.900,75

The costs will be, in this case, the number of aircrafts that will require the residual value over the years multiplied by the residual value.

% Table generated by Excel2LaTeX from sheet 'Cash Flow Acionista'
\begin{table}[H]
  \scriptsize
  \centering
  \caption{Residual Value}
    \begin{tabular}{crccccc}
    \toprule
    \multicolumn{7}{c}{\textbf{COSTS}} \\
    \midrule
    \textbf{ITEM} & \multicolumn{1}{c}{\textbf{DESCRIPTION}} & \textbf{YEAR 1} & \textbf{YEAR 2} & \textbf{YEAR 3} & \textbf{YEAR 4} & \textbf{YEAR 5} \\
    1     & \multicolumn{1}{l}{Residual Value} & \$40.000.000,00 & \$40.000.000,00 & \$20.000.000,00 & \$0,00 & \$0,00 \\
    \bottomrule
    \end{tabular}%
  \label{tab:financeCostResidualValue}%
\end{table}%

The revenue for the company will be selling the service bulletin for the price of the diluted non-recurrent value of the modifications multiplied by the markups.

The markup for the clients that have the contract will be 30\% and the markup for the clients that don't have the contract will be 500\%.

Therefore, the service bulletin price will be:

SB(contract) = \$ 267,136.35

SB = \$ 1,027,447.52


% Table generated by Excel2LaTeX from sheet 'Cash Flow Acionista'
\begin{table}[H]
  \scriptsize
  \centering
  \caption{Revenue (Residual Value Contract)}
    \begin{tabular}{crcccccc}
    \toprule
    \multicolumn{8}{c}{\textbf{REVENUE (RESIDUAL VALUE CONTRACT)}} \\
    \midrule
    \textbf{ITEM} & \multicolumn{1}{c}{\textbf{DESCRIPTION}} & \textbf{YEAR 1} & \textbf{YEAR 2} & \textbf{YEAR 3} & \textbf{YEAR 4} & \textbf{YEAR 5} & \textbf{YEAR 6} \\
    1     & \multicolumn{1}{l}{Service Bulletin sells} & \$0,00 & \$0,00 & \$8.014.090,65 & \$24.042.271,94 & \$5.342.727,10 & \$2.671.363,55 \\
    2     & \multicolumn{1}{l}{Number of sells} & 0     & 30    & 90    & 20    & 10    & 0 \\
    \bottomrule
    \end{tabular}%
  \label{tab:financeResidualRevenue1}%
\end{table}%


% Table generated by Excel2LaTeX from sheet 'Cash Flow Acionista'
\begin{table}[H]
  \scriptsize
  \centering
  \caption{Revenue}
    \begin{tabular}{crcccccc}
    \toprule
    \multicolumn{8}{c}{\textbf{REVENUE}} \\
    \midrule
    \textbf{ITEM} & \multicolumn{1}{c}{\textbf{DESCRIPTION}} & \textbf{YEAR 1} & \textbf{YEAR 2} & \textbf{YEAR 3} & \textbf{YEAR 4} & \textbf{YEAR 5} & \textbf{YEAR 6} \\
    1     & \multicolumn{1}{l}{Service Bulletin sells} & \$0,00 & \$0,00 & \$0,00 & \$92.470.276,69 & \$51.372.375,94 & \$10.274.475,19 \\
    2     & \multicolumn{1}{l}{Number of sells} & 0     & 0     & 90    & 50    & 10    & 0 \\
    \bottomrule
    \end{tabular}%
  \label{tab:revenue2}%
\end{table}%

\textbf{Cash Flow}

% Table generated by Excel2LaTeX from sheet 'Cash Flow Acionista'
\begin{table}[H]
  \tiny
  \centering
  \caption{Cash Flow}
    \begin{tabular}{rcccccc}
    \toprule
    \multicolumn{7}{c}{\textbf{CASH FLOW}} \\
    \midrule
    \multicolumn{1}{r}{\multirow{2}[4]{*}{\textbf{DESCRIPTION}}} & \multicolumn{6}{r}{\textbf{ ANNUAL INCOME STATEMENT DISTRIBUTION}} \\
    \multicolumn{1}{r}{} & \textbf{YEAR 1} & \textbf{YEAR 2} & \textbf{YEAR 3} & \textbf{YEAR 4} & \textbf{YEAR 5} & \textbf{YEAR 6} \\
    (+) Total Gross Revenue (RV) & \$0,00 & \$0,00 & \$8.014.090,65 & \$24.042.271,94 & \$5.342.727,10 & \$2.671.363,55 \\
    (+) Total Gross Revenue & \$0,00 & \$0,00 & \$0,00 & \$92.470.276,69 & \$51.372.375,94 & \$10.274.475,19 \\
    ( -) Operational Costs & \$40.000.000,00 & \$40.000.000,00 & \$20.000.000,00 & \$0,00 & \$0,00 & \$0,00 \\
    (=) Net Entry & \textbf{-\$40.000.000,00} & \textbf{-\$40.000.000,00} & \textbf{-\$11.985.909,35} & \textbf{\$116.512.548,63} & \textbf{\$56.715.103,04} & \textbf{\$12.945.838,74} \\
    \bottomrule
    \end{tabular}%
  \label{tab:cashFlowAcionista}%
\end{table}%

% Table generated by Excel2LaTeX from sheet 'Cash Flow Acionista'
\begin{table}[H]
  \centering
  \caption{Cumulative Cash Flow}
    \begin{tabular}{ccc}
    \toprule
    \textbf{YEAR} & \textbf{CASH FLOW} & \textbf{CUMULATIVE CASH FLOW} \\
    \midrule
    \textbf{0} & -\$41.097.900,75 & -\$41.097.900,75 \\
    \textbf{1} & -\$40.000.000,00 & -\$81.097.900,75 \\
    \textbf{2} & -\$40.000.000,00 & -\$121.097.900,75 \\
    \textbf{3} & -\$11.985.909,35 & -\$133.083.810,10 \\
    \textbf{4} & \$116.512.548,63 & -\$16.571.261,48 \\
    \textbf{5} & \$56.715.103,04 & \$40.143.841,56 \\
    \textbf{6} & \$12.945.838,74 & \textbf{\$53.089.680,29} \\
    \bottomrule
    \end{tabular}%
  \label{tab:financeCumulativeCashFlowShare}%
\end{table}%


\textbf{Payback:}

Payback (fractionated) = Last negative cumulative cash flow / First positive cash flow corresponding to the first positive cumulative cash flow.

Payback = 0.3

Then, the payback will occur approximately in 4 years and 5 months.

\textbf{Net Present Value (NPV):}

Investment = \$ 41,097,900.75
Revenue in the 1st year = -\$ 40,000,000.00
Revenue in the 2nd year = -\$ 40,000,000.00
Revenue in the 3rd year = -\$ 11,985,909.35
Revenue in the 4th year = \$ 116,512,548.63
Revenue in the 5th year = \$ 56,715,103.04
Revenue in the 6th year = \$ 12,945,838.74

NPV = - Investment value + [Revenue in the $1^{st}$ year/$(1 + i_{1})^{1}$] + [Revenue in the $2^{nd}$ year/$(1 + i_{2})^{2}$] + [Revenue in the $3^{rd}$ year/$(1 + i_{3})^{3}$] + [Revenue in the $4^{th}$ year/$(1 + i_{4})^{4}$] + [Revenue in the $5^{th}$ year/$(1 + i_{5})^{5}$] + [Revenue in the $6^{th}$ year/$(1 + i_{6})^{6}$]

i = 3 \% (Market rate)

NPV = \$ 34,679,279.04

\textbf{Internal Rate of Return(IRR):}

The IRR for this example can be calculated using an excel tool, then the internal rate of return will be:

IRR = 11 \%



