The active wing study came from a technology developed by TAMARACK AEROSPACE GROUP. This improvement consist on a mobile compensator (highlighted on figure \ref{fig:ActiveWinglet}), which objective is to compensate the aerodynamic loads on the wing in order to reduce or even eliminate the need of any structural reinforcement.


\begin{figure}[H] % Example image
\center{\includegraphics[width=400px]{Pictures/Aeronautics/ModificationStudies/ActiveWinglet.eps}}
\caption{Induced drag with and without winglet}
\label{fig:ActiveWinglet}
\end{figure}

This Technology is already deployed on smaller aircrafts as the CIRRUS SR22 and CITATIONJET, the benefits of this solutions is due to the winglet gain and also the lack of the structural reinforcement, but it would also have some bad consequences as the difficulty to certificate this technology on a larger and commercial aircraft as the ERJ 145.
The recurrent cost of this modifications (winglet included) is $ 225.000,00 (evaluated due the CITATIONJET installation cost) and  the non-recurrent cost is $ 5.000.000 due to the need fact that this modification was never applied on an aircraft at the size of the ERJ 145.
This technology is a direct concurrent of the winglet with structural reinforcements, so a trade of study was necessary between those two modifications. It was concluded that the Active Winglet is not viable for the project due to the recurrent cost that is too high and to the fact that it was needed a minor structural reinforcement due to the winglet.
