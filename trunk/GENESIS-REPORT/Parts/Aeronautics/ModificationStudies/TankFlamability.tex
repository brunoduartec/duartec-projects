This study had as a background a new regulation due to tank flammability (FAA AR-98). This way, it was necessary to evaluate if the new 145 would be included on this regulation, and if it was, which changes it was necessary to be compliance.
The regulation offers two possibilities, a quantitative and qualitative analyses. The quantitative involves the utilization of Monte Carmo statistical method, basically, what this approach evaluate is the frequency that the fuel is exposed to flammability region, with an evaluation of different routes and the aircraft operation point.
The qualitative method is based on the aircraft geometry, briefly, it evaluate if the aircraft has some of the potential characteristics that maximizes the probability of flammability.  This method was chosen to evaluate the need or not of a flammability solution. To better understand on what we had to analyze, the flowchart \ref{fig:Flammabilityflowchart} was created.

\begin{figure}[H] % Example image
\center{\includegraphics[width=400px]{Pictures/Aeronautics/ModificationStudies/Flammabilityflowchart.eps}}
\caption{Flammability flowchart}
\label{fig:Flammabilityflowchart}
\end{figure}

The ERJ 145 LR wing's are made of aluminum alloy with some titanium reinforcements. The second question, the airplane has a small portion of the tank inside the fuselage (figura \ref{fig:fueltankposition}), despite this aspect, it is reasonable to assume that this isn't a limitation.

\begin{figure}[H] % Example image
\center{\includegraphics[width=400px]{Pictures/Aeronautics/ModificationStudies/fueltankposition.eps}}
\caption{fuel tank position}
\label{fig:fueltankposition}
\end{figure}

Lastly, the 145 LR doesn't has a heat source inside the fuel tank. With all those information, the airplane is already compliance with the requirement, and there is no need to adopt a solution to improve it (if the goal is only to fulfill it).
