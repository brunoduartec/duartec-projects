Winglet is used to reduce induced drag. It has a similar effect of an increase in aspect ratio. The winglet also represents an increase in the total wetted area and, therefore, an increase in parasite drag.
The entire flight must be considered to check if the winglet will be advantageous. Flight conditions with high $C_{L}$ will benefit from a wing with winglet.
The change in induced drag was estimated with the panel method software Tornado by calculating a drag polar for the wing with and without winglet. The constant $k$ is obtained by getting the inclination from the curve $C_{D} = f({C_{L}}^{2})$



\begin{figure}[H] % Example image
\center{\includegraphics[width=400px]{Pictures/Aeronautics/ModificationStudies/Induceddragwinglet.eps}}
\caption{Induced drag with and without winglet}
\label{fig:InducedDrag}
\end{figure}

The change in parasite drag was estimated by considering only the wetted area increase. From the new complete aircraft polar it is possible to see that the winglet reduces the drag for $C_{L} > 0.2830 $


\begin{figure}[H] % Example image
\center{\includegraphics[width=400px]{Pictures/Aeronautics/ModificationStudies/DragPolar.eps}}
\caption{Drag Polar}
\label{fig:DragPolar}
\end{figure}
