The study of this change came from a development already deployed by EMBRAER on LEGACY 2002, this change resulted on a reduction of about 4% in the parasitic drag, with that background in mind, a great potential was detected but a detailed study for the new 145 was necessary.
Firstly, an analysis of the modifications already implemented on LEGACY 2002 and a technical visit on the airplane was held, with that study, a division of 3 different packs (simple, medium and complete) was built in order to bring to the team various configurations, each one of those packs were analyzed in the point of view of costs and the reduction of the parasite drag.
\textbf{Simple improvement}
On this pack the focus was on concepts already developed by EMBRAER in order to reduce the indirect cost of the total change. For that we used some of the implementations employed on LEGACY 2002, listed on table below.

% Table generated by Excel2LaTeX from sheet 'Sheet1'
\begin{table}[htbp]
  \centering
  \caption{List of changes on simple aerodynamic improvement}
    \begin{tabular}{crr}
    \toprule
    \multirow{2}[2]{*}{} & \multicolumn{1}{c}{\multirow{2}[2]{*}{Options}} & \multicolumn{1}{c}{\multirow{2}[2]{*}{Detailed}} \\
    \midrule
          & \multicolumn{1}{c}{} & \multicolumn{1}{c}{} \\
    \multirow{21}[42]{*}{Simple Improvement} & Steps and Gaps & Wing Sealing of Gaps and Steps \\
          & Steps and Gaps & Nose and main landing gear gaps and steps seals \\
          & Steps and Gaps & Gaps seals on movable boom parts \\
          & Steps and Gaps & New fairing ("closure") for the internal flap FTF (lower surface root) \\
          & Steps and Gaps & New aileron upper surface mechanism fairing design \\
          & Steps and Gaps & Wing hard points removal \\
          & Steps and Gaps & Wing root leading edge and trailing edge fillets \\
          &       &  \\
          & Inlet / Outlet & Nosa avionics Bay NACA inlet \\
          & Inlet / Outlet & Nose eletronic bay NACA inlet \\
          & Inlet / Outlet & air conditioning NACA inlet \\
          & Inlet / Outlet & Wing to fuselage fairing NACA inlet and grill outlet \\
          & Inlet / Outlet & Wing tank vents \\
          &       &  \\
          & Miscellaneous & Door handles removal \\
          & Miscellaneous & New ski fairing, red beacon and drain removal \\
          & Miscellaneous & Leading edge polishing \\
          & Miscellaneous & Removal of anti-skid strips on left wing \\
          & Miscellaneous & New top red beacon \\
          & Miscellaneous & Main landing gear wheel cap \\
          & Miscellaneous & Wing root leading edge and trailing edge fillets \\
    \multirow{3}[6]{*}{Median Improvement} &       & Pylon to nacelle fillet \\
          & Antenna & Embed ADF antenna \\
          &       & Inlet/Outlet Improvement \\
    \multirow{2}[4]{*}{Complete Improvement} &       & Legacy 500 nose \\
          &       & Leading edge improvement \\
    \bottomrule
    \end{tabular}%
  \label{tab:simpleaerodinamicchanges}%
\end{table}%

Some of those opportunities are better detailed on the figures \ref{fig:Nacelleandcargodoorhandles}, \ref{fig:Jackhardpoints},\ref{fig:Wingfillets} and \ref{fig:NoseNACAinlet}

\begin{figure}[H] % Example image
\center{\includegraphics[width=400px]{Pictures/Aeronautics/ModificationStudies/Nacelleandcargodoorhandles.eps}}
\caption{Nacelle and cargo door handles}
\label{fig:Nacelleandcargodoorhandles}
\end{figure}

\begin{figure}[H] % Example image
\center{\includegraphics[width=400px]{Pictures/Aeronautics/ModificationStudies/Jackhardpoints.eps}}
\caption{Jack hard points}
\label{fig:Jackhardpoints}
\end{figure}

\begin{figure}[H] % Example image
\center{\includegraphics[width=400px]{Pictures/Aeronautics/ModificationStudies/Wingfillets.eps}}
\caption{Wing fillets}
\label{fig:Wingfillets}
\end{figure}

\begin{figure}[H] % Example image
\center{\includegraphics[width=400px]{Pictures/Aeronautics/ModificationStudies/NoseNACAinlet.eps}}
\caption{Nose NACA inlet}
\label{fig:NoseNACAinlet}
\end{figure}

It is important to be aware that those changes would bring a greater benefit in LEGACY 2002 than on our new 145 due to the different MACH speed of those two, Mc = 0.80 and 0.78 respectively. Despite this, some of those modifications would still bring benefits to the new 145, for the concept study this preliminary analyses is still viable to use this study to estimate the COC reduction. A more detailed study will be conducted on EP.
For this improvement, a initial estimate of -0,8\% due to the reduction of 10 drag counts (lower than what as obtained on LEGACY 2002) and an increase of 35 kg to the aircraft weight.


\textbf{Median Improvement}
For the second pack, we focused on changes that were not implemented on LEGACY 2002 but some potential have been seen on those changes at a relative low development cost. Those changes are listed on the table below

% Table generated by Excel2LaTeX from sheet 'Sheet2'
\begin{table}[htbp]
  \centering
  \caption{List of changes on median aerodynamic improvement}
    \begin{tabular}{crr}
    \toprule
    \multirow{2}[2]{*}{} & \multicolumn{1}{c}{\multirow{2}[2]{*}{Options}} & \multicolumn{1}{c}{\multirow{2}[2]{*}{Detailed}} \\
    \midrule
          & \multicolumn{1}{c}{} & \multicolumn{1}{c}{} \\
    \multirow{4}[8]{*}{Median Improvement} & Miscellaneous & Pylon to nacelle fillet \\
          & Antenna & Embed ADF antenna \\
          & Inlet / Outlet & Inlet/Outlet Improvement \\
          & Simple Improvement & Simple Improvement \\
    \bottomrule
    \end{tabular}%
  \label{tab:changesmedian}%
\end{table}%

For this improvement, a initial estimate of -1,1\% due to the reduction of 15 drag counts and an increase of 50 kg to the aircraft weight.



\textbf{Complete improvement}
Lastly, this pack aims to eliminate some major shock waves problems on the airplane, that is on the nose and the position of the leading edge

% Table generated by Excel2LaTeX from sheet 'Sheet3'
\begin{table}[htbp]
  \centering
  \caption{List of changes on complete aerodynamic improvement}
    \begin{tabular}{crr}
    \toprule
    \multirow{2}[2]{*}{} & \multicolumn{1}{c}{\multirow{2}[2]{*}{Options}} & \multicolumn{1}{c}{\multirow{2}[2]{*}{Detailed}} \\
    \midrule
          & \multicolumn{1}{c}{} & \multicolumn{1}{c}{} \\
    \multirow{4}[8]{*}{Complete Improvement} & Miscellaneous & Legacy 500 nose \\
          & Miscellaneous & Leading edge improvement \\
          & Simple Improvement & Simple Improvement \\
          & Median Improvement & Median Improvement \\
    \bottomrule
    \end{tabular}%
  \label{tab:changescomplete}%
\end{table}%

For this improvement, a initial estimate of -1,5\% due to the reduction of 20 drag counts and an increase of 75 kg to the aircraft weight.
 The configurations chosen by the team were the simple aerodynamic improvement due to his high ration between cast and drag reduction, this modification can be evaluated on the figure \ref{fig:Aerodynamicimprovementreview}

\begin{figure}[H] % Example image
\center{\includegraphics[width=400px]{Pictures/Aeronautics/ModificationStudies/Aerodynamicimprovementreview.eps}}
\caption{Aerodynamic improvement review}
\label{fig:Aerodynamicimprovementreview}
\end{figure}






