One objective of this modification was to reduce the original aircraft takeoff length in 150 m. One possibility is to improve flap design to achieve a higher CLmax. To verify the actual performance and quantify the improvement needed, the flap $\Delta CL$,$\Delta CD$,$\Delta CM$ were estimated using the method proposed by Stinton (1983).

The increments for the flapped wing are given by:
                    $
                    \\\Delta C_{L}=\lambda_{C} \lambda_{\beta} \lambda_{b}
                    \\\Delta C_{D0}=\delta_{C} \delta_{\beta} \delta_{b}
                    $



\begin{figure}[H] % Example image
\center{\includegraphics[width=400px]{Pictures/Aeronautics/ModificationStudies/flap-CmCl.eps}}
\caption{Flap $\frac{ \Delta C_{Mac}}{ \Delta C_{L}}  $ analysis}
\label{fig:flap-CmCl}
\end{figure}

Where the $\lambda s$ and $\delta s$ depend on flap type (double slotted in this case) and geometry.
The data used in this method was obtained for a wing with aspect ratio of 6, she there is a correction factor for wings with different aspect ratio:

                        $  (\Delta C_{L} )_{A} \approx F_{A} ( \Delta C_{L} )_{6}         $

The estimative of CLmax was compared to the actual CLmax for the different flap positions $\beta$

\begin{figure}[H] % Example image
\center{\includegraphics[width=400px]{Pictures/Aeronautics/ModificationStudies/FlapCLmaxestimations.eps}}
\caption{Flap CLmax estimations}
\label{fig:FlapCLmaxestimations}
\end{figure}

To reduce the runway length in 150 m with a takeoff weight for a 400 nm range, the CLmax with 22$\degree$ flap must increase about 11.2\%. The estimated CLmax for this flap position is only 1.5\% higher, so a major flap modification would be necessary to fulfill this requirement.
