To determine the engine's performance, a simulation software was necessary. The studies here presented were made using the Gasturb 10 (GASTURB10.EXE, 2004).
All engine data was obtained from EASA TCDS, ICAO engine exhaust emissions databank, Embraer Internal studies (DAP-GH8-070921, ERJ 155,1997).
The data is presented in the \ref{tab:enginedata}

% Table generated by Excel2LaTeX from sheet 'Plan1'
\begin{table}[htbp]
  \centering
  \caption{Literature engine data}
    \begin{tabular}{cc}
    \toprule
    Manufacturer & Rolls Royce \\
    \midrule
    Model & AE 3007 A1 \\
    Static Thrust & \multirow{2}[2]{*}{7.580} \\
    (SL ISA) [lb] &  \\
    Length [m] & 2,92 \\
    Width [m] & 1,41 \\
    Height [m] & 1,17 \\
    Weight [lb] & 1657 \\
    Cruise TSFC & \multirow{2}[2]{*}{0,6764} \\
    (37000ft M=0,76 ISA)  [lb/(lb*h)] &  \\
    Price [US\$Million] & 2,2 \\
    \bottomrule
    \end{tabular}%
  \label{tab:enginedata}%
\end{table}%

According to the methodology presented by Kurzke's article (KURZKE, 2005) , and using component efficiencies suggested by Mattingly (MATTINGLY, 2002), the following parameters were initially calibrated to match the engines thrust at static sea-level and ISA+15ºC conditions:

\begin{itemize}
\item Compressor/Fan Pressure Ratio;
\item Turbine Entry Temperature;
\item Inlet air mass flow.
\item Component efficiencies
\item Spool speeds
\item Bleed air mass flow
\end{itemize}
Once a representative engine model was obtained, off-design studies were performed in order to estimate the engine's performance at the other flight stages, in order to supply the inputs to the Aeronautics and AMS teams, such as traction, TSFC and bleed air temperatures and pressures at the engine's extraction ports.
Here we have some examples of the results generated by the software, to illustrate the quality of the model:
The figure \ref{fig:SLSandISA} shows the output of the simulation for an sea-level-static at ISA+15ºC, the design point of the simulation. As we can see, the net thrust at these conditions (FN) matches the TCDS data.

\begin{figure}[H] % Example image
\center{\includegraphics[width=400px]{Pictures/Aeronautics/Methodology/AE3007A1_SLS_ISA+15.eps}}
\caption{Model output at SLS and ISA+15 conditions}
\label{fig:SLSandISA}
\end{figure}


The figure X.X (segunda) represents the engine TSFC versus the engines net thrust as we vary the relative HPC spool speed, in other words, we change the thrust lever position, from 80 to 100\% open. Again, we can see that the model output matches the conditions found in the literature (Cruise TSFC at 37000ft and M=0,76 of approximately  0,6776 [lb/(lb*h)]) (DAP-GH8-070921, ERJ 155,1997).


\begin{figure}[H] % Example image
\center{\includegraphics[width=400px]{Pictures/Aeronautics/Methodology/TSFC.eps}}
\caption{TSFC versus Net Thrust at 37000ft and M=0,76}
\label{fig:TSFC}
\end{figure}


Figures X.X and X.X show the bleed air temperatures and pressures, varying with the altitude, some of the inputs for the AMS team.


\begin{figure}[H] % Example image
\center{\includegraphics[width=400px]{Pictures/Aeronautics/Methodology/Bleed.eps}}
\caption{Bleed Air Temperatures and Pressure versus Altitude}
\label{fig:BleedAir}
\end{figure}


\begin{figure}[H] % Example image
\center{\includegraphics[width=400px]{Pictures/Aeronautics/Methodology/Fan.eps}}
\caption{Outer Fan Exit temperature and pressure versus altitude.}
\label{fig:FanAir}
\end{figure}


