The performance methodology used to analyze the aircraft, was based in the concepts presented by engineer Jos\'e Renato O. Melo. And is described as follows below.
	The main parameters checked was:

\begin{itemize}
  \item Climb: rate of climb, time, consumption and distance as a function of altitude;
  \item Especific Range: as a function of speed, in typical altitudes;
  \item Descent : time, consumption and distance as a function of altitude;
  \item Holding : consumption as a function of speed(Equivalent Air Speed) and altitude
\end{itemize}


       (Data for ISA atmosphere and weight  between 70\% and 100\% of MTOW)
- Time and block consumption :
(for maximum number of pax, in Long Range and in maximum speed cruise, as a function of flight phase, until the maximum fuel in typical altitudes;

\begin{itemize}
  \item Payload x Range : in Long Range and maximum cruise speed, in typical  altitude;
  \item Take-off weight limited by gradient (WAT) : ISA to ISA+30, SL to 8000 ft;
  \item Take-off field length x TOW: ISA to ISA+15, Altitude sea level, 4000 ft;
  \item Landing field length x LDW :  sea level.
\end{itemize}


\textbf{Definitions}

\begin{figure}[H] % Example image
\center{\includegraphics[width=400px]{Pictures/Aeronautics/Methodology/forces.eps}}
\caption{Airplane Forces}
\label{fig:forces}
\end{figure}

x-axis:

$L-W\cos\gamma = 0 \Rightarrow L = W\cos\gamma$

y - axis:

$T - D - W\sin\gamma = W \frac{W}{g} \frac{dV}{dt} $


\textbf{Climb}

Gradient is a parameter to check if an aircraft is able to overcome an obstacle. Gradient of climb:


$\sin\gamma = \frac{T-D}{W} - \frac{1}{g}\frac{dV}{dt}$

$\sin\gamma = \frac{T-D}{W}$ or $\sin\gamma = \frac{T}{W} - \frac{C_{D}}{C_{L}}$

For small values of $\gamma$, $\sin\gamma = \gamma$

Rate of climb, measure the time it takes an aircraft to run a vertical distance . Equations:


$ROC = \frac{dH}{dt} = V\sin\gamma \therefore ROC = \frac{T-D}{W}V - \frac{V}{g}\frac{dV}{dH}\frac{dH}{dt} \therefore \frac{ \frac{T-D}{W}V   }{ 1 + \frac{V}{g}\frac{dV}{dH} }$


Total time to climb was calculated to a specific altitude, 39000 ft (from 0 ft). The altitude step was 1000 ft, for calculations. And total time was the sum of all steps. For each step the variation in time is equal to:

$\Delta = \frac{\Delta H}{ROC}$

Distance of climb to a variation (step) of altitude, is equal to:

$ \Delta d_{c} = V_{media} \Delta$

Total distance is the sum of all altitude variations.
	Consumption in the climb to a specific variation of altitude, is equal to:

$ \Delta C = \dot{m} \Delta t$ , where $ \dot{m}_{f} (fuel flow) = TSFC.T $

Total consumption is the sum of all consumptions of the each variation in altitude.



\textbf{Specific Range}
Another parameter to analyze is the Specific Range, defined as:

$SR = \frac{TAS}{\dot{m}}$ , where $\dot{m} = TSFC*T$ (for jets) and $ T = D = \frac{C_{D}}{C_{L}}W$

$C_{D} = C_{D0} + kC_{L}^{2}+C_{DW}$

To determine the Mach relative to best SR, the calculations were made for a range of Mach from 0.3 to 0.8.

\textbf{Descent}
	To analyze this phase the variation in altitude started from 0ft to 35000ft and used steps of 1000ft. The total time was the sum of all times of each step. The time of a 1000ft of altitude variation, is equal to:

$t = \frac{\Delta H}{RD}$\\
$ C = \dot{m}*t$,consumtion\\
$d_{d} = TAS\cos \gamma*t$, distance\\
$\sin \gamma = - \frac{RD}{TAS}$\\
$T = W \sin \gamma + D$, thrust\\
$\dot{m}_{f} = SFS*T$, (SFC in flight idle)


\textbf{Holding}

In this phase the drag is equal to thrust. The fuel flow is:

$\dot{m}_{f} = TSFC*T$
Speed in Holding (EAS), must be greater than 1.3 of stall velocity.

\textbf{Block Time}
Block time was analyzed for a specific number of pax, for two speeds, long range speed and maximum cruise speed.
	For some ranges, in typical cruise altitude, were determined:

\begin{itemize}
  \item Time, consumption and distance of climb;
  \item Time, consumption and distance of descent;
  \item $d_{cr} = d_{T} - d_{c} - d_{d}$ cruise distance is equal to total distance less distance of climb, less distance of descent;
  \item $C_{cr} = \frac{d_{cr}}{SR}$, Cruise consumption is equal to cruise distance divided by the specific range;
  \item $ t_{cr} = \frac{d_{cr}}{v_{cr}} $ , cruise time is the cruise distance divided by the cruise speed;

  \item Block Consumption, is the sum of consumption in taxiing, climb, cruise and descent.
  \item Block time, is the sum of the time taxiing, climb, cruise and descent.
\end{itemize}


\textbf{Fuel Reserves}
Fuel reserves is calculated with the sum of the block consumption(100nm) more holding fuel.
	Holding fuel was determined to a fuel consumption of 45 minutes, in holding condition, with the minimum consumption speed.
	
\textbf{Payload x Range}
Payload x range was considered with MTOW, in long range speed and maximum cruise speed. Then the payload was changed from zero to maximum.
	The total fuel is equal to MTOW, less BOW, less payload:
	
	$m_{ft} = MTOW - BOW - m_{payload}$
	
	Total fuel was limited to maximum fuel. So, the fuel reserves was calculated. In sequence, the block fuel was determined, equal to total fuel less fuel reserves.  Range was found with the reverse process used in topic BLOCK TIME.
	
\textbf{Take off weight limited by gradient}
Simplified equation to climb gradient is:
	$\sin \gamma = \frac{T}{W} - \frac{C_{D}}{C_{L}}$
	
	Requirements FAR:
	1st Segment: flap in take off position, landing gear down,  (velocity of lift off), the value of $\gamma = 0$
	2nd Segment: flap in take off position, landing gear up,  , the minimum value of $ \gamma = 2.4 \% $
	
	$\frac{C_{D}}{C_{L}}$ corresponding to take off configuration and $C_{L}$ of specified speeds $v_{lof}$ and $v_{2}$.
	With thrust, T, as a function of altitude and temperature, the TOW was determined.
	
\textbf{Take off field length x take off weight}
To find the value of take off field length,$S_{TOFL}$,was used the following equation:

$S_{TOFL} = k_{d}\frac{w^{2}}{TSC_{max}\sigma}$

Where w is the weight, T thrust, S wing area,$C_{Lmax}$  maximum lift coefficient in take off condition and $\sigma$ is the relation between local density and sea level density. The factor $ k_{d} $was adjusted with the manual of airplane to one point. All parameters were known, so with a graphic found in manual the value $k_{d}$ was calibrated to a specific weight and take off field length. All others values of take off field length were determined using same $k_{d}$

\textbf{Landingfield length x landing weight}

The method used in this topic, was the same in topic above. With some difference in equation.
$S_{FL}=k_{p}\frac{w}{SC_{Lmax}\sigma}$
Where $S_{FL}$ is the landing field length, and $k_{p}$  is the factor calibrated with the graph founded in manual to one point.
	
	

	
	
	
	
	


