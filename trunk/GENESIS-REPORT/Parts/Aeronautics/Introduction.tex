The Aeronautics / Propulsion study is focused on improvements that could bring a good ration between benefits and modification costs. The first step was to brainstorm opportunities and get feedback from EMBRAER engineers in order to raise our pool of modifications to study.
The second phase consisted on data collection, technical visits, technical reunions, literature study, analysis of requirements and build our calculation template. Those steps were necessary to obtain the costs and benefits of each modification. With those data at hand, a trade of was made not only within the aeronautic / propulsion team but with all the GENESIS team in order to decide which modification would go ahead.
The third and last step consisted on evaluation how the modification selected would impact on the plane performance and flight quality.
It's important to highlight this is an EP study, most of the modifications are in a level of depth just enough to be able to evaluate their macro contribution to the aircraft, most of the details would be developed on future project phase.

% Table generated by Excel2LaTeX from sheet 'Plan1'
\begin{table}[htbp]
  \centering
  \caption{Modification Selected}
    \begin{tabular}{rr}
    \toprule
    Modification & Status \\
    \midrule
    Winglet &  \\
    Active Winglet &  \\
    Aerodynamic Improvements &  \\
    Fuselage Stretch &  \\
    Flap  &  \\
    Single Pilot &  \\
    Tank Flammability &  \\
    Engine Change &  \\
    Alternative Fuel &  \\
    APU Replacement &  \\
    \bottomrule
    \end{tabular}%
  \label{tab:Modification Selected}%
\end{table}%

Some of the most important parameters used are listed bellow, as well the new aircraft design.


\begin{figure}[H] % Example image
\center{\includegraphics[width=400px]{Pictures/Aeronautics/Introduction/aircraftdesign.eps}}
\caption{Aircraft Design}
\label{fig:aircraftdesign}
\end{figure}


COLOCAR TABELA AQUI


The aeronautic / Propulsion report consists on a explanation of the methodology employed on the project for the most important calculations and computational models (6.2), followed by an explanation of each modification (6.3) and in the end the selected changes will be the inputs to evaluate the airplane performance and flight quality (6.4 and 6.5).
