
In order to decrease the DMC was proposed a better way to access the system on the aircraft.
This was a result from the technical meeting in Gavi�o Peixoto, so in the EP the most deep studies started.

Some insights about the problem as proposed as solution, like create a access window in the places
where has a high incidence of change, both in case of MTBUR and in the schedule maintenance.

This subject was discussed in the mentoring, and the mentor said the ERJ-145 don't have problem in accessibility,
because when they discuss this subject with the operators they say that ERJ-145 is a friendly to perform the changes, in terms of accessibility.

With the ERJ-145 maintenance manual, was realized that the aircraft has a good division on the floor of the airplane,
like showed in the figure \ref{fig:zonas_chao}. It is divided in zones and sub-zones, so the main problem was in the type of questions that was made in the technical meeting,
that guided the maintenance team to conclude the aircraft has a problem in accessibility. Once the questions was always like "what's the most problem in the maintenance/accessibility?".
 So the mechanicals showed where the problem is, not the ERJ-145 has a problem in this item.

%Incluindo figuras
\begin{figure}[H]
	%comando para centralizara  figura
	\centering
	%comando para a inclus�o da figura
	\includegraphics[width=200px]{Pictures/3MASU/zona_chao.eps}
	%Legenda
	\caption{ERJ-145 Floor panel division.}
	%Nome para se referir na figura no texto
	\label{fig:zonas_chao}
\end{figure}

After new discussion about this new interpretation about the technical meeting, the direction changed determine what take a considerable amount of time to remove. The result of this task
was a guide line when the interior team specify new seats that needed to be easy to remove (quick change), the same guide line in the case of monuments inside the aircraft. And the problem
in the case of the bathroom occurs when the check C needed to be performed to check some system that is bellow the bathroom, and it need to be all removed, including the side panels. So to
reduce a little the effort to remove the bathroom, a new guide line was proposed to create a new type of side panel.

\begin{itemize}	
	\item \textbf{Economic Analysis:}
\end{itemize}	

With the side panel change, the time won is speculated in 3 hours. Based on the information that was gathered in the field, the time to remove the bathroom is estimated in 18 hours, so with
the modification the time will reduce to 15 hours. Accumulating this change plus the extend in the time of check C, the cost per flight hour reduce from US\$ 0.47 to US\$ 0.33, a gain in US\$ 0.14.
Like showed in the table \ref{tab:eco_ninja_01}.

%Montando uma tabela
%Iniciando o ambiente de tabela
\begin{table}[H]
	%Indicando que ela deve ficar ao centro
	\centering
	%Legenda
	\caption{Man Hour Cost in the Bathroom Modification}
	%Desenhando a tabela
	\begin{tabular}{|c|c|c|c|c|}
		\hline
		 & \multicolumn{2}{|c|}{\textbf{Cost M/H:}} & \multicolumn{2}{|c|}{65.00} \\
		\cline{2-5}
		 & \textbf{Time} & \textbf{Cost} & \textbf{Interval} & \textbf{FH} \\
		\hline
		\textbf{Actual} & 36 & \$ 2,340.00 & 5,000 & \$ 0.47 \\
		\hline
		\textbf{New} & 30 & \$ 1,950.00 & 6,000 & \$ 0.33 \\
		\hline
	\end{tabular}
	%Dando um nome de refer�ncia a tabela
	\label{tab:eco_ninja_01}
	%Para referenciar no texto
	%\ref{tab:matriz_1}
\end{table}

In the case of galley, with the change and create a better access to the bolt that hold the door bearing, that is need to be accessed to perform a structural check each 10,000 cycles. And the check is estimated to be performed in 30 man hour plus 10 man hour to remove / install the galley. Since this task don't be more required, so the cost will reduce from US\$ 0.35 to US\$ 0.26, a gain in US\$ 0.09. Like showed in the table \ref{tab:eco_ninja_02}.

%Montando uma tabela
%Iniciando o ambiente de tabela
\begin{table}[H]
	%Indicando que ela deve ficar ao centro
	\centering
	%Legenda
	\caption{Man Hour Cost in Access the Bolt in the Galley Change}
	%Desenhando a tabela
	\begin{tabular}{|c|c|c|c|c|}
		\hline
		 & \multicolumn{2}{|c|}{\textbf{Cost M/H:}} & \multicolumn{2}{|c|}{65.00} \\
		\cline{2-5}
		 & \textbf{Time} & \textbf{Cost} & \textbf{Interval} & \textbf{FH} \\
		\hline
		\textbf{Actual} & 40 & \$ 2,600.00 & 7,518 & \$ 0.35 \\
		\hline
		\textbf{New} & 30 & \$ 1,950.00 & 7,518 & \$ 0.26 \\
		\hline
	\end{tabular}
	%Dando um nome de refer�ncia a tabela
	\label{tab:eco_ninja_02}
	%Para referenciar no texto
	%\ref{tab:matriz_1}
\end{table}

Last case studied is the access to the GPS / TCAS system unit control. With the information gathered in the field, the time spent on the system check was about 2 hours, and take 4 hours to remove the galley and more 4 hours to mount it again. Since this monument will be removed, and the system will be placed in another region with more easy accessibility, the estimated time lost in access time will be 1 hour. So this change will reduce from US\$ 0.13 to US\$ 0.03, a gain in US\$ 0.10. Like showed in the table \ref{tab:eco_ninja_03}.

%Montando uma tabela
%Iniciando o ambiente de tabela
\begin{table}[H]
	%Indicando que ela deve ficar ao centro
	\centering
	%Legenda
	\caption{Man Hour in Access GPS/TCAS in the Galley Change}
	%Desenhando a tabela
	\begin{tabular}{|c|c|c|c|c|}
		\hline
		 & \multicolumn{2}{|c|}{\textbf{Cost M/H:}} & \multicolumn{2}{|c|}{65.00} \\
		\cline{2-5}
		 & \textbf{Time} & \textbf{Cost} & \textbf{Interval} & \textbf{FH} \\
		\hline
		\textbf{Actual} & 10 & \$ 650.00 & 5,000 & \$ 0.13 \\
		\hline
		\textbf{New} & 3 & \$ 195.00 & 6,000 & \$ 0.03 \\
		\hline
	\end{tabular}
	%Dando um nome de refer�ncia a tabela
	\label{tab:eco_ninja_03}
	%Para referenciar no texto
	%\ref{tab:matriz_1}
\end{table}

With this changes the reduce on costs in a gain performance in access time to some systems is US\$ 0.33 US\$/FH.
