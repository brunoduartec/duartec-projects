
Another way to reduce the DMC is to increase the Mean Time Between Unscheduled Removals (MTBUR). This indicator shows the interval between two unexpected component failures, or in other words how long a component will operate without unnoticed failing.

The relation between DMC and MTBUR is shown in the equation below:

\[DMC_{unsch} =\left(\frac{TPA}{MTBUR} \right)\cdot \left(\% _{repair} \cdot ROC+\% _{scrap} \cdot PList+\% _{NFF} \cdot NFF_{FEE} +MH_{unsch\_ repair} \cdot LaborRate_{unsch} \right)\]

Where:
	TPA = Total parts per aircraft

	MTBUR = Mean Time Between Unscheduled Removal of the part

	\% repair = Percentage of repaired parts

	ROC = Repair Operational Cost

	\% scrap = Percentage of scraped parts

	PriceList = Price of a new part

	\% NFF = Percentage of Non Fault Found parts

	NFFfee = Non Fault Found Fee

    MHunsch\_repair = Man-hour time used in the part

    LaborRateunsch = Cost of the labor per hour

Increasing the MTBUR means that the component is more reliable than before and to achieve this there are four possible ways:

\begin{itemize}
	\item Improving or changing system design (redundancy, fault isolation, etc.)
	\item Change component design
	\item Change component for another more reliable
	\item Enhance maintenance procedures (for example guaranteeing that repairs are being done correctly)
\end{itemize}

The maintenance team selected the less reliable components of each ATA system based on the EMBRAER Service Performance Monthly Report of December of 2012, then each team worked in their own solutions using these inputs. To this project two ways of increase the MTBUR were chosen: change the components for more reliable ones or to improve system design.

\begin{figure}[H]
	\centering
	\includegraphics[width=400px]{Pictures/3MASU/Top10.eps}
	\caption{URR of the Top 10 less reliable ATA systems}
	\label{fig:Top10Maintenance}
\end{figure}

From the figure above the maintenance team considered the last 12 months of the Top 10 unreliable systems and each technology determined a list with the current component low reliability systems to make the trade-off analysis for exchanging the component and improve the reliability of the aircraft.

% Table generated by Excel2LaTeX from sheet 'Plan1'
\begin{table}[htbp]
  \centering
  \caption{Components MTBUR analysis}
    \begin{tabular}{rcccccccc}
    \toprule
    \multicolumn{1}{c}{\multirow{2}[4]{*}{\textbf{Description ( SPES )}}} & \textbf{MTBUR} & \textbf{New} & \multirow{2}[4]{*}{\textbf{Old Cost by airplane}} & \multirow{2}[4]{*}{\textbf{New Cost by airplane}} & \multirow{2}[4]{*}{\textbf{Total 10 year gain by airplane}} & \multirow{2}[4]{*}{\textbf{Trade-Off}} & \multirow{2}[4]{*}{\textbf{\%DMC Red.}} & \multirow{2}[4]{*}{\textbf{\%CASM Red.}} \\
    \midrule
    \multicolumn{1}{c}{} & \textbf{145} & \textbf{MTBUR} &       &       &       &       &       &  \\
    \multicolumn{1}{c}{\textbf{Text}} & \textbf{FH} & \textbf{} & \textbf{USD} & \textbf{USD} & \textbf{} & \textbf{} & \textbf{} & \textbf{} \\
    \textbf{STARTER GENERATOR - APU} & 4497  & 20000 & \$18.306 & \$3.398 & \$5.633 & Exchange & 0,15\% & 0,05\% \\
    \textbf{BATTERY, 44AH,NI-CD} & 9548  & 16280 & \$3.652 & \$2.146 & \$15.976 & Exchange & 0,12\% & 0,04\% \\
    \textbf{STATIC INVERTER} & 7854  & 23230 & \$2.304 & \$608 & \$84  & Exchange & 0,01\% & 0,00\% \\
    \textbf{GCU - GENERATOR CONTROL UNIT (APU)} & 4523  & 13000 & \$12.749 & \$4.440 & \$1.240 & Exchange & 0,06\% & 0,02\% \\
    \textbf{GUST LOCK ACTUATOR} & 18254 & 21905 & \$4.143 & \$3.466 & \$278 & Exchange & 0,00\% & 0,00\% \\
    \textbf{CABLAGE COMMAND} & 5000  & 10000 & \$10.090 & \$5.050 & \$3.990 & Exchange & 0,03\% & 0,01\% \\
    \textbf{FLAP TRANSMISSION BRAKE} & 9967  & 18000 & \$75.299 & \$41.778 & \$45.513 & Exchange & 0,22\% & 0,07\% \\
    \textbf{AILERON DAMPER} & 22496 & 9999999 & \$9.066 & \$0   & \$14.329 & Exchange & 0,06\% & 0,02\% \\
    \textbf{SPOILER CONTROL UNIT (EQUIPPED)} & 14884 & 22915 & \$2.686 & \$1.750 & \$36  & Exchange & 0,01\% & 0,00\% \\
    \textbf{BRAKE CONTROL UNIT} & 4500  & 16898 & \$39.575 & \$10.549 & \$7.012 & Exchange & 0,19\% & 0,06\% \\
    \textbf{WHEEL SPEED TRANSDUCER } & 60074 & 753685 & \$4.547 & \$367 & \$2.505 & Exchange & 0,03\% & 0,01\% \\
    \textbf{DOME LIGHT} & 80000 & 133333 & \$1.507 & \$919 & -\$1.946 & No Exchange & 0,00\% & 0,00\% \\
    \textbf{COCKPIT LIGHTS} & 70000 & 116667 & \$1.403 & \$241 & \$1.634 & Exchange & 0,01\% & 0,00\% \\
    \textbf{UPPER RED STROBE LIGHT (ANTI-COLLISION)} & 15951 & 79755 & \$5.406 & \$1.523 & \$241 & Exchange & 0,03\% & 0,01\% \\
    \textbf{WING TIP WHITE STROBE LIGHT} & 22000 & 110000 & \$6.324 & \$1.134 & \$589 & Exchange & 0,03\% & 0,01\% \\
    \textbf{INVERTER LAMP (CABIN PAX AND LAVATORY)} & 75000 & 125000 & \$21.469 & \$4.884 & \$714 & Exchange & 0,11\% & 0,03\% \\
    \textbf{READING LIGHT ASSY} & 200000 & 333333 & \$3.060 & \$1.035 & -\$118 & No Exchange & 0,01\% & 0,00\% \\
    \textbf{LANDING LIGHTS NLG} & 10000 & 2500000 & \$5.170 & \$9   & \$3.461 & Exchange & 0,03\% & 0,01\% \\
    \textbf{LANDING LIGHTS WINGS} & 15000 & 3750000 & \$3.667 & \$18  & \$248 & Exchange & 0,02\% & 0,01\% \\
    \textbf{TAXI LIGHTS NLG} & 6000  & 1500000 & \$8.017 & \$6   & \$7.010 & Exchange & 0,05\% & 0,02\% \\
    \multicolumn{1}{c}{\textbf{}} & \textbf{} & \textbf{} & \textbf{\$265.632} & \textbf{\$95.403} & \textbf{\$85.436} & \textbf{} & \textbf{1,20\%} & \textbf{0,36\%} \\
    \bottomrule
    \end{tabular}%
  \label{tab:componentsMTBUR}%
\end{table}%

The main objective of the table above is to perform a comparison between the maintenance costs between old and new components, so there was a trade-off analysis wich analyzed the exchange or non exchange the new component by each technology, the modification would be sustainable and profitable, generating a significant reduction or improvement to other systems indirectly.

From this situation, it was found that there was an improvement in MTBUR values, these values were based on values used by competitors as Boieng, Airbus, Bombardier and Embraer. Therefore it was possible to estimate the new MTBUR's for all components.

The maintenance team assume a investment to development the new components, in some cases the values is expensive and the modification is not profitable, besides that for other systems is necessary to exchange the components for example: in the last one it is necessary to exchange for reduce the fuel consumption.

With these solutions the maintenance team estimates that the reduction in the DMC will be approximately 1.20 \% and the CASM will be 0.36 \%, the exact value depends on the final alternative provided by all the other teams.
