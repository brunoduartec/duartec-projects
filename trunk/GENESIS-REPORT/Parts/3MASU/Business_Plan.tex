
At the beginning of the ERJ-145 project, EMBRAER has developed a system of partnership with its key suppliers, high cost component suppliers. These partners are responsible for providing components to production and marketing and services of maintenance for the aircraft. With the imminent reduction in ERJ-145 operation these suppliers need to help EMBRAER reducing costs in order to keep the fleet flying and prolong the input amount.

The main partners of this project are Roll-Royce, Hamilton Sundstrand, ELEB-Liebher and Goodrich that are responsible for development of AE 3007 engine, Nose and Main Landing Gear, Wheels/Brakes and APU. The engine is the most expensive system of the project, and today it is responsible for 50 \% of the maintenance cost of the aircraft, the other suppliers represent 10 \%.

In 2009, EMBRAER proposed to Rolls Royce an increase of the time between engine overhauls of 20,000 flight hours for 30,000 flight hours, trying to reduce the engine maintenance cost. This change results in an increase of 97 \% of the Life Limited Parts (LLP) cost, and therefore most of the operators have joined the Total Care program, but it is very expensive for operators with small fleets. Today it is around US\$ 1,500,000.00.

In an attempt to keep the fleet in operation, it will be proposed to Rolls Royce a reduction of the price of the Total Care and LLP costs, in order to make the operation economically feasible.

With an estimated cost of US\$ 300.00 that represent 50 \% of the maintenance per flight hour for the Total Care program and a current membership of 85 \% of aircraft, it is calculated an estimated amount that Rolls Royce should receive for services in the next years.

From this situation, it will be taken a study that will assuming a premise that the business plan was accepted by the Partner and so there will be a reduction of 10 \% of the DMC Cost by Rolls Royce, that will increase the amount of Profits in 139 \% in 10 years. The detailed study is shown on the figure below.

\begin{figure}[H]
	\centering
	\includegraphics[width=400px]{Pictures/3MASU/RollsRoyce.eps}
	\caption{Estimated amount of Rolls Royce Services}
	\label{fig:RollsRoyce}
\end{figure}

The figure above shows that the estimated perspective is more profitable than the current estimated, therefore increase the fleet flying and prolong the life-cycle of the airplane until 2035 and maintain the operation profitable.

Others important project partners are Hamilton Sundstrand (APU), ELEB-Liebher (landing gear) and  Goodrich (wheels and brakes). Following the same reasoning of the business plan proposed to Rolls Royce, will be presented to these partners another plans to propose the DMC reduction.

Assuming a premise that the business plan was accepted by other partners and so there will be a reduction of 2.08 \% of the DMC Cost by Hamilton Sundstrand, ELEB-Liebher and Goodrich. The detailed study is shown on the figures below.

\begin{figure}[H]
	\centering
	\includegraphics[width=400px]{Pictures/3MASU/ELEBLiebher.eps}
	\caption{Estimated amount of Eleb-Leibher Services}
	\label{fig:ELEBLiebher}
\end{figure}

\begin{figure}[H]
	\centering
	\includegraphics[width=400px]{Pictures/3MASU/Goodrich.eps}
	\caption{Estimated amount of Goodrich Services}
	\label{fig:Goodrich}
\end{figure}

\begin{figure}[H]
	\centering
	\includegraphics[width=400px]{Pictures/3MASU/HamiltonSundstrand.eps}
	\caption{Estimated amount of Hamilton Sundstrand Services}
	\label{fig:HamiltonSundstrand}
\end{figure}

In conclusion, to achieve these values it is necessary to negotiate with all suppliers and show them the importance of reducing the cost of maintenance of the aircrafts, so as to keep the fleet flying and thus increase the profits of the partners around 100 \% until 2036.
