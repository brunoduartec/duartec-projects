To be possible to increase the number of passengers, it is necessary to change the aircraft's LOPA. Besides resizing the galley, reducing the size of pitch from 31 inches to 29 inches is also needed to achieve the project's goal. One concern the fact that decreasing the pitch can also bring discomfort to the passengers, so the solution was chose a new seat that provides an enjoyable experience for passengers even reducing the pitch size.

\begin{figure}[H]
\center{\includegraphics[width=400px]{Pictures/Interior/SlimSeat_Isometric.eps}}
\caption{Slim Seat}
\label{fig:slimseatisometric}
\end{figure}

The C\&D Zodiac, under Embraer's specification, has developed a new seat, known as Slim Seat, as seen in Figure \ref{slimseatisometric} and Figure \ref{slimseatside}, which has certain characteristics what matters to the EMB-145's modernization, such as 2.5 kg less in each dual seat configuration in comparison to the original version, called Elite, which have 24 kg.

\begin{figure}[H]
\center{\includegraphics[width=400px]{Pictures/Interior/SlimSeat_Side.eps}}
\caption{Slim Seat's side view}
\label{fig:slimseatside}
\end{figure}

The Slim Seat was initially developed for the ERJ-190's aircraft family, so some changes are necessary to proper compatibility with the EMB-145. These changes are summarized in reducing the width of the seat and the side rail's fixation. Fortunately, all these modifications have been made and, therefore, it is not necessary to redesign these seats. Also, the Slim Seat has already been certified.
