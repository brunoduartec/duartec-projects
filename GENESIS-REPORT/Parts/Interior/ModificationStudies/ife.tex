% 5.2.2 - IFE
Although the installation of an In Flight Entertainment System (IFE) does not reflect directly in the reduction of operating costs or in the increase of the revenue of the airline, it is a subject that must be taken into consideration, as it can increase value and perception of the aircraft among passengers.

As a first analysis, all modern IFE systems were considered. for video entertainment, In Seat Video Monitors were compared to the Flip Down. The In Seat monitors would increase too much the weight of the aircraft, which is already in a critical state, and the cost would be to high. And the flip down monitors would reduce the usable height of the interior, that is not so big already.

The typical flight for the ERJ-145 is of about one hour. That means that there is maximum half an hour of cruise. The use of monitors would be minimum, in just that time, and they turn out not being necessary, comparing the few benefits with the large costs and weigh.
The needs of passengers in the future must be considered. A persona workshop showed that today's passenger are increasingly worried about connectivity. Everyone wants to stay connected all the time. And everyone brings their tablets and smart phones anywhere. Passengers are worried about internet and a place to charge their batteries.

To fulfill the needs mentioned, a trade-off study was made for Wi-Fi and power outlets. Values such as weight, costs and power consumption were obtained. The table bellow shows that data.

% Table generated by Excel2LaTeX from sheet 'Plan1'
\begin{table}[htbp]
  \centering
  \caption{IFE components}
    \begin{tabular}{ccccc}
    \toprule
    \textbf{IFE} & \textbf{Component} & \textbf{Power Usage (W)} & \textbf{Weigh (kg)} & \textbf{Cost (US\$)} \\
    \midrule
    \multirow{2}[1]{*}{Wi-Fi} & Terrestrial Broadband & 250   & 40    & 100,000.00 \\
          & Wireless Access Point & 34    & 9     & 15,000.00 \\
    \multirow{2}[1]{*}{Power Outlets} & AC In-Seat Power Box  & max 5000 & 38    & 15,000.00 \\
          & Master Control Unit & 8.6   & 7.5   & 40,000.00 \\
    \bottomrule
    \end{tabular}%
  \label{tab:ifecomponents}%
\end{table}%

In the table above, the price shown is for 20 AC In-Seat Power Boxes. As each box is capable of supplying 3 outlets, this number of boxes is enough. The results are 284 W, 49 kg and US\$ 115,000.00 for the Wi-Fi and 5 kW, 45.5 kg and US\$ 55,000.00 for the AC power outlets.

Considering these values for the Wi-Fi, the cost is too high, and goes against the main purpose of the project, despite the persona studies. The cost of US\$ 115,000.00 with no direct income increase goes harshly against the maintenance of the aircraft in the market. Besides that, the operation cost would increase, as the airliner would have to pay for some internet service. And again, as the flight time is usually short, the need for the internet is not so evident. Therefore, this change will not be implemented.

A similar analysis can be made for the AC power outlets. The total cost was of  US\$ 55,000.00, just for the material. Development, certification and installation costs are not included yet. Again, this change does not increase the income of the airliner, and even increase the maintenance costs. The main reason of these high prices are that AC power outlets require inverter boxes, as the electrical system of the ERJ-145 is 28 V DC.

With that in mind, another solution was developed. A DC power outlet would not need inverters and the circuitry would be much simpler, and the power consumption smaller. A research was made to prove that DC power outlets would satisfy the passengers.

Today's electronics can almost always be charged by some kind of USB ports. The most modern smart phones and tablets can connect to a computer via USB, and come with chargers that are simple adapters from the standard AC wall outlet to 5 V USB, just with higher current capability than a PC USB port. In this way, this kind of outlet would be enough for the ERJ-145. The only limitation would be that this kind of power source would not fit laptop computers.

In nearly half an hour of cruise per flight, passengers will rarely have time for turning on computers, and work on them. People would rather play some game in their more portable devices, easier to carry. The most popular tablet nowadays, Apple's iPad, can be charged with full capacity in an USB outlet that can give a current of 2 A.

Besides that, even the sales of notebooks is falling, and the tablet sales are increasing. Studies show that in the year 2013 the sales of tablets will be higher in the world than the sales of notebook computers. In 2012, this already happened in the US and in China. This data was obtained from the NPD display search. The image below shows that predictions.

\begin{figure}[H] % Example image
\center{\includegraphics[width=300px]{Pictures/Interior/tabletvsnotebook.eps}}
\caption{Notebook and Tablet sales comparison for the next years}
\label{fig:tabletvsnotebook}
\end{figure}

% 5.2.2.1 - Power Outlets
\textbf{Power Outlets (RI2)}

The graph and the text above describes why the USB ports will be enough as IFE for the ERJ-145. The present section has as purpose of describing how the solution will be implemented, with details such as position of the outlets, power consumption, and thermal dissipation.

% 5.2.2.1.1 - Outlet Position
\textit{Outlet Position}

A key factor when putting power outlets in aircrafts is their position. The position must be of easy access, safe, and do not obstruct passenger movement. Three positions were at first considered. In the back of the seat in front of the passenger, under the seat, or in the armrest.

The pros of the back of the seat USB outlet is the ease of access, and the obvious location. The passenger will easily find the outlet and charge their equipment. The cons are that the wire will obstruct the passage, and the equipment will be far from the user, if his or her cable is short.

The second possibility is under the Seat. Although, this was discarded, because of the access being hard for passengers, and the outlet would be hard to find. The cable would also limit the usage of the device.

As a last possibility, the outlet could be placed in the armrest. This position is easily visible and accessible, and close to the passenger, making the device easy to use while charging. And there is another advantage, as the slim seat comes prepared for earphones, the same structure can be used to the USB outlet, and the cables for the port can be placed where the cables for the earphones would.

Therefore, considering the aspects for the three possibilities, and the pros and cons, the last alternative was chosen. The image below shows a possible implementation of the USB power outlets.



\begin{figure}[H] % Example image
\center{\includegraphics[width=400px]{Pictures/Interior/tomada_usb}}
\caption{Implementation of a USB power outlet in the armrest}
\label{fig:usbpoweroutlet}
\end{figure}

% 5.2.3.3 - Power Usage
\textit{Power Usage}

For the determination of how much power will be needed in each USB port, the specifications of the electronic devices expected to be charged must be considered. As the target of this ports are portable electronics, in other words, smart phones and tablets, the worst case are big tablets, that needs a higher current to charge.

The common large tablets found today, such as the Apple's iPad and Sansung's Galaxy Tab, charges at full capacity with 5 V and 2 A. A 10 W USB port would suffice. With this in mind and 60 ports, one per passenger, the total power consumption in the worst case scenario will not exceed 600 W. The availability of this power is discussed with more detail in the electrical systems section. But, for instance, the power savings of the LED lamps replacement can be used to power the IFE system.

% 5.2.3.5 - Thermal Dissipation
\textit{Thermal Dissipation}

USB outlets consumes power, and so, dissipate heat. For the AMS system analysis, this must be observed. With people changing their electronic devices, the current will dissipate heat, and so will the portable device. Although, with a maximum power usage of 600 W, the heat increase will not exceed this value. Again, a more complete analysis of the heat in the cabin is presented at the AMS section of the report.
