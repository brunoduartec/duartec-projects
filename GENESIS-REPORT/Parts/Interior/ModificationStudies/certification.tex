The ERJ 145 like showed in the figure 07 has two exits type I and two type III. Thus, increasing the number of passenger, there is no necessity in to increase or put more emergency exits, because according with FAR Part 25 Sec 25.807, that treats about emergency exits, the quantity, disposition and types of emergency exits, that already exist on ERJ 145 satisfy these requirements.

\begin{figure}[H]
\center{\includegraphics[width=400px]{Pictures/Interior/145-exits.eps}}
\caption{Top view with the types of exits of ERJ 145.}
\label{fig:145-exits}
\end{figure}


Below, are transcribed some paragraphs from Sec 25.807 that show how the ERJ 145 attend the requirements, even increasing to 10 passenger. The figure  show how the new Lopa attends at requirements of item 04 of Sec 25.807.

\textbf{Emergency exits}

(4) For an airplane that is required to have more than one passenger emergency exit for each side of the fuselage, no passenger emergency exit shall be more than 60 feet from any adjacent passenger emergency exit on the same side of the same deck of the fuselage, as measured parallel to the airplane's longitudinal axis between the nearest exit edges.


(g) \emph{Type and number required}. The maximum number of passenger seats permitted depends on the type and number of exits installed in each side of the fuselage. Except as further restricted in paragraphs (g)(1) through (g)(9) of this section, the maximum number of passenger seats permitted for each exit of a specific type installed in each side of the fuselage is as follows:

Type A 110\\
Type B 75\\
Type C 55\\
Type I 45\\
Type II 40\\
Type III 35\\
Type IV 9\\

(5) For a passenger seating configuration of 41 to 110 seats, there must be at least two exits, one of which must be a Type I or larger exit, in each side of the fuselage.


\begin{figure}[H]
\center{\includegraphics[width=400px]{Pictures/Interior/lopameasuresevacuation.eps}}
\caption{New lopa with measures that attend at item 04 - Sec 25.807}
\label{fig:lopameasuresevacuation}
\end{figure}


The evacuation time in the certification of ERJ 145 is 85 seconds. Hence, might be necessary to do another evacuation test or combination of analysis and testing with this new configuration of 60 passenger, for to proof that is possible evacuate all passenger and crew members in 90 seconds, attending the Sec 25.803 that is transcribed below.

(c) For airplanes having a seating capacity of more than 44 passengers, it must be shown that the maximum seating capacity, including the number of crewmembers required by the operating rules for which certification is requested, can be evacuated from the airplane to the ground under simulated emergency conditions within 90 seconds. Compliance with this requirement must be shown by actual demonstration using the test criteria outlined in Appendix J of this part unless the Administrator finds that a combination of analysis and testing will provide data equivalent to that which would be obtained by actual demonstration.

