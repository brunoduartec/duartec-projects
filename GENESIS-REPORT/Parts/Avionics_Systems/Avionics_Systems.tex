\subsection{Introduction}

The avionics system of the ERJ 145 is based on the Honeywell primus 1000 avionics that was designed at the decade of 1980 as a federated architecture that has as its main components a Integrated Computer (IC 600), Data Acquisition Units (DAU), Flight Management System (FMS), Radio Management Units (RMU) and five Display Units (DU).

\begin{figure}[H] % Example image
\center{\includegraphics[width=350px]{Pictures/Avionics_Systems/ERJ145Cockpit.eps}}
\caption{ERJ 145 cockpit.}
\label{fig:ERJ145Cockpit}
\end{figure}

In the last decades, new technologies in navigation and surveillance were developed and they will be required or mandatory in the coming years as the RNP (Required navigation performance) and ADS-B (Automatic dependent surveillance-broadcast) which will be needed for operate in most part of the world. There are also other functionalities that will possibly become mandatory in some parts of the world principally in Europe as the CPDLC and the RF (Radius to fix).
\\The motivation for mostly of modifications on the avionic system was try to attend the new exigencies in communication, navigation and surveillance to make the aircraft able to keep flying without restrictions and with better performance.

\subsubsection{ADS-B - Automatic Dependent Surveillance-Broadcast}
The Automatic dependent surveillance-broadcast is part of the next generation of the airspace control and consists in the aircraft transmitting its information such as identification, altitude, velocity and current position in real time what can be used for improve the air traffic control due to the more precise information compared to the radar-based systems.
\begin{figure}[H] % Example image
\center{\includegraphics[width=350px]{Pictures/Avionics_Systems/ADSBSchema.eps}}
\caption{ADS-B - Automatic dependent surveillance-broadcast.}
\label{fig:ADSBSchema}
\end{figure}

ADS-B-out, that consists only in broadcast the information is foreseen to become mandates already in 2014 in Australia and after in Europe in 2017 for retrofit. The requirement at the United States is foreseen for 2020 but can be anticipated for 2018. The ADS-B-in have no information about any requirement before 2020.

\subsubsection{CPDLC - Controller-pilot Data Link Communications}
The Controller-pilot data link communications tries to solve communication problems due to the crescent number of aircrafts and the limits of the voice communications channels. This technology uses data link to send and receive messages relative to the air traffic control such as flight levels change, lateral deviations and route information.
\begin{figure}[H] % Example image
\center{\includegraphics[width=350px]{Pictures/Avionics_Systems/CPDLCmessages.eps}}
\caption{CPDLC messages.}
\label{fig:CPDLCmessages}
\end{figure}
The CPDLC is foreseen to become mandatory in Europe before the end of the decade  trying to optimize the communication, the air traffic control and improve safety. However there is not any information about that requirement in United States and other regions.


\subsubsection{RNP - Required Navigation Performance}
The RNP is part of the FAA's next-generation performance-based navigation system and utilizes global positioning systems and inertial reference system navigation to fly more precise paths which are previously in the computer database. That navigation functionality can reduce the flight time and fuel consumption due to the optimized flight paths and provide more precise and repeatable flight routes.

\begin{figure}[H] % Example image
\center{\includegraphics[width=350px]{Pictures/Avionics_Systems/NavigationTechnologiesComparison.eps}}
\caption{Navigation technologies comparison.}
\label{fig:NavigationTechnologiesComparison}
\end{figure}

Those advantages can provide a for the aircraft a better operation ensure that it will not be its operation prejudiced due to less precise navigation capabilities.
\\The RF (Radius to fix) consists in a technology part of the performance-based navigation that provides some capabilities for curved paths, approach procedures and increases accuracy during turn in some segments. RF is foreseen to become mandatory in Europe until the end of the decade as part of the Advanced RNP. There is not any other information about the exigency for that modification in other countries.

\subsection{Modification Studies}

Based on the scenario described, the electro-electronic systems team chose the modifications that will be presented below. The first decision was to maintain the current avionic platform discarding the alternative of change the entire system.
\\For try to adapt the aircraft for the new CNS requirements, it will be include in the aircraft Vertical Navigation (VNAV), Required Navigation Performance (RNP),  and Automatic dependent surveillance-broadcast (ADS-B). The addition of those functionalities is described below.

\subsubsection{VNAV - Vertical Navigation (CE1-E1)}
The VNAV (Vertical Navigation) can provide a vertical guidance that can reduce the pilot workload and optimize the fuel consumption due to the improved flight profile. VNAV utilizes pre-programmed flight plans in the FMS to fly vertical profiles which are available during the full flight and can support the approach and landing operations.

\begin{figure}[H] % Example image
\center{\includegraphics[width=350px]{Pictures/Avionics_Systems/AircraftLanding.eps}}
\caption{Aircraft Landing.}
\label{fig:AircraftLanding}
\end{figure}

The inclusion of VNAV will demand some modification in software and hardware on the aircraft. The first one consists in change the GC-550 panel for a new one that includes a VNAV button. It will also need some cable changes to connect the panel to the IC-600. A software modification in the FMS computer is also necessary to include that functionality.

\begin{figure}[H] % Example image
\center{\includegraphics[width=350px]{Pictures/Avionics_Systems/ModifiedGC550.eps}}
\caption{Modified GC-550.}
\label{fig:ModifiedGC550}
\end{figure}

That modification is already available for the Legacy 600/650 and the ERJ 145 shuttle version what can simplify and reduce the development time and cost. It is estimated about 6 months to develop and certificate the VNAV for the others ERJ 145 versions. The pilots training is estimated in a few hours in class and 1 hour in a simulator.
\\The necessary modifications can be made at the operator maintenance center based in some service bulletins issued to the ERJ 145 and Honeywell's service bulletins. The parts supply is made by Honeywell.

\subsubsection{RNP - Required Navigation Performance (CE1-E3)}
It will be implemented the RNP 0.3 that can provide a small separation between aircrafts in the same area. This functionality demands only a software modification in the system. That modification is already available for the Legacy 600/650 and the ERJ 145 shuttle version what can simplify and reduce the development time and cost. It is estimated about 6 months to develop and certificate RNP for the others ERJ 145 versions. The pilots training is estimated 1 hours in class.  The necessary modifications can be made at the operator maintenance center based in some service bulletins issued to the ERJ 145 and Honeywell's service bulletins.

\subsubsection{LPV - Localizer Performance with Vertical Guidance (CE1-E2)}

LPV utilizes the WAAS (Wide Area Augmentation System), which improves the GPS accuracy using ground stations, to provide guidance on approaches operations. The functionality improves approaches capabilities supplying ILS operations without the required structure and supporting operations in poor weather conditions. 300 airports in United States are expected to adopt this procedure every year.

\begin{figure}[H] % Example image
\center{\includegraphics[width=350px]{Pictures/Avionics_Systems/WideAreaAugmentationSystem.eps}}
\caption{Wide Area Augmentation System.}
\label{fig:WideAreaAugmentationSystem}
\end{figure}

The functionality demands the replacement of the GPS system for a more precise equipment what includes new antenna and a new receiver. The equipments can be installed in the original positions, however some cable modifications could be needed. A software modification in the FMS computer is also necessary to enable LPV functionalities.

\begin{figure}[H] % Example image
\center{\includegraphics[width=350px]{Pictures/Avionics_Systems/GPSReceiverLocation.eps}}
\caption{GPS receiver location.}
\label{fig:GPSReceiverLocation}
\end{figure}

That modification is already available for the Legacy 600/650 and the ERJ 145 shuttle version what can simplify and reduce the development time and cost. It is estimated about 6 months to develop and certificate LPV for the others ERJ 145 versions. The pilots training is estimated in 1 hour in class.
\\The necessary modifications can be made at the operator maintenance center based in some service bulletins issued to the ERJ 145 and Honeywell's service bulletins. The parts supply is made by Honeywell.


\subsubsection{ADS-B - Automatic Dependent Surveillance-Broadcast (CE1-E4)}
The ADS-B implementation will demand some equipment modification and connection changes. The GPS system have to be actualized for a more precise system. The connection routing in the IC 600 and some connections between the equipments will also demand modifications. It will also be necessary software modifications and updates in the IC600, RMU and transponder.
\\That modification still in development for the Legacy 600/650 family and it is not available for the ERJ 145 shuttle version, so it is estimated about 1 year to develop and certificate ADS-B for the ERJ 145. The pilots training is estimated in a few hours in class.
\\The necessary modifications can be made at the operator maintenance center based in some service bulletins issued to the ERJ 145 and Honeywell's service bulletins. The parts supply is made by Honeywell.


\subsubsection{Trade-off analysis}
The decision of keeping the current avionics system and include some functionalities was made based on the analysis of the advantages and problems of each solution.
\\The original avionics with some modifications have the advantage of the cost and time of work due to the smaller modifications involved and the functionalities are already developed or in development for the Legacy 600/650 and for the ERJ145 Shuttle version. It is also possible consider a price negotiation with Honeywell due to the number of aircrafts. However, it is not possible to attend all mandates with this proposal, so it is necessary to consider a permission for operate or problems to fly in Europe.
\\The option of change all the avionics system can solve the problems with navigation, communication and surveillance, however it is a more expensive solution due to the necessity of develop the complete platform for the aircraft and change all the equipments of a aircraft in service. The possibility of a support to the development and certification campaign by the manufacturer of the avionics can be a alternative to reduce costs.
\\The choice for keeping the current avionics was based on a comparison between the presented analysis and the business model of the project.



\subsection{Avionics Costs Summary}
The functionalities that will be implemented are already developed for the Legacy 600/650 family what can reduce the development and certification costs. So it is estimated a non recurrent cost between \$ 1,000,000 and \$ 1,500,000.
\\The implementation of the modifications in each aircraft will depend on the parts supplied for the Honeywell and man-hour needed to modify the aircrafts. The recurrent cost is estimated between \$ 300,000 and \$ 400,000 what can change depending on negotiations with Honeywell.


\begin{figure}[H] % Example image
\center{\includegraphics[width=400px]{Pictures/Avionics_Systems/CostSum1.eps}}
\label{fig:AVIOCostSum1}
\end{figure}

\begin{figure}[H] % Example image
\center{\includegraphics[width=400px]{Pictures/Avionics_Systems/CostSum2.eps}}
\label{fig:AVIOCostSum2}
\end{figure}


\subsection{Conclusion}

The modifications described in this work have the objective to try to attend the new requirements in communication, navigation and surveillance and allow  the aircraft to keep flying without constraints.
\\After a trade-off analysis, it was decides to implement a set of new functions (VNAV, RNP, LPV and ADS-B) that can improve the aircraft navigation capabilities, reduce flight time and fuel consumption and improve safety. However, it is not possible to attend all the requirements foreseen due to limitations of the avionic system, so it was necessary to consider some operational scenarios.
\\The cost of development and certification for those functionalities is small due the service bulletins that already exist to Legacy 600/650 and ERJ 145 shuttle version. The time required to modify the aircraft and train the pilots is short because there is no great changes on the system architecture and interface.










