% 10.4 - Electrical Load Analysis
To make the ERJ-145's electrical load analysis the group received some material referring to the current electrical load of the aircraft. Such material is confidential and cannot be presented outside Embraer, therefore, the numbers shown in this report were manipulated and do not picture the reality of the aircraft's electrical loads.

The following table displays all the ERJ-145's continuous electrical loads measured in current, during each flight phase.

% Table generated by Excel2LaTeX from sheet 'Plan1'
\begin{table}[htbp]
  \centering
  \scriptsize
  \caption{Current Electrical Loads}
    \begin{tabular}{rccccccccc}
    \toprule
    \multicolumn{1}{c}{ATA Chapter} & Taxi  & Take-off and Climb & Cruise & Landing & Taxi (2 gen. fail) & Take-off and Climb (2 gen. fail) & Cruise  (2 gen. fail) & Landing  (2 gen. fail) & Electrical Emergency \\
    \midrule
    21 - Air Coditioning and Pressurization & 96    & 94    & 91    & 91    & 31    & 29    & 27    & 27    & 0 \\
    22 - Auto-Flight & 4     & 4     & 4     & 4     & 4     & 4     & 4     & 4     & 0 \\
    23 - Communications & 21    & 20    & 18    & 20    & 19    & 19    & 17    & 19    & 9 \\
    24 - Electrical Power & 31    & 31    & 22    & 22    & 14    & 14    & 14    & 14    & 5 \\
    25 - Equipment / Furnishings & 15    & 14    & 14    & 0     & 0     & 0     & 0     & 0     & 0 \\
    26 - Fire Protection & 1     & 1     & 1     & 1     & 1     & 1     & 1     & 1     & 0 \\
    27 - Flight Controls & 8     & 7     & 5     & 11    & 8     & 7     & 5     & 11    & 1 \\
    28 - Fuel & 94    & 94    & 94    & 94    & 72    & 72    & 72    & 72    & 22 \\
    29 - Hydraulic Power & 0     & 0     & 0     & 0     & 104   & 104   & 83    & 104   & 0 \\
    30 - Ice and Rain Protection & 232   & 260   & 236   & 260   & 152   & 180   & 156   & 180   & 9 \\
    31 - Indicating / Recording System & 34    & 34    & 34    & 34    & 34    & 34    & 34    & 34    & 11 \\
    32 - Landing Gear & 13    & 11    & 7     & 11    & 13    & 11    & 7     & 11    & 6 \\
    33 - Lights & 221   & 141   & 110   & 141   & 138   & 107   & 88    & 107   & 4 \\
    34 - Navigation & 36    & 41    & 41    & 41    & 37    & 41    & 41    & 41    & 9 \\
    35 - Oxygen & 0     & 0     & 0     & 0     & 0     & 0     & 0     & 0     & 0 \\
    36 - Pneumatic & 5     & 2     & 5     & 5     & 4     & 2     & 4     & 4     & 2 \\
    38 - Water / Waste & 4     & 4     & 5     & 4     & 4     & 4     & 5     & 4     & 0 \\
    45 - Diagnostic and Maintenance & 0     & 0     & 0     & 0     & 0     & 0     & 0     & 0     & 0 \\
    49 - Airborne Auxiliary Power & 0     & 0     & 0     & 0     & 0     & 0     & 0     & 0     & 0 \\
    76 - Engine Controls & 3     & 0     & 0     & 0     & 6     & 3     & 3     & 3     & 3 \\
    77 - Engine Indicating & 0     & 0     & 0     & 0     & 0     & 0     & 0     & 0     & 0 \\
    78 - Exhaust & 0     & 0     & 0     & 0     & 0     & 0     & 0     & 0     & 0 \\
    80 - Starting & 0     & 0     & 0     & 0     & 0     & 0     & 0     & 0     & 0 \\
    \textbf{TOTAL} & \textbf{821} & \textbf{763} & \textbf{690} & \textbf{743} & \textbf{642} & \textbf{636} & \textbf{562} & \textbf{638} & \textbf{83} \\
    \bottomrule
    \end{tabular}%
  \label{tab:CurrentEleLoads}%
\end{table}%

The modifications that directly affect the electrical system in terms of electrical load were three: changes in the lighting system from the current lamps to LED, addition of 5 V power jacks for passengers and the addition of an ECS Controller. The USB power jacks provide 2 A at 5 V. Since the project contemplates 60 jacks (one for each passenger), and the estimated efficiency of each converter is 80 \%, the total power consumed by the system is 750 W. This power at 28 V means 27 A of electrical current at a maximized used of the system, and since it will be added to the two shed buses, it will stop providing power when 2 generators fail or at electrical emergency. The power jacks will be supplied during all flight phases except landing.  The ECS Controller consumes 160 W at 28 V, which means it consumes a current of 6 A at all flight phases, besides while the airplane is in electrical emergency. These three modifications will change the electrical load analysis' table, and the results are shown bellow.

% Table generated by Excel2LaTeX from sheet 'Plan2'
\begin{table}[htbp]
  \centering
  \scriptsize
  \caption{New Electrical Loads}
    \begin{tabular}{rcccccccp{2cm}p{2cm}}
    \toprule
    \multicolumn{1}{c}{ATA Chapter} & Taxi  & Take-off and Climb & Cruise & Landing & Taxi (2 gen. fail) & Take-off and Climb (2 gen. fail) & Cruise  (2 gen. fail) & Landing  (2 gen. fail) & Electrical Emergency \\
    \midrule
    21 - Air Coditioning and Pressurization & 96    & 94    & 91    & 91    & 31    & 29    & 27    & 27    & 0 \\
    22 - Auto-Flight & 4     & 4     & 4     & 4     & 4     & 4     & 4     & 4     & 0 \\
    23 - Communications & 21    & 20    & 18    & 20    & 19    & 19    & 17    & 19    & 9 \\
    24 - Electrical Power & 31    & 31    & 22    & 22    & 14    & 14    & 14    & 14    & 5 \\
    25 - Equipment / Furnishings & 42    & 41    & 41    & 0     & 0     & 0     & 0     & 0     & 0 \\
    26 - Fire Protection & 1     & 1     & 1     & 1     & 1     & 1     & 1     & 1     & 0 \\
    27 - Flight Controls & 8     & 7     & 5     & 11    & 8     & 7     & 5     & 11    & 1 \\
    28 - Fuel & 94    & 94    & 94    & 94    & 72    & 72    & 72    & 72    & 22 \\
    29 - Hydraulic Power & 0     & 0     & 0     & 0     & 104   & 104   & 83    & 104   & 0 \\
    30 - Ice and Rain Protection & 238   & 266   & 242   & 266   & 158   & 186   & 162   & 186   & 9 \\
    31 - Indicating / Recording System & 34    & 34    & 34    & 34    & 34    & 34    & 34    & 34    & 11 \\
    32 - Landing Gear & 13    & 11    & 7     & 11    & 13    & 11    & 7     & 11    & 6 \\
    33 - Lights & 116   & 63    & 77    & 63    & 76    & 45    & 72    & 45    & 4 \\
    34 - Navigation & 36    & 41    & 41    & 41    & 37    & 41    & 41    & 41    & 9 \\
    35 - Oxygen & 0     & 0     & 0     & 0     & 0     & 0     & 0     & 0     & 0 \\
    36 - Pneumatic & 5     & 2     & 5     & 5     & 4     & 2     & 4     & 4     & 2 \\
    38 - Water / Waste & 4     & 4     & 5     & 4     & 4     & 4     & 5     & 4     & 0 \\
    45 - Diagnostic and Maintenance & 0     & 0     & 0     & 0     & 0     & 0     & 0     & 0     & 0 \\
    49 - Airborne Auxiliary Power & 0     & 0     & 0     & 0     & 0     & 0     & 0     & 0     & 0 \\
    76 - Engine Controls & 3     & 0     & 0     & 0     & 6     & 3     & 3     & 3     & 3 \\
    77 - Engine Indicating & 0     & 0     & 0     & 0     & 0     & 0     & 0     & 0     & 0 \\
    78 - Exhaust & 0     & 0     & 0     & 0     & 0     & 0     & 0     & 0     & 0 \\
    80 - Starting & 0     & 0     & 0     & 0     & 0     & 0     & 0     & 0     & 0 \\
    \textbf{TOTAL} & \textbf{749} & \textbf{718} & \textbf{690} & \textbf{671} & \textbf{586} & \textbf{580} & \textbf{552} & \textbf{582} & \textbf{83} \\
    \bottomrule
    \end{tabular}%
  \label{tab:NewEleLoads}%
\end{table}%

The LED reduces the electrical load during all flight phases, depending on which lamp is used at which phase. Together with the interiors team, the electrical team was able to calculate the electrical load reduction on each flight phase. The information used to do so, was the reduction of each lamp, the phases during which they are used, and the electrical bus they are attached to. For example, the reading lights, the taxi lights and the galley lights are fed through the shed buses, and stop functioning when two generators fail, leading to the conclusion that their load reduction does not count when two or more engines fail. The main reductions come from the landing light and the taxi lights, which are halogens.
This table summarizes the load analysis and proves that the electrical loads added with the modifications the teams proposed don't overload the generation capacity and can be easily provided during all flight phases. Furthermore, the modifications provide electrical load savings, therefore fuel savings.
