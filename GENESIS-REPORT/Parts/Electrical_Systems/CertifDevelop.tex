To validate the electrical modifications the following certification tests should be performed, and each modification will have the subsequent man-hour cost.

Exchanging the batteries: (250:00 hours)

\begin{enumerate}
    \item Energization with battery 1
    \item Energization with battery 2
    \item Backup battery system
    \item Dc bus 1 short circuit
    \item Dc bus 2 short circuit
    \item Central dc bus short circuit
    \item Essential dc bus 1 short circuit
    \item Backup dc bus1 short circuit
    \item Backup essential dc bus short circuit
\end{enumerate}

Exchanging the APU Generator: (350:00 hours)	

\begin{enumerate}
\item APU generator capacity
\item Transient limits and stability APU starter/generator
\item Voltage regulation /ripple APU starter/generator
\item Feeder fault protection
\item Starter/generator reverse current protection
\item Starter/generator overvoltage protection
\item Overexcitation
\item Equalization
\end{enumerate}

Exchanging the static inverter: (80:00 hours)


\begin{enumerate}
 \item THD (total harmonic distortion)
\item Overload
\item Overvoltage
\item Undervoltage
\item Crest factor
\item Short circuit
\end{enumerate}

The test rig used before to validate the ERJ-145 family was disassembled, so a new RIG is necessary. Building this test rig results some additional costs, estimated below.

% Table generated by Excel2LaTeX from sheet 'Plan2'
\begin{table}[htbp]
  \centering
  \caption{Add caption}
    \scriptsize
    \begin{tabular}{rp{4cm}rr}
    \toprule
    \textbf{Activity} & \textbf{Description} & \textbf{Duration  (MH)} & \textbf{Cost (US\$)} \\
    \midrule
    RIG project & Electric and Mechanical Project  & \#\#\#\#\#\#\# &   \\
    Fabrication & Wiring / Electrical Boxes and Panels & \#\#\#\#\#\#\# &   \\
    Mounting & Wiring / Electrical Boxes and Panels & \#\#\#\#\#\#\# &   \\
    Experiment & Battery exchange & \#\#\#\#\#\#\# &   \\
    Experiment & APU / GCU Generator exchange & \#\#\#\#\#\#\# &   \\
    Experiment & Static Inverter exchange & 80:00:00 &   \\
    Building Construction & 120 $m^{2}$ (command room / machine room, Electrical Load Bench room) &       & US\$ 117,000.00 \\
    Buying & 2 Engines (50 HP / 12000 RPM) &       & US\$ 193,600.00 \\
    Buying & Electrical Load Bench de 1000 A &       & US\$ 49,720.00 \\
    Buying & Airplane components such as Generator, GCU, Batteries, Backup batteries, Relay box, Electric Panel, etc.  &       & US\$ 120,000.00 \\
    Mechanical HW & Racks, Panels, metallic board. &       & US\$ 17,350.00 \\
    Electrical HW & Fio, Conector, Borne, Sensors, etc &       & US\$ 35,000.00 \\
    \bottomrule
    \end{tabular}%
  \label{tab:addlabel}%
\end{table}%


\begin{itemize}
 \item This quotation does not include instrumentation.
 \item  It is considered that the main component tested was the starter/generator, therefore, all the RIG construction costs will be paid by this modification. And the other three modifications will cost an additional US\$ 5,000.00 each to buy the components for the RIG.
\end{itemize}

A flight test should be performed at ISA+35 to assure that the starter/generator is well refrigerated during cruise. Since it is too hard to reach ISA+35, the test is performed at different other temperatures and, furthermore, the acquisition data is extrapolated to 50 $\degree$C.


