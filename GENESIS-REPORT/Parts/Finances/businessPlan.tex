The Embraer Profit Estimative graph was determined according with some assumptions made by the team Genesis.

The residual value is \$ 2 mi and the number of aircraft that have the contract is 200 aircraft. The number of aircraft that will make the
modification and those that will not (will prefer the residual value) are always complementary, adding the same 200 aircraft.

The price that Embraer will buy the SB is \$ 41,097,900.75, thus generating a price of \$ 267,136.35 per aircraft, equivalent to 30 \% of
the purchase price of the SB to the airlines that have the residual value waranty.

As there are other ERJ145 aircraft that were not sold under RV waranty, the team adopted three diffetent scenarios. The realistic believes
that around 100 to 150 aircraft will buy the package, pessimist considers that no one aircraft will buy the package and the optimistic considers
200 aircraft that will buy the package.

These purchases for those who do not have the contract will be buying by \$ 1,027,447.52, representing a markup of 500 \% on the purchase price of the SB.


\begin{figure}[H] % Example image
\center{\includegraphics[width=400px]{Pictures/Finances/EmbraerProfitEstimative.eps}}
\caption{Embraer profitability}
\label{fig:EmbProfEstim}
\end{figure}

The X-axis is the value of aircraft under contract to be sold. The difference of the analyzed value in relation to the adopted value by the team Genesis, 200 aircraft,
is the number of aircraft that request the residual value waranty.

The Y axis is the profit of Embraer considering all the assumptions mentioned previously.

The curves on the graph represent the number of aircraft without residual value that will be sold. The space between the green curves, continuous and dashed, represents a realistic scenario. 
Above and below this space are respectively the curves representing the optimistic and pessimistic scenario.
