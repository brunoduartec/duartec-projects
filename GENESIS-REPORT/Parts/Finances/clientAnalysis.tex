Initially, will be presented the analysis for the clients, based on the Express Jet case. The following data were calculated using some Express Jet parameters and others parameters of the project:

YIELD = 0.14;
Distance = 831 km;
COC [\$/seat.km] = 0.064;
Load Factor = 75 \%;
Frequency (per year) = 1,950 flights.
Ticket Price = YIELD x Distance = 116.34
CASK reduction = 22 \% (considering the worst case = 18 \%)

\textbf{Investments, Costs and Revenues}

According to the Table \ref{tab:costsCompilation}, we have the recurrent and non recurent costs below:

TOTAL:
Recurrent Value = \$ 1,630,930.00
Non-recurrent Value = \$ 35,737,305.00

All the recurrent values of the modification will be charged by the outsourced workshops,
considering a markup of 15 \%.

The non-recurrent values will be invested by EMBRAER and diluted by 200 (considered scenario). After that, EMBRAER will charge a markup of 30\%.

Then, the values are rearranged applying the markups and the diluted value:

Recurrent Value = Modifications recurrent value sum x Outsourced markup (15 \%)  Recurrent Value = \$ 1,875,569.50

Non-recurrent diluted Value = ((Modifications non-recurrent sum x Outsourced markup(15\%)) / Number of aircrafts in the scenario (200)) x EMBRAER markup (30\%) Non-recurrent diluted Value = \$ 267,136.35

Based on these values, the investment cost for each aircraft is \$ 2,142,705.85.

The annual costs are calculated this way: CASK x Distance x Number of seats x Frequency (per year).

The annual costs raises 5 \% in 10 years.

The annual revenues are calculated this way: Ticket price x Number of seats x Frequency (per year) x Load Factor.

% Table generated by Excel2LaTeX from sheet 'Cash Flow Operador'
\begin{table}[H]
  \scriptsize
  \centering
  \caption{Operational Cost}
    \begin{tabular}{rrrrrrr}
    \toprule
    \multicolumn{7}{c}{\textbf{OPERATIONAL COST}} \\
    \midrule
    \textbf{ITEM} & \textbf{AIRCRAFT} & \textbf{YEAR 1} & \textbf{YEAR 2} & \textbf{YEAR 3} & \textbf{YEAR 4} & \textbf{YEAR 5} \\
    \multicolumn{1}{c}{1} & COC (ERJ 145 GENESIS) & \$5.102.472,96 & \$5.130.820,03 & \$5.159.167,10 & \$5.187.514,18 & \$5.215.861,25 \\
    \multicolumn{1}{c}{2} & COC (ERJ 145) & \$5.185.440,00 & \$5.214.248,00 & \$5.243.056,00 & \$5.271.864,00 & \$5.300.672,00 \\
          &       &       &       &       &       &  \\
    \textbf{ITEM} & \textbf{AIRCRAFT} & \textbf{YEAR 6} & \textbf{YEAR 7} & \textbf{YEAR 8} & \textbf{YEAR 9} & \textbf{YEAR 10} \\
    \multicolumn{1}{c}{1} & COC (ERJ 145 GENESIS) & \$5.244.208,32 & \$5.272.555,39 & \$5.300.902,46 & \$5.329.249,54 & \$5.357.596,61 \\
    \multicolumn{1}{c}{2} & COC (ERJ 145) & \$5.329.480,00 & \$5.358.288,00 & \$5.387.096,00 & \$5.415.904,00 & \$5.444.712,00 \\
    \bottomrule
    \end{tabular}%
  \label{tab:financeCostOperation1}%
\end{table}%


% Table generated by Excel2LaTeX from sheet 'relatorio'
\begin{table}[H]
  \scriptsize
  \centering
  \caption{Revenue}
    \begin{tabular}{crrrrrr}
    \toprule
    \multicolumn{7}{c}{\textbf{REVENUE}} \\
    \midrule
    \textbf{ITEM} & \textbf{AIRCRAFT} & \textbf{YEAR 1} & \textbf{YEAR 2} & \textbf{YEAR 3} & \textbf{YEAR 4} & \textbf{YEAR 5} \\
    1     & Ticket sells (ERJ 145 GENESIS) & \$10.420.303,73 & \$10.420.303,73 & \$10.420.303,73 & \$10.420.303,73 & \$10.420.303,73 \\
    2     & Ticket sells (ERJ 145) & \$8.683.586,44 & \$8.683.586,44 & \$8.683.586,44 & \$8.683.586,44 & \$8.683.586,44 \\
          &       &       &       &       &       &  \\
    \textbf{ITEM} & \textbf{AIRCRAFT} & \textbf{YEAR 6} & \textbf{YEAR 7} & \textbf{YEAR 8} & \textbf{YEAR 9} & \textbf{YEAR 10} \\
    1     & Ticket sells (ERJ 145 GENESIS) & \$10.420.303,73 & \$10.420.303,73 & \$10.420.303,73 & \$10.420.303,73 & \$10.420.303,73 \\
    2     & Ticket sells (ERJ 145) & \$8.683.586,44 & \$8.683.586,44 & \$8.683.586,44 & \$8.683.586,44 & \$8.683.586,44 \\
    \bottomrule
    \end{tabular}%
  \label{tab:financeRevenue}%
\end{table}%


\textbf{Cash Flow}


% Table generated by Excel2LaTeX from sheet 'relatorio'
\begin{table}[H]
  \scriptsize
  \centering
  \caption{Cash Flow ERJ 145 Genesis}
    \begin{tabular}{rrrrrr}
    \toprule
    \multicolumn{6}{c}{\textbf{CASH FLOW ERJ 145 GENESIS}} \\
    \midrule
    \textbf{DESCRIPTION} & \multicolumn{5}{c}{\textbf{ ANNUAL INCOME STATEMENT DISTRIBUTION}} \\
    \textbf{} & \textbf{YEAR 1} & \textbf{YEAR 2} & \textbf{YEAR 3} & \textbf{YEAR 4} & \textbf{YEAR 5} \\
    (+) Total Gross Revenue & \$10.420.303,73 & \$10.420.303,73 & \$10.420.303,73 & \$10.420.303,73 & \$10.420.303,73 \\
    ( -) Operational Costs & \$5.102.472,96 & \$5.130.820,03 & \$5.159.167,10 & \$5.187.514,18 & \$5.215.861,25 \\
    (=) Net Entry & \$5.317.830,77 & \$5.289.483,69 & \$5.261.136,62 & \$5.232.789,55 & \$5.204.442,48 \\
          &       &       &       &       &  \\
          & \textbf{YEAR 6} & \textbf{YEAR 7} & \textbf{YEAR 8} & \textbf{YEAR 9} & \textbf{YEAR 10} \\
    (+) Total Gross Revenue & \$10.420.303,73 & \$10.420.303,73 & \$10.420.303,73 & \$10.420.303,73 & \$10.420.303,73 \\
    ( -) Operational Costs & \$5.244.208,32 & \$5.272.555,39 & \$5.300.902,46 & \$5.329.249,54 & \$5.357.596,61 \\
    (=) Net Entry & \$5.176.095,41 & \$5.147.748,33 & \$5.119.401,26 & \$5.091.054,19 & \$5.062.707,12 \\
    \bottomrule
    \end{tabular}%
  \label{tab:CashFlow}%
\end{table}%



% Table generated by Excel2LaTeX from sheet 'Cash Flow Operador'
\begin{table}[H]
  \scriptsize
  \centering
  \caption{Cumulative Cash Flow ERJ 145 Genesis}
    \begin{tabular}{crr}
    \toprule
    \textbf{YEAR} & \multicolumn{1}{c}{\textbf{CASH FLOW}} & \multicolumn{1}{c}{\textbf{CUMULATIVE CASH FLOW}} \\
    \midrule
    \textbf{0} & -\$2.142.705,85 & -\$2.142.705,85 \\
    \textbf{1} & \$5.317.830,77 & \$3.175.124,91 \\
    \textbf{2} & \$5.289.483,69 & \$8.464.608,60 \\
    \textbf{3} & \$5.261.136,62 & \$13.725.745,22 \\
    \textbf{4} & \$5.232.789,55 & \$18.958.534,77 \\
    \textbf{5} & \$5.204.442,48 & \$24.162.977,25 \\
    \textbf{6} & \$5.176.095,41 & \$29.339.072,66 \\
    \textbf{7} & \$5.147.748,33 & \$34.486.820,99 \\
    \textbf{8} & \$5.119.401,26 & \$39.606.222,25 \\
    \textbf{9} & \$5.091.054,19 & \$44.697.276,44 \\
    \textbf{10} & \$5.062.707,12 & \textbf{\$49.759.983,56} \\
    \bottomrule
    \end{tabular}%
  \label{tab:CumulCash}%
\end{table}%


% Table generated by Excel2LaTeX from sheet 'relatorio'
\begin{table}[H]
  \scriptsize
  \centering
  \caption{Cash Flow ERJ 145}
    \begin{tabular}{rrrrrr}
    \toprule
    \multicolumn{6}{c}{\textbf{CASH FLOW ERJ 145}} \\
    \midrule
    \textbf{DESCRIPTION} & \multicolumn{5}{c}{\textbf{ ANNUAL INCOME STATEMENT DISTRIBUTION}} \\
    \textbf{} & \multicolumn{1}{c}{\textbf{YEAR 1}} & \multicolumn{1}{c}{\textbf{YEAR 2}} & \multicolumn{1}{c}{\textbf{YEAR 3}} & \multicolumn{1}{c}{\textbf{YEAR 4}} & \multicolumn{1}{c}{\textbf{YEAR 5}} \\
    (+) Total Gross Revenue & \$8.683.586,44 & \$8.683.586,44 & \$8.683.586,44 & \$8.683.586,44 & \$8.683.586,44 \\
    ( -) Operational Costs & \$5.185.440,00 & \$5.214.248,00 & \$5.243.056,00 & \$5.271.864,00 & \$5.300.672,00 \\
    (=) Net Entry & \$3.498.146,44 & \$3.469.338,44 & \$3.440.530,44 & \$3.411.722,44 & \$3.382.914,44 \\
          &       &       &       &       &  \\
          & \multicolumn{1}{c}{\textbf{YEAR 6}} & \multicolumn{1}{c}{\textbf{YEAR 7}} & \multicolumn{1}{c}{\textbf{YEAR 8}} & \multicolumn{1}{c}{\textbf{YEAR 9}} & \multicolumn{1}{c}{\textbf{YEAR 10}} \\
    (+) Total Gross Revenue & \$8.683.586,44 & \$8.683.586,44 & \$8.683.586,44 & \$8.683.586,44 & \$8.683.586,44 \\
    ( -) Operational Costs & \$5.329.480,00 & \$5.358.288,00 & \$5.387.096,00 & \$5.415.904,00 & \$5.444.712,00 \\
    (=) Net Entry & \$3.354.106,44 & \$3.325.298,44 & \$3.296.490,44 & \$3.267.682,44 & \$3.238.874,44 \\
    \bottomrule
    \end{tabular}%
  \label{tab:CashFlow2}%
\end{table}%


% Table generated by Excel2LaTeX from sheet 'Cash Flow Operador'
\begin{table}[htbp]
  \scriptsize
  \centering
  \caption{Cumulative Cash Flow ERJ 145}
    \begin{tabular}{crr}
    \toprule
    \textbf{YEAR} & \multicolumn{1}{c}{\textbf{CASH FLOW}} & \multicolumn{1}{c}{\textbf{CUMULATIVE CASH FLOW}} \\
    \midrule
    \textbf{0} & \$2.000.000,00 & \$2.000.000,00 \\
    \textbf{1} & \$3.498.146,44 & \$5.498.146,44 \\
    \textbf{2} & \$3.469.338,44 & \$8.967.484,88 \\
    \textbf{3} & \$3.440.530,44 & \$12.408.015,31 \\
    \textbf{4} & \$3.411.722,44 & \$15.819.737,75 \\
    \textbf{5} & \$3.382.914,44 & \$19.202.652,19 \\
    \textbf{6} & \$3.354.106,44 & \$22.556.758,63 \\
    \textbf{7} & \$3.325.298,44 & \$25.882.057,06 \\
    \textbf{8} & \$3.296.490,44 & \$29.178.547,50 \\
    \textbf{9} & \$3.267.682,44 & \$32.446.229,94 \\
    \textbf{10} & \$3.238.874,44 & \textbf{\$35.685.104,38} \\
    \bottomrule
    \end{tabular}%
  \label{tab:CumulCash2}%
\end{table}%

\textbf{Payback}

It's the number of years or months required for corresponding disbursement to the initial investment is recovered, or even equaled and surpassed by the cumulated net entries

Payback (fractionated ) = Last negative cumulative cash flow / First positive cash flow corresponding to the first positive cumulative cash flow.

Payback (GENESIS)= 0.4

Then, the payback will occur approximately in 5 months.

\textbf{Net Present Value (NPV)}

The NPV is obtained by discounting cash flows at a specified rate, bringing thus all the values for the initial situation - to a net present value.

This specified rate usually corresponds to a rate of minimum return that must be achieved by the project.

From the point of view of NPV, the project is considered viable when the calculation result is greater than zero, because it means that the project will give a higher return than the specified rate.

Investment = \$ 2,142,705.85
Revenue in the $1^{st}$ year = \$ 5,317,830.77
Revenue in the $2^{nd}$ year = \$ 5,289,483.69
Revenue in the $3^{rd}$ year = \$ 5,261,136.62
Revenue in the $4^{th}$ year = \$ 5,232,789.55
Revenue in the $5^{th}$ year = \$ 5,204,442.48
Revenue in the $6^{th}$ year = \$ 5,176,095.41
Revenue in the $7^{th}$ year = \$ 5,147,748.33
Revenue in the $8^{th}$ year = \$ 5,119,401.26
Revenue in the $9^{th}$ year = \$ 5,091,054.19
Revenue in the $10^{th}$ year = \$ 5,062,707.12

NPV = - Investment value + [Revenue in the $1^{st}$ year/$(1 + i_{1})^{1}$] + [Revenue in the $2^{nd}$ year/$(1 + i_{2})^{2}$] + [Revenue in the $3^{rd}$ year/$(1 + i_{3})^{3}$] + [Revenue in the $4^{th}$ year/$(1 + i_{4})^{4}$] + [Revenue in the $5^{th}$ year/$(1 + i_{5})^{5}$] + [Revenue in the $6^{th}$ year/$(1 + i_{6})^{6}$] + [Revenue in the $7^{th}$ year/$(1 + i_{7})^{7}$] + [Revenue in the $8^{th}$ year/$(1 + i_{8})^{8}$] + [Revenue in the $9^{th}$ year/$(1 + i_{9})^{9}$] + [Revenue in the $10^{th}$ year/$(1 + i_{10})^{10}$]

i = 3 \% (Market rate)

NPV (GENESIS) = - \$ 2,142,705.85 + \$ 5,162,942.49 + \$ 4,985,845.69 + \$ 4,814,685.30 + \$ 4,649,265.74 + \$ 4,489,397.80 + \$ 4,334,898.41 + \$ 4,185,590.47 + \$ 4,041,302.63 + \$ 3,901,869.12 + \$ 3,767,129.56

NPV (GENESIS) = \$ 42,190,221.35

NPV (ERJ 145) = \$ 2,000,000.00 + \$ 3,396,258.68 + \$ 3,270,184.22 + \$ 3,148,572.73 + \$ 3,031,271.20 + \$ 2,918,131.71 + \$ 2,809,011.34 + \$ 2,703,771.93 + \$ 2,602,279.99 + \$ 2,504,406.50

NPV (ERJ 145) = \$ 30,793,915.05

\textbf{Internal Rate of Return(IRR)}

The IRR calculation involves calculating the interest rate that would make the NPV value goes to zero.

The IRR must be greater than the rate of pay received by the application of the project budget value in another application.

If the IRR is greater than the market rate of return, the project is viable.

In this example, to calculate the IRR, will be subtracted the ERJ 145 revenue of the GENESIS revenue, in order to avoid a wrong calculation.

Using an excel tool, the internal rate of return can be obtained:
IRR = 85 \%.
