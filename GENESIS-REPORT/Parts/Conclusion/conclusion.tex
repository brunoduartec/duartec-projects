Studies done by Genesis team have shown that the 30-60 seats market segment is shrinking, reducing the value of any aircraft of this segment such as ERJ 145.
The tendency of the airliners decision is to replace their 50 seats fleet by 75 seats segment

In addition, analysis have shown that ERJ 145 presents an important contractual problem that can turn to a financial problem to Embraer, making the company to spend up to 400 million dolars, in a realistic scenario, to pay the residual value warranty to their clients.

Therefore, developing a retrofit to ERJ 145 is essential to Embraer. Keeping the ERJ 145 operating in the market is also important to Embraer's reputation, specially if the remaining fleet of the 30-60 seats segment is mostly composed by Embraer's aircraft.

This study has proposed a solution that acts in four fronts and offers a CASM reduction to ERJ 145, as shown in the figure below:

\begin{figure}[H] % Example image
\center{\includegraphics[width=350px]{Pictures/Conclusion/GraphCASMReductionbyFronts.eps}}
\caption{CASM Reduction by Fronts.}
\label{fig:GraphCASMReductionbyFronts}
\end{figure}


Each front proposed several modifications, each of them is responsible for a part of the CASM reduction, this detailed analysis is shown below

\begin{figure}[H] % Example image
\center{\includegraphics[width=350px]{Pictures/Conclusion/GraphCASMReductionDistribution.eps}}
\caption{CASM Reduction by Fronts.}
\label{fig:GraphCASMReductionDistribution}
\end{figure}


The point of view of Embraer, the package is economically feasible because it has the capacity to supply in part or cancel the residual value warranty question and even provide profit to the company

For the airlines, the final result was a modifications package that given to ERJ145 the power to increase the profit margin and, in addition, this package has enabled the company to remain longer with the aircraft before it reaches its useful operating life limit date.

The Genesis team finished this project with an accomplishment sense as an educational view according to the goals that the team had at the begin of the project.

Teamwork and studies since SPEC phase until the preliminary design phase were done hardly by Genesis team.

It is understood that the feeling of accomplishment is due to the way of how the team faced the problem as a real one. Seeking solutions considering cost minimization and sometimes even facing into contractual problems
with companies that were partners when ERJ145 was launched.

Therefore, Genesis team concludes and informs that this solution is highly applicable to solve the Embraer and the ERJ 145 clients question and recommends the project continuity.

