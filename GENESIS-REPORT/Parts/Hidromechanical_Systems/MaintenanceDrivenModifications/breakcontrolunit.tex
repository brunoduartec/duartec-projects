The Brake Control unit (BCU) shall contain all the circuitry required  by  the electronically  signaled  brake  system  to  perform  the brake pressure control and monitoring,  anti-skid, touchdown protection, locked wheel protection, in flight brake (spin down), built in test,  and fault isolation. Also, it shall incorporate the required interface for communication with the central alerting computer (EICAS) and Central Maintenance Computer (CMC) for indication purposes, excitement of the pedal transducers and power supply for the pressure transducers of the brake lines. Details regarding the BCU installation are shown in the figure \ref{fig:bcu}.

\begin{figure}[H] % Example image
\center{\includegraphics[width=400px]{Pictures/Hidromechanical_Systems/BCUassembly.eps}}
\caption{Brake Control Unit installation � EMB-145 Illustrated Parts Catalog}
\label{fig:bcu}
\end{figure}

The table \ref{tab:CUR} resumes data from the Service Performance Monthly Report (SPMR) of December/2012, and show the Mean Time Between Unscheduled Removals (MTBUR) related to some BCU Part Numbers (P/N�s) installed on some ERJ-145 aircraft.


% Table generated by Excel2LaTeX from sheet 'Plan1'
\begin{table}[htbp]
  \centering
  \caption{Component Unscheduled Removals - Brake Control Unit (December/2012)}
    \begin{tabular}{cc}
    \toprule
    Part Number & MTBUR (hours) \\
    \midrule
    42-951-3 & 7,723 \\
    42-951-5 & 5,426 \\
    42-951-6 & 4,034 \\
    142-093 & 14,116 \\
    \bottomrule
    \end{tabular}%
  \label{tab:CUR}%
\end{table}%


Obeying RBAC/RBHA rules, since 2002 all aircrafts must incorporate some additional parameters to the Flight Data Record (FDR), including some which the BCU deals with, such as brake valve pressure, pedals pressure and wheel speed. So, in compliance with the issues above, the P/N 142-093 was made, and can be noticed by the table "xxx" that this component performs better MTBUR. From April/2006, the P/N 42-951-6 was released for replacement on the fleet, but it doesn�t incorporate all the signal capabilities the last P/N provides and also, performs worse MTBUR rates.

According to the facts exposed, the cheaper and effective solution to increase the BCU time between failures on the ERJ-145 fleet is to issue a Service Bulletin justifying the component P/N 142-093 installation as well its requiring cables.
