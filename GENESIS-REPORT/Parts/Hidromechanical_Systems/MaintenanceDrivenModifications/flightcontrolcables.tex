There are two material options for control cables in ERJ-145, the most common and designed from the ground is the stainless steel. On the other hand, there is the option of carbon steel.
There are disadvantages and benefits from both materials. The stainless steel has no problem with corrosion and is a very hard material, but it compromises the life of the sheaves. There are some concerns with carbon steel about corrosion, although with a proper lubrication this problem is mitigated.
Another advantage from carbon steel material is that it's life is improved. The stainless steel has a 5,000 hours lifetime while with the carbon material it's raised to 10,000 hours (possibly 15,000).

This modification brings a considerable maintenance cost drop, impacting in a DMC reduction of 0.03 \%, that means \$ 0.2016 (USD) in economy by flight hour per airplane. This means a gain of \$ 3,990.00 (USD) after ten years of operation by each aircraft in the fleet. Nevertheless the proposal could bring a considerable maintenance cost reduction, mainly because of the task time. To exchange the command cables its necessary to disassemble almost the whole interior, this is the most expensive part of the task and involves a lot of man-hour. The modification would imply in much less time with the aircraft grounded.

Doubling the lifetime of the component, not only the acquisition cost will become lower, but also the man-hour needed for the task (taking the whole life of the aircraft into consideration).

Although carbon steel cables need a lubrication after 5000 hours, this procedure is much cheaper than the replacement of the system.

At Embraer, the Product Change Request (PCR) numbered as PCR 90995 contains the analysis of the modification and probable advantages. This modification doesn't request a Service Bulletin (SB) and is quite easy for the operator to implement. The procedure is to buy the new specified component through the Illustrated Parts Catalog (IPC) and install it. The Aircraft Maintenance Manual (AMM) already contains the different procedures for both materials.

Referring to the ruder control (\ref{fig:RCC}), the only change is the PN 145-21026-403 for the 140-06637-401.

\begin{figure}[H] % Example image
\center{\includegraphics[width=400px]{Pictures/Hidromechanical_Systems/RuderControlCables.eps}}
\caption{Ruder control cables}
\label{fig:RCC}
\end{figure}

As the density is almost the same for both materials, there is no gain in weight for the aircraft.
