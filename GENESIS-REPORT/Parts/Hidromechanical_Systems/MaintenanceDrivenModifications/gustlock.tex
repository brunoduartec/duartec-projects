The first reason of this modification was to reduce the maintenance costs. It was concluded that melted ice penetrated the gust-lock picking and then it became ice again. This fault resulted in the gust-lock actuator fail. The electric engine could not support the load to disengage the lock and an excessive current appeared. The component is shown in the \ref{fig:GL}.

\begin{figure}[H] % Example image
\center{\includegraphics[width=400px]{Pictures/Hidromechanical_Systems/Gust-lock.eps}}
\caption{Gust-lock: Gust-lock component}
\label{fig:GL}
\end{figure}

The root-cause of the event was attacked. To avoid this problem, the application of grease was studied, this would result in the impermeabilization of the lock, avoiding water infiltration.
According to the SPMR of the ERJ-145 the current MTBUR of the component is 18254 flight hours. The proposal is to increase this parameter to up to 21905 flight hours, reducing at least 45 unscheduled removals in ten operational years for each aircraft. The economy generated by the modification is about \$ 0.027 (USD) per flight hour per aicraft, when only maintenance costs due to unscheduled removals is considerate, and \$ 278.00 (USD) per aircraft in economy for spare itens during ten years.

Although the economy is very modest, the indirect cost reduction for the operator is considerable. The gust-lock fail is a "no go fault", which could imply in delays or even cancellations of flights. These problems mean huge loses for the airlines.

The modification increase the reliability of the component and the aircraft dispatchability, mainly in cold regions.
