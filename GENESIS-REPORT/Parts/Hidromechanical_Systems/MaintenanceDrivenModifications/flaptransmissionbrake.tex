The discussion involving the Flap Transmission Brake (FTB) was due to the anticorrosive melting. This problem implicates in a drop inside the brake component making it fail. The problem occurs because of the dihedral angle of the wing, which makes the melted fluid flows to lower positions. \ref{fig:FTB} shows the position of the element.

\begin{figure}[H] % Example image
\center{\includegraphics[width=400px]{Pictures/Hidromechanical_Systems/FlapTransmissionBrake.eps}}
\caption{Flap Transmission Brake: Position of the FTB in the wing}
\label{fig:FTB}
\end{figure}

To avoid such a problem, it is necessary a different way to protect the flexible shaft that transmits torque to the flaps. The utilization of a solid anticorrosive is suggested. Cor-Ban 27L is previously taken as an example and is a strong candidate for this application.

Cor-Ban 27L has been formulated to function as a non-drying, wide temperature range, corrosion inhibiting compound for static joint applications on a multitude of substrates. A thixotropic, buff-colored film can be achieved on any metal surface and will maintain a flexible protective coating that will withstand high ultra violet exposure in highly corrosive environments. Cor-Ban 27L has been approved by aeronautical companies as a substitute for current BMS 3-27 materials in some applications where it is specified on the drawing.
The Embraer documentation about the Cor-Ban 27L is treated in Embraer E944758D specification.

As a result of this modification, no more faults would be met in the FTB because of a leakage of the anticorrosive, boosting the component reliability, as well as its MTBUR.
Although the Cor-Ban 27L is treated herein as the substitute of the current anticorrosive, further analyses are required to choose the appropriate material, targeting a similar result to that outlined above.

The proposed modification is introduced via the supplier Service News Letter (SNL), allowing the application of a different anticorrosive, presenting revisions at the Aircraft Maintenance Manual (AMM). Parker Company is responsible for this document and there is no need of a Service Bulletin released by Embraer, leading for zero non-concurrent costs for the improvement.
