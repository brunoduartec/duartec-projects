The current ERJ-145 air conditioning system meets the needs of the 50 passengers and of the flight attendant in the aircraft cabin.
Some of the proposed changes include an increase of 10 passengers and a flight attendant, 3 more windows and a new cabin volume.
To determine the necessity of the PACKs update in order to attend the new conditions, a thermal model on AMESim (Figure \ref{fig:thermalmodel}) was used, to compare the thermal behavior of the aircraft before and after the modification.

 \begin{figure}[H] % Example image
\center{\includegraphics[width=400px]{Pictures/ECS/ThermalModel.eps}}
\caption{Thermal Model - ERJ-145 Thermal Model on AMESim}
\label{fig:thermalmodel}
\end{figure}

The increase of 11 persons (10 passengers and 1 flight attendant) was modeled considering a heat load of 100 W per person, and their seats.
The outside radiation through the windows in the system (34 before to 37 after the modification) was considered, and each window was considered like two acrylic plates with a small spacing, and a thin layer of polyvinyl butyral.
Heat exchange through the walls of the aircraft, piping and floor were also modeled.
Thermal load of the avionics and of the interior lights were also considered. The lights were replaced with LED lamps, which led to a reduction of 2,400 W in the cabin.
However, 60 USB ports were included, which added 300 W in the system.
The ERJ-145 has 50% of cabin air from recirculation which is mixed with the hot air (bleed) and the cold air. This condition was maintained in all simulated models.
Internal cabin pressure was maintained at 8000 feet during the cruise.
A thermal balance analysis indicates that there are no major changes in system dynamics, once the thermal load due to the addition of passenger is compensated by the exchange of lights. The PACKs have sufficient flow to supply the addition of 11 people.
As can be seen in the graphic of Figure Thermal Curve, the temperature difference inside the  modified aircraft cabin falls by 2 ° C, at cruise, when compared with the original aircraft.

 \begin{figure}[H] % Example image
\center{\includegraphics[width=400px]{Pictures/ECS/ThermalCurve.eps}}
\caption{Thermal Curve - Temperatures for the ERJ-145 and ERJ-145 modified}
\label{fig:thermalcurve}
\end{figure}


The system behavior was analyzed in different heating and cooling situations: the aircraft at low altitude, high altitude, low speed, maximum cruise speed, in very cold days and very warm ones.
Modified ERJ-145 presented the same thermal behavior that the original aircraft in all situations analyzed, and the temperatures differences were similar to that found in the first flight condition modeled.
The studies also indicated that the comfort of passengers has not changed on the modified aircraft.
So it was not necessary to change or upgrade the air conditioning PACKs, also considering the low rates of exchange and maintenance of the equipment.
A new configuration for air distribution in the cabin will be determined using a mock-up.
The certification of the air conditioning system will be made with the following Means of Compliance:

\begin{itemize}
  \item MoC 1 - Design Review
  \item MoC 2 - Calculation/ Analysis/ Similarity Analysis
  \item MoC 5 - Ground Test on Aircraft
  \item MoC 6 - Flight Tests
  \item MoC 7 - Inspection
  \item MoC 8 - Simulation
  \item MoC 9 - Equipment Qualification
\end{itemize}