The regional airline industry is undergoing a lot of changes. These changes have been stimulated as much by the 2008 crisis as by the still fragile recovery under way at present of the airline industries. The changes are comprehensive, especially in the U.S. market, and cover the type of aircraft utilized, to the business models under which carriers operate, to the very makeup of the industry itself.

In a new study, "The Market for Regional Transport Aircraft", indicates that the regional aircraft market is entering a period of transition and transformation. The regional airline industry is consolidating as large holding companies gain control of multiple carriers. This is especially true in the large U.S. market, where companies such as Pinnacle, Republic, and SkyWest each control a number of airline brands.

Regional airlines are continuing to buy ever-larger-capacity aircraft. The heyday of the 50-seat jet market has come and gone. Large numbers of 50-seat jets still populate the fleets of North American regional airlines. However, high fuel costs and other factors have made the operation of these 50-seat regional jets uneconomic.
	
Meanwhile, opportunities for growth in the regional airline sector will be somewhat limited, especially in mature air travel markets such as the U.S. and Europe. The majors themselves are consolidating, and regional carriers will be competing for a shrinking number of opportunities to provide hub feeder service to the majors' networks.

One way for regional carriers to grow revenue in the years ahead will be to fly larger aircraft. In the U.S. in particular, though, scope clauses generally prevent regional carriers partnered with major airlines from flying jets larger than 70 or 76 seats.

It's possible to see that the COC is directly proportional to total crew expenses, total maintenance costs and airport fees.
	
The airlines companies are attracted by the profit obtained through the difference between the Yield and the CASM. the bigger the difference the bigger the profit.

Being the CASM a value as a function of number of pax and the flight distance, if the aircraft makes a flight through a specific route, the number of passengers is a determinant factor to an airline company to choose the aircraft.

This way, this work has the aim of become attractive the retrofit of ERJ 145. For this goal to be reached it's necessary increase the airlines profit reducing the COC or the CASM values.

If the airline doesn't change the fleet and the routes to be flight, a reduction of COC entails a reduction of CASM. But the profitability can also be reached by another way. If the aircraft operating costs maintains the same or even grows not so significantly, for a given route the airline can choose to acquire an aircraft with a higher number of pax.
The proposed modification that increases the number of pax decreases significantly the CASM. 

The project will be focused in the American market because more than 65 \% of the fleet in service is in there.

The American market seeks a high CASM reduction, in order to compete with the new airplanes with more capacity and compete with the CRJ200, but don't need a higher range or TOFL reduction
