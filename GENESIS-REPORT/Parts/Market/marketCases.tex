To be able to determine the value of the modernization, a research was conducted to establish how much the airline companies are willing to pay for it.

Were taken into account the following parameters:

\begin{itemize}
  \item YIELD [\$/seat.km]
  \item Load factor [\%]Capacity [num]
  \item Distance [km]
  \item Frequency (per year)
  \item COC [\$/seat.km]
  \item Aircraft Value
  \item Investment/Overhaul
  \item New Aircraft Price
  \item ERJ145 Selling Price
  \item Residual Value	
  \item Revenue
  \item Costs
  \item Profit
  \item Investment
  \item Cash flow
  \item Net Present Value (NPV)
  \item Investment rate of Return (IRR)
\end{itemize}

These parameters are compared with the 76 PAX aircrafts (E175 and CRJ900), the CRJ200 and the ATR72 over time.

The result of this analysis is a graph that provides the market value for a certain amount of CASM reduction provided by the modernization package.

\begin{figure}[H]
  \centering
  % Requires \usepackage{graphicx}
  \includegraphics[width=400px]{Pictures/Market/OperatorsInvestmentPrevision.eps}\\
  \caption{Operators investment prevision}
  \label{OperInvestPrevis}
\end{figure}

To define the value, first must be set the CASM reduction provided by the modernization:
CASM reduction = 22 \% (considering the worst case = 18 \%).
	Then, considering the worst case (EXPRESS JET), the value for 18 \% of CASM reduction is \$ 3,300,000.00. 

This means that for 18 \% of CASM reduction plus \$ 3,300,000.00 is equivalent to buying a new 76 PAX aircraft for EXPRESS JET.
