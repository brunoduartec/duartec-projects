\textbf{Overview}\\
There are two ways to make one aircraft more profitable: make it more economical or increase the income that it produces. The fuselage stretch has the second propose: allow the plane to fly with more passengers. Immediately consequences of this change are: weight increase, structural limitations, aircraft balance and takeoff distance.
In the past EMBRAER has done the inverse process: the ERJ-145 were reduced to produce the ERJ-135 and the ERJ-140. The main motivation of these reduction was the scope clauses of US airliners. The initial analyze of how a fuselage stretch impacts the current aircraft performance will be done comparing the performance of those three aircrafts: ERJ-135/140/145. The first analysis will be very a simple one and aims to relate how the fuel burned increase with the fuselage stretch and consequently with the increase of passengers number.
	Table \ref{tab:BOW} compares the basic operation weight of those three aircraft and its passengers capacity:

% Table generated by Excel2LaTeX from sheet 'Sheet3'
\begin{table}[htbp]
  \centering
  \caption{BOW and Pax Number of the ERJ Family}
    \begin{tabular}{cccccc}
    \toprule
          & Seat Rows & \multicolumn{2}{c}{BOW [kg]} & \multicolumn{2}{c}{Pax} \\
    \midrule
    ERJ-145 & 16    & 12114 & -     & 50    & - \\
    ERJ-140 & 14    & 11808 & -2.50\% & 44    & -12\% \\
    ERJ-135 & 12    & 11501 & -5.10\% & 37    & -26\% \\
    \bottomrule
    \end{tabular}%
  \label{tab:BOW}%
\end{table}%

It will be supposed that the addition of one seat row increases the BOW by 150kg and adds 3 extra passengers. According to the FAA, one flight crew member is needed for each 50 passengers. By adding more seat rows, one extra flight crew member will be necessary and consequently the BOW increases by another 90kg.

% Table generated by Excel2LaTeX from sheet 'Sheet3'
\begin{table}[htbp]
  \centering
  \caption{BOW and ZFW Estimation}
    \begin{tabular}{rrccrrrr}
    \toprule
    \multicolumn{1}{c}{} & \multicolumn{1}{c}{Seat Rows} & \multicolumn{2}{c}{BOW [kg]} & \multicolumn{2}{c}{Pax} & \multicolumn{2}{c}{ZFW [kg]} \\
    \midrule
    ERJ-145 & 16    & 12114 & -     & 50    & -     & 16254 &  \\
    +1 Row & 17    & 12350 & 2.00\% & 57    & 14\%  & 17480 & 7\% \\
    +2 Rows & 18    & 12500 & 3.20\% & 60    & 20\%  & 17900 & 10\% \\
    +3 Rows & 19    & 12650 & 4.50\% & 63    & 26\%  & 18320 & 13\% \\
    +4 Rows & 20    & 12800 & 5.70\% & 66    & 32\%  & 18740 & 16\% \\
    \bottomrule
    \end{tabular}%
  \label{tab:BOWandZFW}%
\end{table}%


For each configuration fuel consumption estimation is made for different route ranges. It is compared the percentage of fuel increase with respect to the increase of PAX number.


\begin{figure}[H] % Example image
\center{\includegraphics[width=400px]{Pictures/Aeronautics/ModificationStudies/FuelincreasePAXincrease.eps}}
\caption{: Fuel increase versus PAX increase}
\label{fig:FuelincreasePAXincrease}
\end{figure}

As Table \ref{tab:BOWandZFW} shows, a fuselage stretch seems to be something interesting to reduce the cost per assent of the aircraft, because the percentage of fuel increase is less then the percentage of passenger increase. Furthermore, some costs, such as pilot, engine maintenance keep constant.
Finally, the first analysis shows that an increase of fuselage is a valid option for the project objective and it will more detailed studied.


\textbf{MZFW Restriction}

In an aircraft there are some structural restriction, such as the Maximum Zero Fuel Weight. By adding more payload to the aircraft and increasing the aircraft empty weight there will be an increase of the ZFW. The MZWF of the ERJ-145 is 17900kg and an increase of the MZFW is not considered to avoid structural reinforcements. Considering the estimated values of ZFW, it is possible to eliminate two configurations of stretched aircraft: 19 and 20 seat rows:

% Table generated by Excel2LaTeX from sheet 'Sheet3'
\begin{table}[htbp]
  \centering
  \caption{ZFW Restriction}
    \begin{tabular}{cccccccr}
    \toprule
          & \textbf{Seat Rows} & \multicolumn{2}{c}{\textbf{BOW [kg]}} & \multicolumn{2}{c}{\textbf{Pax}} & \multicolumn{2}{c}{\textbf{ZFW [kg]}} \\
    \midrule
    \textbf{ERJ-145} & 16    & 12114 & -     & 50    & -     & 16254 &  \\
    \textbf{+1 Row} & 17    & 12350 & + 2,0\% & 57    & 14\%  & 17480 & \multicolumn{1}{c}{7\%} \\
    \textbf{+2 Rows} & 18    & 12500 & + 3,2\% & 60    & 20\%  & 17900 & \multicolumn{1}{c}{10\%} \\
    \textbf{+3 Rows} & 19    & 12650 & + 4,5\% & 63    & 26\%  & 18320 & \multicolumn{1}{c}{13\%} \\
    \textbf{+4 Rows} & 20    & 12800 & + 5,7\% & 66    & 32\%  & 18740 & \multicolumn{1}{c}{16\%} \\
    \bottomrule
    \end{tabular}%
  \label{tab:ZFWRestriction}%
\end{table}%



\textbf{The Stretch Position}

There are three ways to stretch the fuselage:

\begin{itemize}
  \item Stretch the fuselage in front the wing
  \item Stretch the fuselage back the wing
  \item Stretch the fuselage in front and back the wing
\end{itemize}


The first option is discarded due stability issues.
The third options seems to be best one concerning stability and performance. However it is the most expensive one, because there would be necessary two cuts on the existing aircraft fuselage.
The second option, seems to be a good solution, because it is simple (only one cut on the fuselage) and there are no significant stability issues. The position of the CG will go backwards, making the aircraft more efficient. However, after some preliminary analyzes, it was concluded that when stretching the fuselage by two seat rows between the wing and the engine, the position of the after CG will in the limit of the critical position. So, it was decided that the add-on fuselage part would not have the length of exactly two seat rows, but it would be a little bit shorter. The reduction of the seat pitch would allow to accommodate all the 60 seats in the fuselage, as shows the figure \ref{fig:LOPA}
