The analysis of a single pilot airplane came from a tendency in the aeronautic industries of drones and some executive airplanes that are certified to just one pilot, also this modification could bring a great improvement for the operator.
	In a work with hydromechanics team, it was realized the technical possibility of this kind of modification due to the high reliability of the new autopilot. The aerodynamic team was responsible to analyze the certification impediments of this kind of modification on the new 145.
	After research FAR 25: airworthiness standards and operational requirements (Part 135). No clear restriction was obtained on FAR 25, but the following limitation due to operational requirement as noted (135.99 - Composition of flight crew)


\begin{enumerate}
  \item No certificate holder may operate an aircraft with less than the minimum flight crew specified in the aircraft operating limitations or the Aircraft Flight Manual for that aircraft and required by this part for the kind of operation being conducted.
  \item No certificate holder may operate an aircraft without a second in command if that aircraft has a passenger seating configuration, excluding any pilot seat, of ten seats or more.
\end{enumerate}


	The need of a second in command could force a co-pilot or a flight attendant that could fulfill this requirement and this would difficult the operation of the aircraft. Although, having an exception of those requirement could open the possibility to have a single pilot airplane in terms of certification, its not a wise decision to base a project in a possibility of have an regulation exception.
	In conclusion, the single pilot airplane was put on hold until there is a improvement in terms or regulation.
