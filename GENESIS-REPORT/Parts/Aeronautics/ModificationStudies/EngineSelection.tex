The propulsion system of the ERJ 145 Family is composed by two Allison/Rolls Royce AE 3007 family engines, models A1, A3, A1/3, A1P and A1E. According with Daly and Gunston (2006), this engine is a turbofan and was developed in the beginning of 1990.

The engine that currently equips the ERJ-145LR (the most common version of the 145 family and with higher presence on the fleet) is the Allison/Rolls-Royce AE3007 A1. It develops a SLS Thrust of 7.580 pounds, and all the calculations and comparisons was made according with this engine.

With the goals established in the SPEC phase in mind (CASM reduction in 10 \%, TOFL without modifications to the MTOW of the original ERJ 145 and 60 passengers), the propulsion studies sought to understand the ERJ 145 scenario, identifying the engines of the fleet, as well as their technical features.

The selection started with a study to identify a possible engine to equip the airplane. To offer a wide range of suggestions, researches were made to identify engines available on the market, or in a final stage of development, with static thrust rates ranging from 7,500 to 10,000 pounds. So, the following engines were detected, and are presented in \ref{tab:EngineThrust}

% Table generated by Excel2LaTeX from sheet 'Plan1'
\begin{table}[htbp]
  \centering
  \caption{Engines with static thrust rates ranging from 7,500 to 10,000 pounds.}
    \begin{tabular}{cc}
    \toprule
    \textbf{Manufacturer} & \textbf{Model} \\
    \midrule
    \multirow{6}[12]{*}{General Electric} & CF 34-1A \\
          & CF3 4-3A \\
          & CF 34-3A1 \\
          & CF 34-3A2 \\
          & CF 34-3B \\
          & CF 34-3B1 \\
    Honeywell & HTF 7250G \\
    Pratt Whitney & PW 800 \\
    \multirow{8}[16]{*}{Rolls-Royce} & AE 3007 A \\
          & AE 3007 A1/1 \\
          & AE 3007 A1 \\
          & AE 3007 A1/2 \\
          & AE 3007 A1/3 \\
          & AE 3007 A1P \\
          & AE 3007 A1E \\
          & AE 3007 A2 \\
    Snecma & Silvercrest \\
    \bottomrule
    \end{tabular}%
  \label{tab:EngineThrust}%
\end{table}%

The availability of information and the technologic level of the engines were the first criteria of selection. The absence of reliable and precise information about the PW800 prevented a more detailed study of its benefits. The CF34-3 series was developed before the AE3007 series, assumedly having the same, or worse performance levels of the latter. It was also considered that this engine family was evaluated in the original ERJ-145 design. This facts, plus the estimated non-recurrent costs of such modification excluded it of any further analysis.

The models A, A1/1, A1/2, A1/3 and A1P of the AE 3007 family engines were excluded from the analysis because they present thrust lower than the engine reference, the AE 3007 A1.
The next selection criteria was the cruise TSFC of the remaining engines. To determine this value for each engine a simulation software was necessary. The studies here presented were made using the Gasturb 10. Engines data was obtained from EASA TCDS, ICAO engine exhaust emissions databank, Embraer internal studies and, in the case of the Snecma Silvercrest engine, that has not yet been certified, manufacturers press releases. The obtained results are showed in \ref{tab:TSFC}

% Table generated by Excel2LaTeX from sheet 'Plan1'
\begin{table}[htbp]
  \centering
  \caption{Table 2: Obtained TSFC results in the Gas Turb 10 software simulations.}
    \begin{tabular}{cccccc}
    \toprule
    \textbf{Manufacturer} & \textbf{Rolls Royce} & \textbf{Rolls Royce} & \textbf{Rolls Royce} & \textbf{Snecma} & \textbf{Honeywell} \\
    \midrule
    \textbf{Model} & AE 3007 A1 & AE 3007 A1E & AE 3007 A2 & Silvercrest & HTF 7250 G \\
    \textbf{Static Thrust} & \multirow{2}[2]{*}{7.580} & \multirow{2}[2]{*}{8.110} & \multirow{2}[2]{*}{8.997} & \multirow{2}[2]{*}{9.500} & \multirow{2}[2]{*}{7.765} \\
    \textbf{(SL ISA) [lb]} &       &       &       &       &  \\
    \textbf{Cruise TSFC} & \multirow{3}[2]{*}{0,6838} & \multirow{3}[2]{*}{0,6764} & \multirow{3}[2]{*}{0,6667} & \multirow{3}[2]{*}{0,59} & \multirow{3}[2]{*}{0,6275} \\
    \textbf{(37000ft M=0,76 ISA)} &       &       &       &       &  \\
    \textbf{[lb/(lb*h)]} &       &       &       &       &  \\
    \bottomrule
    \end{tabular}%
  \label{tab:TSFC}%
\end{table}%

The following criteria in engine selection was the engine acquisition price. The initial estimation was made by the methodology proposed by Roskam (1990), which relates directly the engine price with it installed static thrust and requires corrections to values of present days. The \ref{tab:Prices} presents the found values.

% Table generated by Excel2LaTeX from sheet 'Plan1'
\begin{table}[htbp]
  \centering
  \caption{Table 3: Obtained acquisition prices results calculated by Roskam method.}
    \begin{tabular}{cccccc}
    \toprule
    \textbf{Manufacturer} & \textbf{Rolls Royce} & \textbf{Rolls Royce} & \textbf{Rolls Royce} & \textbf{Snecma} & \textbf{Honeywell} \\
    \midrule
    \textbf{Model} & AE 3007 A1 & AE 3007 A1E & AE 3007 A2 & Silvercrest & HTF 7250 G \\
    \textbf{Price [US\$ million - 1 engine]} & 2,534,967.93 & 2,927,286.77 & 3,078,872.32 & 3,096,200.36 & 2,589,666.76 \\
    \bottomrule
    \end{tabular}%
  \label{tab:Prices}%
\end{table}%

Non recurrent costs also was estimated, according with the Embraer's engineering mentors. The acquisition of new engines, such as Silvercrest and HTF 7250G were estimated with US\$ 40 millions of non recurrent costs. These values are composed by number of engineers, his salaries, time of design, tests and certification values.
Lastly, all these analysis allowed to chose the best option to the project. Due to the highly recurrent and non recurrent costs of the engines HTF 7250G and Silvercrest, they were excluded among the options. The AE 3007 A1E and A2 were considered without recurrent costs (due to the fact of other airplanes in Embraer already uses this engines models) and with approximately US\$ 5 millions of non recurrent costs. Putting these two options in analysis with options of other areas (hydromechanicals, electricals, etc), and observing that the reduction in TSFC was not too relevant, they were excluded of the options because they prevented to achieve the SPEC goals. Thus, the project remains with the AE 3007 A1 engine, as the option that most collaborates to reach the SPEC targets. The figure \ref{fig:Rational} that follows, shows the rational of all the activities described above.

\begin{figure}[H] % Example image
\center{\includegraphics[width=400px]{Pictures/Aeronautics/LLA_F1.eps}}
\caption{Activities Rational - Propulsion Team}
\label{fig:Rational}
\end{figure}
