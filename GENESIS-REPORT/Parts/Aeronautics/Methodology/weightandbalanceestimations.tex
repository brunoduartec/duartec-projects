After the Conceptual Studies done during the first project part, more accurate values related to the weight of each part were obtained. These more approximate data was available in Embraer(2012) and is shown in the table below:

% Table generated by Excel2LaTeX from sheet 'Plan1'
\begin{table}[htbp]
  \centering
  \caption{Group weight estimations}
    \begin{tabular}{cc}
    \toprule
    \textbf{GROUP} & \textbf{WEIGHT(kg)} \\
    \midrule
    Wing  & 2244 \\
    Horizontal Tail & 212 \\
    Vertical Tail & 159 \\
    Wing-Stub & 42 \\
    Fuselage & 2436 \\
    Pylon & 159 \\
    Nacelle & 384 \\
    Nose Gear & 169 \\
    Main Landing Gear & 691 \\
    Systems & 1730 \\
    Interior & 774 \\
    Seats & 600 \\
    Furnishing & 264 \\
    \bottomrule
    \end{tabular}%
  \label{tab:groupweight}%
\end{table}%

 The engines and APU weights used were the ones published by the manufactures:
                Engine Weight: 809 kg (each)
                APU: 82 kg
After estimating these weights, the operational items were also considered to calculate the Basic Operating Weight and compare with the real airplane. These values were obtained with the EMBRAER ERJ145 Weight and Balance Manual (2005) and are shown below:

% Table generated by Excel2LaTeX from sheet 'Plan1'
\begin{table}[htbp]
  \centering
  \caption{Operating Items Weight}
    \begin{tabular}{rr}
    \toprule
    \textbf{Operating Items} & \textbf{Weight} \\
    \midrule
    Crew/Baggage/Manuals & 271 kg \\
    Unusable Fuel & 36 kg \\
    Catering & 122 kg \\
    Oils/Water Fluids & 89 kg \\
    Total Operating Items & 518 kg \\
    \bottomrule
    \end{tabular}%
  \label{tab:operationweight}%
\end{table}%

When adding the EW (11564kg) with the Operating Items Weight (518kg) a value of 12082kg is obtained. However, according to the EMBRAER - EMB145 Airport Planning Manual (2007), the aircraft BOW is of 12114kg, with a difference of 32kg.
To correct this error and obtain more precise data, a correction factor was adopted in the main structures (wing, horizontal tail, vertical tail and fuselage). So a correction factor of 0,63\% was applied and new weight values were calculated:


% Table generated by Excel2LaTeX from sheet 'Plan1'
\begin{table}[htbp]
  \centering
  \caption{Final Weight Estimations}
    \begin{tabular}{rr}
    \toprule
    \textbf{BEW} & \textbf{11596 kg} \\
    \midrule
    Wing  & 2258 kg \\
    Elevator & 213 kg \\
    Rudder & 160 kg \\
    Systems & 1730 kg \\
    Nose Gear & 169 kg \\
    Main Landing Gear & 691 kg \\
    Fuselage & 2452 kg \\
    Nacelle & 384 kg \\
    Engine & 1618 kg \\
    Pyllon & 159 kg \\
    Interior & 774 kg \\
    Equipment/Furnishing & 264 kg \\
    APU   & 82 kg \\
    WFF   & 42 kg \\
    Seats & 600 kg \\
    \textbf{Operating Items} & \textbf{518 kg} \\
    Crew/Baggage/Manuals & 271 kg \\
    Unusable Fuel & 36 kg \\
    Catering & 122 kg \\
    Oils/Water Fluids & 89 kg \\
    \textbf{Basic Operating Weight} & \textbf{12114 kg} \\
    \bottomrule
    \end{tabular}%
  \label{tab:finalweightestimations}%
\end{table}%


\textbf{Center of Gravity Estimation}
Another important data to estimate is the location of the Center of Gravity. This is crucial to evaluate the modifications and how they will affect the controllability, maneuverability and trimming.
The CG position of main structures were determined according to Raymer (1992). The pylon, nacelle, engines and landing gear were considered at 50\% of each component, the interior with the same position of fuselage and the systems CG at 33% of the fuselage.
The table in sequence shows the CG position of each component and the position relative to the leading edge of the MAC (datum is localized at airplane nose):

% Table generated by Excel2LaTeX from sheet 'Plan1'
\begin{table}[htbp]
  \centering
  \caption{CG position of groups}
    \begin{tabular}{rrcc}
    \toprule
    \textbf{ Group} & \textbf{CG Location} & \textbf{Xcg [mm]} & \textbf{Xcg [\% of MAC]} \\
    \midrule
    Fuselage & 50\% of Length & 13852 & -26\% \\
    Wing  & 40\% of MAC & 15717 & 39\% \\
    Pylon & Center & 22421 & 273\% \\
    Horizontal Tail & 40\% da MAC & 26810 & 426\% \\
    Vertical Tail & 40\% da MAC & 28303 & 478\% \\
    Wing Stub & Center & 15256 & 23\% \\
    Nose Landing Gear & Center & 2217  & -432\% \\
    Main Landing Gear & Center & 16631 & 71\% \\
    APU   & Center & 27143 & 438\% \\
    Systems & 33\% of Fus. Length & 9358  & -183\% \\
    Interior & 50\% of Fus. Length & 13852 & -26\% \\
    Furnishing & 50\% of Fus. Length & 13852 & -26\% \\
    Seats & Center of each seat & 13485 & -39\% \\
    Engine Dry Weight & Center & 22421 & 273\% \\
    Nacelle & Center & 22421 & 273\% \\
    Oper. Items &       & 11341 & 273\% \\
    \textbf{BOW} & \textbf{} & \textbf{15482} & \textbf{31\%} \\
    \bottomrule
    \end{tabular}%
  \label{tab:cgposition}%
\end{table}%



With these values, the position of fuel and seats centers of gravity, it was possible to determine the CG Envelope of the original airplane and how it moves while the plane is loaded (Figure 6 1). The passenger load was made from back to front and front to back, explaining the appearance of the three curves in the figure, representing each one of the airplane seat lines.


\begin{figure}[H] % Example image
\center{\includegraphics[width=400px]{Pictures/Aeronautics/Methodology/CG_Envelope.eps}}
\caption{CG Envelop of the original airplane}
\label{fig:compressiblepolar}
\end{figure}


\begin{figure}[H] % Example image
\center{\includegraphics[width=400px]{Pictures/Aeronautics/Methodology/CG_EnvelopeMAC.eps}}
\caption{CG Envelope of the original airplane, in \% of MAC}
\label{fig:envelopeCGoriginal}
\end{figure}

The CG envelope and weights of the modified airplane will be further discussed, after proposing the modifications and integrating the results of each design area. However the methodology will be the same for the studies done for the original airplane.

