The performance methodology used to analyze the aircraft, was based in the concepts presented by engineer José Renato O. Melo. And is described as follows below.
	The main parameters checked was:

\begin{itemize}
  \item Climb: rate of climb, time, consumption and distance as a function of altitude;
  \item Especific Range: as a function of speed, in typical altitudes;
  \item Descent : time, consumption and distance as a function of altitude;
  \item Holding : consumption as a function of speed(Equivalent Air Speed) and altitude
\end{itemize}


       (Data for ISA atmosphere and weight  between 70\% and 100\% of MTOW)
- Time and block consumption :
(for maximum number of pax, in Long Range and in maximum speed cruise, as a function of flight phase, until the maximum fuel in typical altitudes;

\begin{itemize}
  \item Payload x Range : in Long Range and maximum cruise speed, in typical  altitude;
  \item Take-off weight limited by gradient (WAT) : ISA to ISA+30, SL to 8000 ft;
  \item Take-off field length x TOW: ISA to ISA+15, Altitude sea level, 4000 ft;
  \item Landing field length x LDW :  sea level.
\end{itemize}


\textbf{Definitions}

\begin{figure}[H] % Example image
\center{\includegraphics[width=400px]{Pictures/Aeronautics/Methodology/forces.eps}}
\caption{Airplane Forces}
\label{fig:forces}
\end{figure}

x-axis:

$L-W\cos\gamma = 0 \Rightarrow L = W\cos\gamma$

y - axis:

$T - D - W\sin\gamma = W \frac{W}{g} \frac{dV}{dt} $


\textbf{Climb}

Gradient is a parameter to check if an aircraft is able to overcome an obstacle. Gradient of climb:


$\sin\gamma = \frac{T-D}{W} - \frac{1}{g}\frac{dV}{dt}$

$\sin\gamma = \frac{T-D}{W}$ or $\sin\gamma = \frac{T}{W} - \frac{C_{D}}{C_{L}}$

For small values of $\gamma$, $\sin\gamma = \gamma$
