Flight and handling qualities are an important analysis to do in the preliminary design, as this can decide the project viability. Taking it into account, a the longitudinal flight characteristics were analyzed, based on the calculations proposed by Irving (1966) and Melo (2013).

The main characteristics calculated were the ones related to flight qualities (static margin and neutral point), elevator control (elevator and stabilizer angles to trim and maneuver) and pilot handling (column forces to maneuver).

\textbf{Automatic Tab Modeling}
\\
In order to do a more realistic modeling of the trimming and maneuver characteristics, an automatic tab was modeled in a simple way: it was considered that its angle is proportional to the elevator angle by a factor called $K_{\delta}$:

\textbf{Trimming}
\\
The main results in the trimming analysis are the angles, lift coefficients and static margin of the airplane. The airplane lift coefficient (CL) can be defined with the equation (\ref{eq:X2}), and be related with the wing-body and horizontal tail lift coefficient:

\begin{equation}
    C_{L}=\frac{2W}{\rho_{0}V_{E}^{2}S}=C_{Lwb}+C_{Lh}\frac{S_{h}}{S}
    \label{eq:X2}
\end{equation}

The wing-body and horizontal tail lift coefficients can also be related when calculating the airplane global pitching moment coefficient, and the horizontal tail lift coefficient can be determined:

\begin{equation}
    C_{Lh}=\frac{C_{m0wb}+(h-h_{0})C_{L}-(z/c)(T/W)C_{L}}{V'_{H}}
    \label{eq:X3}
\end{equation}

To calculate the horizontal tail angle of attack necessary to achieve this lift coefficient, is necessary to relate the elevator and tab angles, as the trimming is done with zero force in the column (stick-free), and the elevator will not be aligned with the horizontal stabilizer.

As the column force is zero, the hinge moment coefficient is also null and a relation between elevator and horizontal stabilizer can be established:

\begin{equation}
    C_{H}=b_{1}\alpha_{h}+b_{2}\delta_{e}+b_{3}\delta_{t}=0
    \label{eq:X4}
\end{equation}

\begin{center}
    $\delta_{e}=\frac{-b_{1}\alpha_{h}}{b_{2}+b_{3}K_{\delta}}$
\end{center}

Calculating the horizontal tail lift coefficient:

\begin{equation}
    C_{Lh}=a_{1}\alpha_{h}+a_{2}\delta{e}+a_{3}\delta{t}
    \label{eq:X5}
\end{equation}

The angle of attack can be determined replacing the equation \ref{eq:X4} and \ref{eq:X3} in the equation \ref{eq:X5}:

\begin{equation}
    \alpha_{h}=\frac{C_{Lh}}{a_{1}+a_{2}(\frac{b_{1}}{b_{2}+b_{3}K_{\delta}})}
    \label{eq:X6}
\end{equation}

The wing-body lift coefficient is calculated replacing the equation \ref{eq:X3} in the equation \ref{eq:X2}, and the angle of attack with the equation below:

\begin{equation}
    \alpha_{wb}=\frac{C_{Lwb}}{a_{wb}}+\alpha_{0}
    \label{eq:X7}
\end{equation}

\textbf{Automatic Tab Modeling}
\\
In order to do a more realistic modeling of the trimming and maneuver characteristics, an automatic tab was modeled in a simple way: it was considered that its angle is proportional to the elevator angle by a factor called $K_{\delta}$:

\begin{equation}
    \frac{\delta_{T}}{\delta_{wb}}=K_{\delta}
    \label{eq:X1}
\end{equation}

Finally, the trimming angle (incidence angle) of the horizontal tail is calculated:

\begin{equation}
    i_{h}=\alpha_{h}-\alpha_{wb}+\frac{\partial\varepsilon}{\partial\alpha}(\alpha_{wb}-\alpha_{0})
    \label{eq:X8}
\end{equation}

\textbf{Static Margin}
\\
The fixed-stick static margin ($K_{N}$) is defined as the difference between the neutral point and CG position:

\begin{equation}
    K_{N}=h_{n}-h
    \label{eq:X9}
\end{equation}

And the neutral point is calculated according to Melo (2013) and Irving (1966):

\begin{equation}
    h_{n}=h_{0}+\frac{V'_{H}}{1+F_{H}}\frac{a_{1}}{a_{WB}}(1-\frac{d\varepsilon_{H}}{d\alpha})+(z/c)(T/W)
    \label{eq:X10}
\end{equation}

Where:

\begin{center}
$F_{H}=\frac{S_{H}}{S}\frac{a_{1}}{a_{WB}}(1-\frac{d\varepsilon_{H}}{d\alpha})$ and $V'_{H}=\frac{l'_{H} S_{H}}{\bar{c} S}$
\end{center}

The free-stick static margin can be calculated replacing by the following expression:

\begin{equation}
    a'_{1}=a_{1}(1-\frac{b_{1} a_{2}}{b_{2} a_{1}})
    \label{eq:X11}
\end{equation}

\textbf{Maneuver}
\\
Two different maneuvers were considered in the analysis: the pull-up maneuver, when the pilot pull the column in order to achieve the maximum load factor determined by the flight envelope, and the speed maneuver, when a 10 \% increase/decrease in the equivalent airspeed is achieved only with pitching control.

\begin{itemize}
    \item[\textbf{a.}] \textbf{Load Factor Maneuver}\\
    To calculate the coefficients in this maneuver, firstly the aircraft $C_{L}$ is determined using equation \ref{eq:X2} multiplying the weight by the load factor. Then the tail and wing-body lift coefficients are calculated following the equations \ref{eq:X3} and \ref{eq:X2}.
    As this is a maneuver, the tail incidence angle is the same of the trimmed condition, so this new $C_{Lh}$ will be achieved changing the elevator angle ($\delta_{e}$):
    \begin{equation}
        \delta_{e}=\frac{C_{Lh}-a_{1} \alpha_{h}}{a_{2}}
        \label{eq:X12}
    \end{equation}
    Where $\alpha_{h}=i_{h}+\alpha{wb}+\frac{\partial\varepsilon}{\partial\alpha}(\alpha_{wb}-\alpha{0})$, and $i_{h}$ is the same of the trimmed condition
    With the elevator angle, the pilot force is calculated:
    \begin{equation}
        F_{P}=G[\frac{\rho V^{2}}{2}S_{e}\bar{c}_{e}(b_{1}\alpha_{h}+b_{2}\delta_{e}+b_{3}\delta_{t})]
        \label{eq:X13}
    \end{equation}
    \item[\textbf{b.}] \textbf{Speed Maneuver}\\
    The speed maneuver calculations are similar to the ones presented in the load factor maneuver. The difference is between the aircraft global lift coefficient, when the equation \ref{eq:X2} is used with a unitary load factor, but with a speed different from the trimmed one.
    Then the analysis is done also using equations \ref{eq:X3}, \ref{eq:X12} and \ref{eq:X13}, considering that the perturbed speed is used to calculate the pilot force.
    
    
\end{itemize} 