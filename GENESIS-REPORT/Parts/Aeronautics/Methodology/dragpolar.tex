 Results from previous CFD computations of the ERJ-145 wing-fuselage configuration were used for wave drag computation. The drag rise ($C_{D_{w}}$) (\ref{wavedrag}) was added to the correspondent subsonic drag polar point.


\begin{figure}[H] % Example image
\center{\includegraphics[width=400px]{Pictures/Aeronautics/Methodology/wavedrag.eps}}
\caption{Wave Drag}
\label{fig:wavedrag}
\end{figure}

Results for original and modified aircraft compressible drag polars are for $h=35000$ ft and $W=20000$ kg shown in figure. The reduction of parasite drag from aerodynamic improvements was greater than the increase from fuselage extension and winglets, leading to a global reduction of parasite drag. Induced drag was also reduced by the winglets. The wave drag was considered unchanged, since the modifications should not influence this component. The total drag reduction represents a reduction in required cruise thrust (figure).

\begin{figure}[H] % Example image
\center{\includegraphics[width=400px]{Pictures/Aeronautics/Methodology/compressible_polar.eps}}
\caption{Compressible Polar}
\label{fig:compressiblepolar}
\end{figure}

\begin{figure}[H] % Example image
\center{\includegraphics[width=400px]{Pictures/Aeronautics/Methodology/required_thrust.eps}}
\caption{Required Thrust}
\label{fig:requiredthrust}
\end{figure}
