% ---------- Takeoff ---------- %
\subsubsection{Takeoff}

Most variants of ERJ-145 have 9\degree and 22\degree flap positions available for takeoff. For ISA+0\celsius, there is a minimum climb gradient limitation with flaps 22. This happens for a takeoff weight of 20,000 kg. Above this point the aircraft must takeoff with flaps 9, that is not climb gradient limited, but increases the required takeoff field length. Some versions of ERJ-145 uses 9\degree and 18\degree flap positions for takeoff. This extends the weight where the climb gradient limitation occurs to 20,500 kg. Beyond this weight, takeoff must be done with flaps 9.

The ERJ-145G does not have this climb gradient limitation for operating at ISA+15\celsius and sea level, which is the condition for Santos Dumont airport, at Rio de Janeiro. This is possible because the combined effect of flaps 18 instead of 22 and winglets have increased the L/D in 9 \%, when climbing in takeoff configuration. So it is possible to takeoff with maximum takeoff weight and flaps 18, reducing the takeoff filed length in 200 m in this condition.

\begin{figure}[H] % Example image
\center{\includegraphics[width=400px]{Pictures/Aeronautics/Performance/fig1_takeoff_isa0.eps}}
\caption{Takeoff ISA 0\degree}
\label{fig:fig1_takeoff_isa0}
\end{figure}

\begin{figure}[H] % Example image
\center{\includegraphics[width=400px]{Pictures/Aeronautics/Performance/fig2_takeoff_isa15.eps}}
\caption{Takeoff ISA 15\degree}
\label{fig:fig2_takeoff_isa15}
\end{figure}

% ---------- Climb ---------- %
\subsubsection{Climb}

The time to climb to 35,000 ft was reduced from 30 min to 25 min. The climb ratio at the maximum altitude (37,000 ft) was increased in 51 \%.

\begin{figure}[H] % Example image
\center{\includegraphics[width=400px]{Pictures/Aeronautics/Performance/fig3_time2climb.eps}}
\caption{Time to climb}
\label{fig:fig3_time2climb}
\end{figure}

\begin{figure}[H] % Example image
\center{\includegraphics[width=400px]{Pictures/Aeronautics/Performance/fig4_rate_of_climb.eps}}
\caption{Rate of climb}
\label{fig:fig4_rate_of_climb}
\end{figure}

% ---------- Cruise ---------- %
\subsubsection{Cruise}

The ERJ-145 has 10 extra seats. In order to compare specific ranges of the original and modified aircraft in a fair way, the full PAX scenario was chosen. Both with fuel for a 400 nm mission after the climb phase. That is, 17,865 kg for the original ERJ-145 and 19,068 kg for the ERJ-145G. Specific range as a function of Mach number at 35,000 ft is shown in Figure \ref{fig5_specific_range}. The aerodynamic improvements were enough to overcome the weight increment due to structural mass and extra passengers, resulting in a lower overall fuel consumption when compared to the original aircraft.

\begin{figure}[H] % Example image
\center{\includegraphics[width=400px]{Pictures/Aeronautics/Performance/fig5_specific_range.eps}}
\caption{Specific range}
\label{fig:fig5_specific_range}
\end{figure}

% ---------- Hold ---------- %
\subsubsection{Hold}

Fuel consumption for holding at 10,000 ft is about 9 \% lower than the original aircraft. The optimum equivalent speed for minimum fuel flow is 170 kt for the ERJ-145G, while 180 kt for the original ERJ-145.

\begin{figure}[H] % Example image
\center{\includegraphics[width=400px]{Pictures/Aeronautics/Performance/fig6_fuel_flow_copy.eps}}
\caption{Fuel flow}
\label{fig:fig6_fuel_flow_copy}
\end{figure}

% ---------- Descent ---------- %
\subsubsection{Descent}

Time to descent is essentially the same for both aircraft. About 12 minutes to descent from 35,000 ft to sea level (Figure ).

\begin{figure}[H] % Example image
\center{\includegraphics[width=400px]{Pictures/Aeronautics/Performance/fig7_time2descent.eps}}
\caption{Time to descent}
\label{fig:fig7_time2descent}
\end{figure}

% ---------- Landing ---------- %
\subsubsection{Landing}

Landing performance is not altered.

% ---------- Payload x Range ---------- %
\subsubsection{Payload x Range}

The maximum zero fuel weight for was increased for the ERJ-145G to allow 60 passengers with 100 kg. Aerodynamic improvements have increased the range for the ERJ-145G, when comparing aircrafts with same payload. This might not be a fair comparison, though. Aircrafts with full PAX configurations are presented in next section.

\begin{figure}[H] % Example image
\center{\includegraphics[width=400px]{Pictures/Aeronautics/Performance/fig8_payload_range_max.eps}}
\caption{Payload x Range for maximum cruise}
\label{fig:fig8_payload_range_max}
\end{figure}

\begin{figure}[H] % Example image
\center{\includegraphics[width=400px]{Pictures/Aeronautics/Performance/fig9_payload_range_long.eps}}
\caption{Payload x Range for long cruise}
\label{fig:fig9_payload_range_long}
\end{figure}

% ---------- Block Time & Block Fuel ---------- %
\subsubsection{Block Time \& Block Fuel}

The modifications increased about 230 kg in basic weight. The weight of the extra 10 passengers is 1,000 kg. This factors contributed to reduce the range for aircraft at full occupation. The range is reduced to 1248 nm in maximum cruise speed (Figure \ref{fig10_block_fuel_max}) and 1323 nm in long range speed (Figure \ref{fig11_block_fuel_long}).

Market analysis has shown that this ranges are still sufficient for the missions performed and expected to be performed.

\begin{figure}[H] % Example image
\center{\includegraphics[width=400px]{Pictures/Aeronautics/Performance/fig10_block_fuel_max.eps}}
\caption{Block fuel for maximum cruise}
\label{fig:fig10_block_fuel_max}
\end{figure}

\begin{figure}[H] % Example image
\center{\includegraphics[width=400px]{Pictures/Aeronautics/Performance/fig11_block_fuel_long.eps}}
\caption{Block fuel for long range}
\label{fig:fig11_block_fuel_long}
\end{figure}

\begin{figure}[H] % Example image
\center{\includegraphics[width=400px]{Pictures/Aeronautics/Performance/fig12_block_time_max.eps}}
\caption{Block time for maximum cruise}
\label{fig:fig12_block_time_max}
\end{figure}

\begin{figure}[H] % Example image
\center{\includegraphics[width=400px]{Pictures/Aeronautics/Performance/fig13_block_time_long.eps}}
\caption{Block time for long range}
\label{fig:fig13_block_time_long}
\end{figure}
