In order to improve the ERJ 145's dispatchability, the team considered study the possibility of inclusion and change in time repairs of some items of MMEL.
During the EC it was decided that the team would focus on components which presents higher unscheduled removals, that is, smaller MTBUR.
The MTBUR considered have to be only for the last months and not the accumulated because it aggregates some problems which already have been solved or improved.
The Table \ref{tab:tableURRMEL} below shows the components with smaller MTBUR according to Service Performance Monthly Report (SPMR) of december, 2012 for ERJ-145 family.

% Table generated by Excel2LaTeX from sheet 'Plan1'
\begin{table}[H]
  \centering
  \tiny
  \caption{Components with the highest number of unscheduled removals.}
    \begin{tabular}{ccc}
    \toprule
    \textbf{ATA} & \textbf{Component} & \textbf{PN} \\
    \midrule
    23 - Communications & ATTENDANT HANDSET/CRADLE & 8/1/5483 \\
    23 - Communications & RMU PANEL & 7013270-965 \\
    23 - Communications & DIGITAL AUDIO PANEL & 7511001-939 \\
    24 - Electrical Power & MAIN GENERATOR,400A/28VDC & 30086-011 \\
    24 - Electrical Power & BATTERY 44AH NI-CD & 32248-001 \\
    24 - Electrical Power & APU STARTER GENERATOR & 23080-013A \\
    24 - Electrical Power & GENERATOR CONTROL UNIT GCU (APU) & 51525-014A \\
    26 - Fire Protection & ENGINE FIRE BOTTLE & 33600057-5 \\
    26 - Fire Protection & EXTINGUISHER, PORTABLE & 33600057-5 \\
    26 - Fire Protection & AFT FIRE BOTTLE 224 CI & 33700027-1 \\
    26 - Fire Protection & CONTROL MODULE & 33700027-1 \\
    26 - Fire Protection & AFT FIRE BOTTLE BAGGAGE 630 CI & 34600073-1 \\
    26 - Fire Protection & WASTE DISPOSAL FIRE EXTINGUISHER & 34600073-1 \\
    26 - Fire Protection & FIRE EXTINGUISHING BOTTLE & 30100050-5 \\
    26 - Fire Protection & WATER FIRE EXTINGUISHER & 30100050-5 \\
    26 - Fire Protection & HALONAIRE PORTABLE EXTINGUISHER & RT-A1200 \\
    26 - Fire Protection & TRANSMITTER ELT & 453-0150 \\
    26 - Fire Protection & FIRE EXTINGUISHER & 466090 \\
    27 - Flight Controls & UNIT, SPOILER CONTROL, EQUIPPED  & 140-05321-611 \\
    27 - Flight Controls & FLAP TRANSMISSION BRAKE  & 363500-1003 \\
    27 - Flight Controls & ACTUATOR-AILERON  & 418800-1007 \\
    27 - Flight Controls & FLAP ELETRONIC CONTROL UNIT  & 363100-1011 \\
    27 - Flight Controls & HORIZONTAL STABILIZER ACTUADOR  & 362200-1015 \\
    27 - Flight Controls & FLAP ELETRONIC CONTROL UNIT  & 363100-1005 \\
    27 - Flight Controls & AOA SENSOR  & 0861DT5 \\
    27 - Flight Controls & STALL PROTECTION COMPUTER  & 0020BN4 \\
    27 - Flight Controls & MODULE ASSY TRIM CONTROL  & 145-34177-405 \\
    27 - Flight Controls & SPOILER CONTROL UNIT EQUIPPED  & 145-33022-409 \\
    27 - Flight Controls & BRAKE VALVE AERODINAMIC  & 360550-1003 \\
    27 - Flight Controls & HOR. STAB.,CONTROL UNIT  & 362100-1013 \\
    27 - Flight Controls & GROUND SPOILER VALVE  & 360450-1001 \\
    29 - Hydraulic Power & HYDRAULIC PUMP (MODEL PV3-032-11)  & 971808 \\
    29 - Hydraulic Power & PUMP, DC ELECTRIC MOTOR DRIVEN HYDRAULIC  & 971533 \\
    31 - Indication/Recording Systems & DISPLAY UNIT DU-870 - EICAS/PFD/MFD & 7014300-901 \\
    31 - Indication/Recording Systems & CLOCK,DIGITAL & GMT4190-011 \\
    32- Landing Gear & TEMPERATURE SENSOR ASSY,BRAKE  & 0132AFU-2 \\
    32- Landing Gear & MAIN WHEEL ASSY  & 3-1641 \\
    32- Landing Gear & BRAKE CONTROL UNIT  & 42-951-6 \\
    32- Landing Gear & BRAKE ASSY  & 2-1707-1 \\
    32- Landing Gear & NOSE WHEEL ASSY 6.50-8  & 3-1551 \\
    32- Landing Gear & BRAKE ASSY  & 2-1707 \\
    32- Landing Gear & LANDING GEAR ELETRONIC UNIT  & 355-022-003 \\
    32- Landing Gear & MAIN WHEEL ASSY  & 3-1631-1 \\
    32- Landing Gear & WHEEL SPEED TRANSDUCER  & 140-197 \\
    32- Landing Gear & LEG OLEO STRUT ASSY, (LEFT)  & 2309-2002-623 \\
    34 - Navigation & RADIO ALTIMETER & 7001840-937 \\
    \bottomrule
    \end{tabular}%
  \label{tab:tableURRMEL}%
\end{table}%


During the EP, the team was divided in two fronts. The first was the study of the change from repair categories smaller (A and B) to greater (C) of some items already existing in the MMEL and present in table \ref{tab:tableURRMEL}. The second front would study items that could be included in MMEL, considering components which would be exchanged.
The items from table \ref{tab:tableURRMEL} that already are in the MMEL are show in the table \ref{tab:includedMMEL}. The component "portable fire extinguishers" in this table include all kind of portable fire extinguishers which are required by FAR.

% Table generated by Excel2LaTeX from sheet 'Plan1'
\begin{table}[H]
  \centering
  \tiny
  \caption{Items from table \ref{tab:tableURRMEL} which are included in MMEL.}
    \begin{tabular}{ccccc}
    \toprule
    \textbf{Component} & \textbf{Item MMEL} & \textbf{Category} & \textbf{Number Installed} & \textbf{Number Required for Dispatch} \\
    \midrule
    ATTENDANT HANDSET/CRADLE & 23-31-03 & B     & -     & - \\
    \multirow{2}[3]{*}{APU STARTER GENERATOR} & 24-34-01 & C     & 1     & 0 \\
          & 24-34-01 & C     & 1     & 0 \\
    PORTABLE FIRE EXTINGUISHERS & 26-23-01 & D     & -     & - \\
    HYDRAULIC PUMP (MODEL PV3-032-11) & 38654 & C     & 2     & 0 \\
    \multirow{2}[3]{*}{CLOCK,DIGITAL} & 31-21-01 & C     & 1     & 0 \\
          & 31-21-01 & A     & 1     & 0 \\
    TEMPERATURE SENSOR ASSY,BRAKE & 32-40-01 & C     & 4     & 3 ou 0 \\
    BRAKE ASSY & 32-49-00 & C     & 8     & 4 \\
    \multirow{3}[5]{*}{DISPLAY UNIT DU-870 - EICAS/PFD/MFD} & 34-22-01 & B     & 5     & 1 \\
          & 34-22-01 & C     & 2     & 1 \\
          & 34-22-01 & C     & -     & 0 \\
    \multirow{3}[5]{*}{RADIO ALTIMETER} & 34-31-00 & A     & 1     & 0 \\
          & 34-31-00 & C     & 2     & 1 \\
          & 34-31-00 & A     & 2     & 0 \\
    \bottomrule
    \end{tabular}%
  \label{tab:includedMMEL}%
\end{table}%

Considering the components in the table \ref{tab:includedMMEL} and the information displayed on it, was founded that it would be impossible change the repair categories of these components. The majority of components that was considered in EC with great potential to change the time repairs already were in the higher time repair possible. It isn't possible to change a repair category from C to D because the difference between the times is large (from 10 to 120 days) and an aeronautical authority would not permit this. Besides, all items in the table \ref{tab:includedMMEL} that could be a D category (in this case only the portable fire extinguishers) already are included in this category. The components with a time repair category A already have an optional case in which the time repair is B or C as showed in the table \ref{tab:includedMMEL} and in the figures \ref{fig:MMELExample} and \ref{fig:MMELExample2}. All because ERJ-145 is an aircraft which have been in the market for a long time, so to many changes had already been made to improve MMEL.

\begin{figure}[H]
	\centering
	\includegraphics[width=400px]{Pictures/3MASU/fig1.eps}
    \caption{Procedures in MMEL (FAA, revision 15) for clocks failure.}
	\label{fig:MMELExample}
\end{figure}

\begin{figure}[H]
	\centering
	\includegraphics[width=400px]{Pictures/3MASU/fig2.eps}
	\caption{Procedures in MMEL (FAA, revision 15) for radio altimeter system failure.}
	\label{fig:MMELExample2}
\end{figure}

Each component which has been analyzed for replacement in accord with the MTBF also has been analyzed for inclusion in MMEL. The table below shows these items.
Analyzing each one separately the team reached the conclusion that any one of them could not be included in MMEL. Following there are the explanation for this. For a better organization the components are handled in groups hereafter considering only some components as example - for others which didn't appears the same explanations could be applied. For a better description and explanation about the components see the chapter regarding the corresponding system.

% Table generated by Excel2LaTeX from sheet 'Plan1'
\begin{table}[htbp]
  \centering
  \tiny
  \caption{Replacement components list.}
    \begin{tabular}{cccc}
    \toprule
    \textbf{PN PRIME} & \textbf{Description ( SPES )} & \textbf{ATA} & \textbf{System} \\
    \midrule
    \textbf{Text} & \textbf{Text} & \textbf{Number} & \textbf{Name} \\
    816604-2 & PACK VALVE & 21    & AIR CONDITIONING \\
    145-25698-407 & AIR DISTRIBUTION VALVE & 21    & AIR CONDITIONING \\
    7021170-951 MOD.A & GC-500 (FLIGHT GUIDANCE CONTROLLER ) & 22    & AUTO FLIGHT \\
    23080-013A & STARTER GENERATOR - APU & 24    & ELECTRICAL POWER \\
    442CH1 & BATTERY, 44AH,NI-CD & 24    & ELECTRICAL POWER \\
    SB200 & STATIC INVERTER & 24    & ELECTRICAL POWER \\
    51525-014A & GCU - GENERATOR CONTROL UNIT (APU) & 24    & ELECTRICAL POWER \\
    501-1228-04 & BACKUP BATTERY & 24    & ELECTRICAL POWER \\
    363100-1011 & FECU - FLAP ELECTRONIC CONTROL UNIT & 27    & FLIGHT CONTROLS \\
    0861DT5 MOD.1 & AAS - ANGLE OF ATTACK SENSOR & 27    & FLIGHT CONTROLS \\
    418800-9007 & HYDRAULIC AILERON ACTUATOR (WING) & 27    & FLIGHT CONTROLS \\
    363110-1009 & FLAP SELECTOR LEVER & 27    & FLIGHT CONTROLS \\
    145-33022-409 & SPOILER CONTROL UNIT (EQUIPPED) & 27    & FLIGHT CONTROLS \\
    PV3-032-11 & ENGINE DRIVEN HYDRAULIC PUMP & 29    & HYDRAULIC \\
    971533 MOD A & ENGINE DC ELECTRIC MOTOR DRIVEN HYDRAULIC PUMP & 29    & HYDRAULIC \\
    0132AFU-1 & BRAKE TEMPERATURE SENSOR & 32    & LANDING GEAR \\
    42-951-6 & BRAKE CONTROL UNIT  & 32    & LANDING GEAR \\
    140-197 & WHEEL SPEED TRANSDUCER  & 32    & LANDING GEAR \\
    B-6245-305 & DOME LIGHT & 33    & LIGHTS \\
    30-2510-3 & EMERGENCY LIGHT (WING/FUSGL) & 33    & LIGHTS \\
    2LA 006 265-00 & COCKPIT LIGHTS & 33    & LIGHTS \\
    30-2438-3 & STABILIZER NAVIGATION LIGHT & 33    & LIGHTS \\
    30-2437-2 & NAVIGATION LIGHT ASSY (GREEN) & 33    & LIGHTS \\
    30-2437-1 & NAVIGATION LIGHT ASSY (RED) & 33    & LIGHTS \\
    60-5096-1 & STROBE LIGHT POWER SUPPLY (WING AND STABILIZER TIP) & 33    & LIGHTS \\
    30-2675-1 & UPPER RED STROBE LIGHT (ANTI-COLLISION) & 33    & LIGHTS \\
    5434-00 & ELPU (EMERGENCY LIGHT POWER UNIT) & 33    & LIGHTS \\
    30-2827-1 & WING TIP WHITE STROBE LIGHT & 33    & LIGHTS \\
    189-720 & INVERTER LAMP (CABIN PAX AND LAVATORY) & 33    & LIGHTS \\
    5500-01 & READING LIGHT ASSY & 33    & LIGHTS \\
    50-0291-1 & LANDING LIGHTS NLG & 33    & LIGHTS \\
    Q4681 & LANDING LIGHTS WINGS & 33    & LIGHTS \\
    D6490-17 & TAXI LIGHTS NLG & 33    & LIGHTS \\
    816624-10 & THERMOSTAT-FAN AIR CONTROL,ENGINE & 36    & PNEUMATIC \\
    816610-1 & FAN AIR VALVE REGULATION & 36    & PNEUMATIC \\
    163500-109 & WINDSHIELD LH & 56    & WINDOWS \\
    163500-110 & WINDSHIELD RH & 56    & WINDOWS \\
    \bottomrule
    \end{tabular}%
  \label{tab:repComponentList}%
\end{table}%


ATA 27 - Fight Controls


Some components like Flap Electronic Control Unit and Spoiler Control Unit should be redundant  to be included in MMEL. But it would be necessary an implementation in the electronic and avionic systems, which means the development of new systems - and high costs. Furthermore, it would be required a new unit of each one, meaning the necessity of space which there is not available in the ERJ-145 family.


ATA 32 - Landing Gear


The requires above described are also applied for brake temperature sensor. The figure below shows the space problem: each main landing gear have two spaces for two sensors (number 40 indicated in the figure): one for the inboard wheel and other for the outboard wheel. As noted from figure \ref{fig:breakEx3} the system could be considered redundant at first sight: the inboard and outboard wheels from left main landing gear - for example - theoretically are submitted to the same temperature. Then, if one sensor fails, the other could be used to indicate the temperature of the wheel without sensor. But the system logic compares the measures from sensors to avoid a wrong and an absurd indication from one of them. Therefore the system can not be considered redundant and can not be included in the MMEL.

The angle of attack sensor, the wheel speed transducer and the thermostat-fan air control have a similar logic for measure and can not be included in MMEL for the same reason.

\begin{figure}[H] % Example image
\center{\includegraphics[width=400px]{Pictures/3MASU/fig3.eps}}
\label{fig:breakEx1}
\end{figure}

\begin{figure}[H] % Example image
\center{\includegraphics[width=400px]{Pictures/3MASU/fig4.eps}}
    \caption{Brake temperature sensor position.}
\label{fig:breakEx3}
\end{figure}





ATA 33 - Lights

The MMEL already includes some conditions for aircraft dispatch with fail of navigation lights, taxi lights, landing lights (the following figure). However, the system changes and, for this case, the MMEL items have to be recertified.

\begin{figure}[H]
	\centering
	\includegraphics[width=400px]{Pictures/3MASU/fig5.eps}
	\caption{Example of items from ATA 33 present in MMEL.}
	\label{fig:MMELExample3}
\end{figure}

Conclusion

The MMEL of ERJ-145 family is already refined because it is in the market for a long time (more than 15 years). So there is no more new items or improvements that could be incorporated, considering the components studied.

The unique changes that could be made in the MMEL of ERJ-145 involve only redundancy. Some redundancy are optional - the operator can choose, like the quantity of portable fire extinguishers, or the quantity of some systems (single or dual), like radio altimeter. These options are already available for customers. But other redundancy which could be made are not possible to be implemented because of space restrictions and the necessity of changes in several systems - which means a high cost for development.
