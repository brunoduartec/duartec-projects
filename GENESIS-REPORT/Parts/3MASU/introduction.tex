Maintenance and Maintainability are two important parameters that influence directly the airplane life cycle and operational cost. They have to be considered in a new airplane design so as in an airplane modification. These considerations are essential for the design success.

Maintainability means the ease with which the airplane can be safely repaired in the least amount of time. It provides design characteristics that facilitates the airplane maintenance and results in minimum maintenance costs in monitoring, dispatchability, availability, servicing and other operations. Maintenance are corrective, predictive and preventive actions for replace or maintain the airplane in airworthy conditions. It includes inspections, servicing, modifications, tests and repairs.

It is necessary to have in mind that maintainability does not eliminate the needs for aircraft maintenance and it should not be overestimated. This could affect indirectly the costs-benefits. Maintainability and optimized procedures of maintenance turn the airplane economically viable and more competitive.

For EMBRAER ERJ 145 family the Direct Maintenance Cost (DMC) increased strongly in last years. This happened in part because systems technologies are older and maintenance always increases with the airplane life time. The EMBRAER ERJ-145 was conceived in 1990 decade and its basics systems became obsolete despite some modifications that had been incorporated. But the main cause of DMC increase is related with the propulsion system maintenance cost. The propulsion system has an expensive maintenance due to it obsolescence and the way that the supplier contract was made. It represents approximately 50 \% of the maintenance cost.

Other systems from EMBRAER ERJ 145 airplane also have an expensive maintenance due to supplier contracts, like avionics and landing gear systems. Further, the high prices of EMBRAER limited life parts can also be pointed as causes of the high maintenance cost.

According to ERJ 145's Owner's Operators Guide nowadays the ERJ-145's maintenance program is optimized for operators flying 2,500 flight hours (FH) and 2,500 flight cycles (FC) per year. Consequently operators with lower or higher utilizations will find maintenance planning more complicated. The ERJ 145's Owner's Operators Guide also says that the ERJ 145 has a base maintenance cycle of 20,000 FH and many aircraft have been though their first base cycle. The ERJ 145 structure was designed for three or even four base maintenance cycles during their service lives. The points described above are better discussed in the next items.

First, important concepts are presented and after that, the possibilities of DMC reduction are shown.
