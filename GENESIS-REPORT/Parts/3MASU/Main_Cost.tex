
The Maintenance Cost is composed by Direct Maintenance Cost (DMC) and the Indirect Maintenance Cost (IMC). The figure below shows the structure of Maintenance Costs.

\begin{figure}[H]
	\centering
	\includegraphics[width=400px]{Pictures/3MASU/dmc.eps}
	\caption{The structure of Maintenance Costs}
	\label{fig:MaintenanceCost}
\end{figure}

The Indirect Maintenance Cost (IMC) is the maintenance cost which contributes to the program costs through infrastructure, stock parts, logistics, human resources, administration, engineering, facilities, etc.

The Direct Maintenance Cost (DMC) of commercial aircraft is one of the major aeronautical ownership costs. The DMC is composed by the costs of maintenance performance, product performance, repair, parts and labor. These costs are needed to maintain the aircraft flying and operating in a safety flight and are used in aircraft engines, systems and structures on scheduled and unscheduled basis.

To Calculate the DMC is necessary to know a lot of information about the costs of the aircraft maintenance. Among them are: the costs of line checks, A-checks, Base checks, the costs of landing gear, wheels, brakes, APU, thrust reversers, engines and LRU component support and maintenance.

According to the 2012 Owner's guide \cite{OwnersGuide} and information provided by EMBRAER employees the value estimated of DMC for ERJ145 family is around US\$ 600.00 per flight hour.
The figure below shows the project aircraft's DMC relative to competitors� values:

\begin{figure}[H]
	\centering
	\includegraphics[width=400px]{Pictures/3MASU/DMCpassenger.eps}
	\caption{Direct Maintenance Cost (DMC) Competitors }
	\label{fig:DMCCompetitorsFig}
\end{figure}

\begin{table}[htbp]
  \centering
  \caption{Direct Maintenance Cost (DMC) Competitors}
    \begin{tabular}{ccc}
    \toprule
    \textbf{Aircraft} & \textbf{DMC [US\$]} & \textbf{US\$/FH/PAX} \\
    \midrule
    ERJ145 & 600   & 12 \\
    EMB 170 & 783   & 9.79 \\
    EMB 175 & 837   & 9.51 \\
    EMB 190 & 771   & 6.76 \\
    EMB 195 & 811   & 6.65 \\
    CRJ 100 & 743   & 14.86 \\
    CRJ 200 & 783   & 15.66 \\
    CRJ 700 & 838   & 10.74 \\
    CRJ 900 & 940   & 10.44 \\
    \textbf{ERJ-145 Genesis} & 510   & 8.5 \\
    \bottomrule
    \end{tabular}%
  \label{tab:DMCCompetitors}%
\end{table}%

According to the figure, the ERJ145 has the lowest DMC cost in relation to its principal competitor in the 50-seats aircrafts and the third relative value per seat , but is necessary to reduce the operational cost to maintain the airplane flying and with a cost cheaper to operate.

Since some of them are new projects and have a higher capacity than the 50-seats the DMC divided by the number of passengers is lower.
Nowadays, the costs of DMC to the ERJ145 family are very expensive for its sector and the operation is unfeasible to some operators because the costs of engine maintenance and the fuel consumption are too expensive and not profitable.

From this situation, it will be taken a study that will assume the 60-seats on the ERJ145 Genesis, this quantity contributed to reduce de DMC and the CASM per passenger.
After this analysis the new DMC making the ERJ145 Genesis more competitive against its competitors with 70-seats and the aim of the maintenance team is to reduce 15 \% of the DMC to achieve a 4.50 \% reduction in the Operational Cost. The relation between DMC and Operational Cost is going to be seen below.

\subsubsection{Operating Cost}

The Total Operating Cost (TOC) is based in all cost involved in the operation of the aircraft. The TOC is divided in two parts: direct and indirect costs.

The Indirect Operating Cost (IOC) is composed by the indirect costs per flight associated with the aircraft operation, which includes the passenger and cargo organization, flight service, general management and others.

The Direct Operating Cost (DOC) is the costs per flight hour directly associated with aircraft operation. Mainly involves the costs of fuel, maintenance, crew, leasing, depreciation, aircraft's insurance and airport fees.It is important to consider the same method used to calculate the DMC to calculate the  Cash Operating Cost(COC).

The COC is composed by Landing Fee, Aircraft Handling, Maintenance, Crew flight and Fuel. The value of COCestimated is US\$ 2,000.00/h and this contributors are shown below:

\begin{figure}[H]
	\centering
	\includegraphics[width=400px]{Pictures/3MASU/coc.eps}
	\caption{Cash Operating Cost}
	\label{fig:DMCost}
\end{figure}
