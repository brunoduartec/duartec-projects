
The major factor that influence the maintenance is the human factor. Since the repair is made by humans, it is needed to understand what influence in their task, and try to take some corrective actions.

One of the problems is the access to the system that need to be repaired, not the ones outside of the plane, but that ones who need to remove all the interior to do the inspection or repair. According to mechanics of the Service Centre of Gavi�o Peixoto's, they have an inspection task to check one system that is behind the galley, so they need to dismounted this galley, check the system and mount it again. To perform this task they spend two hours in the checking and about ten to twelve hours to dismount and mount the galley. And this can cause another problem with changing the focus of technical during the perform of the task.

Other problem is in knowing what component has failed and where is it, because not all the indications are recorded in the ERJ 145 Central Maintenance Computer (CMC), they lose a lot of time in interpreting the small quantity of data, finding the procedures and starting the task.

Besides the human factor, other problem is the high costs to maintain spare parts, and, depending in where these parts are stored, it can cause some unwanted wait to receive them.
In conclusion, to ensure a reliable maintenance, operators need to guarantee that the conditions where the mechanics work are the best possible, reducing labor fatigue, and hazardous environments. The aircraft manufacture needs to design easy access to components and plan a clear Maintenance Review Board so that the operators and the mechanics can understand what a task is and how it is done.
